\section{Transitivity operations}\label{trans-op}

This section discusses operations that bring about some change in the \isi{transitivity} of a verb. The \isi{transitivity operations} distinguish between \isi{argument structure} modifying and \isi{argument structure} preserving operations, just as the \isi{valency alternations} distinguish between argument-structure modifying and preserving alternations (see  \sectref{valclass}). Some of the operations change the semantics of a predicate by introducing or removing certain arguments, while other operations are related to requirements of information flow, thereby promoting or demoting certain participants syntactically. Not all of these operations are overtly marked, and detransitivizations have no dedicated marking at all.  

This section is organized as follows: \sectref{detrans} discusses two unmarked detransitivations: the passive and the antipassive, §\sectref{caus}—\sectref{middle} discuss the \isi{transitivity operations} that are overtly marked on the verb: the causative, the \isi{benefactive}, the reflexive, the  reciprocal, and the middle. Most of these operations involve the attachment of a function verb  (V2) to the lexical verb (see also Chapter \ref{verb-verb}). \sectref{t-sap} deals with a pattern of complex predication that is required by certain pragmatically marked scenarios (with referentially high P or T arguments). Finally, \sectref{stemchange} introduces some stem alternations that must have been productive valency-increasing morphology  in an earlier stage of Kiranti. As these stem alternations are marked, they do not fit into the previous section (on unmarked alternations); and since they are no longer productive, they do not fit into the section on productive \isi{transitivity operations} either. But as they provide a glimpse into the history of \isi{transitivity operations} in Yakkha, they are included in this section.

\subsection{Unmarked detransitivizations}\label{detrans}

The detransitivizations, although morphologically unmarked and thus  formally identical to lability (see \sectref{labile}), have to be carefully distinguished from lability, because the arguments are merely demoted with regard to some of their morphosyntactic properties; they are not removed semantically. The detransitivizations are syntactic operations; they are less restricted, whereas the \isi{labile verbs} build a closed lexical class. The formal identity of \isi{detransitivization} and lability may lead to overlaps and ambiguities (discussed below).

%This section is structured as follows: \sectref{detr-pass} introduces the passive detransitivizations, \sectref{detr-antip} the antipassive, and \sectref{detrans-polite} discusses the syncretism between detransitivized forms and verbs with first person plural arguments, which was probably motivated historically by a politeness strategy.

\subsubsection{The passive detransitivization}\label{detr-pass}

Transitive verbs can be intransitively inflected, and thus receive a passive reading. In the passive, the P argument is the pragmatically salient argument and gets promoted to the intransitive subject syntactically, i.e. it becomes the sole agreement triggering argument. The  A argument can still be expressed, but it does not trigger agreement and has to be in an oblique \isi{case} (the \isi{ablative} \emph{=bhaŋ \ti =haŋ}). I cannot make strong claims about the naturalness of obliquely expressed A arguments in the passive, as there is not a single example of this use of the \isi{ablative} in my recorded natural language data.\footnote{Remember that clauses with all arguments overtly expressed are exceedingly rare in Yakkha.}  It is possible that the \isi{ablative} is calqued upon the \ili{Nepali} postposition \emph{dvārā} \rede{by means of}.\footnote{\citet[123]{Ebert1997A-grammar} provides examples of similar uses of the \isi{ablative} in the closely related Athpare language, but it seems to be a marginal option in Athpare, too.}  In  \Next and \NNext, the (a) examples illustrate the formal properties of the passive, with the corresponding transitive active clauses in the (b) examples.

\ex. \ag. na wa magman=bhaŋ sis-a=na\\
			this chicken  Magman{\sc =abl} kill{\sc [3sg]-pst=nmlz.sg}\\
		\rede{This chicken was killed by Magman.}
	\bg. magman=ŋa na wa sis-u=na  \\
		Magman{\sc =erg}	this chicken  kill{\sc -3.P[pst]=nmlz.sg}\\
		\rede{Magman killed this chicken.}
		
	\ex. \ag.  na paŋ a-phu=bhaŋ cog-a=na \\
	this house  {\sc 1sg.poss-}elder\_brother{\sc =abl} make{\sc [3sg]-pst=nmlz.sg}\\
	\rede{This house was built by my elder brother.}
	\bg. a-phu=ŋa na paŋ cog-u=na \\
	 {\sc 1sg.poss-}elder\_brother{\sc =erg} this house make{\sc -3.P[pst]=nmlz.sg}\\
	\rede{My elder brother built this house.}	


As for \isi{three-argument verbs}, in verbs of the \isi{double object frame}, both T and G arguments can be promoted to subject  status, illustrated in \Next. Just as in the monotransitive verbs, the inflection changes to intransitive in the passive voice. 

\ex. \ag. na phuŋ nda=bhaŋ khut-a=na\\
this  flower {\sc 2sg=abl} bring{\sc [3sg]-pst=nmlz.sg}\\
\rede{This flower was brought by you.} (T → S)
\bg. ka phuŋ (nda=bhaŋ) khut-a-ŋ=na\\ 
{\sc 1sg} flower {\sc 2sg=abl} bring{\sc -pst-1sg=nmlz.sg}\\
\rede{I was brought a flower by you.} (G → S)
 

The situation is different for \isi{three-argument verbs} of the \isi{indirective frame} (with G in the \isi{locative}) and the \isi{secundative frame} (with T in the instrumental). Here, only the argument in the  unmarked \isi{nominative} can be promoted to subject. The passive of the \isi{indirective frame} is shown in \Next.

\ex. \ag. on siŋ=be thund-a=na\\
horse  tree{\sc =loc} tie{\sc [3sg]-pst=nmlz.sg}\\
\rede{The horse was tied to the tree.}
\bg. babu=ŋa on siŋ=be thund-u=na\\
boy{\sc =erg} horse  tree{\sc =loc}  tie{\sc -3.P[pst]=nmlz.sg}\\
\rede{The boy tied the horse to the tree.}


The passive is typically employed when the transitive object is more salient than the subject in a particular section of discourse. In \Next from a narrative, the passive is not motivated by a topical patient, but by the unknown identity of the agent, as the whole sentence was uttered in surprise and all elements in it were equally new. The story is about a bride who wants to take some megaliths from her maternal home to her in-laws' village. In the course of the narration, the girl has to solve various tasks and is confronted with many difficulties, but finally she succeeds: one morning, the people from the in-laws' village find one of the rocks in their village, rocks that actually belong miles further downhill. As they have no clue about how this happened, the passive is used to avoid reference to the agent in the utterance in \Next. 

\exg.namthaluŋ=beʔ=na  luŋkhwak nhe  ket-a-ma, eko!\\
Namthalung\_rock{\sc =loc=nmlz.sg} stone here bring\_up{\sc [3sg]-pst-prf}, one\\
\rede{The rock of Namthalung was brought here, one (of them)!}\source{37\_nrr\_07.085}


The passive is also used when the speaker wants to be unspecific about the reference of the agent, comparable to using the indefinite pronoun \emph{one} in English or \emph{man} in German. In  \Next[a], no overt A argument is possible. Distinct agreement and \isi{negation} suffixes as well as the choice of the nonpast allomorph \emph{-men} in \Next[a] (vs.  \emph{-wa}) show that the inflection is intransitive (compare with transitive \Next[b]). Note, however, that the form in (a) can also have a first person nonsingular A reading, discussed below in \sectref{detrans-polite}.

\ex. \ag. i=ya=ca cok-ma n-ya-me-n=na\\
what{\sc =nmlz.nc=add} do{\sc -inf} {\sc neg-}be\_able{\sc [3sg]-npst-neg=nmlz.sg}\\
\rede{One cannot do anything (about it).} or\\
\rede{We cannot do anything (about it).}
\bg. (kaniŋ) i=ya=ca cok-ma n-yas-wa-m-nin=na \\
{\sc 1pl[erg]} what{\sc =nmlz.nc=add} do{\sc -inf} {\sc neg-}be\_able{\sc -npst[3.P]-1pl.A-neg.pl=nmlz.sg}\\ 
\rede{We cannot do anything (about it).}


There are several verb stems that are ambiguous between inchoative and passive readings when they are detransitivized, i.e. between \isi{argument structure} modifying and \isi{argument structure} preserving detransitivizations. This ambiguity is found for all events that can be conceptualized either with or without external causation, for instance \emph{kept} \rede{stick, glue}, \emph{ek} \rede{break},  \emph{pek} \rede{shatter}, \emph{hos} \rede{open}, \emph{her} \rede{dry}, \emph{lond} \rede{come/take out}. The context provides clues about whether an agent is implied or not. In \Next, the ambiguity can be resolved by including an oblique-marked A argument to distinguish the passive in (a) from the ambiguous reading in (b). 

\ex.\ag.na jhyal phuaba=bhaŋ peg-a=na\\
this window  last\_born\_male{\sc =abl}  shatter{\sc [3sg]-pst=nmlz.sg}\\
\rede{This window was shattered by Phuaba (the youngest of the brothers).}
\bg.na jhyal imin peg-a=na?\\
this window how shatter{\sc [3sg]-pst=nmlz.sg}\\
\rede{How did this window break?} or \\
\rede{How was this window broken?}

Another way to distinguish inchoative from passive readings is the specification of the predicate by a second verbal stem (V2). In \sectref{labile}, it was mentioned that certain V2 in complex predicates are sensitive to \isi{transitivity}. Some V2 occur only in inchoative readings, and thus they rule out a  passive reading. Consider \Next, and how the meaning changes from the caused \isi{motion} of a liquid in \Next[a] and \Next[b] to spontaneous \isi{motion} in \Next[c].

\ex.\ag.chemha tuks-a=ha\\
liquor pour{\sc [3sg]-pst=nmlz.nc} \\
\rede{The liquor was poured.}
\bg.uŋ=ŋa khorek=pe maŋcwa tuks-u=ha\\
{\sc 3sg=erg} bowl{\sc =loc} water pour{\sc -3.P[pst]=nmlz.nc}\\
\rede{He poured the water into the bowl.}
\bg.chemha tuks-a-khy-a=ha\\
liquor pour{\sc [3sg]-pst-V2.go-pst=nmlz.nc} \\
\rede{The liqour spilled/ran over.} not: *\rede{The liquor was poured.}


\subsubsection{The antipassive detransitivization}\label{detr-antip}

The antipassive \isi{detransitivization}, just like the passive, is expressed simply by intransitive inflection. Potential ambiguities are  resolved by the context. Antipassives are  found in many Kiranti languages, e.g. in \ili{Puma}, in \ili{Chintang}, in Belhare \citep{Bickel2011Detrans, Schikowski2013_Thesis}, in \ili{Bantawa} \citep[221ff.]{Doornenbal2009A-grammar}\todo{Do not use ff. Give full page ranges}, and in Athpare \citep[122]{Ebert1997A-grammar}. In Yakkha, as in most Kiranti languages, the P argument may still be expressed overtly in the antipassive \isi{detransitivization}. The P is in the \isi{nominative}, just as it would be in transitive clauses, but the verb is inflected intransitively and agrees only with the agent, whose \isi{case} marking changes from \isi{ergative} to \isi{nominative}.

The choice of this construction is related to the referential status of the P argument. If it is non-referential, indefinite or non-specific, the odds for the use of the antipassive are higher. General statements, for instance,  tend to be in the antipassive. If one uses a detransitivized verb in a question as in \Next, it will be understood as inquiring about the habit of a person, not as a question about a specific situation. This is why it is not possible to anchor this clause temporally (except for purposes of irony). 

	\exg.\label{ex-raksi}(*hen=go) chemha uŋ-meʔ=n=em ŋ-uŋ-meʔ=n=em? \\
(*today{\sc =top}) liquor  drink{\sc [3sg]-npst=nmlz.sg=alt} {\sc neg-}drink{\sc [3sg]-npst=nmlz.sg=alt}\\
\rede{Does she drink raksi or not (*today)?}

If the statement is made rather about the manner of the event than about the result, the antipassive is likely to be used as well. Compare detransitivized \emph{cokma} \rede{do, make} and \emph{cekt} \rede{talk, speak} in \Next[a] and  \NNext[a] with the transitive uses in  \Next[b] and \NNext[b].\footnote{The stem alternations in (b) are phonologically triggered; they do not encode \isi{transitivity}.}

\ex. \ag. khatniŋ=go liŋkha ekdam cog-a=nuŋ cog-a=nuŋ\\
but{\sc =top} a\_clan  very do{\sc [3sg]-pst=com} do{\sc [3sg]-pst=com}\\
\rede{as the Linkha man worked and worked} \source{11\_nrr\_01.008}
\bg. uŋ=ŋa na paŋ cog-uks-u=na\\
   {\sc 3sg=erg} this house   do{\sc -prf-3.P[pst]=nmlz.sg}\\
\rede{He has made this house.}

\ex. \ag. menuka=le ucun=nuŋ ceŋ-meʔ=na\\
Menuka{\sc =ctr} nice{\sc =com} speak{\sc [3sg]-npst=nmlz.sg}\\
\rede{Menuka talks nicely!}
\bg.nnakha nak-se            ŋ-gheʔ-me=hoŋ          ceʔya n-jekt-wa\\
		those ask{\sc -sup} {\sc 3pl-}go{\sc -npst=seq}	matter	{\sc 3pl.A-}speak{\sc -npst[3.P]}\\
	\rede{After they go there to ask (for the girl), they discuss the matter.} \source{25\_tra\_01.007}


The antipassive is also found in procedural descriptions, as the speaker refers to general facts rather than to specific situations. Example \Next[a] provides  a description of the long and highly formalized wedding procedure. It may also play a role whether an event has  already been introduced in a text. In \Last[b], which was uttered shortly before \Next[a] in the same recording session, the verb \emph{cekma} \rede{talk} is introduced with the transitive inflection and with an overt P argument \emph{ceʔya} \rede{matter}, but when it is taken up again, the detransitivized form is used and the object is omitted (see \Next[a]). Similarly, the stem \emph{lend \ti lem} \rede{exchange} had been introduced in the transitive inflection and taken up in the intransitive form in  \Next[a]. In \Next[b], though, the object  \emph{sala} cannot be omitted for lexical reasons; the verb \emph{sala lend} \rede{discuss} (lit. \rede{exchange matters}) is a  fixed expression.

\ex.\ag. kuʈuni=ci panca bhaladmi=ci jammai sala   n-lem-me, n-jeŋ-me\\
	matchmaker{\sc =nsg} an\_official\_rank  respected\_elder{\sc =nsg} all matter {\sc 3pl-}exchange{\sc -npst}, {\sc 3pl-}talk{\sc -npst}		\\
	\rede{The matchmakers, the officials, the respected elders, all discuss, they talk.} \source{25\_tra\_01.017}
\bg. ŋ-khaep cim=nuŋ sala lend-u-ga-iǃ\\
	{\sc 2sg.poss-}interest be\_satisfied{\sc =com} matter exchange{\sc -3.P[imp]-2=emph}\\
	\rede{Talk until you are satisfied!}

As for \isi{three-argument verbs}, in the double object class  either T or  G can be demoted. Example \Next[a] shows a clause where the G argument is demoted, and \Next[b] shows a clause where the T argument is demoted.  

\ex. \ag.nhe maŋcwa m-bi-me-n=ha\\
here water {\sc neg-}give{\sc [3sg]-npst-neg=nmlz.nsg}	\\
\rede{They do not serve water here.} (T → P)
 \bg.nhe ghak m-bi-me-n=ha, yaŋ kap-khuba se=ppa\\
here all {\sc neg-}give{\sc [3sg]-npst-neg=nmlz.nsg}, money own{\sc -nmlz} {\sc restr=emph} \\
\rede{They do not serve everyone here, only the rich people.} (G → P)


As already mentioned, there is  the option to retain the demoted argument in a  detransitivized clause (see example \ref{ex-raksi} above). In \Next, the \isi{noun phrase} \emph{kulpitrici} has a generic reading, similar to incorporated nouns, but the noun can still be inflected for nonsingular. 

\exg. ochoŋ=ha cayoŋwa  pahile   kulpitri=ci  m-bim-me\\
new{\sc =nmlz.nc} food at\_first ancestor{\sc =nsg} {\sc 3pl-}give{\sc -npst}\\
\rede{They give the new food to the ancestors at first.}\source{01\_leg\_07.137}
 

Actually, this clause is ambiguous between passive and antipassive reading. This example could as well mean \rede{The ancestors are given the new food at first}. The interpretation has to be inferred from the context. In this particular example, \emph{kulpitrici} was not yet active in discourse. The word order, too, speaks against the topicality of this constituent and the passive interpretation. The plural agreement on the verb is triggered by the (non-overt) A argument that does not change throughout the text: \emph{yakkhaci} \rede{the Yakkha people}. Ambiguities between passivized and antipassivized clauses in Yakkha are always encountered when both A and P have third person, the same \isi{number} features and are both equally low in referential salience. Consider the verb \emph{kheps \ti khem} \rede{hear, listen} in \Next, for example. Both verb forms are inflected intransitively, but \Next[a] is a passive, while \Next[b] is an antipassive.

Another regularity noticed is that, since antipassives often express generic statements about the world as such, they are more likely to be in the nonpast, while passives more often occur with past morphology.

\ex.\ag. ceʔya kheps-a-m=ha\\
matter hear{\sc [3sg]-pst-prf=nmlz.nc}\\
\rede{The matter has been heard.} \source{18\_nrr\_03.004}
\bg. Dilu  reɖio khem-meʔ=na?\\
Dilu radio  hear{\sc [3sg]-npst=nmlz.sg}\\
\rede{Does Dilu listen to the radio (generally)?}	


A  phenomenon that is related to the antipassive \isi{detransitivization} is the frequent omission of nonsingular marking on nouns generally, and the omission of a verbal plural marker for S and P arguments (see \sectref{number-1}). The nouns in \Next[a] are not marked for plural, which does  not generally prevent them from triggering plural agreement on the verb in Yakkha.\footnote{If a noun is not marked for plural, it may still trigger plural agreement on the verb. The only constraint found for the relatively freely organized agreement in Yakkha is that when the noun has plural marking, it has to trigger plural marking in the verb, too.} The verb shows object agreement with the nouns, but the suffix \emph{-ci} referring to the plural \isi{number} is missing. The full form would be  \emph{yogamacuci}. The agreement is not missing because the reference is singular, but because the nouns are non-referential. The predicate refers to the general activity of porcupine hunting and pangolin hunting, and it is not even clear yet if the hunt will be successful in this particular \isi{case}. The same effect is found for agreement with S arguments in intransitive verbs, shown in \Next[b]. The fully inflected verb form would be \emph{ŋgammehaci}, but the verb here functions as a marker of evidentiality, woven into a narration in order to release the speaker of being fully responsible for the content of the utterance, and thus the A of the verb has no clear reference.


\ex. \ag. yakpuca   yog-a-ma-c-u, phusa    yog-a-ma-c-u\\
porcupine   look\_for{\sc -pst-prf-du.A-3.P}, pangolin   look\_for{\sc -pst-prf-du.A-3.P}\\
\rede{They (dual) looked for porcupines, they looked for pangolins (but did not hunt any.)} \source{22\_nrr\_05.015}
\bg. maŋmaŋ-miŋmiŋ m-maks-a-by-a-ma ŋ-gam-me=pa\\
	amazed{\sc -echo} {\sc 3pl-}be\_surprised{\sc -pst-V2.give-pst-prf} {\sc 3pl-}say{\sc -npst=emph}\\
	\rede{They (plural) were utterly surprised, people say.} \source{22\_nrr\_05.031}
	


\subsubsection{Syncretisms of \isi{detransitivization} and {\sc 1nsg} re\-fe\-rence}\label{detrans-polite}

Both the passive and the antipassive may have a second reading, with the omitted argument having first person nonsingular reference. Thus, the \isi{passive construction} may also refer to 1{\sc nsg} A arguments, while the anti\isi{passive construction} may refer to 1{\sc nsg}  P arguments. This system is slowly replacing the older, more complex verbal \isi{person marking} for first person nonsingular arguments. When speakers are confronted with the detransitivized forms out of the blue, usually the first interpretation that is offered is the one with first person nonsingular arguments, and not with generic arguments. 

As for the antipassive forms, the detransitivized verbal inflection is in many cases already identical to the forms with 1{\sc nsg}  P reading; it has replaced various formerly present markers for 1{\sc nsg}  P argument (see \tabref{antip-tab}, and \tabref{omruwa} in \sectref{verb-infl}).
Given that the nominalizing clitics \emph{=na} and \emph{=ha} are optional markers, the intransitive forms with 2sg and 3sg S arguments are identical to most of the transitive forms with first person P argument (the shaded cells in \tabref{antip-tab}). The same syncretism is found in the \isi{imperative} forms (also shown in \tabref{antip-tab}).


The only way to formally differentiate between the antipassive and first \isi{person } P arguments is the presence of an ergative-marked A argument in transitive active clauses, as shown in \Next. As overtly realized arguments are rare, the only  element that could distinguish the two constructions is often missing.  

\ex.\ag.cyaŋkuluŋ=ci=ŋa     tuʔkhi m-bi-me-n\\
ancestral\_god{\sc =nsg=erg} pain {\sc neg-}give{\sc [3A;1.P]-npst-neg}\\
\rede{The ancestors will not make us suffer.}\source{01\_leg\_07.117}
\bg.uŋci=ŋa   phophop=na  sumphak=pe=se   camyoŋba pim-me=ha mit-a-ma-ci\\
{\sc 3nsg=erg} upside\_down{\sc =nmlz.sg} leaf{\sc =loc=restr} food  give{\sc [3A;1.P]-npst=nmlz.nsg} think{\sc -pst-prf-du}\\
\rede{They (dual) thought: They only give {\bf us} the food in an upside-down leaf plate.}\source{22\_nrr\_05.053} 


\noindent
\begin{table}[htp!]
\begin{center}
%{\footnotesize
\begin{tabular}{llll}
 \lsptoprule
 \multicolumn{4}{c}{{\sc indicative}}\\
		\midrule
	A>P	& 	{\sc 1sg}  &	 {\sc 1nsg}  & {\sc intransitive} \\
 \midrule
{\sc 1sg} 	&\multicolumn{2}{r}{}&-ŋ(=na)\\
 % \cline{1-1} \cline{4-4}			
{\sc 1du.excl}  	& \multicolumn{2}{r}{}&-ci-ŋ(=ha)\\
 % \cline{1-1} \cline{4-4}  	
{\sc 1pl.excl}	 	&\multicolumn{2}{r}{(reflexive)}&-i-ŋ(=ha)\\
 % \cline{1-1} \cline{4-4}			
{\sc 1du.incl}  	& \multicolumn{2}{r}{}&-ci(=ha)\\
 % \cline{1-1} \cline{4-4}			
{\sc 1pl.incl}  	&	\multicolumn{2}{r}{}&-i(=ha)\\
  \midrule
{\sc 2sg}  	&	 -ŋ-ka(=na)&\cellcolor[gray]{.8}&\cellcolor[gray]{.8}-ka(=na)\\
 % \cline{1-2}  
{\sc 2du}  	&  \multicolumn{2}{r}{\cellcolor[gray]{.8}} &-ci-ka(=ha)\\
 % \cline{1-1} 	
{\sc 2pl}	 	&  \multicolumn{2}{r}{\cellcolor[gray]{.8}-ka(=ha)} &-i-ka(=ha)\\
  \midrule			
{\sc 3sg} 	 		& -ŋ(=na)&\cellcolor[gray]{.8}&\cellcolor[gray]{.8}(=na/=ha)\\
  % \cline{1-2}  			
{\sc 3du}	 	&   \multicolumn{2}{r}{\cellcolor[gray]{.8}} &-ci(=ha)\\
  % \cline{1-2}  
{\sc 3pl}	 	&  \multicolumn{2}{r}{\cellcolor[gray]{.8}(=ha)} &n-...(=ha=ci)\\
\midrule
 \multicolumn{4}{c}{{\sc imperative}} \\
\midrule
	A>P	& 	{\sc 1sg}  &	 {\sc 1nsg}  & {\sc intransitive} \\
\midrule
 {\sc 2sg} &-aŋ& \cellcolor[gray]{.8}&\cellcolor[gray]{.8}-a\\
 % \cline{1-2}  
 {\sc 2du} & \multicolumn{2}{r}{\cellcolor[gray]{.8}} &-a-ci\\
  % \cline{1-1} 	
 {\sc 2pl}&\multicolumn{2}{r}{\cellcolor[gray]{.8}-a}& -a-ni\\
\lspbottomrule
\end{tabular}
%}
\end{center}
\caption{Antipassive-{\sc 1nsg.P} syncretisms}\label{antip-tab}
\end{table}


\ex.\ag.cyaŋkuluŋ=ci=ŋa     tuʔkhi m-bi-me-n\\
ancestral\_god{\sc =nsg=erg} pain {\sc neg-}give{\sc [3A;1.P]-npst-neg}\\
\rede{The ancestors will not make us suffer.}\source{01\_leg\_07.117}
\bg.uŋci=ŋa   phophop=na  sumphak=pe=se   camyoŋba pim-me=ha mit-a-ma-ci\\
{\sc 3nsg=erg} upside\_down{\sc =nmlz.sg} leaf{\sc =loc=restr} food  give{\sc [3A;1.P]-npst=nmlz.nsg} think{\sc -pst-prf-du}\\
\rede{They (dual) thought: They only give {\bf us} the food in an upside-down leaf plate.}\source{22\_nrr\_05.053} 


The lack of agreement marking for certain participants in a language with otherwise abundant agreement morphology is suspicious and calls for an explanation. Considering the broader Kiranti perspective, the \isi{equation} of generic or indefinite reference with first person undergoers developed independently in many Eastern Kiranti languages, with different morphological realizations. In \ili{Puma}, a Southern-Central Kiranti language, the antipassive is marked by the prefix \emph{kha-}, and this marker is also found as regular 1nsg P agreement prefix.  \citet[6]{Bickeletal2005Generics} note further that indefinite pronouns and generic nouns with the meaning \rede{people} have developed into first person patient markers in \ili{Limbu} and Belhare. 

Bickel and Gaenszle suggest a functional motivation. The speaker, in the patient role, downplays or minimizes the reference to himself as a politeness strategy.  Bickel and Gaenszle, for the (geographically) Southern Kiranti languages, relate this to contact with the Maithili (Indo-Aryan) speaking Sena principalities in the 17th-18th centuries.\footnote{The alliances of Kirat (Kiranti) kings and Sena kings began in Makwanpur, and were later extended eastward to Vijayapur. Kiranti military power probably helped the Sena rulers to defend their rule against others, e.g. the Mughals. Kiranti chiefs also acted as judicial officers in Vijayapur \citep[76]{Pradhan1991The-Gorkha}.} In the course of this contact, Hindu religion and custom had a strong impact on Kiranti traditions and languages (see also \citet{Gaenszleetal2005Worshipping} on \ili{Chintang}). The in\isi{tense} contact with  spoken and written Maithili probably introduced formal registers with grammaticalized honorific distinctions, which the Kiranti languages were lacking. Thus, the speakers resorted to the strategy of identifying first person with an indefinite reference. Particularly striking is  the \isi{exclusive} choice of the patient role for this \isi{equation}, throughout the Central and Eastern Kiranti languages (except for Yakkha, where this strategy got extended to A arguments, too). Bickel and Gaenszle explain this with the sensitive role of patients, especially recipients, in Kiranti societies.\footnote{Exchanging gifts, money and alcoholic beverages is formalized in weddings and funerals, and has a social function.} As the Yakkha territory is located to the north of the core contact zone, this pattern must have spread from the south into the Yakkha speaking areas.


The \isi{equation} of a passive interpretation with a first person agent is, to my knowledge, not a Kiranti-wide pattern, although one can easily imagine that it has developed in analogy to the antipassive \isi{equation}. In \Next, the detransitivized clause  can have two interpretations. 

\exg.kisa sis-a=na\\
deer kill{\sc [3sg]-pst=nmlz.sg}\\
\rede{The deer was killed./We killed the deer.}
 

The passive interpretation is generally less accessible, the default interpretation in such cases implies a first person nonsingular A argument. The passive was even completely rejected in \Next[a]. Instead, speakers  offered \Next[b], with a first person nonsingular A argument. Note that the \isi{ablative} marking on A is optional.\footnote{The verb is \emph{lumbiʔma} \rede{tell}, a \isi{benefactive} of \rede{tell}. Derived transitives do not behave differently from non-derived transitives, at least not with regard to the \isi{passive construction}.} Just as in the antipassive, the downgraded argument can still be overt, but it does not trigger verbal agreement. The motivation for downgrading the first person participant is  probably the same politeness strategy as in the antipassive: omission of explicit reference to first person nonsingular by agreement markers, without changing anything else in the structure. Another example is provided in \Next[c].  


\ex.\ag.*Numa u-ppa=bhaŋ tablik lut-a-by-a=na.\\
Numa {\sc 3sg.poss-}father{\sc =abl} story tell{\sc [3sg]-pst-V2.give-pst=nmlz.sg}\\
Intended: \rede{Numa was told a story by her father.}
\bg.Numa kaniŋ(=bhaŋ) tablik lut-a-by-a=na\\
Numa  {\sc 1pl(=abl)} story tell{\sc [3sg]-pst-V2.give-pst=nmlz.sg}\\
\rede{Numa was told a story  by us.} OR \rede{We told  Diana a story.}
\bg.hoŋma  yokhaʔla chekt-haks-a=na.\\
river across block{\sc [3sg]-V2.send-pst=nmlz.sg} \\
\rede{We redirected the river.} 


The passive politeness construction is definitely younger than the antipassive politeness construction, as the antipassive forms have become the standard verbal inflection for first person patients (see \sectref{paradigmtables} for \isi{paradigm tables}). Marginally, one can find different, more complex forms in these paradigm cells, with overt agreement markers for all participants, especially in the generation of older speakers. But speakers always pointed out that the less complex forms are more common, and this is also borne out by the corpus data. In a paper by \citet[425]{Gvozdanovic1987How}, who made the first agreement paradigm of Yakkha available, the first person patient forms still distinguish singular, dual and plural \isi{number} of both A and P (cf. \sectref{verb-infl} for the details). Apparently, these forms got replaced by the less explicit politeness forms, as they were semantically bleached due to overuse. The antipassive became the default form to indicate first person P arguments. One could speculate whether the same is going to happen with the passive forms, as the passive already seems to be the less salient interpretation for these forms.




\subsection{The causative construction}\label{caus}

Causatives are constructed morphologically, by attaching the suffix \emph{met \ti meʔ} to the stem of the lexical verb. The marker has developed from a lexical verb \emph{met} \rede{make, do, apply}, in the same way as in other Kiranti languages (\ili{Limbu}, \ili{Puma}, \ili{Bantawa}, \ili{Chintang}, see \citealt{Driem1987A-grammar, Bickeletal2006The-\ili{Chintang}, Doornenbal2009A-grammar}); in Yakkha its lexical meaning got narrowed down to \rede{tie cloth around the waist}. 

Both direct and indirect causation can be expressed by the causa\-tive. The causative marker is only used to introduce an animate causer to the verb frame, never inanimate causes such as weather phenomena, illnesses and other circumstances. The intentionality of the causer, however, is not relevant in the Yakkha causative formation. With some verbs, the causative marker is found to have an applicative function,  where instead of a causer, a P argument is added to the \isi{argument structure} (discussed further below).


The causative derivation applies to both intransitive and transitive verbs, deriving minimally a monotransitive predicate. The S/A argument of the underived predicate becomes the P argument of the causative predicate, while a causer is added and becomes the A argument in the \isi{causative construction}. The causer triggers  subject agreement accordingly, and is marked by the \isi{ergative} \isi{case}. In \Next, the \isi{case} marking changes from oblique in (a) to \isi{ergative} in (b), as the role of the tiger changes from stimulus to causer.

\ex.\ag.hari kiba=nuŋ kisit-a=na\\
Hari tiger{\sc =com} be\_afraid{\sc -pst[3sg]=nmlz.sg}\\
\rede{Hari was afraid of the tiger.}
\bg.kiba=ŋa hari kisi-met-u=na\\
tiger{\sc =erg} Hari be\_afraid{\sc -caus-3.P[pst]=nmlz.sg}\\
\rede{The tiger frightened Hari.}


Arguments (other than the causer) retain their respective cases (\isi{nominative}, instrumental, \isi{locative}), so that the causative derivation yields different three-argument frames. The standard \isi{monotransitive frame} (with the P argument in the \isi{nominative}) results in the \isi{double object frame} in the causative, with the former A becoming the G argument and the P becoming the T (outlined in \figref{fig-caus} and illustrated by \Next). 

\begin{figure}[htp]
\begin{center}
{\small
\begin{tabular}{ll}
\lsptoprule
{\sc underived}&{\sc causative} \\
\midrule
& A\\
S/A →&P/G \\
P →& T\\

\lspbottomrule
\end{tabular}
}
\end{center}
\caption{Mapping of roles in the causative derivation}\label{fig-caus}
\end{figure}
 
 

 
\ex.\ag.ka phoʈo soʔ-wa-ŋ=na\\
{\sc 1sg[erg]} photo look{\sc -npst[3.P]-1sg.A=nmlz.sg}\\
\rede{I will look at the photo.}
\bg. ka nda phoʈo soʔ-meʔ-me-nen=na\\
{\sc 1sg[erg]} {\sc 2sg} photo look{\sc -caus-npst-1>2=nmlz.sg}\\
\rede{I will show you the photo./ I will make you look at the photo.} 
 
 
 
More examples of causatives resulting in double object constructions are provided in \Next. All of them illustrate  that the causer triggers subject agreement and the causee triggers object agreement on the verb. 

\ex.\ag. uŋ=ŋa ka chem luʔ-met-a-ŋ=na\\
{\sc 3sg=erg} {\sc 1sg} song tell{\sc -caus-pst-1sg.P=nmlz.sg}\\
\rede{He made me sing a song.}
\bg. ka mim-meʔ-me-nen=na\\
{\sc 1sg[erg]} remember{\sc -caus-npst-1>2=nmlz.sg} \\
\rede{I will remind you.}
\bg.ŋkha yapmi=ci    koi  namphak   si-met-uks-u-ci, \\
those person{\sc =nsg} some wild\_boar kill{\sc -caus-prf-3.P[pst]-3nsg.P}\\
\rede{Those people, (he) made some of them hunt wild boar, ...}\source{27\_nrr\_06.033}

If the underived verb is a \isi{motion} verb, the causative results in a three-argument construction of the \isi{indirective frame}, with the G in the \isi{locative} \isi{case}.

\ex.\ag. a-kamnibak ten=be tas-u=na\\
{\sc 1sg.poss-}friend  village{\sc =loc} arrive{\sc -3.P[pst]=nmlz.sg}\\
\rede{My friend arrived in the village.}
\bg. m-baŋ=be  ta-met-i\\
{\sc 2sg.poss-}house{\sc =loc} arrive{\sc -caus-compl[3.P;imp]}\\
\rede{Deliver it (the rock) at your home.}\source{37\_nrr\_07.011}


If the causative is applied to non-canonically marked constructions such as they are found in the expression of experiential events, the \isi{experiencer}  becomes the causee. Experiencers can be coded as standard objects (shown in \Next[a]). Despite this non-canonical marking, the \isi{experiencer} becomes the causee and triggers object agreement in the \isi{causative construction} (see \Next[b]). The stimulus, formally identical to A arguments in the non-causative predicate, is in the instrumental \isi{case} and does not trigger agreement in the causative.\footnote{The instrumental is homophonous with the \isi{ergative} but it is clearly not an \isi{ergative} here, in the classical definition of marking an A as opposed to S and P.} Thus, in causatives of the object \isi{experiencer} construction, nothing changes for the \isi{experiencer}; it remains the argument that is coded as object.\footnote{Marginally, the right context provided, interpretations with the stimulus as the causee are also possible, e.g. \emph{ʈailorŋa tek khopmetuha} \rede{The tailor made the fabric be enough} (i.e. he cut the pieces carefully).} 

\ex.\ag.siŋ=ŋa ŋ-khot-a-ŋa-n=na\\
	fire\_wood\sc{=erg} \textsc{neg}-have\_enough\sc{-pst-1sg.P-neg=nmlz.sg}\\
	\rede{I do not have enough fire wood.} 
	\bg. ka uŋci cama=ŋa khoʔ-met-wa-ŋ-ci-ŋ=ha\\
	\sc{1sg[erg]} \sc{3nsg} food\sc{=ins}	have\_enough\sc{-caus-npst[3.P]-1sg.A-nsg.P-1sg.A=nmlz.nsg}\\
		\rede{I serve them food.} (lit. \rede{I make them have enough food.})

		
In the possessive \isi{experiencer} construction, the \isi{experiencer} does not even trigger agreement in the underived verb (see \Next[a]), but is still treated as object by the verbal agreement of the causative verb in \Next[b]. Thus, the causative shows that the morphosyntax of Yakkha is not sensitive to \isi{case} marking or agreement but to generalized semantic roles. As the \isi{experiencer} is the A in \Last and the S in \Next, it becomes the causee in the respective causative constructions. 

	\ex.\ag.a-sokma hips-a-by-a=na\\
	{\sc 1sg.poss-}breath thrash{\sc [3sg]-pst-V2.give-pst=nmlz.sg}\\	
	\rede{I am annoyed.}   
	 \bg.khem=nuŋ rajiv=ŋa a-sokma him-met-a-g=haǃ\\
	Khem{\sc =com}  Rajiv{\sc =erg} {\sc 1sg.poss-}breath whip{\sc -caus-pst-2.A[1.P]=nmlz.nsg}\\
	\rede{Khem and Rajiv(, you) annoy me!} 


Some causatives have idiomatic, lexicalized meanings, such as the verb \emph{yokmeʔma} \rede{tell about something, make someone curious}. The lexical meaning of \emph{yokma} is \rede{search}. Another instance is \emph{incameʔma} \rede{play with someting}. Literally, it translates as \rede{make something revolve for fun} (see \Next).

 
\ex. \ag.muŋri caram=be i-ca-ya=na\\
Mungri yard{\sc =loc} revolve-\sc{V2.eat-pst[3sg]=nmlz.sg}\\
\rede{Mungri played in the yard.}
\bg.khaʔla (nasa) in-ca-met-uks-u-ŋ=niŋ\\
	like\_this (fish) revolve-\sc{V2.eat-caus-prf-3.P[pst]-1sg.A=ctmp}\\
	\rede{while I played (with the fish) like this}   \source{13\_cvs\_02.026}
 

Some instances of causative morphology have applicative interpretations. The argument added is not a causer, but an object. Consider \Next, where no specific causee argument can be identified, as the interpretation of \Next[b] is not \rede{she makes her spread the umbrella}. The causative here is used to distinguish whether an action is performed on oneself or on another participant. The underived verb \emph{hamma} \rede{spread over} always refers to covering oneself, and not to spreading something at another location. A second factor might be the pragmatic saliency and frequency of the causative-marked event. Covering a person with a blanket or umbrella is performed more often than encouraging the person to cover oneself.\footnote{Spreading an umbrella over the bride and groom is an integral component of the wedding ceremony.} Another verb following this pattern is  \emph{waʔmepma} \rede{dress someone/ help to put on clothes} (see \Next[c]). The \isi{causative construction}, in such cases, indicates the causation of a state (being dressed, being covered).  


\ex.\ag. ka phopma haps-wa-ŋ=na\\ 
{\sc 1sg[erg]} blanket spread{\sc -npst[3.P]-1sg.A=nmlz.sg}\\
\rede{I cover myself with a blanket.}
\bg.beuli=ga=ca  u-nuncha pʌrne=ŋa   chata    ham-met-wa\\
bride{\sc =gen=add} {\sc 3sg.poss-}younger\_sibling falling{\sc =erg} umbrella spread{\sc -caus-npst[3.P]}\\
\rede{Equally, a younger sister of the bride spreads an umbrella (over the bride).} \source{25\_tra\_01.053} 
\bg.ka a-nuncha tek waʔ-met-u-ŋ=na\\
{\sc 1sg[erg]}  {\sc 1sg.poss-}younger\_sibling cloth put\_on{\sc -caus-3.P[pst]-1sg.A=nmlz.sg}\\
\rede{I helped my younger sister to get dressed./ I dressed my younger sister.}


The causative of the transitive verb \emph{koʔma} \rede{walk (from place to place)} also deviates  from the classic causative function. The meaning of \emph{koʔmeʔma} is \rede{walk someone around}, i.e. not just making someone walk but actually walking with them and guiding them (see \Next). 

\exg.ka a-kamniwak=ci  koʔ-met-wa-ŋ-ci-ŋ=ha\\
{\sc 1sg[erg]}  {\sc 1sg.poss-}friend{\sc =nsg} walk{\sc -caus-npst-1sg.A-nsg.P-1sg.A=nmlz.nsg} \\
\rede{I walked my friends around.}
 
  
 
The verbs derived by the causative behave identical to simple stems in most respects. The A argument can be a privileged syntactic argument in constructions that select S/A pivots. It can, for instance, undergo a \isi{participant nominalization} with the \isi{nominalizer} \emph{-khuba} that derives nouns or noun phrases with the role of S or A \Next.  
 
 \ex.\ag.cok-khuba\\
do{\sc -nmlz}\\
\rede{doer, someone who does (something)}
\bg.hiʔwa=be    camyoŋba ca-mek-khuba  \\
wind{\sc =loc} food eat{\sc -caus-nmlz}\\
\rede{someone (a bird) who feeds (his children) in the air}\source{21\_nrr\_04.003}




\subsection{The \isi{benefactive} construction}\label{benefactive}

The \isi{benefactive} is marked by the suffix \emph{-t}\footnote{This marker is a remnant of a  formerly productive Proto-\isi{Tibeto-Burman}  transitivizing suffix. Apart from its employment in the \isi{benefactive} function, this marker is not productive in Yakkha; it has been re-analyzed as part of the stem in Kiranti languages in general, see \sectref{stem}.} attached to the lexical root and by the V2 \emph{-piʔ} \rede{give}, resulting in a \isi{complex predicate} that has a beneficiary argument in addition to the arguments of the lexical verb. The suffix \emph{-t} is usually added to yet unaugmented stems such as \emph{cok} \rede{do} or \emph{soʔ} \rede{look} (becoming \emph{cokt} and \emph{sot}, respectively). Stems with an augmented \emph{-s}, however, do not provide a coherent picture. The stems with the bilabial stop in the coda, like \emph{haps} \rede{spread, distribute} and \emph{tups} \rede{meet, find}, do not host the suffix \emph{-t} in the \isi{benefactive} derivation, but stems with the velar stop in the coda, like \emph{leks} \rede{overturn} and \emph{haks} \rede{send}, for instance, change to \emph{lekt} and \emph{hakt}. More examples are necessary to find out if this is a phonological regularity. Hence, the use of this suffix is both functionally and phonologically conditioned. Adding \emph{-t} to the lexical stem in the \isi{benefactive} could be a strategy to distinguish the \isi{benefactive} from the other uses of the V2 \emph{piʔ} (discussed in \sectref{V2-give}).

The addition of  a beneficiary argument changes the marking and behavioral properties of a verb. The beneficiary is promoted to an argument; it is in the unmarked \isi{nominative} \isi{case} and triggers object agreement, illustrated by \Next. Both intransitive and transitive verbs can undergo the \isi{benefactive} derivation (see \Next). The latter result in double object constructions.  

 \ex. \ag. ceŋ    pok-t-a-by-a-ŋ lu-ks-u\\
upright stand\_up-{\sc ben-imp-V2.give-imp-1sg.P} tell{\sc -prf-3.P[pst]}\\
\rede{Stand upright for me, he told him.} \source{27\_nrr\_06.18}
 \bg. ka chem lu-t-a-by-a-ŋ\\
 {\sc 1sg} song tell-\sc{ben-imp-V2.give-imp-1sg.P}\\
 \rede{Sing me a song.} 
 
 If the beneficiary has nonsingular \isi{number}, it also triggers the third person nonsingular agreement with object arguments that is typically found  on infinitives with a deontic reading (see \sectref{obl}).

\exg. yenda taŋ-biʔ-ma=ci\\
millet\_mash take\_out-{\sc V2.give-inf[deont]=nsg}\\
\rede{The millet mash (for  brewing beer) has to be taken out for them.}
 
 
Example \Next is a nice semantic minimal pair illustrating that the \isi{benefactive} is inappropriate in non-\isi{benefactive} contexts. While in \Next[a] the \isi{benefactive} derivation of the stem \emph{haps} \rede{spread, distribute} is possible and necessary, in \Next[b] the verb has to be used without the \isi{benefactive} derivation. The G argument  \emph{ten-ten} \rede{villages}  in \Next[a] gains something from the event of distributing, which is not the \isi{case} for the G argument \emph{klas-ci} \rede{classes} in \Next[b]. Furthermore, this example shows that the \isi{benefactive} derivation does not necessarily change the \isi{argument realization} of a verb that already has three arguments.

 \ex. \ag. sarkar=ŋa yaŋ ten-ten=be ŋ-haps-u-bi-ci=ha\\
government{\sc =erg} money village{\sc -redup=loc}	{\sc 3pl.A-}distribute-{\sc 3.P[pst]-V2.give-nsg.P=nmlz.nsg}\\
\rede{The government (plural) distributed the money among the villages.}
 \bg. uŋci=ŋa picha=ci klas=ci=be ŋ-haps-u-ci=ha\\
{\sc 3nsg=erg} child{\sc =nsg} class{\sc =nsg=loc}	{\sc 3pl.A-}distribute-{\sc 3.P[pst]-nsg.P=nmlz.nsg}\\
\rede{They distributed the children among the classes.}


The events denoted by the \isi{benefactive} derivation do not necessarily happen to the advantage of the \rede{beneficiary}, as \Next shows. The crucial semantic component of the \isi{benefactive} is a volitional, intentional agent, acting in order to bring about an event that affects the \rede{beneficiary}, either in desirable or in undesirable ways. Example \Next also offers insight into the morphological structure of complex predicates. The \isi{benefactive} applies to an already \isi{complex predicate}, consisting of the lexical stem \emph{pek} \rede{shatter} and the V2 \emph{-haks} \rede{send} (which adds a notion of irreversibility to the meaning of the lexical verb). However, the existing complex structure is not opaque to the derivational morphology, as the \isi{benefactive} suffix \emph{-t} attaches to both stems.

\ex.\ag. ak=ka pek-t-hak-t-a-by-a-ŋ=na loʔwa, nna=ga=ca  pek-t-hak-t-u-bi-wa-ŋ=ha\\
{\sc 1sg.poss=gen} shatter-{\sc ben-V2.send-ben-pst-V2.give-pst-1sg.P=nmlz.sg} as,   that{\sc =gen=add} shatter-{\sc ben-V2.send-ben-3.P-V2.give-npst-1sg.A=nmlz.nsg}\\
\rede{As he destroyed mine, I will also destroy his (house, family etc.).} \source{21\_nrr\_04.029}
\bg.eko yapmi=ga o-keŋ en-d-hak-t-u-bi=na\\
one person{\sc =gen} {\sc 3sg.poss-}tooth	uproot{\sc -ben-V2.send-ben-3.P-V2.give[3.P;pst]=nmlz.sg}\\
\rede{He pulled out some man's tooth.} 
 

Two examples from a narrative are provided in \Next. The beneficiary here has first person nonsingular reference, and is thus not indexed on the verb by overt markers (cf. \sectref{verb-infl} on \isi{person marking}).

\ex.\ag. aniŋ=ga          khaʔla=na   piccha=ca    tups-a-by-a-ga\\
	{\sc 1pl.excl.poss=gen} in\_this\_way{\sc =nmlz.sg} child{\sc =add} find{\sc -pst-V2.give-pst-2[1.P]}\\
\rede{You also found (this) our daughter for us.} \source{22\_nrr\_05.115}
\bg. kanciŋ nakt-a-ŋ-c-u-ŋ=na=cen ina baŋniŋ, na phophop=na sumphak cilleŋ  lek-t-a-by-a,\\
 {\sc 1du} ask\_for{\sc -pst-excl-du.A-3.P.pst-excl=nmlz.sg=top} what  {\sc top}, this upside\_down{\sc =nmlz.sg} leaf    facing\_up overturn{\sc -ben-imp-V2.give-imp[1.P]}\\
\rede{As for what we asked you for: this overturned leaf plate, turn it on the right side for us.}\source{22\_nrr\_05.126--7}


The beneficiary does not only trigger agreement on the verb. Plenty of examples show that the \isi{benefactive} verb can undergo the reciprocal derivation\footnote{The reciprocal is constructed by the suffix \emph{-khusa} attached to the (last) stem of a verb and the verb \emph{cama} \rede{eat} as auxiliary. Although the  reciprocal derivation of a \isi{benefactive} predicate still has two arguments, the person inflection in the \isi{reciprocal construction} always shows the intransitive morphology.} when an action is performed bidirectionally, and the (minimally) two participants have each the role of agent and beneficiary/maleficiary, as shown in \Next.  

\ex. \ag. piccha=ci caram pheŋ-bi-khusa ca-me-ci=ha\\
 child{\sc =nsg} yard sweep-{\sc V2.give-recip} eat{\sc .aux-npst-[3]du=nmlz.nsg}\\
\rede{The children sweep the yards for one another.}\footnote{First person dual \isi{inclusive} and third person dual are  identical in intransitively inflected verbs.}
\bg. kanciŋ moja pham-bi-khusa ca-me-ci=ha\\
 {\sc 1du} sock knit-{\sc V2.give-recip} eat{\sc .aux-npst-[1]du=nmlz.nsg}\\
\rede{We knit socks for each other.}
 \bg. anciŋ-cuwa=ci uk-nim-bi-khusa ca-ya-ŋ-ci-ŋ=ha\\
 {\sc 1du.excl.poss}-beer{\sc =nsg} drink-{\sc compl-V2.give-recip} eat{\sc .aux-pst-excl-du-excl=nmlz.nsg}\\
 \rede{We (dual, \isi{exclusive}) accidentally drank out each other's beer!}
 
 
An operation that is not available for verbs derived by the \isi{benefactive} is reflexivization. Expressing propositional content such as in  \Next by the form \emph{*thum-\textbf{bi}-ca-me-ŋ=na} is ungrammatical. The semantics of the Yakkha \isi{benefactive} entail that the benefactor and the beneficiary must not have the same reference. The expression of actions for oneself can be achieved simply by attaching the reflexive morphology (the V2 \emph{-ca}) to the verb stem, without prior \isi{benefactive} derivation (cf. also \sectref{refl}).

\exg.kurta thun-ca-me-ŋ=na \\
	long\_shirt sew{\sc -V2.eat-npst-1sg=nmlz.sg} 	\\
	\rede{I sew a \emph{kurta} for myself.}



