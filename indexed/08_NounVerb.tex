
\chapter{Noun-verb predicates}\label{noun-verb}

This chapter deals with idiomatic combinations of a noun and a verb. These predicates occupy a position somewhat between word and phrase. Lexically, a \isi{noun-verb predicate} always constitutes one word, as its meaning is not directly predictable from its individual components (with varying degrees of metaphoricity and abstraction). But since the nouns enjoy considerable morphosyntactic freedom, speaking of  noun incorporation here would be misleading. About 80 noun-verb predicates are attested so far, with rougly two thirds referring to experiential events.

There are two main morphologically defined patterns for noun-verb predicates. In the first pattern (Simple noun-verb predicates) the predicate consists of a noun and a verb that are juxtaposed in N-V order, such as \emph{lam phakma} \rede{open way, give turn} or \emph{tukkhuʔwa lamma} \rede{doze off}  (discussed in \sectref{simple-noun-verb}).\footnote{The same pattern is also used as a strategy to incorporate \ili{Nepali} nouns into the Yakkha morphology, with a very small class of light verbs, namely \emph{cokma} \rede{make}, \emph{wama} \rede{exist, be} and \emph{tokma} \rede{get}, cf. \sectref{cop}. } The second pattern (the \isi{experiencer-as-possessor construction}) is semantically more restricted and also  different morphologically. It expresses experiential events, with the \isi{experiencer} coded as possessor, as for instance \emph{hakamba keʔma} \rede{yawn} (literally \rede{someone's yawn to come up}). This pattern is discussed in \sectref{nv-comp-poss}. 

\section{Simple noun-verb predicates}\label{simple-noun-verb}

Most of the \isi{simple noun-verb predicates} are relatively transparent but fixed collocations. They denote events from the semantic domains of natural phenomena, e.g., \emph{nam phemma} \rede{shine [sun]}, \emph{taŋkhyaŋ kama} \rede{thunder} (literally \rede{the sky shouts}), some culturally significant actions like \emph{kei lakma} \rede{dance the drum  dance} and also verbs that refer to experiential events and bodily functions, such as \emph{whaŋma tukma} \rede{feel hot} (literally \rede{the sweat (or heat) hurts}). Experiential concepts are, however, more frequently expressed by the \isi{experiencer-as-possessor construction}.\footnote{There is no clear explanation why some verbs expressing bodily functions, like  \emph{chipma chima} \rede{urinate}  belong to the \isi{simple noun-verb predicates}, while most of them belong to the experiencer-as-possessor frame. Some verbs show  synonymy across these two classes, e.g., the two lexemes with the meaning  \rede{sweat}: \emph{whaŋma lomma} (literally \rede{(someone's) sweat comes out}, an experiencer-as-possessor predicate, with the \isi{experiencer} coded as possessor of \emph{whaŋma}) and \emph{whaŋmaŋa lupma} (literally \rede{the sweat disperses}, a simple \isi{noun-verb predicate}).} 

\tabref{simple-nv-tab} provides some examples of \isi{simple noun-verb predicates}. Lexemes in square brackets were not found as independent words beyond their usage in these compounds. Some verbs, like weather verbs (e.g., \emph{nam phemma} \rede{shine [sun]} and \emph{wasik tama} \rede{rain}) and some \isi{experiential predicates} (e.g., \emph{wepma sima} \rede{be thirsty} and \emph{whaŋma tukma} \rede{feel hot}), for instance, do not allow the expression of additional arguments; their \isi{valency} is zero (under the assumption that the nouns belonging to the predicates are different from full-fledged arguments). If overt arguments are possible, they behave like the arguments of standard intransitive or standard transitive verbs. They trigger agreement on the verb, and they take \isi{nominative} or \isi{ergative} \isi{case} marking (see \sectref{frames}). 

\begin{table}[htp]
\resizebox{\textwidth}{!}{
\begin{tabular}{lll}
\lsptoprule
{\sc predicate} & {\sc gloss} &{\sc literal translation}\\
\midrule
\emph{cabhak lakma} &\rede{do the paddy dance} &(paddy – dance)\\
\emph{chakma pokma} & \rede{troubled times to occur} &(hardship – strike)\\
\emph{chipma chima}& \rede{urinate} &(urine – urinate)\\
\emph{cuŋ tukma}& \rede{feel cold} &(cold – hurt)\\
\emph{himbulumma cama} &\rede{swing} &(swing – eat)\\
\emph{hiʔwa phemma} &\rede{wind blow} &(wind –  be activated [weather])\\
\emph{hoŋga phaŋma} & \rede{crawl} &([{\sc stem}] – [{\sc stem}])\\
\emph{kei lakma} &\rede{do the drum dance} &(drum – dance)\\
\emph{laŋ phakma} &\rede{make steps} &(foot/leg – apply) \\
\emph{lam phakma} &\rede{open way, give turn} &(way – apply/build)\\
\emph{lambu lembiʔma} & \rede{let pass} &(way – let–give)\\
\emph{muk phakma} & \rede{help, serve} &(hand – apply)\\
\emph{nam ama} &\rede{sit around all day} &(sun – make set)\\
\emph{nam phemma} &\rede{be sunny} &(sun – be activated [weather])\\
\emph{phiʔma phima}& \rede{fart} &(fart – fart)\\
\emph{sak tukma}& \rede{be hungry} &(hunger – hurt)\\
\emph{setni keʔma}& \rede{stay awake all night} &(night – bring up)\\
\emph{sokma soma}& \rede{breathe} &(breath – breathe)\\
\emph{susuwa lapma}& \rede{whistle} &([whistle] – call)\\
\emph{tukkhuʔwa lapma} &\rede{doze off} &([{\sc stem}] – call)\\
\emph{(\ti lamma)} &  & \\
\emph{taŋkhyaŋ kama} &\rede{thunder} &(sky –  call)\\
\emph{uwa cama} &\rede{kiss} &(nectar/liquid – eat)\\
\emph{wa lekma} &\rede{rinse} &(water – turn)\\
\emph{wasik tama} &\rede{rain} &(rain – come)\\
\emph{wepma sima} & \rede{be thirsty} &(thirst – die)\\    %: ka wepma siangna
\emph{wepma tukma}& \rede{by thirsty} &(thirst – hurt)\\
\emph{wha pokma}  &\rede{septic wounds to occur} &(septic wound – infest)\\
\emph{whaŋma tukma}  &\rede{feel hot} &(heat/sweat – hurt)\\
\emph{yak yakma} &\rede{stay over night} &([{\sc stem}]  – stay over night)\\
\emph{yaŋchan chiʔma} &\rede{regret} &([{\sc stem}] – get conscious)\\
\midrule
\emph{chemha=ŋa sima} &\rede{be intoxicated, be drunken} &(be killed by alcohol)\\
\emph{cuŋ=ŋa sima} &\rede{freeze} &(die of cold)\\
\emph{sak=ŋa sima } &\rede{be hungry} &(die of hunger) \\      %: ka saknga siangna
\emph{whaŋma=ŋa lupma}  &\rede{sweat} &(heat/sweat – disperse/strew)\\
\lspbottomrule
\end{tabular}
}
\caption{Simple noun-verb predicates}\label{simple-nv-tab}
\end{table}

%= 34

The predicates vary as to whether the noun or the verb carries the semantic weight of the predicate, or whether both parts play an equal role in establishing the meaning of the construction. In verbs like  \emph{wepma sima} \rede{be thirsty} (lit. \rede{thirst – die}) or \emph{wasik tama} \rede{rain} (lit. \rede{rain – come}), the noun carries the semantic weight,\footnote{This is the reason why noun-verb collocations  have also become known as \isi{light verb} constructions (after \citet{Jespersen1965_Modern}, who used this term for English collocations like \emph{have a rest}).} while in verbs like  \emph{kei lakma} \rede{drum – dance}, \emph{cabhak lakma} \rede{paddy – dance}, the nouns merely modify the verbal meaning. The nouns may stand in various thematic  relations to the verb:  in \emph{wepma sima} \rede{be thirsty}, the noun has the role of an effector, in predicates like \emph{hiʔwa phemma} \rede{wind – blow} it is closer to an agent role. In \emph{saya pokma} \rede{head-soul – raise},\footnote{\rede{Raising the head soul} is a ritual activity undertaken by specialists to help individuals whose physical or psychological well-being is in danger.} it is a patient.

There are also a few constructions in which the noun is etymologically related to the verb, such as  \emph{chipma chima} \rede{urinate}, \emph{sokma soma} \rede{breathe} and \emph{phiʔma phima} \rede{fart} (cognate object constructions). The nouns in these constructions do  not contribute to the overall meaning of the predicate.\footnote{Semantically empty nouns are also attested in the \isi{experiencer-as-possessor construction} (cf. below). They are called “eidemic" in \citet{Bickel1995In-the-vestibule, Bickel1997The-possessive}; and “morphanic” (morpheme orphans) in  \citet{Matisoff1986Hearts}.} 

Concerning \isi{stress assignment} and the \isi{voicing} rule (see \sectref{stress} and \sectref{morphophon}), noun and verb do not constitute a unit. Both the noun and the verb carry equal stress, even if the noun is monosyllabic, resulting in adjacent stress, as in \emph{ˈsak.ˈtuk.ma}. As for \isi{voicing}, if the initial stop of the verbal stem is preceded by a nasal or a vowel, it remains voiceless. This stands in contrast to the verb-verb predicates (see Chapter \ref{verb-verb}), which are more tightly fused in other respects, too. Compare, for instance, \emph{cuŋ tukma} \rede{be cold} (N+V, \rede{cold—hurt}) with  \emph{ham-biʔma} \rede{distribute among people} (V+V, \rede{distribute—give}). 

There are different degrees of morphological fusion of noun and verb, and some nouns may undergo operations that are not expected if they were incorporated. They can be topicalized by means of the particle \emph{=ko} (see \Next[a]), and two nouns selecting the same \isi{light verb} may also be coordinated (see \Next[b]). Such examples are rare, though. Note that \Next[a] is from a collection of proverbs and sayings, in which rhythm and rhyming constraints could lead to the insertion of particles such as \emph{=ko}. In most cases the noun and verb occur without any intervening material. The noun may also be modified independently, as the spontaneously uttered sentence in \Next[c] shows. Typically, the predicates are modified as a whole by adverbs, but here one can see that the noun may also be modified independently by adnominal modifiers. The modifying phrase is marked by a \isi{genitive}, which is never found on adverbial modifiers.

\ex.\ag.makkai=ga cama, chiʔwa=ga    khyu,   cabhak=ko  lak-ma, a-ŋoʈeŋma=jyu.\\
maize{\sc =gen} cooked\_grains nettle{\sc =gen} curry\_sauce paddy{\sc =top} dance{\sc -inf} {\sc 1sg.poss-}female\_in-law{\sc =hon}\\
\rede{Corn mash, nettle sauce, let us dance the paddy dance, dear sister-in-law.} \source{12\_pvb\_01.008}
\bg.kei=nuŋ cabhak lak-saŋ ucun n-joŋ-me.\\
drum{\sc =com} paddy dance{\sc -sim} nice {\sc 3pl-}do{\sc -npst}\\
\rede{They have a good time, dancing the drum dance and the paddy dance.}\footnote{The interpretation of \rede{dancing the paddy dance with drums} can be ruled out here, because the drums are not played in the paddy dance.} \source{01\_leg\_07.142}
	\bg. a-phok tuk=nuŋ=ga  sak tug-a=na.\\
	{\sc 1sg.poss}-stomach hurt{\sc =com=gen} hunger hurt{\sc -pst[3sg]=nmlz.sg}	\\
	\rede{I am starving.} (literally \rede{A hunger struck (me) that makes my stomach ache.}) 

	
Some of the nouns may even trigger agreement on the verb, something which is also unexpected from the traditional definition of compounds, which entails that compounds are one unit lexically and thus morphologically opaque  (see e.g., \citealt{Fabb2001Compounding}). Example \Next[a] and \Next[b] are different in this respect:\footnote{Many Yakkha verbs have inchoative-stative Aktionsart so that the past inflection refers to a state that still holds true at the time of speaking.} while predicates that contain the verb \emph{tukma} \rede{hurt} are invariably inflected for third person singular (in other words, the noun \emph{whaŋma} triggers agreement; overt arguments are not possible), predicates containing \emph{sima} \rede{die}  show agreement with the overtly expressed (\isi{experiencer}) subject in the unmarked \isi{nominative} \Next[b].\footnote{The same \isi{argument realization} is found in the Belhare cognates of these two verbs \citep{Bickel1997The-possessive}. For the details of \isi{argument realization} in Yakkha see Chapter \ref{verb-val}.} Some meanings can be expressed by either frame (compare \Next[b] and \Next[c]), but this is not a regular and productive alternation.
  
\ex.\ag.whaŋma tug-a=na.\\
sweat hurt{\sc [3sg]-pst=nmlz.sg}\\
\rede{I/you/he/she/it/we/they feel(s) hot.}
\bg.ka wepma sy-a-ŋ=na.\\
{\sc 1sg} thirst die{\sc -pst-1sg=nmlz.sg}\\
\rede{I am thirsty.}
\bg.wepma tug-a=na.\\
thirst hurt{\sc [3sg]-pst=nmlz.sg}\\
\rede{I/you/he/she/it/we/they is/are thirsty.}

Note that if \emph{wepma} in  \Last[b] were a regular verbal argument, an \isi{instrumental} \isi{case} would be expected, since it is an effector with respect to the verbal meaning. And indeed, some noun-verb predicates require an \isi{instrumental} or an \isi{ergative} \isi{case} on the noun (see \Next).\footnote{Yakkha has an \isi{instrumental}/\isi{ergative} syncretism. Therefore, in intransitive predicates \emph{=ŋa} is interpreted as instrumental; in transitive predicates it is interpreted as \isi{ergative}.} 

\ex.\ag. (chemha=ŋa) sis-a-ga=na=i?\\
		(liquor{\sc =erg}) kill{\sc -pst-2.P=nmlz.sg=q}\\
		\rede{Are you drunk?}
	\bg. sak=ŋa n-sy-a-ma-ŋa-n=na.\\
			hunger{\sc =ins} {\sc neg-}die{\sc -pst-prf-1sg-neg=nmlz.sg}\\
		\rede{I am not hungry.}
	
	
Some verbs participating in noun-verb predicates have undergone semantic changes. Note that in \Last the nouns \emph{sak} and \emph{chemha} do not have the same status with regard to establishing the semantics of the whole predicate. The verbal stem \emph{sis} \rede{kill} in (a) has already acquired a metaphorical meaning  of \rede{be drunk, be intoxicated} (with the \isi{experiencer} coded like a standard object). The noun is frequently omitted in natural speech, and if all arguments are overt, the \isi{experiencer} precedes the stimulus, just like in Experiencer-as-Object constructions (see \Next and Chapter \ref{tr-objex}).\footnote{The same development has taken place in Belhare \citep[151]{Bickel1997The-possessive}.} In contrast to this, the stem \emph{si} in (b) is not polysemous; the noun is required to establish the meaning of the construction. 
	
\exg. ka macchi=ŋa haŋd-a-ŋ=na.\\
	{\sc 1sg}	pickles/chili{\sc =erg} taste\_hot{\sc -pst-1sg.P=nmlz.sg}\\
		\rede{The pickles/chili tasted hot to me.}


Despite a certain \isi{degree} of morphosyntactic freedom, the nouns are not full-fledged arguments. It is not possible to demote or promote the noun via \isi{transitivity operations} such as the causative or the passive, or to extract it from the noun-verb complex via \isi{relativization} (see ungrammatical \Next).
 
\ex.\ag.*lakt-i=ha cabhak\\
dance{\sc -1pl[pst]=nmlz.nsg} paddy\\
Intended: \rede{the paddy (dance) that we danced}
\bg.*tug-a=ha sak\\
hurt{\sc [3sg]-pst=nmlz.nsg} hunger\\
Intended: \rede{the hunger that was perceivable}


To sum up, \isi{simple noun-verb predicates} behave like one word with respect to lexical semantics, adjacency (in the overwhelming majority of examples), extraction possibilities for the noun (i.e., the lack thereof). They behave like two words as far as \isi{clitic} placement (including \isi{case}), coordination, modifiability, stress and \isi{voicing} are concerned. Thus, they are best understood as lexicalized phrases.


\section{Experiencer-as-possessor constructions}\label{nv-comp-poss}

Following a general tendency of languages of South and \isi{Southeast Asia}, Yakkha has a dedicated construction for the expression of experiential concepts, including emotional and cognitive processes, bodily functions, but also human character traits and their moral evaluation. In Yakkha, such concepts are expressed  by predicates that are built from a noun and a verb, whereby the noun is perceived as the location of this  concept, i.e., the “arena” where a physiological or psychological experience unfolds \citep[8]{Matisoff1986Hearts}. These nouns are henceforth referred to as psych-nouns, but apart from referring to emotions and sensations, they can also refer to body parts and excreted substances. Example \Next illustrates the basic pattern:

\exg.u-niŋwa tug-a=na.\\
{\sc 3sg.poss-}mind hurt{\sc [3sg]-pst=nmlz.sg}\\
\rede{He was/became sad.}


The verbs come from a rather small class;  they denote the manner in which the \isi{experiencer} is affected by the event, many of which refer to \isi{motion} events. The \isi{experiencer} is morphologically treated like the possessor of the \isi{psych-noun}; it is indexed by \isi{possessive prefixes}. The expression of experiential concepts by means of a possessive metaphor is a characteristic and robust feature of Kiranti languages (cf.  the “possessive of experience" in \citet{Bickel1997The-possessive}, “emotive predicates" in \citet[72]{Ebert1994The-structure},  and “body part emotion verbs" in \citealt[219]{Doornenbal2009A-grammar}), but this  is also found beyond Kiranti in  South-East Asian languages, including Hmong-Mien, Mon-Khmer and Tai-Kadai languages \citep{Matisoff1986Hearts, Bickel2004The-syntax}. In other \isi{Tibeto-Burman} languages, such as \ili{Newari}, Balti and Tibetan, for instance, experiencers are marked by a dative \citep{Beyer1992_Tibetan, Genetti2007_Newari, Read1934Balti}, an option which is not available, at least not by native morphology, in most Kiranti languages.


Experiencer-as-possessor constructions are not the only option to express experiential events. The crosslinguistic variation that can be found within \isi{experiential predicates} is also reflected in the language-internal variation of Yakkha. We have seen simple noun-verb predicates in \sectref{simple-noun-verb} above. Other possibilities are simple verbal stems like \emph{haŋma} \rede{taste hot/have a spicy sensation} (treating the \isi{experiencer} like a standard P argument), \emph{eʔma} \rede{perceive, like, have an impression, have opinion} (treating it like a standard A argument) and the  historically complex  verb \emph{kisiʔma} \rede{be afraid} (treating it like a standard S argument). Verbs composed of several verbal stems may also encode experiential notions, such as \emph{yoŋdiʔma} \rede{be scared} (a compound consisting of the roots for \rede{shake} and \rede{give}). It is the \isi{experiencer-as-possessor construction} though that constitutes the biggest class of \isi{experiential predicates}. About  fifty verbs have been found so far (cf. Tables \ref{tab-exp1} through \ref{tab-exp2c}), but probably this list is far from exhaustive. 

This section is organized as follows:  the various possibilities of \isi{argument realization} within the experiencer-as-possessor frame are introduced in \sectref{poss-e1} introduces, \sectref{poss-e3} looks at the principles behind the semantic composition of possessive \isi{experiential predicates}, and \sectref{poss-e2} deals with the morphosyntax of these predicates and with the behavioral properties of experiencers as non-canonically marked S or A arguments. 


\subsection{Subframes of argument realization}\label{poss-e1}

A basic distinction can be drawn between predicates of intransitive \isi{valency} and transitive or labile\footnote{See also \sectref{labile}.} \isi{valency}. Within this  basic distinction, the verbs can be further divided into various subframes of \isi{argument realization} (see Tables \ref{tab-exp1} through  \ref{tab-exp2c} at the end of the section). In all classes, the \isi{experiencer} is marked as possessor of the \isi{psych-noun}, i.e., as possessor of a sensation or an affected body part.

In the class of intransitive verbs, the \isi{psych-noun} triggers third \isi{person marking} on the verb, as in \Last and \Next. Intransitive verbs usually do not have an overt \isi{noun phrase} referring to the \isi{experiencer}; only the possessive prefix identifies the reference of the \isi{experiencer}. When the \isi{experiencer} has a special pragmatic status, and is thus marked by a discourse particle, it can be overtly expressed  in either the \isi{nominative} or in the \isi{genitive} (compare example \ref{kacaayupma} and \ref{ukkaseopomma} below). As this is quite rare, the reasons for this alternation are not clear yet.

In some cases, the noun is conceptualized as nonsingular, triggering  the according  \isi{number} markers on the verb as well (see \Next[a]). One verb in this group is special in consisting of  two nouns and a verb  (see \Next[b]). Both nouns take the possessive prefix. Their respective full forms would be \emph{niŋwa} and \emph{lawa}. It is not uncommon that the nouns get reduced to one \isi{syllable} in noun-verb predicates.

\ex.\ag.a-pomma=ci ŋ-gy-a=ha=ci.\\
{\sc 1sg.poss-}laziness{\sc =nsg} {\sc 3pl-}come\_up{\sc -pst=nmlz.nsg=nsg}\\
\rede{I feel lazy.}
\bg.a-niŋ a-la sy-a=na.\\
{\sc 1sg.poss-}mind {\sc 1sg.poss-}spirit die{\sc [3sg]-pst=nmlz.sg}\\
\rede{I am fed up/annoyed.}



The transitive group can be divided into five classes (cf. Tables \ref{tab-exp2}, \ref{tab-exp2b} and \ref{tab-exp2c} on pages \pageref{tab-exp2}--\pageref{tab-exp2c}). In all classes, the \isi{experiencer} is coded as the possessor of the \isi{psych-noun} (via \isi{possessive prefixes}), and hence this does not need to be explicitly stated in the schematic representation of \isi{argument realization} in the table. 

In class (a) the \isi{experiencer} is realized like a standard transitive subject (in addition to being indexed by \isi{possessive prefixes}): it  triggers transitive subject agreement and has \isi{ergative} \isi{case} marking (only overtly marked if it has third person reference and is overt, which is rare). The stimulus is unmarked and triggers object agreement (see \Next[a]). 

Class (b) differs from class (a) in that the \isi{psych-noun} triggers object agreement, invariably third person and in some cases, third person plural (see \Next[b]). No stimulus is expressed in class (b). This class has the highest number of members. 

 \ex.\ag. uŋ=ŋa   u-ppa             u-luŋma  tukt-uks-u=na.\\
{\sc 3sg=erg} {\sc 3sg.poss-}father {\sc 3sg.poss-}liver pour{\sc -prf-3.P=nmlz.sg}\\
\rede{He loved his father.} (literally \rede{He poured his father his liver.})
\bg.\label{ex-yupmaci}a-yupma=ci cips-u-ŋ-ci-ŋ=ha.\\
 {\sc 1sg.poss-}sleepiness{\sc =nsg} complete{\sc -3.P[pst]-1sg.A-3nsg.P-1sg.A=nmlz.nsg}\\
\rede{I am well-rested.} (literally \rede{I completed my sleep(s).})

Predicates of class (c) show three possibilities of \isi{argument realization}. One possibility is an unexpected pattern where the stimulus triggers object agreement, while the \isi{psych-noun} triggers subject agreement, which leads, oddly enough, to  a literal translation \rede{my disgust brings up bee larvae} in \Next[a]. Despite the subject agreement on the verb, the  psych-nouns in this class do not host an \isi{ergative} \isi{case} marker,  an option that is available, however, for verbs of class (d). The \isi{experiencer} is indexed only  by the possessive prefix in this frame; overt \isi{experiencer} arguments were not found. The stimulus can be in the \isi{nominative} or in the \isi{ablative} in class (c), but if it is in the \isi{ablative}, the verb is blocked from showing object agreement with the stimulus, showing 3>3 agreement instead (see \Next[b]). The third option of \isi{argument realization} in class (c) is identical to class (a) (cf. the comments in \Next[a] and (b)). Reasons or conditions for these alternations, for instance  in different  configurations of the referential properties of the arguments, could not be detected. 

\ex.\ag.thaŋsu=ga u-chya=ci a-chippa ket-wa-ci=ha.\\
bee{\sc =gen} {\sc 3sg.poss-}child{\sc =nsg} {\sc 1sg.poss-}disgust bring\_up{\sc -npst-3nsgP=nmlz.nsg}\\
\rede{I am disgusted by the bee larvae.} \\
(same: \emph{thaŋsuga ucyaci achippa ketwaŋciŋha} - (1{\sc sg}>3{\sc pl}, class (a)))
\bg.njiŋda=bhaŋ a-sokma hips-wa=na!\\
{\sc 2du=abl} {\sc 1sg.poss-}breath whip-{\sc npst[3A>3.P]=nmlz.sg}\\
\rede{I get fed up by you.} \\
(same: \emph{njiŋda asokma himmeʔnencinhaǃ} - 1{\sc sg}>2{\sc du}, class (a))

\newpage
In class (d), the \isi{psych-noun} also triggers transitive subject agreement, and it exhibits \isi{ergative} marking. The object agreement slot can be filled either by  the stimulus  or by the \isi{experiencer} argument (see \Next[a]).\footnote{There are (at least) two concepts, \emph{saya} and \emph{lawa}, that are related to or similar to \rede{soul} in Yakkha and the Kiranti metaphysical world in general.  \citet{Gaenszle2000Origins} writes about these two (and other) concepts in Mewahang (also Eastern Kiranti, Upper Arun branch): 

\begin{quote}
The concept of \emph{saya} is understood to be a kind of “vital force" that must be continually renewed (literally “bought") by means of various sacrificial rites. [...] The vital force \emph{saya} makes itself felt [...] not only in subjective physical or psychic states but also, and in particular, in the social, economic, religious and political spheres - that is, it finds expression in success, wealth, prestige and power. The third concept, \emph{lawa} (cf. \citet[165]{Hardman1981The-psychology}, \citealt[299]{Hardman_phd_Conformity}) is rendered by the \ili{Nepali} word \emph{sāto} (\rede{soul}). This is a small, potentially evanescent substance, which is compared to a mosquito, a butterfly or a bee, and which, if it leaves the body for a longer period, results in loss of consciousness and mental illness. The shaman must then undertake to summon it back or retrieve it. \citep[119]{Gaenszle2000Origins} 
\end{quote}
}

Class (e) is exemplified by \Next[b]. Here, the \isi{experiencer} is the possessor of a body part  which triggers object agreement on the verb. Some verbs may express an effector or stimulus overtly. Others, like \emph{ya limma} \rede{taste sweet} cannot express an overt A argument, despite being inflected transitively (see \Next[c]). This pattern is reminiscent of the transimpersonal verbs (treated in \sectref{tr-imp}).

 \ex.\ag.\label{ex-lawa}a-lawa=ŋa naʔ-ya-ŋ=na.\\
 {\sc 1sg.poss-}spirit{\sc =erg} leave{\sc -V2.leave-pst-1sg.P=nmlz.sg}\\
 \rede{I was frozen in shock.} (literally \rede{My spirit left me.})
 \bg. (cuŋ=ŋa) a-muk=ci khokt-u-ci=ha.\\
 (cold{\sc =erg}) {\sc 1sg.poss-}hand{\sc =nsg} chop{\sc -3.P[pst]-3nsg.P=nmlz.nsg}\\
\rede{My hands are tingling/freezing (from the cold).} (literally \rede{The cold chopped off my hands.})
\bg.a-ya limd-u=na.\\
{\sc 1sg.poss-}mouth taste\_sweet{\sc -3.P[pst]=nmlz.sg}\\
\rede{It tastes sweet to me.}

Many of the transitive verbs are attested also with intransitive inflection without further morphological marker of decreased \isi{transitivity}, i.e., they show a lability alternation (see \Next).

 \ex.\ag.n-lok  khot-a-ŋ-ga=na=i?\\
	{\sc 2sg.poss}-anger scratch{\sc -pst-1sg.P-2.A=nmlz.sg=q}	\\
	\rede{Are you angry at me?}
\bg.  o-lok khot-a=na.\\
{\sc 3sg.poss-}anger scratch{\sc [3sg]-pst=nmlz.sg}\\
\rede{He/she got angry.}


For two verbs, namely \emph{nabhuk-lemnhaŋma} \rede{dishonor (self/others)} (literally \rede{throw away one's nose}) and \emph{nabhuk-yuŋma} \rede{uphold moral} (literally \rede{keep one's nose}), there is one more constellation of participants, due to their particular semantics. The \isi{experiencer} can either be identical to the agent or different from it, as the social consequences of morally transgressive behavior usually affect more people than just the agent (e.g., illegitimate sexual contacts, or an excessive use of swearwords).\footnote{This concept is particularly related to immoral behavior of women. It is rarely, if ever, heard that a man \rede{threw away his nose}.} The morphosyntactic consequences of this are that the verbal agreement and the possessive prefix on the noun may either have the same conominal or two different conominals. Taken literally, one may \rede{throw away one's own nose} or \rede{throw away somebody else's nose} (see \Next). Note that due to the possessive \isi{argument realization} it is possible to have partial coreference, which is impossible in the standard transitive verbal inflection (cf. \sectref{verb-infl}).


\ex. \ag. u-nabhuk lept-haks-u=na.\\
{\sc 3sg.poss-}nose throw{\sc -V2.send-3.P[pst]=nmlz.sg}\\
\rede{She dishonored herself.} 
\bg. nda eŋ=ga nabhuk(=ci) lept-haks-u-ci-g=haǃ\\
{\sc 2sg[erg]} {\sc 1pl.incl.poss=gen} nose({\sc =nsg}) throw{\sc -V2.send-3.P[pst]-3nsg.P-2.A=nmlz.nsg}\\
\rede{You dishonored us all (including yourself)!} 

In \sectref{simple-noun-verb} cognate object constructions like \emph{chipma chima} \rede{urinate} were discussed. In these cases, the noun is cognate to the verb and does not actually make a semantic contribution to the predicate. Such developments are also found in the experiencer-as-possessor frame. Example \Next[a] and \Next[b] are two alternative ways to express the same propositional content. Note the change of \isi{person marking} to third person in (b). The noun \emph{phok} \rede{belly} is, of course, not etymologically related to the verb in this case, but it also does not make a semantic contribution. Further examples are \emph{ya limma} \rede{taste sweet} (\emph{ya} means \rede{mouth}) and \emph{hi ema} \rede{defecate} (\emph{hi} means \rede{stool}).

\ex.\ag.ka khas-a-ŋ=na.\\
{\sc 1sg} be\_full{\sc -pst-1sg=nmlz.sg}\\
\rede{I am full.}
\bg.a-phok khas-a=na.\\
{\sc 1sg.poss-}belly be\_full{\sc [3sg]-pst=nmlz.sg}\\
\rede{I am full.}

All frames of \isi{argument realization} with examples are provided in Tables \ref{tab-exp1} through \ref{tab-exp2c}.\footnote{Stems in square brackets in the tables were not found as independent words beyond their use in these collocations.}

\begin{table}[p]
\begin{tabularx}{\textwidth}{lll}
\lsptoprule
{\sc predicate} & {\sc gloss} &{\sc literal translation}\\
\midrule
\multicolumn{3}{l}{\{(S[{\sc exp]-nom/gen}) V-s[3]\}}\\
\midrule
\emph{chipma lomma}&\rede{have to pee}&(urine – come out)\\ 
\emph{hakamba keʔma}&\rede{yawn}&(yawn – come up)\\ 
\emph{hakchiŋba keʔma}&\rede{sneeze}&(sneeze – come up)\\ 
\emph{heli lomma}&\rede{bleed}&(blood – come out)\\ 
\emph{hi lomma}&\rede{have to defecate}&(shit – come out)\\ 
\emph{laŋ miŋma}&\rede{twist/sprain leg}&(leg – sprain)\\ %other limbs?
\emph{laŋ sima}&\rede{have paraesthetic leg}&(leg – die)\\ %other limbs?
\emph{miʔwa uŋma}&\rede{cry, shed tears}&(tear – come down)\\ 
\emph{niŋ-la sima}&\rede{be fed up}&([mind] – [spirit] – die)\\ 
\emph{niŋwa kaŋma}&\rede{give in, surrender}&(mind – fall)\\
\emph{niŋwa khoŋdiʔma}&a)\rede{be mentally ill}& (mind – break down)\\
&b)\rede{be disappointed/sad}&\\
\emph{niŋwa ima}&\rede{feel dizzy}&(mind – revolve)\\ 
\emph{niŋwa tama}&\rede{be satisfied, content}&(mind – come)\\%nniŋda jacpe pas leksighabhoŋ aniŋwa tayana, nniŋdanuŋ aniŋwa tayana
\emph{niŋwa tukma}&\rede{be sad, be offended}&(mind – be ill/hurt)\\
\emph{niŋwa wama}&\rede{hope}&(mind – exist)\\ 
\emph{phok kama}&\rede{be full}&(stomach – be full/saturated)\\
\emph{pomma keʔma}&\rede{feel lazy}&(laziness – come up)\\
\emph{saklum phemma}&\rede{be frustrated}&(frustration – be activated)\\ 
\emph{ʈaŋ pokma}&\rede{be arrogant, naughty}& (horn – rise)\\
\emph{yuncama keʔma}&\rede{have to laugh, chuckle}&(laugh – come up)\\ 
\emph{yupma yuma}&\rede{be tired}&(sleepiness – be full)\\
\lspbottomrule
\end{tabularx}\\

\caption{Intransitive experiencer-as-possessor predicates}\label{tab-exp1} 
\end{table}



%\pagestyle{empty}


\begin{table}%[p] 
\begin{tabularx}{\textwidth}{lll}
\lsptoprule
{\sc predicate} & {\sc gloss }& {\sc literal translation}\\
\midrule
\multicolumn{3}{l}{Class (a): \{A{\sc [exp]-erg} P{\sc [stim]-nom} V-a[A].p[P]\}}\\
\midrule
\emph{chik ekma}&\rede{hate}&(hate – make break)  \\%(\ti intr.)
\emph{lok khoʔma}&\rede{be angry at}&(anger – scratch) \\%(\ti intr.)
\emph{luŋma kipma}&\rede{be greedy}&(liver – cover tightly) \\%(\ti intr.)
\emph{luŋma tukma}&\rede{love, have compassion}&(liver – pour)\\
\emph{na hemma}&\rede{be jealous}& ([jealousy] – [feel]) \\
\lspbottomrule
\end{tabularx} 
\caption{Transitive experiencer-as-possessor predicates, Class (a)}\label{tab-exp2} 
\end{table}

\begin{table}%[p] 
\begin{tabularx}{\textwidth}{lXX}
\lsptoprule
{\sc predicate} & {\sc gloss }& {\sc literal translation}\\
\midrule
\multicolumn{3}{l}{Class (b): \{A{\sc [exp]-erg} P{\sc [noun]-nom} V-a[A].p[3]\}}\\
\midrule 
\emph{hi ema}&\rede{defecate}&(stool-defecate)\\ 
\emph{iklam saŋma}&\rede{clear throat, harrumph}&(throat – brush)\\ 
\emph{khaep cimma}&\rede{be satisfied, lose interest}& ([interest] –\newline be completed)\\
\emph{miʔwa saŋma}&\rede{mourn (ritually)}&(tear – brush)\\
\emph{nabhuk lemnhaŋma}&\rede{dishonor self/others}&(nose – throw away)\\
\emph{nabhuk yuŋma}&\rede{uphold own/\newline others' moral}&(nose – keep)\\
\emph{niŋwa chiʔma}&\rede{see reason, get grown up}&(mind – [get conscious])\\ 
\emph{niŋwa cokma}&\rede{pay attention}&(mind – do)\\ 
\emph{niŋwa hupma}&\rede{unite minds, decide together}&(mind – tighten, unite)\\ 
\emph{niŋwa lapma}&\rede{pull oneself together}&(mind – hold)\\ 
\emph{niŋwa lomma}&\rede{have/apply an idea}&(mind – take out)\\
\emph{niŋwa piʔma}&\rede{trust deeply}&(mind – give)\\ 
\emph{niŋwa yuŋma}&\rede{be careful}&(mind – put)\\ 
\emph{saya pokma}&\rede{raise head soul (ritually)}&(head soul –  raise)\\ %need examples
\emph{semla saŋma}&\rede{clear throat, clear voice}&(voice – brush)\\ 
\emph{sokma soma}&\rede{breathe}&(breath – breathe)\\ 
\emph{yupma cimma}&\rede{be well-rested}&(sleepiness –\newline be completed)\\ 
\lspbottomrule
\end{tabularx}
\caption{Transitive experiencer-as-possessor predicates, Class (b)}\label{tab-exp2b} 
\end{table}

\begin{table}%[p] 
\begin{tabularx}{\textwidth}{lp{3.5cm}l}
\lsptoprule
{\sc predicate} & {\sc gloss }& {\sc literal translation}\\
\midrule
\multicolumn{3}{l}{Class (c): \{P{\sc [stim]-nom} V-a[3].p[P]\} \ti \{P{\sc [stim]-abl} V-a[3].p[3]\} \ti \{Class (a)\} }\\
\midrule 
\emph{chippa keʔma}&\rede{be disgusted}&(disgust – bring up) \\
\emph{niŋsaŋ puŋma}&\rede{lose interest, have enough}&([interest] – [lose])\\
\emph{sokma himma}&\rede{be annoyed, be bored}&(breath – whip/flog)  \\
\emph{sap thakma}&\rede{like}& ([{\sc stem}] – send up)\\
\midrule
\multicolumn{3}{l}{Class (d): \{A{\sc [noun]-erg}  P{\sc [stim]-nom} V-a[3].p[A/P]\}}\\
\midrule 
\emph{niŋwa=ŋa cama}&\rede{feel sympathetic}&(mind=\textsc{erg} – eat)\\
\emph{niŋwa=ŋa mundiʔma}&\rede{forget}&(mind=\textsc{erg} – forget) \\
\emph{hop=ŋa khamma}&\rede{trust}&([{\sc stem}]-\textsc{erg} – chew)\\
\emph{niŋwa=ŋa apma}&\rede{be clever, be witty}&(mind=\textsc{erg} – bring)\\
\emph{lawa=ŋa naʔnama}&\rede{be frozen in shock, be scared stiff }&(spirit=\textsc{erg} – leave)\\
\midrule
\multicolumn{3}{l}{Class (e): \{P{\sc [stim]-erg} V-a[3].p[3]\}}\\
\midrule 
\emph{muk khokma}&\rede{freezing/stiff hands}&(hand-chop) \\
\emph{miʔwa saŋma} & (part of the death ritual) &(tear - brush off)\\
\emph{ya limma} (transimp.)& \rede{taste good} &(mouth - taste sweet)\\
\lspbottomrule
\end{tabularx} 
\caption{Transitive experiencer-as-possessor predicates, Classes (c)--(e)}\label{tab-exp2c} 
\end{table}

\pagestyle{scrheadings}

\subsection{Semantic properties}\label{poss-e3}

The experiencer-as-possessor predicates are far less transparent and predictable than the \isi{simple noun-verb predicates}. The nouns participating in this structure refer to abstract psychological or moral concepts like \emph{lok} \rede{anger}, \emph{yupma} \rede{sleepiness} and \emph{pomma} \rede{laziness}, or they refer to body parts or inner organs which are exploited for experiential metaphors. The lexeme \emph{luŋma} \rede{liver}, for instance, is used in the expression of love and greed, and \emph{nabhuk} \rede{nose} is connected to upholding (or eroding) moral standards. The human body is a very common source for psychological metaphors, or as  Matisoff observed: 

\begin{quote}
[...] it is a universal of human metaphorical thinking to equate mental operations and states with bodily sensations and movements, as well as with physical qualities and events in the outside world. \citep[9]{Matisoff1986Hearts} 
\end{quote} 

In Yakkha, too, psychological concepts are treated as concrete tangible entities that can be possessed, moved or otherwise manipulated. Many verbs employed in experiencer-as-possessor predicates are verbs of \isi{motion} and caused \isi{motion}, like \emph{keʔma} (both \rede{come up} and \rede{bring up}, distinguished by different stem behavior), \emph{kaŋma} \rede{fall}, \emph{haŋma} \rede{send}, \emph{lemnhaŋma} \rede{throw}, \emph{pokma} \rede{raise} or \emph{lomma} (both  \rede{take out} and \rede{come out}). Other verbs refer to physical change (both spontaneous and caused), such as \emph{khoŋdiʔma} \rede{break down}, \emph{himma} \rede{whip/flog} or \emph{kipma} \rede{cover tightly}. Most of the predicates acquire their experiential semantics only in the particular idiomatic combinations. Only a few verbs have intrinsic experiential semantics, like \emph{tukma} \rede{hurt/be ill}.


\subsection{Morphosyntactic properties}\label{poss-e2}

\subsubsection{Wordhood vs. phrasehood}

Experiencer-as-possessor predicates host both nominal and verbal morphology. A possessive prefix (referring to the \isi{experiencer}) attaches to the noun, and the verbal inflection attaches to the verb. The verbal inflection always attaches to the verbal stem, so that the verbal prefixes stand between the noun and the verb (see \Next). It has been shown above that some of the  psych-nouns can  be inflected for \isi{number} as well as trigger plural morphology on the verb, and that others may show \isi{case} marking (see \ref{ex-yupmaci} and \ref{ex-lawa}).
\largerpage

\exg. a-luŋma n-duŋ-meʔ-nen=na.\\
		{\sc 1sg.poss}-liver {\sc neg-}pour{\sc -npst-1>2=nmlz.sg}\\
	\rede{I do not love you./I do not have compassion for you.}


The \isi{experiencer} argument, which is always indexed by the possessive prefix on the noun, is rarely expressed overtly. It may show the following properties: it is in the \isi{nominative} or in the \isi{genitive} when the \isi{light verb} is intransitive, and in the \isi{ergative} in predicates that show transitive subject agreement with the \isi{experiencer} argument (class (a) and (b)). 

Noun and verb have to be adjacent, as shown by the following examples. Constituents like \isi{degree} adverbs and \isi{quantifiers}  (see \Next[a] and \Next[b]) or question words (see \Next[c]) may not intervene. 

\ex.\ag. tuknuŋ u-niŋwa (*tuknuŋ) tug-a-ma, {\hspace{-.4cm}\ob\dots\cb}\\
 completely {\sc 3sg.poss}-mind  (*completely) hurt{\sc [3sg]-pst-prf}\\
 \rede{She was so sad, ...} \source{38\_nrr\_07.009}
 \bg. ka khiŋ pyak a-ma=ŋa u-luŋma  (*khiŋ pyak) tuŋ-me-ŋ=na!\\
 {\sc 1sg} so\_much much {\sc 1sg.poss-}mother{\sc =erg} {\sc 3sg.poss-}liver (*so\_much much) pour{\sc [ 3sg.A]-npst-1sg.P=nmlz.sg}\\
 \rede{How much my mother loves me!}\source{01\_leg\_07.079}
 \bg. ijaŋ n-lok (*ijaŋ) khot-a-ŋ-ga=na=i?\\
 why	{\sc 2sg.poss}-anger (*why) scratch{\sc -pst-1sg.P-2sg.A=nmlz.sg=q}\\
 \rede{Why are you angry at me?}


Information-structural clitics, usually attaching to the rightmost element of the phrase, may generally stand between noun and verb, 
but some combinations were judged better than others (compare \Next[a] with \Next[b]). Compare also the impossible \isi{additive focus} particle \emph{=ca} in \Next[a] with the \isi{restrictive focus} particle \emph{=se} and the contrastive particle \emph{=le} in \NNext. Overtly expressed \isi{experiencer} arguments  may naturally also host topic and focus particles, just like any other constituent can. This is shown e.g., by \Next[b], \NNext[c] and \NNext[d].

%*** \Next[a] how then?? kaca ayupmaci ....? -yes

\ex. \ag. a-yupma=ci(*=ca) n-yus-a=ha=ci.\\
	{\sc 1sg.poss}-sleepiness{\sc =nsg(*=add)} {\sc 3pl.A}-be\_full{\sc -pst=nmlz.nsg=nsg}\\
	Intended: \rede{I am tired, too (in addition to being in a bad \isi{mood}).} 
	\bg.u-ʈaŋ=ca pog-a-by-a=na.\\
	{\sc 3sg.poss-}horn{\sc =add} rise{\sc [3sg]-pst-v2.give-pst=nmlz.sg}\\
	\rede{She is also naughty.}
	\bg. ka=ca a-yupma=ci n-yus-a=ha=ci.\label{kacaayupma}\\
	{\sc 1sg=add} {\sc 1sg.poss}-sleepiness{\sc =nsg} {\sc 3pl.A}-be\_full{\sc -pst=nmlz.nsg=nsg}\\
	Only: \rede{I am also tired (in addition to you being tired).}  (not e.g., \rede{I am tired in addition to being hungry.})
	

\ex.\ag. a-saklum=ci=se m-phen-a-sy-a=ha=ci.\\
{\sc 1sg.poss}-need{\sc =nsg=restr} {\sc 3pl}-be\_activated-{\sc pst-mddl-pst=nmlz.nsg=nsg}	\\
\rede{I am just pining for it.} 
\bg. uŋ=ŋa   u-ma             u-chik=se  ekt-uks-u-sa.\\
{\sc 3sg=erg} {\sc 3sg.poss-}mother {\sc 3sg.poss-}hate{\sc =restr} make\_break{\sc -prf-3.P[pst]-pst.prf}\\
\rede{He had nothing but hate for his mother.}\source{01\_leg\_07.065}
\bg.ka=go a-sap=le thakt-wa-ŋ=na.\\
{\sc 1sg=top} {\sc 1sg.poss-}[stem]{\sc =ctr} send\_up{\sc -npst[3.P]-1sg.A=nmlz.sg}\\
\rede{But I like it.} (said in contrast to another speaker)
\bg.  uk=ka=se  o-pomma=ci ŋ-gy-a=ha=ci.\label{ukkaseopomma}\\
{\sc 3sg=gen=restr} {\sc 3sg.poss-}laziness{\sc =nsg} {\sc 3pl.A}-come\_up{\sc -pst=nmlz.nsg=nsg}\\
\rede{Only he was lazy (not the others).}


The noun can even be omitted, in \isi{case} it was already active in discourse, such as in the question-answer pair in \Next. It is, however, not possible to extract the noun from the predicate to relativize on it, neither with the \isi{nominalizer} \emph{-khuba} nor with the  nominalizers \emph{=na} and \emph{=ha} as shown in \NNext (cf. Chapter \ref{ch-nmlz}).  Furthermore, in my corpus there is not a single example of a noun in a possessive experiential construction that is modified independently. The predicate is always modified as a whole, by adverbial modification. A certain \isi{degree} of  morphological freedom does not imply that the noun is a full-fledged argument.


 	\ex.\ag. ŋkha mamu=ci n-sap thakt-u-ci-g=ha=i?\\
	those girl{\sc =nsg} {\sc 2sg.poss-[stem]} send\_up-{\sc 3.P[pst]-3nsg.P-2.A=nmlz.nsg=q}\\
	\rede{Do you like those girls?}
	\bg. thakt-u-ŋ-ci-ŋ=ha!\\
	send\_up-{\sc 3.P[pst]-1sg.A-3nsg.P-1sg.A=nmlz.nsg}\\
	\rede{I do!} 

	\ex.\ag.*kek-khuba (o-)pomma\\
	come\_up-{\sc nmlz[S/A]} ({\sc 3sg.poss-})laziness\\
	Intended: \rede{the laziness that comes up}
	\bg.*ky-a=na (o-)pomma\\
	come\_up{\sc -pst=nmlz.sg} ({\sc 3sg.poss-})laziness\\
	Intended: \rede{the laziness that came up}

	
The noun-verb complex as a whole may serve as input to derivational processes, such as the creation of \isi{adjectives} by means of a \isi{reduplication} and the \isi{nominalizer} \emph{=na} or \emph{=ha}, shown in \Next. 

\ex.\ag.uŋ tuknuŋ    luŋma-tuk-tuk=na       sa-ya-ma.\\
{\sc 3sg} completely liver-{\sc redupl-}pour{\sc =nmlz.sg}  be{\sc [3]-pst-prf}\\
\rede{She was such a kind (loving, caring) person.}\source{01\_leg\_07.061}
\bg.ikhiŋ chippa-ke-keʔ=na  takabaŋ!\\
how\_much disgust-{\sc redupl-}come\_up{\sc =nmlz.sg} spider\\
\rede{What a disgusting spider!}
\bg.nna chik-ʔek-ʔek=na  babu\\
that hate-{\sc redupl-}make\_break{\sc =nmlz.sg} boy\\
\rede{that outrageous boy}


Wrapping up, just as we have seen above for the \isi{simple noun-verb predicates}, the noun and the verb build an inseparable unit for some processes, but not for others; the predicates show both word-like and phrasal properties. Semantically, of course, noun and verb build one unit, but  they can be targeted by certain morphological and syntactic processes: the nonsingular marking on psych-nouns, psych-nouns triggering agreement, the possibility of hosting phrasal clitics, and the partial ellipsis. The ambiguous status of these predicates is also reflected in their phonology: noun and verb are two units with respect to stress and \isi{voicing}.  

Another feature distinguishes the possessive \isi{experiencer} predicates from compounds: nouns in compounds are typically generic (\citealt[66]{Fabb2001Compounding}, \citealt[156]{Haspelmath2002Understanding}). As the noun in the possessive \isi{experiential predicates} hosts the possessive prefix, its reference is made specific. The contiguity of noun and verb, the derivation of \isi{adjectives} and the restrictions on extraction and modification also clearly show that noun and verb are one unit. All these conflicting properties of Yakkha add further support to approaches that question the notion of the word as opaque to morphosyntactic processes (as e.g., stated in the Lexical Integrity Principle). The possessive \isi{experiential predicates} may best be understood as lexicalized phrases, such as  the predicates discussed in \sectref{simple-noun-verb} above.  


\subsubsection{Behavioral properties of the experiencer arguments}\label{poss-e4}

Experiencers as morphologically downgraded, non-canonically marked subjects do not necessarily have to be downgraded in other parts of the grammar. As observed  by \citet{Bickel2004The-syntax}, \isi{Tibeto-Burman} languages, in contrast to Indo-Aryan languages, show a strong tendency to treat  experiencers as full-fledged arguments  syntactically. Yakkha confirms this generalization. In syntactic constructions that select pivots, the \isi{experiencer} argument is chosen, regardless of the fact that  it is often blocked from triggering verbal agreement. The \isi{nominalizer} \emph{-khuba} (S/A arguments) selects the \isi{experiencer}, because it is the most agent-like argument in the clause (see \Next). As the ungrammatical \Next[c] shows, the stimulus cannot be nominalized by \emph{-khuba}.

%***checked this from notes 2011: chippa kekkhuba camyoŋba not possibleǃǃ ( rather chippakekeʔna?), hangkhuba machi, *hangkhuba mamu (from KS, MM judged this as ungrammatical)

 \ex.\ag.takabaŋ u-chippa kek-khuba mamu\\
spider {\sc 3sg.poss-}disgust come\_up{\sc -nmlz} girl\\
\rede{the girl who is disgusted by spiders}
\bg.o-pomma kek-khuba babu\\
{\sc 3sg.poss-}laziness come\_up{\sc -nmlz} boy\\
\rede{the lazy fellow}
\bg.*chippa kek-khuba camyoŋba \\
disgust come\_up{\sc -nmlz} food \\
Intended: \rede{disgusting food} (only: \emph{chippakekeʔna})

Another process that exclusively selects S and A arguments is the converbal \isi{clause linkage}, which is marked by the suffix \emph{-saŋ}. It implies that two (or more) events happen  simultaneously, and it requires the referential identity of the S and A arguments in both clauses. Example \Next illustrates that this also holds for \isi{experiencer} arguments.

\ex.\ag. o-pomma kes-saŋ kes-saŋ kam cog-wa.\\
{\sc 3sg.poss-}laziness come\_up{\sc -sim} come\_up{\sc -sim} work do{\sc -npst[3.P]}\\
\rede{He does the work lazily.}
\bg.uŋ lok khos-saŋ lukt-a-khy-a=na.\\
{\sc 3sg} anger scratch{\sc -sim} run{\sc [3sg]-pst-V2.go-pst=nmlz.sg}\\
\rede{He ran away angrily.}
 

In causatives, the \isi{experiencer} is the causee, as is evidenced by the verbal marking in  \Next. There  is no overt marking for 1.P, but the reference is retrieved from the opposition to the other forms in the paradigm — with third person object agreement, the inflected form would have to be \emph{himmetugha}.

\exg. khem=nuŋ manoj=ŋa a-sokma him-met-a-g=haǃ\\
Khem{\sc =com} Manoj{\sc =erg} {\sc 1sg.poss-}breath whip{\sc -caus-pst-2.A[1.P]=nmlz.nsg}\\
\rede{Khem and Manoj (you) annoy me!}


The last syntactic property discussed here is the agreement in complement-taking verbs that embed infinitives, as for instance \emph{yama} \rede{be able} or \emph{tarokma} \rede{begin}, shown in \Next. Basically, the complement-taking verb mirrors the agreement that is found in the embedded verb. Those predicates whose \isi{experiencer} arguments do not trigger agreement in the verb  do not show agreement in the complement-taking verb either. Other restrictions  are semantic in nature, so that, for instance, \rede{I want to get lazy} is not possible, because being lazy is not conceptualized as something one can do on purpose. Thus, the agreement facts neither confirm nor contradict the generalization made above. A more interesting \isi{case} is the periphrastic \isi{progressive} construction, with the lexical verb in the \isi{infinitive} and an intransitively inflected auxiliary \emph{-siʔ} (infinitial form and auxiliary got fused into one word). The auxiliary selects the \isi{experiencer} as agreement triggering argument (see \Next[b]).
\largerpage

\ex.\ag.ka nda a-luŋma tuk-ma n-ya-meʔ-nen=na.\\
{\sc 1sg[erg]} {\sc 2sg} {\sc 1sg.poss-}liver pour{\sc -inf} {\sc neg-}be\_able-{\sc npst-1>2=nmlz.sg}\\
\rede{I cannot love you./I cannot have pity for you.}
\bg.nda ka ijaŋ n-lok khoʔ-ma-si-me-ka=na?\\
 {\sc 2sg[erg]} {\sc 1sg} why {\sc 2sg.poss-}anger scratch{\sc -inf-aux.prog-npst-2=nmlz.sg}\\
\rede{Why are you being angry at me?}


 	