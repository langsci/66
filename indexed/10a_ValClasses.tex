\section{Valency alternations} \label{valclass}
 
The frames introduced in \sectref{frames} show various alternations.  Different types of alternations have to be distinguished: some just change the \isi{argument realization}, e.g., differential \isi{case} marking, which is triggered by pragmatic factors such as  scenario classes. Other alternations, e.g., the inchoative-causative lability, change the \isi{argument structure}. 

The  \isi{labile verbs} will be discussed  in \sectref{labile},  \sectref{three-arg}  deals with the alternations among the three-argument frames.\footnote{For alternations found among the experiencer-as-possessor predicates see \sectref{nv-comp-poss}.} 


\subsection{Lability}\label{labile}

\largerpage[-1]
Labile verbs are characterized by variable \isi{transitivity} of the same verbal stem, which is not brought about by means of a morphological derivation. \citet[224]{Letuchiy2009Labile} classifies \isi{labile verbs}  into different types: the inchoative/causative alternation, the reflexive alternation, the reciprocal alternation, the passive (extremely rare) and the converse type. According to this classification, Yakkha has the inchoative/causative\footnote{The inchoative/causative type is equated with \rede{labile} in \citet{Haspelmath1993More}, whose definition of \isi{labile verbs} is more restrictive.} and the reflexive.\footnote{Furthermore, Yakkha shows morphologically unmarked detransitivizations that can have both passive and antipassive interpretations, but they do not change the semantic roles of the arguments and hence they are not lexical alternations. They are treated below in \sectref{trans-op} on \isi{transitivity operations}. Letuchiy acknowledges the passive-type as labile, but considers unmarked antipassives \emph{quasi-lability}, because his crucial defining feature for lability is a change of the semantic roles. But if semantic role change is required, his inclusion of the passive alternation is misleading. In passives, the semantic roles do not change; the undergoer of \rede{beat} does not have different semantic roles in the active vs. the passive voice.} The current lexical database contains 77 \isi{labile verbs}. The inchoative/causative alternation is patient-preserving; the reflexive alternation is agent-preserving (see also \citealt[223]{Letuchiy2009Labile}). 

As lability is defined by the absence of morphological marking, it is hard to tell which form of a labile pair is the basic form.  The intransitive verb can be considered the basic form semantically and formally, as less participants are involved in the event, and as the verb hosts less \isi{inflectional morphology} than the transitive verb.\footnote{From a first impression, there are definitely also differences in frequency among the \isi{labile verbs}. Some are rather used transitively and some intransitively, depending on which function of a verb is more plausible in natural discourse. The existing corpus is not big enough for significant statistic analyses.} 
 
\subsubsection{Inchoative-causative lability}

By far the majority of the \isi{labile verbs} belong to the inchoative-causative class, a fact that goes along with the crosslinguistic findings in \citet{Letuchiy2009Labile}. The intransitive verbs denote states or spontaneous changes of state. No agent or causer argument is entailed in the verbal semantics.\footnote{Notably, inchoative (\rede{anticausative} in \citealt{Creissels2012_Lability}) readings do not always express events that do not have an agent or a causer argument. Sometimes, the A is merely not relevant for a certain event, and thus it is not part of the underlying concept of the event, and has to be left unexpressed, as shown, e.g., for facilitative readings of anticausatives in Tswana by \citet{Creissels2012_Lability}.} In the corresponding transitive verb, a causer argument that brings about the event is added, and the P argument  corresponds to the S of the intransitive verb. Examples \Next[a], \Next[c] and \Next[e] show the inchoative verbs with S undergoing a spontaneous change of state, while \Next[b], \Next[d] and \Next[f] show the corresponding transitive verbs with an A argument bringing about that change of state. The verb \emph{cimma}, meaning both \rede{learn} and \rede{teach}, basically belongs to the same alternation, but it has one additional argument. The intransitively inflected verb has two arguments and the transitively inflected verb has three arguments (see \sectref{itr-teach}).  

\ex. \ag. dailo hos-a=na.\\
door open{\sc [3sg]-pst=nmlz.sg}\\
\rede{The door opened.}
\bg. a-ppa=ŋa dailo hos-uks-u=na.\\
{\sc 1sg.poss-}father{\sc =erg} door open{\sc -prf-3.P=nmlz.sg}\\
\rede{Father has opened the door.}
\bg. siŋ eg-a=na.\\
	wood break{\sc [3sg]-pst=nmlz.sg}\\
	\rede{The piece of wood broke.}  
	 \bg.uŋ=ŋa siŋ eg-u=na.\\ 
	{\sc 3sg=erg} wood break{\sc -3.P[pst]=nmlz.sg}		\\ 
	\rede{He broke the piece of wood.} 
	\bg.phuama yupma=ci=bhaŋ cend-a=na.\\
last-born\_girl  sleepiness{\sc =nsg=abl} wake\_up{\sc [3sg]-pst=nmlz.sg}\\
\rede{Phuama woke from her sleep.}
\bg.ka uŋ cend-u-ŋ=na.\\
{\sc 1sg[erg]}  {\sc 3sg}  wake\_up{\sc -3sg.P[pst]-1sg.A=nmlz.sg}\\
\rede{I woke her up.}  

\largerpage[-1]
There are border cases of lability. In Yakkha, many events are  expressed by complex predicates.  In these predicates, the first stem contains the lexical verb, such as the labile stem \emph{khiks \ti khiŋ} \rede{stretch, grow} in \Next. The second verbal stem is from the closed class of  \isi{function verb}s (V2s, see Chapter \ref{verb-verb}); they specify the verbal semantics, for instance with regard to the temporal structure. In \Next[a], the V2 \emph{-kheʔ} \rede{go} emphasizes the \isi{telicity} of the event. It is sensitive to \isi{transitivity}, too. The V2   \emph{-kheʔ} is only compatible with intransitive interpretations (see ungrammatical \Next[b]). Thus, complex predication can have the secondary function of indicating \isi{transitivity} features. 

\ex.\ag. ikhiŋ khiks-a({\bf -khy}-a)=naǃ\\
how\_much stretch{\sc [3sg]-pst(-V2.go-pst)=nmlz.sg}\\
\rede{How tall she becameǃ}
\bg.a-laŋ=ci khiŋ({\bf *-kheʔ})-ma=ci.\\
{\sc 1sg.poss-}leg{\sc =nsg} stretch{\sc (*-V2.go)-inf[deont]=nsg}\\
Intended: \rede{I have to stretch my legs.}


\subsubsection{Reflexive lability} 

The stems of this class alternate between a transitive reading and an intransitive reading with reflexive semantics. Strictly speaking, no argument is removed in reflexives, but the A and P have identical reference and collapse into one single intransitive subject role formally \citep[1134]{Haspelmath2004_Valency}. In the transitive reading, an external P argument is added. Typically, the verbs undergoing this alternation refer to actions involving the body. The examples in \Next  illustrate the reflexive alternation with three verb pairs. 

\ex. \ag. uŋci=ŋa men-ni-ma=nuŋ cum-a-ŋ=na.\\
	 {\sc 3nsg=erg} {\sc neg-}see{\sc -inf=com.cl} hide{\sc -pst-1sg=nmlz.sg}\\
	\rede{I hid, so that they cannot see (me).} 
 	\bg.ripu=ŋa khorek cum-u=na.\\ 
	Ripu{\sc =erg} bowl hide{\sc -3.P[pst]=nmlz.sg}		\\
	\rede{Ripu hid the bowl.}	
	\bg.            ka=ca         mimiʔ   wasiʔ-a-ŋ=hoŋ, ...\\
	{\sc 1sg=add} a\_little wash{\sc -pst-1sg=seq}\\
	\rede{After washing myself a little, ...}  \source{40\_leg\_08.050}
	 \bg.a-nuncha wasiʔ-wa-ŋ=na.\\ 
	{\sc 1sg.poss-}younger\_sibling wash{\sc -npst[3.P]-1sg.A=nmlz.sg}\\ 
	\rede{I wash my little sister.}  
\bg.a-chya (tek=ŋa) ept-a=na.\\
	{\sc 1sg.poss-}child (cloth{\sc =ins}) cover{\sc [3sg]-pst=nmlz.sg}\\
\rede{My child covered itself (with the blanket).}  
\bg.yenda ept-a-n-u-m.\\
millet\_mash cover{\sc -pst-pl-3.P[imp]-2pl.A}\\
	\rede{Cover the millet mash.}  


\subsection{Alternations in three-argument verbs}\label{three-arg}

Alternations in \isi{three-argument verbs} are mostly conditioned by pragmatic factors such as topicality or the referential properties of the arguments.\footnote{My investigation of referentiality effects in \isi{three-argument verbs} (see also \citealt{Schackow2012_Referential}) has been inspired by the EUROBabel project Referential Hierarchies in Morphosyntax (RHIM) and a questionnaire on three-argument constructions, designed by Anna Siewierska and Eva van Lier (not published).} Typically, in events with three arguments, the G arguments (goals, recipients) are animate, definite and thus also more topic-worthy, whereas the T arguments have a strong tendency to be inanimate, indefinite and thus less topic-worthy. Events in which this expected scenario is reversed are more marked pragmatically, and this could be reflected in the morphosyntax of the clause (\citealt{Dryer1986Primary, Siewierska2003Person}, \citealt{Haspelmath2004Explaining, Haspelmath2005Argument, Haspelmath2007Ditransitive}, \citealt{Malchukovetal2010Ditrans-overview}). Some of the referential effects are found exclusively in \isi{three-argument verbs} in Yakkha, for instance a \isi{case} of hierarchical agreement, where the T and the G argument compete for an agreement slot. One has to distinguish between argument-based alternations, i.e., effects that are conditioned by the referential properties of only one argument, and scenario-based alternations, i.e., effects that are conditioned by the properties of both T and G in relation to each other.


\subsubsection{The spray-load alternation}

One class of verbs shows alternations between the indirective and the \isi{secundative frame}, also known as \emph{spray-load alternation} \citep{Levin1993_English, Malchukovetal2010Studies, Malchukovetal2015_Valency}. Either the T argument is in the \isi{instrumental} \isi{case} and the G  triggers object agreement on the verb (for the \isi{secundative frame}, see \Next[a]), or the G argument is in the \isi{locative} and the T triggers object agreement (for the \isi{indirective frame}, see \Next[b]). 

	\ex. \ag. ka makai=ŋa dalo ipt-wa-ŋ=na.\\
	{\sc 1sg[erg]} corn{\sc =ins} sack fill{\sc -npst[3.P]-1sg=nmlz.sg}	\\
	\rede{I filled the sack with corn.}  (secundative)
	 \bg. gagri=be maŋcwa ipt-u.\\
	pot{\sc =loc} water fill{\sc -3.P[imp]}\\ 
	\rede{Fill the water into the pot.}   (indirective) 
 
The verb \emph{ipma} \rede{fill} in \Last can only have inanimate G arguments. Verbs with a greater variability of possible arguments may show restrictions on this alternation. Some verbs, for instance, block the \isi{secundative frame} when the G argument is inanimate, e.g.,  \Next[a], which renders the \isi{indirective frame} the only possibility (see \Next[b]). In order to license the \isi{secundative frame}, the G argument has to have the potential to be affected by the event \Next[c].  
The verb \emph{lupma} \rede{scatter, disperse, strew} provides another example of this restriction. Again, the \isi{secundative frame} is the preferred option for animate G arguments, while the indirective is used when inanimate G arguments are involved \NNext[b] (context: the preparation of millet beer). In \NNext[a], the G argument is non-overt, but it has human reference, which can be inferred from the context: a funeral. 
\largerpage

\ex.	\ag.  *ka maŋcwa luŋkhwak=ŋa lept-u-ŋ=ha.\\
	{\sc 1sg[erg]} water stone{\sc =ins}	throw{\sc -3.P[pst]-1sg=nmlz.nsg}	\\
	Intended: \rede{I threw a stone into the water.}   (*secundative)
	\bg.  ka lunkhwak maŋcwa=be  lept-u-ŋ=na.\\
	{\sc 1sg[erg]} stone  water{\sc =loc}  	throw{\sc -3.P[pst]-1sg=nmlz.sg}\\
 \rede{I threw a stone into the water.}  (indirective)  
 \bg. ka nda luŋkhwak=ŋa lep-nen=na.\\
 {\sc 1sg[erg]} {\sc 2sg} stone{\sc =ins}	throw{\sc [pst]-1>2=nmlz.sg}	\\
 \rede{I threw a stone at/to you.} (secundative)

\ex. \ag. kham=ŋa lupt-u-ga=i.\\ 
	soil{\sc =ins} scatter{\sc -3.P[imp]-2=emph} \\
	\rede{Cover him with sand.}   
	\bg. yenda=be khawa lupt-u-g=ha=i?\\
	millet\_mash{\sc =loc} yeast disperse{\sc -3.P[pst]-2=nmlz.nsg=q}   \\
	\rede{Did you add the yeast to the millet mash?}   
	 
 
\subsubsection{Alternations related to the animacy of G}\label{loc-alt}
 
One could see in the spray-load alternation that the unmarked \isi{nominative} is preferred for animate, sentient G arguments. For some verbs, this results in alternations between the \isi{double object frame} and the \isi{indirective frame}. In \Next[a], the G argument is human, moreover it is a speech-act participant, and thus the highest on the referential hierarchy \citep{Silverstein1976Hierarchy}. Hence, the \isi{double object frame} is chosen, the verb agrees with G, and both T and G are in the \isi{nominative}. In \Next[b], the G has third person inanimate reference, and the frame changes to indirective, with G in the \isi{locative}, and T triggering the agreement.\footnote{There is no \isi{number} hierarchy at work in these alternations. The \isi{number} of T is not the crucial factor, but nonsingular was chosen to illustrate the agreement.}  

 \ex. \ag. ka nda sandhisa khuʔ-nen=na.\\
{\sc 1sg[erg]}  {\sc 2sg}  present{\sc }	bring{\sc [pst]-1>2=nmlz.sg}\\
\rede{I brought you a present.}	
 	\bg. uŋ=ŋa  kitab(=ci) iskul=be khut-u-ci=ha.\\
		 {\sc 3sg=erg}  book{\sc (=nsg)} school{\sc =loc} bring{\sc -3.P[pst]-3nsg.P=nmlz.nsg}\\
	\rede{He brought the books to school.}  

	
Some verbs only change  the \isi{case} marking of G without changing the agreement.  The verb \emph{hambiʔma} \rede{distribute} is a \isi{benefactive} derivation of \emph{hamma} \rede{distribute, divide, spread}. In the typical scenario, the G argument is referentially high, the T argument is low, and the \isi{argument realization} follows the \isi{double object frame}, as in \Next[a]. When the G argument changes to  inanimate reference, as in example \Next[b], it has to  be in the \isi{locative} \isi{case}, but the verb does not change to the \isi{indirective frame}; and thus the agreement remains with G. Furthermore, instead of using the nonsingular marker \emph{=ci} on the G argument \emph{ten} \rede{village}, it is marked for nonsingular \isi{number} by \isi{reduplication}, which indicates a plurality of subevents. This kind of plural marking is not encountered when the G argument is human, as shown in example \Next[c].

\ex. \ag. ka nniŋda phoʈo(=ci) ham-biʔ-meʔ-nen-in=ha.\\
		{\sc 1sg[erg]} {\sc 2pl} photo{\sc (=nsg)} distribute{\sc -V2.give-npst-1>2-2pl=nmlz.nsg}	\\
	\rede{I distribute the photos among you.}  
 	\bg. sarkar=ŋa yaŋ ten-ten=be ŋ-haps-u-bi-ci=ha.\\
	government{\sc =erg} money village-village{\sc =loc}	{\sc 3pl.A-}distribute{\sc -3.P[pst]-V2.give-3nsg.P=nmlz.nsg}\\
	\rede{The government distributed the money among the villages.}  
		\bg. ka piccha=ci yaŋ haps-u-bi-ŋ-ci-ŋ=ha.\\
	{\sc 1sg[erg]}  child{\sc =nsg} money{\sc }	distribute{\sc -3.P[pst]-V2.give-1sg.A-nsg.P-1sg.A=nmlz.nsg}\\
	\rede{I distributed the money among the children.} 

	
\subsubsection{Scenario-based alternations}\label{scen-based}

Not only \isi{case} marking, but also the verbal \isi{person marking} can be subject to reference-based alternations. The Yakkha verb agrees with only one object, so that there is the potential for competition between T and G arguments as to which argument will trigger the agreement. The universal tendency for agreement to be triggered by arguments that are speech act participants, animate or topical has  already been mentioned by \citet{Givon1976Topic}. This tendency can lead to hierarchical \isi{alignment} of agreement, understood as agreement that is not determined by \isi{syntactic role}s but by the referential properties of the arguments \citep[66]{Nichols1992Language}. This is well-studied for monotransitive verbs, but not for \isi{three-argument verbs}.\footnote{The most prominent example for hierarchical \isi{alignment} in ditransitives is the Yuman language Jamul Tiipay (\citealt[162--163]{Miller2001A-grammar}, discussed, e.g., in \citealt[348]{Siewierska2003Person}).} 
 
There are two verbs of the double object class which allow animate/human T arguments, namely  \emph{soʔmeʔma} \rede{show} and \emph{cameʔma} \rede{feed}. Etymologically, both verbs are causatives, but they show the same behavior as non-derived verbs. Usually, the verb shows object agreement with G in this frame (see \Next[a]), but when G has third person reference and T is a speech act participant (\textsc{sap}), the verb agrees with T instead of G.  The \isi{case} marking of G also changes to \isi{locative}, so that the verb now belongs to the \isi{indirective frame} (see \Next[b]).

\ex. \ag. a-ni=ŋa  ka  u-phoʈo soʔmet-a-ŋ=na.\\
		{\sc 1sg.poss-}elder.sister{\sc =erg} {\sc 1sg} {\sc 3sg.poss-}photo  show{\sc -pst-1sg.P=nmlz.sg}\\
	\rede{My elder sister showed me her photo.} (T[3]→G[\textsc{sap}])
\bg. ka nda appa-ama=be soʔmeʔ-nen=na.\\
		 {\sc 1sg[erg]}  {\sc 2sg}  mother-father{\sc =loc} show{\sc [pst]-1>2=nmlz.sg}\\
		\rede{I showed you to my parents.} (T[\textsc{sap}]→G[3])
		
This alternation is scenario-based, as it only applies in the T[\textsc{sap}]→G[3] constellation. In \Next, both T and G are are speech-act participants, and the agreement remains with the G argument. This scenario is also pragmatically marked, which is why \isi{locative} marking on G is possible (though not obligatory) here.
	
\exg.	uŋ=ŋa  ka  nniŋda(=be) soʔmet-i-g=ha.\\
		{\sc 3sg=erg}  {\sc 1sg}  {\sc 2pl(=loc)}  show{\sc [3sg.A;pst]-2pl-2=nmlz.nsg}\\
	\rede{He showed me to you (plural).} (T[\textsc{sap}]→G[\textsc{sap}])
 
In some contexts, this may yield more than one interpretation. As it is always the speech-act participant that triggers the agreement, a clause like in \Next is ambiguous. Note that the two verbs differ with respect to the acceptability of the \isi{locative} on G. The effects of the T[\textsc{sap}]→G[\textsc{sap}] scenario are summarized in \figref{t-sap-table}.
 
\begin{figure}[h]	 
\begin{center}
\begin{tabular}{r@{\hskip 2em}cl@{\hskip 2pt}l}
\lsptoprule
& G[\textsc{sap}] & \multicolumn{2}{c}{G[3]}\\
\midrule
\multirow{2}{*}{T[\textsc{sap}]}	& \multirow{2}{*}{V-o[G], G-{\sc loc/nom}}	& {\bf V-o[T], G-{\sc loc}} & \emph{soʔmeʔma} \rede{show}\\
& & {\bf V-o[T], G-{\sc nom}}  & \emph{cameʔma}  \rede{feed}\\
% \cline{1-1} \cline{3-3} 
T[3]	& \multicolumn{2}{c}{\hspace{1cm}V-o[G], G-{\sc nom}}&\\
\lspbottomrule
\end{tabular}\\
\caption{The effects of the T[\textsc{sap}]→G[3] scenario}\label{t-sap-table}
% \todo[inline]{I don't think vertical lines are necessary here. If it should resemble a process, I think → between the columns are better for readability}
\end{center}
\end{figure} 

\exg. ka nda kiba(*=be) cameʔ-meʔ-nen=na.\\
		{\sc 1sg[erg]}  {\sc 2sg}  tiger{\sc (*=loc) } feed{\sc -npst-1>2=nmlz.sg}\\
	\rede{I will feed you to the tigerǃ} (T-agr) OR\\
	\rede{I will feed the tiger to you!} (G-agr)
 


The  T[\textsc{sap}]→G[3] scenario may also restrict alternations.  The verb \emph{nakma} (stem: \emph{nakt}) \rede{ask, beg} alternates (almost) freely between the \isi{double object frame} (see \Next) and the \isi{indirective frame} (see \NNext). It is the only verb that shows this alternation. The argument encoding is conditioned by the question of which argument is central in a given discourse.

	 \ex. \ag.  ka nda chemha nak-nen=na.\\
	{\sc 1sg[erg]} 	 {\sc 2sg}  liquor ask{\sc [pst]-1>2=nmlz.sg}	\\
	\rede{I asked you for liquor.} 
	 \bg. ka i=ya=ca n-nakt-a-ŋa-n!\\ 
	{\sc 1sg}  what{\sc =nmlz.nsg=add} {\sc neg-}ask{\sc -imp-1sg.P-neg}		\\ 
	\rede{Do not ask me for anything!}    \source{27\_nrr\_06:25}
	
\ex. \ag. uŋ=ŋa ka=be unipma nakt-u=ha.\\
		{\sc 3sg=erg} {\sc 1sg=loc}  money   ask{\sc -3.P[pst]=nmlz.nc}	\\
	\rede{He asked me for his money.}   
 	\bg. uŋ=ŋa appa-ama=be ka nakt-a-ŋ=na.\\
	{\sc 3sg=erg} mother-father{\sc =loc}  {\sc 1sg}	ask{\sc -pst-1sg.P=nmlz.sg}	\\
	\rede{He asked my parents for me (i.e., to marry me).}  

However, when the T is a speech act participant and the G is not, as in \Last[b], the \isi{indirective frame} is the only option. Clauses like the one in \Next are ungrammatical. Thus, the particular scenario in which the T is a speech act participant and the G is a third person restricts the alternations in the \isi{argument realization} of this verb.

\exg. *uŋci ka n-nakt-u-n-ci-n.\\
	{\sc 3nsg} {\sc 1sg} {\sc neg-}ask{\sc -3.P[imp]-neg-nsg.P-neg}\\
	Intended: \rede{Do not ask them for me.}  


The preceding section has shown how the \isi{argument realization} in \isi{three-ar\-gu\-ment verbs} can be conditioned by referential factors. The scenario  T[\textsc{sap}]→G[3]  leads to an obligatory change in  person and \isi{case} marking for the verbs \emph{soʔmeʔma} \rede{show} and \emph{cameʔma} \rede{feed}, and to a restriction in the alternation possibilities for the verb \emph{nakma} \rede{ask, beg}. Hierarchical \isi{alignment}, partly combined with inverse marking, is also known from the verbal paradigms of other \isi{Tibeto-Burman} languages, e.g., from rGyalrong \citep{Nagano1984A-historical}, Rawang \citep{LaPolla2007Hierarchical}, and to some extent from other Kiranti languages, too, like Hayu and Dumi \citep{Michailovsky2003Hayu, Driem1993A-grammar}. In the Yakkha verbal \isi{person marking}, however, hierarchical \isi{alignment} as it is found in the \isi{three-argument verbs} shown above is not found in the monotransitive paradigms.\footnote{Several morphemes in Yakkha verbal \isi{person marking} are scenario-sensitive, see \sectref{verb-infl}. However, the \isi{alignment} of the verbal \isi{person marking} in Yakkha is too heterogenous to be captured by one principle or one hierarchy. It also includes \isi{ergative}, accusative, tripartite and neutral \isi{alignment} (cf. also \citealt{Witzlacketal2011_Decomposing} for a Kiranti-wide study).}


 