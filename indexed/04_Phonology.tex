\chapter{Phonology}\label{phon}


This chapter deals with the phoneme inventory and phonological and the morpho\-phonological rules and processes that are relevant in Yakkha. The \isi{orthography} used here is explained  in \sectref{orth}. The examples in this chapter, unlike in the other chapters, have two lines representing the Yakkha data: the upper line shows the data after the application of all phonological and morphophonological rules, and the lower line shows the underlying phonemic material with morpheme breaks. The \isi{orthography} is used in both representations, and \textsc{ipa} is only used when necessary in the explanations in prose. \sectref{phon-inv} presents the phoneme inventory of Yakkha,   \sectref{syllable} deals with the \isi{syllable structure} and  \sectref{loansphon} discusses the treatment of loanwords, as they nicely illustrate the phonological features of Yakkha. \sectref{stress} lays out the conditions by which stress is assigned. The abundant morphophonological processes and their connections to \isi{syllable structure}, stress and diachronic processes are the concern of \sectref{morphophon}. 


\section{Phoneme inventory and allophonic rules}\label{phon-inv}

\subsection{Vowel phonemes}\label{vowelphon}

Yakkha has only five basic vowels; it has two close vowels, the front /i/ and the back /u/, two close-mid vowels, the front /e/ and the back /o/, and an open vowel /a/. In contrast to other Kiranti languages,  there are no central vowels like  /ɨ/, /ʌ/ or /ə/. A chart with the vowel inventory is provided in \figref{fig-vowels}. In addition to these vowels, a  front vowel [ɛ] may occur, but only as the  contracted form of the diphthong /ai/ (see \sectref{diphth}), not in any other environments. Minimal pairs are provided in \tabref{min-pair-v}. Tone, length or nasal articulation do not constitute phonemic contrasts in Yakkha. 

\begin{center}
\begin{figure}
	\begin{tikzpicture}[scale=1]
		\node (a3) at (0, 0) {u};
		\node (a1) at (-4, 0) {i};
		\node (a2) at (-2, 0) {};

		\node (b1) at (-3.4,-1) {e};
		\node (b2) at (-1.7,-1) {};
		\node (b3) at (0,-1) {o};

		\node (c1) at (-2.8,-2) {};
		\node (c2) at (-1.4,-2) {};
		\node (c3) at (0,-2) {};

		\node (d1) at (-2.2,-3) {a};
		\node (d2) at (-1.1,-3) {};
		\node (d3) at (0,-3) {};
		
		 \draw (a1) -- (a2) -- (a3);
		\draw (b1)--(b2)--(b3);
		\draw (c1)--(c2)--(c3);
		\draw (d1)--(d2)--(d3);
		
		\draw (a1)--(b1)--(c1)--(d1);
		\draw (a2)--(b2)--(c2)--(d2);
		\draw (a3)--(b3)--(c3)--(d3);
	\end{tikzpicture}

   \caption{Yakkha vowel phonemes}\label{fig-vowels}
 \end{figure}
\end{center}


 \begin{table}[htp]	
\resizebox*{\textwidth}{!}{
\small	
\begin{tabular}{lllll}
\lsptoprule
{\sc phonemes}	& {\sc examples}\\
\midrule
{\bf /e/ vs. /i/} & \emph{nema} & \rede{lay, sow seed} &\emph{nima} & \rede{know, see}\\
			& \emph{tema} & \rede{lean on an angle} & \emph{tima} & \rede{put down, invest}\\
{\bf /e/ vs. /a/} &\emph{tema} & \rede{lean on an angle} &  \emph{tama} & \rede{come}\\
			&\emph{yepma} & \rede{stand} &  \emph{yapma} & \rede{be rough}\\
{\bf /o/ vs. /u/} & \emph{okma} & \rede{shriek} &  \emph{ukma} & \rede{bring down}\\
			& \emph{hoʔma} & \rede{prick, pierce} &   \emph{huʔma} & \rede{push, stuff}\\
{\bf /o/ vs. /a/} & \emph{thokma} & \rede{spit} &   \emph{thakma} & \rede{weigh, hand/send up}\\
			& \emph{hoʔma} & \rede{prick, pierce} &  \emph{haʔma} & \rede{scrape off/out}\\
{\bf /u/ vs. /i/} & \emph{ukma} & \rede{bring down} &   \emph{ikma} & \rede{chase}\\
			& \emph{umma} & \rede{pull} &   \emph{imma} & \rede{sleep}\\
\lspbottomrule
\end{tabular}
}
\caption{Minimal pairs for vowel phonemes}\label{min-pair-v}

\end{table}


\subsection{Diphthongs}\label{diphth}

Given that adjacent vowels are generally avoided in Yakkha, it does not come as a surprise that \isi{diphthongs}, i.e., adjacent vowels in the same \isi{syllable}, are rare. The four \isi{diphthongs} /ai/, /ui/, /oi/ and /au/ were found, occuring marginally, as in \emph{ŋhai} (a dish made from fish stomach), \emph{hoi!} \rede{enough!},  \emph{uimalaŋ} \rede{steeply downhill}, \emph{(h)au} (a sentence-final \isi{exclamative} particle) and \emph{ambau!} (an \isi{exclamative}  expression indicating that the speaker is impressed by huge or dangerous things). Some speakers pronounce underlying sequences like /ŋond-siʔ-ma/ and /thend-siʔ-ma/ with nasalized \isi{diphthongs}, [ŋoĩsiʔma] and [theĩsiʔma], respectively (instead of the more common pronunciations [ŋonsiʔma] and [thensiʔma]).\footnote{The nasalization is exceptional in these cases. Usually, the prosody of Yakkha supports the opposite process, namely the change of nasal vowels to nasal consonants, e.g., in borrowed \ili{Nepali} lexemes (see \sectref{loansphon}). Nasals may, however, regularly change to nasalization of the preceding vowel in intervocalic environment and before \isi{glides} and \isi{liquids}, as in \emph{mẽ.u.le} (/meN-us-le/) \rede{without entering}  and \emph{mẽ.yok.le} (/meN-yok-le/) \rede{without searching}, see \sectref{nas-son}.} 

Most \isi{diphthongs} have their origin in a multimorphemic or in a multisyllabic environment. The \isi{adverb} \emph{uimalaŋ}, for instance, like many other spatial adverbs in Yakkha, is composed of a stem (diachronically, most probably a noun) and the possessive prefix \emph{u-}. The marginal nature of the \isi{diphthongs} is confirmed also by the fact  that they are found more in names and discourse particles than in lexemes with semantic content, and never in verbal roots. Occasionally, \isi{diphthongs} are just one stage in a larger process of contraction. Consider the inflected form \emph{waiʔ.na} \rede{(he/she/it) exists}, which is also found as [wɛʔ.na]. Its nonpast semantics and synchronically available contracted forms of verbs suggest that [waiʔ.na]  used to be *\emph{[wa.me.na] } historically. \tabref{table-diphth} provides an exhaustive list of lexemes containing \isi{diphthongs} from the more than 2400 lexemes in the lexical database that builds the basis of the current analysis.



 \begin{table}[htp]	
 \begin{center}		
\begin{tabular}{llll}
\lsptoprule
{\bf /au/}&{\bf /oi/}&{\bf /ui/}&{\bf /ai/}\\
\midrule
\emph{(h)au}&\emph{coilikha}&\emph{uimalaŋ}&\emph{ŋhai}\\
({\sc excla})&(a village)&\rede{steeply downhill}&\rede{fish stomach}\\
\emph{ambau!}&\emph{hoiǃ}&\emph{phakkui}&\emph{Yaiten}\\
\rede{holy smoke!}&\rede{enoughǃ}&\rede{pig droppings}&(a village)\\
  & &\emph{waghui} &\emph{lai}\\
 & & \rede{chicken droppings}&({\sc excla})\\
\lspbottomrule
\end{tabular}
\caption{Lexemes containing diphthongs}\label{table-diphth}
\end{center}
\end{table}


\subsection{Consonant phonemes}\label{consphon}

\tabref{fig-consonants} below shows the central and the marginal consonant phonemes of Yakkha. The phones that are not in parentheses clearly have phonemic status; they occur in basic, uninflected stems. The phonemic status of the phones in parentheses is not always straightforward (see discussion below). Where my \isi{orthography} deviates from \textsc{ipa}, this is indicated by angle brackets.

\begin{table}[htp]
\resizebox*{\textwidth}{!}{
\small
\begin{tabular}{lcccccc}
\lsptoprule
			& {\sc bilabial} &	{\sc alveolar} &	{\sc retroflex }	&	{\sc palatal }&	{\sc velar }&		{\sc glottal}\\
\midrule
{\sc Plosives		}& p   	&	t   	&	(ʈ)  	&			&	k   	& ʔ\\
 {\sc asp.}		& ph   &	th  	&	 (ʈh) 	&			&	kh 		&\\

{\sc voiced}& (b)  	&	(d)   	&	 (ɖ) 	&			&	(g)   	&\\
 {\sc voiced-asp.	}	& (bh)   &	(dh)  	&	  (ɖh)	&			&	(gh) 	&\\

{\sc Affricates 	}&		&	ts <c>		&	  	&		 	&			&	\\
{\sc asp.} 	&		&	tsʰ <ch>		&	 	&	  	&			&\\
{\sc  voiced}	&		&	(dz)	 <j>	&	  	&	 &			&	\\
{\sc voiced-asp.} 	&		&	(dzʰ) <jh>		&	 	&	   	&			&\\

{\sc Fricatives }	&		&	 		s&		&		 	&		 	& h	\\
{\sc Nasals	}	&	m 	&	n		&		&			&		ŋ	&	\\
{\sc Nas. asp.}		&	(mh)	&	(nh)		&		&			&		(ŋh)	&	\\
{\sc Rhotics 	}&		&			r&		 &			&			&	\\
{\sc Laterals	}&		&			l&		 &			&			&  	\\
{\sc Glides 	}	& w		&			&		&j <y>			&		 	&	\\
{\sc Glides asp.} & wh	&			&		&			&			&\\
\lspbottomrule
\end{tabular}
}
\caption{Yakkha consonant phonemes}\label{fig-consonants}
\end{table}


\subsubsection{The main phonemic distinctions in the consonants}


Yakkha distinguishes six places of articulation: bilabial, alveolar, retroflex (or post-alveolar), palatal, velar and glottal. Retroflex plosives most probably made their way into Yakkha via \ili{Nepali} loanwords. They are found only in a few Yakkha lexemes, and no proper minimal pairs could be established. The retroflex series lacks a nasal, too. However, in the few words that are found with retroflex stops, they are robust, and pronouncing these words with an alveolar stop is not an option. 

Yakkha fits well into the Eastern branch of Kiranti, for instance in the loss of phonemic  contrast between voiced and unvoiced plosives. Generally, plosives, unless they are found in an environment that triggers \isi{voicing}, are pronounced as voiceless. As always, a few exceptions occur that cannot be explained by some rule. The exact parameters of the \isi{voicing} rule are laid out in  \sectref{voicing}. A robust phonemic contrast exists between aspirated and unaspirated consonants, as it is found in the plosives (except for the glottal stop), the affricate and the bilabial glide /w/. Aspiration of a stem-initial consonant, historically a morphological means to increase the \isi{transitivity} in \isi{Tibeto-Burman} \citep{Michailovsky1994Manner, Jacques2012_Internal, Hill2014_Note}, has become purely phonemic in Yakkha.  The aspirated plosives have a strong fricative component. Three \isi{nasals} are distinguished by their place of articulation: bilabial  /m/, alveolar /n/ and velar /ŋ/. Yakkha has two fricatives /s/ and /h/, and two \isi{liquids}, /l/ and /r/. The rhotic does not occur word-initially. In this position, */r/ has changed to the palatal glide /y/\footnote{I use the grapheme <y> to represent IPA [j]; see \sectref{orth} for the notes on the orthography used in this book.} (see also  \tabref{soundchange} in Chapter \ref{languageintro} and the references therein).\footnote{Furthermore, /y/ may be omitted before /e/ in some lexemes, but this process is subject to considerable individual variation.}  The distribution of the rhotic consonant deserves a closer look, also in the perspective of other Eastern Kiranti languages (see \sectref{rhotic} below). \tabref{min-pair-c} provides minimal pairs for the basic consonant phonemes, mostly from verbal stems or citation forms.


 \begin{table}[htp]
\resizebox*{\textwidth}{!}{
\small
\begin{tabular}{lllll}
\lsptoprule
{\sc phonemes}	& {\sc examples}\\
\midrule
{\bf /k/} vs. {\bf /kh/} & \emph{keʔma} & \rede{come up} & \emph{kheʔma} & \rede{go}\\
			& \emph{kapma} & \rede{carry along, have} & \emph{khapma} & \rede{thatch, cover}\\
{\bf /p/} vs. {\bf /ph/} & \emph{pakna} & \rede{young guy} & \emph{phak} & \rede{pig}\\
		 & \emph{pekma} & \rede{fold} & \emph{phekma} & \rede{slap, sweep}\\
{\bf /t/} vs. {\bf /th/} & \emph{tumma} & \rede{understand} & \emph{thumma} & \rede{tie}\\
 		 & \emph{tokma} & \rede{get} & \emph{thokma} & \rede{hit with horns}\\	
{\bf /c/} vs. {\bf /ch/} & \emph{cikma} & \rede{age, ripen} & \emph{chikma} & \rede{measure, pluck}\\
 		 & \emph{cimma} & \rede{teach} & \emph{chimma} & \rede{ask}\\
{\bf /k/} vs. {\bf /ʔ/} & \emph{okma} & \rede{shriek} & \emph{oʔma} & \rede{be visible}\\
 {\bf /t/} vs. {\bf /ʔ/ }& \emph{-met} & ({\sc caus}) & \emph{-meʔ} & ({\sc npst})\\
 {\bf /p/} vs. {\bf /ʔ/} & \emph{opma} & \rede{consume slowly} & \emph{oʔma} & \rede{be visible}\\
{\bf /t/} vs. {\bf /r/ }& \emph{ot} & \rede{be visible} (stem) & \emph{or} & \rede{peel off}\\
 {\bf /l/} vs. {\bf /r/} & \emph{khelek} & \rede{ant} & \emph{kherek} & \rede{hither}\\
 {\bf /y/} vs. {\bf /w/} & \emph{yapma} & \rede{be uncomfortable} & \emph{wapma} & \rede{paw, scrabble}\\
 		 & \emph{yamma} & \rede{disturb} & \emph{wamma} & \rede{attack, pounce}\\
{\bf /y/} vs. {\bf /l/} & \emph{yapma} & \rede{be uncomfortable} & \emph{lapma} & \rede{accuse, blame}\\
 {\bf /w/} vs. {\bf /wh/ }& \emph{wapma} & \rede{paw, scrabble} & \emph{whapma} & \rede{wash clothes}\\
 		 & \emph{waŋma} & \rede{curve, bend} & \emph{whaŋma} & \rede{boil}\\
{\bf /s/} vs. {\bf /h/}& \emph{sima} & \rede{die} & \emph{hima} & \rede{spread}\\
		& \emph{somma} & \rede{stroke gently} & \emph{homma} & \rede{fit into}\\
{\bf /k/} vs. {\bf /ŋ/}& \emph{pekma} & \rede{break} & \emph{peŋma} & \rede{peel}\\
 		 & \emph{okma} & \rede{shriek} & \emph{oŋma} & \rede{attack}\\
{\bf /ŋ/} vs. {\bf /m/} & \emph{toŋma} & \rede{agree} & \emph{tomma} & \rede{place vertically}\\
			 & \emph{tuŋma} & \rede{pour} & \emph{tumma} & \rede{understand}\\
{\bf  /ŋ/} vs. {\bf /n/} & \emph{=ŋa} & ({\sc erg}) & \emph{=na} & ({\sc nmlz.sg})\\
{\bf /m/} vs. {\bf /n/} & \emph{makma} & \rede{burn} & \emph{nakma} & \rede{beg, ask}\\
& \emph{miʔma} & \rede{think, remember} & \emph{niʔma} & \rede{count, consider}\\
\lspbottomrule
\end{tabular}
}
\caption{Minimal pairs for consonants}\label{min-pair-c}
\end{table}



\subsubsection{Marginal consonant phonemes}

Several of the phonemes occur only marginally,  either in \ili{Nepali} loanwords, or in just a handful of Yakkha lexemes. This basically applies to the already mentioned retroflex plosives and to all voiced obstruents, as \isi{voicing} is generally not distinctive in Yakkha.\footnote{There are quasi minimal pairs such as \emph{apaŋ} \rede{my house} and \emph{abaŋ} \rede{I came}, but both are inflected words and the difference is that \emph{a-} in \emph{apaŋ} is a prefix, and the rule that is responsible for the \isi{voicing} of plosives excludes prefixes.} Some sounds are never found in uninflected lexemes, so that they only emerge as the result of some morphophonological processes that are triggered by the concatenation of morphemes with certain phonological features. Voiced-aspirated consonants and the aspirated \isi{nasals} [mʰ], [nʰ] and [ŋʰ] belong to this group. The marginal sounds are included in parentheses in \tabref{fig-consonants}.  The reader is referred to \sectref{morphophon} for the details of the various morphophonological processes that lead to marginal phonemes. 



\subsubsection{The phonemic status of the glottal stop}

The glottal stop is contrastive, as several minimal pairs in \tabref{min-pair-c}  demonstrate. The glottal stop surfaces only before \isi{nasals} and laterals, so that one can find minimal pairs like  \emph{meŋ.khuʔ.le} \rede{without carrying} and \mbox{\emph{meŋ.khu.le}} \rede{without stealing}, or \emph{men.daʔ.le} \rede{without bringing} and \emph{men.da.le} \rede{without coming}. However, the glottal stop can also be the result of a phonological operation. Unaspirated stops, especially /t/, tend to get neutralized to [ʔ] syllable-finally (aspirated stops do not occur in this position). The glottal stop is also prothesized to vowel-initial words to maximize the onset. In certain grammatical markers, the glottal stop may also  be epenthesized at the end of the \isi{syllable} when it is followed by  nasal consonants or \isi{glides}  (see \Next). This may happen only when the \isi{syllable} is stressed, but the conditions for this epenthesis are not fully understood yet. It never occurs at the end of a word (if the word is defined by the domain to which stress is assigned). 

\ex.\a.\glll tu.mok.peʔ.na ma.mu\\
/tumok=pe=na mamu/\\
Tumok{\sc =loc=nmlz.sg} girl\\
\rede{the girl from Tumok}
\b.\glll men baʔ.loǃ\\
/men pa=lo/\\
{\sc cop.neg} {\sc emph=excla}\\
\rede{Of course notǃ}


 
The glottal stop is less consonant-like than the other plosives. In certain environments, stems that end in a glottal stop may behave identically to  stems consisting of open syllables (CV). For instance, if the stem vowel /e/ or /i/ (of a CV stem or a CVʔ stem) is followed by a vocalic suffix like \emph{-a} (marking past or \isi{imperative}), it changes into a glide /y/ and becomes part of the onset. This process is illustrated by the behavior of \emph{kheʔma} \rede{go} and \emph{piʔma} \rede{give}, cf. \tabref{glottal}. If the stem vowel (of a CV stem or a CVʔ stem) is a back vowel, a glide /y/ is inserted between stem and suffixes. If open or /ʔ/-final stems are followed by the suffix sequence \emph{-a-u}, this sequence of suffixes is not overtly realized. Examples of these processes are provided in \tabref{glottal}, contrasted with the behavior of stems with open syllables and stems that end in /p/, /t/ or /k/. The first column shows the underlying stem, the second column provides the citation form and the gloss, the third column shows the behavior before /l/, by means of the forms of the \isi{negative converb}. The fourth and the fifth columns show the behavior before vowels, by means of intransitive {\sc 3.sg} past forms (in \emph{-a}),\footnote{Or detransitivized, depending on the original \isi{valency} of the stem.} and transitive {\sc 3sg.A>3sg.P} past forms (in \emph{-a-u}).\footnote{The verb \emph{cama} \rede{eat} is the only transitive verb that has an open stem in /a/. It is exceptional in having an ablaut. Open stems are rare, and not all of them are found among both transitive and intransitive verbs, so that some fields of the table cannot be filled.}  

To wrap up, the intervocalic environment distinguishes /ʔ/-final stems from stems that end in /p/, /t/ or /k/, while the \isi{infinitive} and the environment before /l/ distinguishes /ʔ/-final stems from open stems. 

The glottal stop at the end of verbal stems can be reconstructed to */t/, in \isi{comparison} with other Eastern Kiranti languages (cf. \sectref{stem} on the structure of the verbal stems). 

\begin{table}[htp]
\resizebox*{\textwidth}{!}{
\small
\begin{tabular}{lllll}%das gleiche mit longtable für mehrseitige tabellen
\lsptoprule	
\multirow{3}{*}{\sc stem} &\multirow{3}{*}{\sc citation form} & /\_-l& /\_-a& /\_-a-u\\
 &&({\sc neg.cvb})&({\sc 3sg.pst})&({\sc 3sg>3sg.pst})\\
\midrule
\multicolumn{5}{l}{{\bf /ʔ/-final stems}}\\
\midrule
/khuʔ/& \emph{khuʔma} \rede{carry}&\emph{meŋ.khuʔ.le}& \emph{khu.ya.na}&\emph{khu.na}\\
/waʔ/&\emph{waʔma} \rede{wear, put on}&\emph{mẽ.waʔ.le}&\emph{wa.ya.na}&\emph{wa.na}\\
/soʔ/&\emph{soʔma} \rede{look}&\emph{men.soʔ.le}&\emph{so.ya.na}&\emph{so.na}\\
/kheʔ/&\emph{kheʔma} \rede{go}&\emph{meŋ.kheʔ.le}&\emph{khya.na}&-\\
/piʔ/&\emph{piʔma} \rede{give}&\emph{mem.biʔ.le}&\emph{pya.na}&\emph{pi.na}\\
\midrule
\multicolumn{5}{l}{{\bf  V-final stems}}\\
\midrule
/ca/&\emph{cama} \rede{eat}&\emph{men.ja.le}&\emph{ca.ya.na}&\emph{co.na}\\
/a/&\emph{ama} \rede{descend}&\emph{mẽ.a.le}&\emph{a.ya.na}&-\\
/u/&\emph{uma} \rede{enter}&\emph{mẽ.u.le}&\emph{u.ya.na}&-\\
/si/&\emph{sima} \rede{die}&\emph{men.si.le}&\emph{sya.na}&-\\
\midrule
\multicolumn{5}{l}{{\bf /p/-, /t/-, /k/-final stems}}\\
\midrule
/lap/&\emph{lapma} \rede{seize, catch}&\emph{mẽ.lap.le}&\emph{la.ba.na}&\emph{la.bu.na}\\
/yok/&\emph{yokma} \rede{search}&\emph{mẽ.yok.le}&\emph{yo.ga.na}&\emph{yo.gu.na}\\
/phat/&\emph{phaʔma} \rede{help}&\emph{mem.phat.le}&\emph{pha.ta.na}&\emph{pha.tu.na}\\
&&\ti \emph{mem.phaʔ.le}&&\\
\lspbottomrule	
\end{tabular}
}
\caption{The glottal stop stem-finally, compared to vowels and other plosives}\label{glottal}
\end{table}
 


\subsubsection{The status of /r/ in Yakkha and in an Eastern Kiranti perspective}\label{rhotic}

The rhotic /r/ does not occur word-initially in genuine Yakkha lexemes, due to the typical Eastern Kiranti sound change from */r/ to /y/ in word-initial position (see \sectref{genetic} and \citealt{Bickeletal_Firstperson}). There are words like \emph{lok} \rede{anger} and \emph{yok} \rede{place}, but no words starting with /r/.\footnote{There are a few exceptions, such as the binomial (a bipartite noun) \emph{raji-raŋma} which means \rede{wealth of land}. It might be  a word that preserved an archaic phonological structure, or a loan (\emph{rājya} means \rede{kingdom} in \ili{Nepali}). Both options are possible and attested for the ritual register (the \emph{Munthum}) of other Kiranti languages \citep{Gaenszle2011_Binomials}.} It can, however, occasionally be found in complex onsets, and syllable-initially in intervocalic environment. \tabref{r-l} shows that /r/ and /l/ can be found in very similar environments, even though proper minimal pairs are rare. In some instances, intervocalic /r/ can  be traced back to historical */t/, as in the complex predicates in \Next. 


\ex.\ag. pe.sa.ra.ya.na\\ %\glll 
%/pes-a-*ta-a=naʔ/\\
fly{\sc [3sg]-pst-V2.come-pst=nmlz.sg}\\
\rede{It came flying to me.}
\bg. phuŋ chik.tu.ra=na\\ %\glll
%/phuŋ chikt-u-*taʔ-a-u=na/\\
flower pluck{\sc -3.P-V2.bring-pst-3.P=nmlz.sg}\\
\rede{She plucked and brought a flower.}
 

\begin{table}
\begin{tabular}{ll}
\lsptoprule
{\bf /r/}&{\bf /l/}\\
\midrule
\emph{khorek} \rede{bowl}  &\emph{ulippa} \rede{old}\\
 \emph{phiʔwaru} a kind of bird&\emph{chalumma} \rede{second-born daughter}\\
  (Nep.: \emph{koʈerā})&\\
 \emph{tarokma} \rede{start}&\emph{caloŋ} \rede{maize}\\
 \emph{kherek} \rede{this side, hither} & \emph{khelek} \rede{ant}\\
\emph{caram} \rede{yard}& \emph{sala} \rede{talk}\\
\emph{khiriri} \rede{spinning round very fast} & \emph{philili} \rede{jittering}\\
 \emph{phimphruwa} \rede{soap berry}& \emph{aphlum} \rede{hearth stones}\\
  (Nep.: \emph{riʈʈhā})&\\
 \emph{hobrek} \rede{rotten}& \emph{phoplek} \rede{[pouring] at once}\\
 \emph{ʈoprak} \rede{leaf plate}& \emph{khesapla} \rede{a kind of fig tree}\\
\lspbottomrule
\end{tabular}
\caption{The phonemes /r/ and /l/ in similar environments}\label{r-l}
\end{table}

According to  \cite{Driem1990The-fall}, [l] and [r] have a complementary distribution in Eastern Kiranti: [l] occurs word-initially and syllable-initially after stops, and [r] occurs between vowels and as the second component of complex onsets. The complementary distribution of [l] and [r] is a consequence of the general Eastern Kiranti sound change from */r/ to /y/ in word-initial position, which left /r/ only in word-internal position.\footnote{The sound change is evident from correspondences such as Yakkha and \ili{Limbu} \emph{yum} \rede{salt} and its non-Eastern cognates, e.g., \emph{rum} in \ili{Puma} (Central Kiranti, \citealt[393]{Bickeletal2009Puma}) or \emph{rɨm} in Dumi (Western Kiranti, \citealt[412]{Driem1993A-grammar}).} It is plausible that [l] and [r], now partly in complementary distribution, were reanalyzed as an allophones as a consequence of this sound change.  Van Driem's claim, however, could only partly be confirmed for Yakkha. In contrast to (Phedappe) \ili{Limbu} (\citealt{Driem1987A-grammar}, \citealt[688]{Schieringetal2010The-prosodic}) and other languages from the Greater Eastern branch of Kiranti such as Lohorung \citep[85]{Driem1990The-fall}, the rhotic is not found as an allophone of /l/ in intervocalic environment in Yakkha (compare the term for \rede{second-born daughter}, \emph{chalumma} (Yakkha) and \emph{sarumma} (\ili{Limbu}), \ili{Limbu} data from \citet[131]{Driem1985_LimbuKin}). Allophonic variation between /l/ and /r/ was not found for any environment in Yakkha. For instance, the \isi{negative converb} \emph{me(n)...le} does not have an allomorph [me(n)...re] after CV-stems in Yakkha, in contrast to the same \isi{converb} in \ili{Limbu}. Furthermore, the question whether C + /r/ are syllabified as .Cr and C + /l/ as C.l could not be answered satisfactorily for Yakkha, based on auditory and phonological evidence. For instance, /r/ as well as /l/ may trigger \isi{voicing} in a preceding consonant, without any rule becoming apparent from the current data set (see \tabref{r-l}). To sum up, there is more than sufficient evidence for the phonemic status of /r/ in Yakkha.\footnote{The postulation of a phoneme /r/ has implications for a possible \isi{orthography} for future Yakkha materials. One of the current local orthographies, used e.g., in \citet{Kongren2007Yakkha} and in several school books \citep{Jimi2009Engka-Yakkha}, conflated /r/ and /l/ under the grapheme <{\Deva ल}>, the \isi{Devanagari} letter for <l>. This turned out to be very impractical for the readers. It is not only too much abstracted away from the actual pronunciation, but also not justified by the phonological facts. It is my recommendation to change this in future publications, i.e., to write <{\Deva र}> (r) when a sound is pronounced as a rhotic and <{\Deva ल}> (l) when a sound is pronounced as a lateral.}



It is possibly a rather new development that the rhotic may also appear in syllable-final position. As shown in  \Next, it may occur at the end of verbal stems that historically used to have a stem-final /t/-augment (cf. \sectref{stem}). This syllabification is only licensed when the following \isi{syllable} starts with /w/. When the stem is followed by vowel sounds, /r/ will be syllabified as onset. Another process leading to syllable-final \isi{rhotics} is metathesis, which is found in free allophonic variation, as in \emph{tepruki \ti tepurki} \rede{flea} or \emph{makhruna \ti makhurna} \rede{black}. 


\ex.\ag.thur-wa-ŋ=na.\\
sew{\sc -npst[3.P]-1sg.A=nmlz.sg}\\
\rede{I will sew it.}
\bg.nir-wa-ŋ-ci-ŋ=ha.\\
count{\sc -npst-1sg.A-3nsg.P-1sg.A=nmlz.sg}\\
\rede{I will count them.}


 
\subsubsection{Aspirated voiced consonants}\label{asp-voiced}

Aspirated voiced plosives can result from the \isi{voicing} rule (cf. \sectref{morphophon}), or from sequences of morphemes with consonants followed by /h/, as in \Next[a]. In this way, aspirated consonants that are not found in simple lexemes can be created; they always involve a morpheme boundary, at least diachronically.\footnote{An exception is the word \emph{ŋhai} \rede{fish stomach (dish)}, for which no transparent multimorphemic etymology is available.} Another process leading to aspirated voiced consonants is vowel elision. If there is an underlying multimorphemic sequence of the shape /C-V-h-V/, the first vowel gets elided and /h/ surfaces as aspiration of the first consonant (see \Next[b]). 


\ex.\a. \glll khe.i.ŋha\\
/kheʔ-i-ŋ=ha/\\
go{\sc [pst]-1pl-excl=nmlz.nsg}\\
\rede{We went.}
\b. \glll ca.mha.ci\\
/ca-ma=ha=ci/\\
eat{\sc -inf[deont]=nmlz.nsg=nsg}\\
\rede{They have to be eaten.}


The environment that is required for the vowel elision is also provided by other forms of the verbal inflectional paradigm. In \Next, the underlying sequence /-ka=ha/ ([-gaha] due to intervocalic \isi{voicing}) licenses the elision of the preceding vowel, which results in the realization of /h/ as aspiration of [g].

\ex.\a. \glll tun.di.wa.gha\\
/tund-i-wa-ka=ha/\\
understand{\sc [3A]-2.P-npst-2=nmlz.nsg}\\
\rede{He/she/they understand(s) you.}
\b. \glll tum.me.cu.ci.gha\\
/tund-meʔ-ci-u-ci-ka=ha/\\
understand{\sc -npst-du.A-3.P-3nsg.P-2=nmlz.nsg}\\
\rede{You (dual) understand them.}



\section{Syllable structure}\label{syllable}

This section describes the parameters for the possible \isi{syllable}s in Yakkha. The structure of the \isi{syllable} is maximally 	CCVC, i.e., VC, CV,  CCV and CVC are possible as well. If a word-initial \isi{syllable} starts with a vowel, a glottal stop is prothesized to yield a minimal onset. Syllables with CVV structure occur only in the form of \isi{diphthongs} (see \sectref{diphth} above). They are exceedingly rare, and they can generally be traced back to bisyllabic or bimorphemic contexts. Syllables containing \isi{diphthongs} are always open. 

In a simple onset, any consonant can occur, with the exception of  /r/, which got replaced by /y/ diachronically in Eastern Kiranti. Among the complex onsets, two sets have to be distinguished. The first set has the general shape CL, where L stands for  \isi{liquids} and \isi{glides}. In this type of \isi{syllable}, the first consonant can be a plosive, a fricative, an affricate or a nasal, while the second consonant can only be a liquid (/l/ or /r/) or a glide (/y/ or /w/). The onsets containing /y/ or /w/ result from contracted CVCV sequences diachronically. Some alternations between a monosyllabic and a bisyllabic structure, like \emph{cwa \ti cu.wa} \rede{beer}, \emph{chwa \ti chu.wa} \rede{sugarcane}, \emph{nwak \ti nu.wak} \rede{bird} and \emph{yaŋcuklik \ti yaŋcugulik} \rede{ant} suggest this. Comparison with related languages like Belhare and \ili{Chintang} provides further evidence for a former bisyllabic structure: \ili{Chintang} and Belhare have \emph{cuwa} and \emph{cua}, respectively, for \rede{water}, and Belhare furthermore has \emph{nua} for \rede{bird} \citep{Bickel1997Dictionary, Raietal2011_Chintangdict}. For Athpare, both bisyllabic and monosyllabic forms are attested \citep{Ebert1997A-grammar}.

On the other hand, complex onsets are not uncommon in \isi{Tibeto-Burman}. Word-initially, the status of CL sequences as complex onsets is robust, but word-inter\-nally, alternative syllabifications would be theoretically possible. This possibility can be ruled out at least for the clusters involving aspirated plosives, because aspirated plosives may never occur syllable-finally. A  segmentation like [kith.rik.pa] or [aph.lum] would violate the restriction on a well-formed \isi{syllable} coda in Yakkha, so that it has to be [ki.thrik.pa] and [a.phlum] (\rede{policeman} and \rede{hearth}, respectively). For unaspirated plosives, it is hard to tell how they are syllabified. Not all logically  possible onsets occur, and some are only possible in morphologically complex (both inflected and derived) words. Some examples of complex onsets are provided in \tabref{onsets-liq} and \tabref{onsets-gli}. Onset types not shown in the tables do not occur.



\begin{table}		
\begin{tabular}{lll}
\lsptoprule
&{\bf }/l/ &{\bf /r/}\\
\midrule
{\bf /p/}&\emph{i.plik} \rede{twisted}&\emph{ca.pra} \rede{spade}\\
{\bf /ph/} &\emph{a.phlum }  \rede{trad. hearth}&\emph{phim.phru.wa} \rede{soap berry}\\
{\bf /k/}&\emph{saklum}\rede{frustration}&\emph{thaŋ.kra}   \rede{store for grains}\\
{\bf /kh/}&(-)&\emph{ʈu.khruk}  \rede{head}\\
{\bf /s/}&(-)&\emph{mik.srumba} \rede{blind person}\\
{\bf /n/}&\emph{nlu.ya.ha} \rede{they said}&(-)\\
\lspbottomrule
\end{tabular}
\caption{Complex onsets with liquids}\label{onsets-liq}
\end{table}

\begin{table}	
\begin{tabular}{lll}
\lsptoprule
&{\bf /w/}&{\bf /y/}\\
\midrule
{\bf /p/}&(-)&\emph{pyaŋ.na} \rede{He/she gave it to me.}\\
{\bf /ph/}&\emph{tam.phwak} \rede{hair}&\emph{ci.sa.bhya} \rede{It cooled down.} \\
{\bf /t/}&\emph{twa}  \rede{forehead}&(-)\\
{\bf /ʈh/}&\emph{ʈhwaŋ} \rede{smelly} (\textsc{ideoph})&(-)\\
{\bf /c/}&\emph{cwa} \rede{heart}&\emph{cya} \rede{child}\\
{\bf /ch/}&\emph{chwa} \rede{sugarcane}&\emph{op.chyaŋ.me} \rede{firefly}\\
{\bf /k/}&(-)&\emph{kya}   \rede{Come up!}\\ 
{\bf /kh/}&\emph{o.sen.khwak}  \rede{bone}&\emph{khya} \rede{Go!}\\
{\bf /s/}&\emph{swak} \rede{secretly}&\emph{sya.na} \rede{He/she died.}\\
{\bf /n/}&\emph{nwak} \rede{bird}&\emph{(ayupma) nyu.sa.ha} \rede{I am tired.}\\
\lspbottomrule
\end{tabular}
\caption{Complex onsets with glides}\label{onsets-gli}
\end{table}


The second set of onsets has the shape NC, where N stands for an unspecified nasal and C for any stem-initial consonant. This type of onset is  found only when one of the nasal prefixes is attached to a stem, never in monomorphemic syllables, and never in syllables inside a word. The value of the nasal is conditioned by the place of articulation of the following consonant. Based on auditory evidence, I conclude that the nasal is not syllabified. However, as the processes related to  prosody or to morphophonology either exclude prefixes from their domain or they apply across \isi{syllable} boundaries as well, I could not find independent evidence for this claim. The nasal prefixes may have the following morphological content: {\sc 3pl.S/A} and \isi{negation} in verbs (see \Next[a] and \Next[b]), a second person possessive in nouns (see \Next[c]), and a distal relation in spatial adverbs and \isi{demonstratives} (see \Next[d] and \Next[e]). 

\ex.\a.\glll	mbya.gha\\
			/N-piʔ-a-ka=ha/\\
			{\sc 3pl.A-}give{\sc -pst-2.P=nmlz.nsg}\\
			\rede{They gave it to you.} %check
			\b.\glll ŋkhyan.na\\
			/N-kheʔ-a-n=na/\\
			{\sc neg-}go{\sc [3sg]-pst-neg=nmlz.sg}\\
			\rede{He did not go.}	
			\b.\glll	mbaŋ\\
			/N-paŋ/\\
			{\sc 2sg.\textsc{poss}-}house\\
			\rede{your house}
			\b.\glll ŋkhaʔ.la\\
			/N-khaʔ.la/\\
			{\sc dist-}like\_this\\
			\rede{like that}
			\b.\glll  nnhe\\
			/N-nhe/\\
			{\sc dist-}here\\
			\rede{there}



The coda is restricted to \isi{nasals}, unaspirated plosives  and, rarely, /r/ (cf. \sectref{rhotic} above). The plosives are often unreleased or neutralized to [ʔ] in the coda, unless they are at the end of a word. While the glottal stop frequently occurs in \isi{syllable} codas, it  is never found at the end of a phonological word (as defined by the stress domain).  


\figref{syll} summarizes the possible \isi{syllable} in Yakkha. If the form of a morpheme does not agree with the \isi{syllable structure}, several strategies may apply. If, for instance, a verbal stem ends in two consonants (C-s, C-t), as \emph{chimd} \rede{ask} or \emph{yuks} \rede{put}, and a vowel follows the stem in an inflected form, the stem-final consonant becomes the onset of the next \isi{syllable} (see \Next). If a consonant follows the stem, the final consonant of the stem is deleted (see \NNext). 

%\begin{figure}[htp] 
%\begin{tabular}{llcc}
%\lsptoprule
%\multicolumn{2}{l}{{\sc onset}}&{\sc nucleus}&{\sc coda}\\
%\midrule
%\multicolumn{2}{l}{any consonant (except /r/)}&any& unasp. plosive,\\
% \cline{1-2}
%obstruent&+ liquid, glide&vowel&nasal,\\
% \cline{1-2}
%nasal&+ any consonant (except /r/)& & /r/\\
% \hline
%\multicolumn{2}{l}{any consonant (except /r/)}&\multicolumn{2}{c}{diphthong}\\
%\lspbottomrule
%\end{tabular} 
%\caption{The syllable}\label{syll} 
%\end{figure}


 \begin{figure}[htp]	
 \begin{center}		
\begin{tabular}{ll|c|l}
\lsptoprule
\multicolumn{2}{l|}{{\sc onset}}&\multicolumn{1}{l|}{\sc nucleus}&{\sc coda}\\
\hline
\multicolumn{2}{l|}{any consonant (except /r/)}& \multirow{3}{*}{any vowel} & unasp. plosive,\\
\cline{1-2}
obstruent&+ liquid, glide & & nasal,\\
\cline{1-2}
nasal&+ any consonant (except /r/)& & /r/\\
\hline
\multicolumn{2}{l|}{any consonant (except /r/)}&\multicolumn{2}{l}{diphthong}\\
\lspbottomrule
\end{tabular}
\caption{The syllable}\label{syll}
\end{center}
\end{figure}


\ex.\a. \glll chim.duŋ.na\\
/chimd-u-ŋ=na/\\
ask{\sc -3.P[pst]-1sg.A=nmlz.sg}\\
\rede{I asked him.}
\b. \glll chim.daŋ\\
/chimd-a-ŋ/\\
ask{\sc -imp-1sg.P}\\
\rede{Ask meǃ}

\ex.\a. \glll chim.nen.na\\
/chimd-nen=na/\\
ask{\sc -1>2[pst]=nmlz.sg}\\
\rede{I asked you.}
\b. \glll men.chim.le\\
/men-chimd-le/\\
{\sc neg-}ask{\sc -cvb}\\
\rede{without asking}


In certain morphological environments and in fast speech, more complex onsets are possible, with the form NCL (nasal-consonant-liquid/glide), but this is restricted to particular inflected verb forms, namely third person plural or negated nonpast forms of verbs with open stems (or with CVʔ stems) (see \Next). Each part of the onset belongs to another morpheme. The complex cluster is a consequence of the deletion of the stem vowel. This process is further restricted to stems with back vowels (/a/, /u/ and /o/). 



\ex.\a. \glll nlwa.na\\
/N-luʔ-wa=na/\\
{\sc 3pl.A-}tell{\sc -npst[3.P]=nmlz.sg}\\
\rede{They will tell him.}
\b. \glll njwa.ŋan.na\\
/N-ca-wa-ŋa-n=na/\\
{\sc neg-}eat{\sc -npst-1sg.A[3.P]-neg=nmlz.sg}\\
\rede{I will not eat  it.}



\section{The phonological treatment of Nepali and English loans}\label{loansphon}

The phonological features of Yakkha are also reflected by the treatment of \ili{Nepali} and English loans, as shown in Tables \ref{loans-nep} and \ref{loans-eng}. Several processes may apply to adjust non-native lexemes to Yakkha phonology. Apart from the regular processes discussed below, many changes in the vowel qualities can be encountered, but they cannot be ascribed to any regular sound change.

As adjacent vowels are a marked structure in Yakkha, sequences of vowels, as well as  vowels which are separated only by /h/, are typically changed to one vowel. The intervocalic /h/ is, however, not completely lost, but preserved as aspiration of the preceding consonant, as shown by the last three examples of \tabref{loans-nep}. This process  happens irrespective of how the words are stressed in \ili{Nepali}.

Another typical process is the change of nasal vowels to nasal consonants:\footnote{Marginally, nasal vowels may occur in Yakkha, but the environments are highly restricted, and a nasal realization of a vowel is always motivated by an underlying nasal consonant (cf. \sectref{morphophon}).}  \isi{hortative} verb forms like  \emph{jum} \rede{Let's go!} or \emph{herum} \rede{Let's have a look!} seem to have been built in analogy to the shape of Yakkha \isi{hortative} verb forms, which also end in \emph{-um}, at least in the transitive verbs. The words  \emph{ʈhoŋ}, \emph{alenci} and \emph{gumthali} illustrate the same process (and also the change of \isi{diphthongs} to simple vowels). 
 
Some loans show the neutralization of voiced and voiceless consonants that is typical for Eastern Kiranti, e.g., \emph{tukkhi} (from \ili{Nepali} \emph{dukha} \rede{sorrow, pain}). Probably, such  words entered the Yakkha language in an earlier stage of the Nepali-Yakkha contact, when people were not yet bilingual. Nowadays there are many \ili{Nepali} loans in Yakkha that are pronounced as in \ili{Nepali}. 
 
 The word \emph{duru} (from \ili{Nepali} \emph{dudh} \rede{milk}) shows a strategy to satisfy the constraint against aspirated plosives at the end of the \isi{syllable} or word (and against aspirated voiced plosives in general).\footnote{The use of goat and cow's milk and other milk products is very rare in Yakkha culture (noted also by \citealt[128--30]{Russell1992_Yakha}), and, thus, the borrowing of this word is not surprising.} 
 
 Another typical process  encountered was closing word-final open syllables by /k/. For example, \emph{belā}  \rede{time} becomes [belak], \emph{bihāna}  \rede{morning} becomes [bhenik] and \emph{duno \ti duna} \rede{leaf bowl} becomes [donak] in Yakkha. Words that end with consonants other than /k/ may also be modified to end in /k/, e.g., \emph{churuk} \rede{cigarette}, from \ili{Nepali} \emph{churoʈ}.

Some English loanwords, shown in \tabref{loans-eng}, illustrate that complex codas and voiced codas are not acceptable in Yakkha. Word-initial clusters of fricative and plosive are also marked, and a vowel is prothesized to yield a \isi{syllable} that corresponds at least to some of  the prosodic constraints of Yakkha (but this also happens in the pronunciation of \ili{Nepali} native speakers). Finally, as Yakkha has no distinctions of length or tenseness for vowels, the difference between e.g., English \emph{sheep} and \emph{ship} is usually not noticed or produced if such words are borrowed. Both words are pronouned with a short [i], that is however slightly more tense than in English \emph{ship}.\footnote{The words displayed in the tables occurred regularly in at least some speakers' idiolects. Nevertheless, I do not want to make any strong claims about what is borrowed and what results from  code-switching, as this is not the purpose of my study.} 

The words selected here illustrate how some of the principles of the Yakkha sound system and the phonological rules are applied to non-native material. The Yakkha phonology in borrowed lexemes is not equally prominent among speakers. It depends on many factors, most obviously the proficiency in the donor languages and the time-depth of the borrowing.



 \begin{table}[htp]	
 \begin{center}		
\begin{tabular}{lll}
\lsptoprule
{\sc yakkha} 	&{\sc nepali}  &{\sc gloss}\\
\midrule
\emph{jum}  & \emph{ˈ​jā.aũ} & \rede{Let us go.}\\
\emph{herum​}  & \emph{ˈhe.raũ} & \rede{Let us have a look.}\\
\emph{ʈhoŋ} & \emph{ʈhāũ} & \rede{place}\\
\emph{gumthali} & \emph{gaũthali } & \rede{swallow}\\
\emph{alenci} & \emph{alaĩci} & \rede{cardamom}\\
 \emph{tuk.khi} & \emph{dukha} & \rede{sorrow, pain}\\
\emph{du.ru} & \emph{dudh} & \rede{(animals') milk}\\
\emph{chen}  & \emph{ca.ˈhĩ​} & (topic particle)\\
\emph{bhenik} & \emph{bi.ˈhā.na} & \rede{morning}\\
\emph{bhya} & \emph{ˈbi.hā} & \rede{wedding}\\
\lspbottomrule
\end{tabular}
\caption{\ili{Nepali} loans in Yakkha}\label{loans-nep}
\end{center}
\end{table}


\begin{table}[htp]	
 \begin{center}		
\begin{tabular}{ll}
\lsptoprule
{\sc yakkha} 	&{\sc gloss}\\
\midrule
\emph{ˈroʈ} &  \rede{road}\\ 
\emph{ˈphlim} &  \rede{film}\\
\emph{ˈphren} &  \rede{friend}\\ 
\emph{is.ˈʈep} &  \rede{step}\\ 
\emph{is.ˈkul} &  \rede{school}\\ 
\lspbottomrule
\end{tabular}
\caption{English loans in Yakkha}\label{loans-eng}
\end{center}
\end{table}

 

\section{Stress assignment}\label{stress}


This section deals with the rules for \isi{stress assignment} and the domain to which these rules apply.
The rules for \isi{stress assignment} can be laid out as follows: by default, the first \isi{syllable} carries main stress. Closed syllables, however, attract stress. If there are closed syllables, the main stress moves to the last closed \isi{syllable}, as long as it is not the final \isi{syllable} of a word, demonstrated by the examples in \tabref{stresstab1} for nouns,\footnote{Both simple and complex nouns (at least historically) can be found in this table. Their etymology does not affect \isi{stress assignment}, though.} and in \Next for inflected verbal forms. The forms in these examples differ with regard to the position of the last closed \isi{syllable} in the word, and thus, by the condition that makes the stress move from the first \isi{syllable} towards the end (but only up to the penultimate \isi{syllable}). Predicates that consist of more than one verbal stem behave like simple verbs in this respect (see \NNext). 



 \begin{table}[htp]	
 
\begin{tabular}{ll}
\lsptoprule
{\sc Yakkha }& {\sc gloss}\\
\midrule
\emph{ˈom.phu} &\rede{verandah}\\
\emph{ˈkho.rek} & \rede{bowl}\\
\emph{ˈca.ram}& \rede{yard}\\
\emph{ˈko.ko.mek}& \rede{butterfly}\\
\emph{ˈol.lo.bak} &\rede{fast}\\
\emph{ˈtok.ca.li}&\rede{buttocks}\\
\emph{ˈyok.yo.rok}&\rede{beyond, a bit further}\\
\emph{ˈkam.ni.bak}&\rede{friend}\\
\emph{wa.ˈriŋ.ba}&\rede{tomato}\\
\emph{cuʔ.ˈlum.phi}&\rede{stele, pillar, stick}\\
\emph{nep.ˈnep.na}&\rede{short one}\\
\emph{op.ˈchyaŋ.me}&\rede{firefly}\\
\emph{cik.ci.ˈgeŋ.ba}&\rede{Bilaune tree}\\
\lspbottomrule
\end{tabular}
\caption{Default stress}\label{stresstab1}
\end{table}




\ex.\a. \glll ˈtum.me.cu.na\\
/tund-meʔ-ci-u=na/\\
understand{\sc -npst-du.A-3.P=nmlz.sg}\\
\rede{They (dual) understand him.}
\b. \glll ˌndum.men.ˈcun.na\\
/n-tund-meʔ-n-ci-u-n=na/\label{str-ex1}\\
{\sc neg-}understand{\sc -npst-neg-du.A-3.P-neg=nmlz.sg}\\
\rede{They (dual) do not understand him.}
\b. \glll ˌtum.meʔ.ˈnen.na\label{str-ex2}\\
/tund-meʔ-nen=na/\\
understand{\sc -npst-1>2=nmlz.sg}\\
\rede{I understand you.}

\ex.\a. \glll ˈluk.ta.khya.na\\
/lukt-a-kheʔ-a=na/\\
run{\sc -pst-V2.go-pst[3sg]=nmlz.sg}\\
\rede{He ran away.}
\b. \glll luk.ta.ˈkhyaŋ.na\\
/lukt-a-kheʔ-a-ŋ=na/\\
run{\sc -pst-V2.go-pst-1sg=nmlz.sg}\\
\rede{I ran away.}


Examples like \emph{ˈkam.ni.bak} \rede{friend} show that the stress never moves to the final \isi{syllable}, even when the \isi{syllable} is heavy. Patterns where the final \isi{syllable} is stressed are possible though, because prefixes are not part of the stress domain. In monosyllabic nouns that host a possessive prefix, the stress generally remains on the stem, as in  \Next.


\ex.\a. \glll a.ˈpaŋ\\
/a-paŋ/\\
{\sc 1sg.poss-}house\\
\rede{my house}
\b. \glll u.ˈphuŋ\\
/u-phuŋ/\\
{\sc 3sg.poss-}flower\\
\rede{his/her flower}


Yakkha has a category of obligatorily possessed nouns, and some of them, mostly kin terms, have undergone  \isi{lexicalization}. They are all monosyllabic. With regard to  stress, the prefix is no longer distinguished from the stem, as shown by examples like \emph{ˈa.mum} \rede{grandmother}, \emph{ˈa.pum} \rede{grandfather}, \emph{ˈa.na} \rede{elder sister}, \emph{ˈa.phu} \rede{elder brother}.\footnote{In the domain of \isi{kinship}, forms with first person singular inflection are also used in default contexts, when no particular possessor is specified. The default possessive prefix for nouns denoting part-whole relations is the third person singular \emph{u-}.} The words are, however, not morphologically opaque, as the first person possessive prefix \emph{a-} can still be replaced by other prefixes in a given context, and then, the stress pattern changes to the expected one, e.g., \emph{u.ˈmum} \rede{his grandmother}. An example for lexicalized obligatory possession beyond the domain of \isi{kinship} is the word \emph{ˈu.wa} \rede{liquid, nectar, water}.

The shift of stress described above occurs only in monosyllabic kin terms. In bisyllabic words, the stress is again on the first \isi{syllable}  of the stem or on the \isi{syllable} that is closed. Terms like \emph{a.ˈnun.cha} \rede{younger sibling} (both sexes) or \emph{a.ŋo.ˈʈeŋ.ma} \rede{sister-in-law} illustrate this.


As Yakkha is a predominantly suffixing language, there are not many prefixes that could illustrate the fact that the domain of stress does not include prefixes. Apart from the \isi{possessive prefixes}, evidence is provided by reduplicated \isi{adjectives} and adverbs like \emph{pha.ˈphap} \rede{entangled, messy} or \emph{son.ˈson} \rede{slanted, on an angle}. The base for these words are verbal stems, in this case \emph{phaps} \rede{entangle, mess up} and \emph{sos} \rede{lie slanted}. Their stress pattern allows the conclusion that this kind of \isi{reduplication} is a prefixation (for the other morphophonological processes involved cf. \sectref{morphophon}).


Clitics generally do not affect \isi{stress assignment}, since they are attached to the phrase and thus to a unit that is built of words to which stress has already been assigned.\footnote{The term \rede{clitic}  may have two readings: (i) affixes that are categorically unrestricted (represented by the equals sign \rede{=} instead of a hyphen \rede{-}), or (ii) phonologically bound words, like \isi{demonstratives}. The latter are written separately in the \isi{orthography} used in this work, as they  may also appear independently and they have the ability to head phrases.} Examples are provided in \Next for \isi{case} clitics and in \NNext for discourse-structural clitics.


\ex.\a. \glll ˈkho.rek.ci\\
/khorek=ci/\\
bowl{\sc =nsg}\\
\rede{the bowls}
\b. \glll ˈtaŋ.khyaŋ.bhaŋ\\
/taŋkhyaŋ=phaŋ/\\
sky{\sc =abl}\\
\rede{from the sky}
\b. \glll ˈkam.ni.bak.ci.nuŋ\\
/kamnibak=ci=nuŋ/\\
friend{\sc =nsg=com}\\
\rede{with the friends}


\ex.\a. \glll a.ˈyu.bak.se\\
/a-yubak=se/\\
{\sc 1sg.poss-}goods{\sc =restr}\\
\rede{only my goods}
\b. \glll u.ˈkam.ni.bak.ko\\
/u-kamnibak=ko/\\
{\sc 3sg.poss-}friend{\sc =top}\\
\rede{his friend(, though)}


An exception to this rule is the nominalization in \emph{=na} and \emph{=ha}. These nominalizers may attach to the verbal inflection, in relative clauses, complement clauses or in main clauses (see \sectref{nmlz-uni}). They are categorically unrestricted (i.e., taking not only verbal hosts), and not an obligatory part of the verbal inflection. However, if they attach to the verb, they are part of the stress domain. If this was not the \isi{case}, \isi{stress assignment} as in \emph{luk.ta.ˈkhyaŋ.na} \rede{I ran away.} would be unexpected, because then the stress would be on the final \isi{syllable} of the stress domain, which violates the prosodic constraints of Yakkha. The anomalous behavior of the nominalizers is not unexpected in light of the fact that they are being reanalyzed from discourse markers to part of the \isi{inflectional morphology}.\footnote{For instance, they also show \isi{number} agreement with verbal arguments, with \emph{=na} indicating singular and \emph{=ha} indicating nonsingular or non-countable reference.}

It is hard to tell whether there is secondary stress. Even in words with five syllables, like in \Last[b], no secondary stress could be detected.  Secondary stress was clearly audible in compounds such as those shown in \tabref{stresscomp}.  It is found on the first \isi{syllable} of the second part of the compound, while the main stress remains on the first \isi{syllable} of the whole compound. Such compounds may override the general restriction against stress on word-final syllables. In inflected verb forms, secondary stress can be found on the verbal stem, e.g., in \emph{ˌndum.men.cu.ˈŋan.na} \rede{We (dual) do not understand him.}; cf. also examples \ref{str-ex1} and \ref{str-ex2} above.

 

 \begin{table}[htp]	
\resizebox{\textwidth}{!}{
\begin{tabular}{ll}
\lsptoprule
{\sc yakkha} & {\sc gloss}\\
\midrule
\emph{ˈko.len.ˌluŋ} &\rede{marble stone} (\rede{smooth-stone})\\
\emph{ˈpi.pi.ˌsiŋ} & \rede{straw, pipe} (\rede{([redup]suck-wood})\\
\emph{ˈyo.niŋ.ˌkhe.niŋ} & \rede{hither and thither} (\rede{while thither-while hither})\\
\emph{ˈmo.niŋ.ˌto.niŋ} & \rede{up and down} (\rede{while down-while up})\\
\emph{ˈsa.meʔ.ˌchoŋ} &\rede{protoclan} (\rede{clan-top})\\
\emph{ˈlim.bu.ˌkhim} & a clan name, composed of the term for the \ili{Limbu} ethnic group\\
& and a word for \rede{house} in many Kiranti languages\\
\lspbottomrule
\end{tabular}
}
\caption{Stress in compounds}\label{stresscomp}

\end{table}



Finally, one exception to the stress rules has to be mentioned. Yakkha has several triplicated ideophonic adverbs, where the first \isi{syllable} is the base and the second and third \isi{syllable} rhyme on the vowel, but replace the initial consonant with a liquid, a glide or a coronal stop, e.g., [se.re.ˈreː] \rede{drizzling}, or [hi.wi.ˈwiː] \rede{pleasantly breezy} (cf. \sectref{redup}). In addition to the \isi{triplication}, the vowel of the last \isi{syllable} is lengthened, and the stress is always on the last \isi{syllable} in these adverbs.


\section{Morphophonological processes}\label{morphophon}



This section discusses the various morphophonological processes in Yakkha. The domains to which certain rules and processes apply are not always congruent. The existence of more than one phonological domain and the problems for theoretical approaches that assume a prosodic hierarchy have already been discussed for \ili{Limbu}, another Eastern \mbox{Kiranti} language  \citep{Hildebrandt2007Prosodic, Schieringetal2010The-prosodic}. Yakkha adds further support to challenges for the assumption that domains of prosodic rules are necessarily hierarchically ordered. 

The following phonological domains could be identified in Yakkha morpho\-pho\-no\-logy: the rules for \isi{stress assignment} disregard prefixes and phrasal affixes. In contrast, the \isi{vowel harmony} establishes a relation between the prefix and the stem only, ignoring the suffixes. The \isi{voicing} rule has the broadest domain (cf. \sectref{voicing} below). Furthermore, some rules differentiate between morphologically simple and complex words (compounds). The \isi{voicing} rule and also various repair operations of marked structures like adjacent obstruents are sensitive to morpheme boundaries, the latter, more precisely,  to stem boundaries. 

\figref{w-domains} provides an overview of the different domains to which the morphophonological processes apply.\footnote{The morphological structure of the word is slightly simplified in the table, disregarding complex predicates that consist of more than one verbal stem. Complex predicates are treated identically to simple words by the stress rule and the \isi{voicing} rule (except for the behavior of /c/).} \sectref{voicing} deals with  the \isi{voicing} rule. The prefixation of underspecified \isi{nasals} is treated in \sectref{nas-pref}. A \isi{case} of \isi{vowel harmony} is described in \sectref{vow-har}. Adjacent vowels are not preferred in Yakkha, and strategies to avoid such undesirable sequences are treated in \sectref{strat-vow}. \sectref{h-w-m} deals with consonants in intervocalic environments. \sectref{ass-to-obs} describes assimilations. The employment of \isi{nasals} to repair marked sequences of adjacent obstruents as well as adjacent vowels in complex predicates is discussed in \sectref{nas-strat}. Finally, \sectref{sec-nasalcop} is concerned with a process of \isi{nasal copying} which is found in the verbal inflection of many Kiranti languages.


%\footnote{Suffixes that subcategorize exclusively for the stem can only be found in the verbal domain in Yakkha, in inflection as well as in infinitives, converbs and nominalizers. Nominal \isi{case} and \isi{number} markers are phrasal affixes, and furthermore, \isi{number} is optional on nouns.}  


%\begin{figure}[htp]
%\begin{center}
%\begin{tabular}{lllll} 
%	&{\bf prefix}&{\bf stem(s)}&{\bf suffixes}&{\bf clitics}\\
%\midrule
%\cline{3-4}
%(1)	&	&\multicolumn{2}{|c|}{stress assignment}&\\
%\cline{2-5}
%(2-a)&	\multicolumn{4}{|c|}{\isi{voicing}/N\_}\\
%\cline{2-5}
%(2-b)&	&\multicolumn{3}{|c|}{\isi{voicing}/V\_V}\\
%\cline{2-5}
%(3)&\multicolumn{2}{|c|}{vowel harmony}&&\\
%\cline{2-3}
%\end{tabular}
%\caption{Summary of phonological domains}\label{w-domains}
%\end{center}
%\end{figure}


\begin{figure}[htp]
\begin{center}
\begin{tabular}{l|l|l|l|l} 
 \lsptoprule
	&{\bf prefix}&{\bf stem(s)}&{\bf suffixes}&{\bf clitics}\\
\hline
(1)	&\cellcolor[gray]{.8}	&\multicolumn{2}{c|}{stress assignment}&\cellcolor[gray]{.8}\\
\hline
(2-a)&	\multicolumn{4}{c}{voicing/N\_}\\
\hline
(2-b)&\cellcolor[gray]{.8}	&\multicolumn{3}{c}{voicing/V\_V}\\
\hline
(3)&\multicolumn{2}{c|}{vowel harmony}&\cellcolor[gray]{.8}&\cellcolor[gray]{.8}\\
\lspbottomrule
\end{tabular}
\caption{Summary of phonological domains}\label{w-domains}
\end{center}
\end{figure}

\subsection{Voicing}\label{voicing}

In Yakkha, unaspirated plosives and the affricate are voiced in intervocalic and postnasal environments and before \isi{liquids} and \isi{glides}, as schematized in \figref{voic-fig}, where C stands for unaspirated plosives and the affricate, N for \isi{nasals} and L for \isi{liquids} and \isi{glides}. Voicing predominantly  applies at morpheme boundaries, but also inside words that, at least synchronically, cannot be split up further into separate morphemes. The rule is illustrated by example \Next, with the stem-final /k/ of the verb \emph{yokma} \rede{search}, and by \NNext, with the stem-initial /t/ of the verb \emph{tama} \rede{come}. 

\begin{figure}[htp]	
\begin{center}		
\begin{tabular}{l}
\lsptoprule
C.{\sc unvoiced} → C.{\sc voiced}/N\_\\
C.{\sc unvoiced} → C.{\sc voiced}/V\_V\\ 
C.{\sc unvoiced} → C.{\sc voiced}/\_L\\
\lspbottomrule
\end{tabular}
\caption{Voicing rules}\label{voic-fig}
\end{center}
\end{figure}


\ex.\a.\glll  yoknenna.\\
/yok-nen=na/\\
search{\sc -1>2[pst]=nmlz.sg}\\
\rede{I looked for you.}
\b.\glll  yogu\\
/yok-u/\\
search{\sc -3.P[imp]}\\
\rede{Look for itǃ}

\ex.\a.\glll  tameʔna.\\
/ta-meʔ=na/\\
come{\sc [3sg]-npst=nmlz.sg}\\
\rede{He will come.}
\b.\glll  ndamenna\\
/N-ta-meʔ-n=na/\\
{\sc neg-}come{\sc [3sg]-npst-neg=nmlz.sg}\\
\rede{He will not come.}


Some environments containing \isi{liquids} and \isi{glides} that trigger \isi{voicing} are shown in \tabref{liqvoice}, with both monomorphemic and multimorphemic words. Some words are found with either pronunciation, and the current conclusion is that allegro speech leads to \isi{voicing}, and that this became the norm for some words, but not for others. 




\begin{table}[htp]
\begin{tabularx}{\textwidth}{llX}
\lsptoprule
&{\scshape Yakkha} & {\scshape gloss}\\
\midrule
{\bf /pl/}&\emph{taplik \ti tablik} &\rede{story}\\
&\emph{hoblek} & [manner of throwing or pouring] \rede{the whole/ at once}\\
{\bf /pr/}&\emph{hobrek} & \rede{completely [rotten]}\\
 &\emph{khibrum.ba} & \rede{fog} (also derogative for people of Caucasian phenotype)\\
{\bf /tr/}&\emph{hoŋdrup} & \rede{pig as present for in-laws}\\
{\bf /kw/}&\emph{cogwana}& \rede{he does it}\\
{\bf /pw/}&\emph{ubwaha}& \rede{he earns [money]}\\
{\bf /khy/}&\emph{maghyam} &\rede{old woman}\\
{\bf /tr/}&\emph{phetrak \ti phedrak} &\rede{petal}\\
{\bf /pr/}&\emph{capra \ti cabra} &\rede{spade with long handle}\\
{\bf /pl/}&\emph{lupliba \ti lubliba} &\rede{earthquake}\\
\lspbottomrule
\end{tabularx}
\caption{Voicing before \isi{liquids} and glides}\label{liqvoice}
\end{table}


As shown above, the \isi{voicing} rule applies to lexical stems, but it also applies to inflectional morphemes and phrasal affixes (see \Next). Thus, the domain for \isi{voicing} is bigger than the domain that is relevant for stress, as phrasal affixes undergo \isi{voicing}, and as prefixes may trigger \isi{voicing} as well.


\ex.\label{hongma}\a.\glll	hoŋmacibego.\\
	 /hoŋma=ci=pe=ko/\\
		river{\sc =nsg=loc=top}\\
		\rede{in the rivers(, though)}
\b.\glll tummecuganabu.\\
/tum-meʔ-c-u-ka=na=pu/\\
understand{\sc -npst-du-3.P.-2.A=nmlz.sg=rep}\\
\rede{(People say that) you (dual) understand him/her.}


After this outline of the basic properties of \isi{voicing} in Yakkha, let us now turn to its details. 
The \isi{voicing} rule needs further specification for prefixes. While nasal prefixes trigger \isi{voicing}, vocalic prefixes are excluded from the \isi{voicing} domain, irrespective of other factors such as stress. I have shown in \sectref{stress} above that \isi{voicing} is triggered neither in \emph{a.ˈpaŋ} \rede{my house} nor in \emph{ˈa.pum} \rede{(my) grandfather}. Only prefixes that consist of a nasal trigger \isi{voicing}, as shown in \Next.


\ex.\a. \glll mbaŋ\\
/N-paŋ/\\
{\sc 2sg.poss-}house\\
\rede{your house}
\b. \glll ŋ-gamnibak\\
/N-kamnibak/\\
{\sc 2sg.poss-}friend\\
\rede{your friend}

		 		 
In \sectref{stress} on \isi{stress assignment}, I mentioned reduplicated \isi{adjectives} and adverbs. They also provide further evidence for the restriction of the \isi{voicing} rule to nasal prefixes. I will exemplify this with the two \isi{adjectives} \emph{bumbum} \rede{compact and heavy} and \emph{tutu} \rede{far up} (cf. \sectref{redup} for more examples). The base of the adjective \emph{bumbum} has the corresponding verbal stem \emph{pups \ti pum} \rede{fold, press, tuck up}, while the base of \emph{tutu} is the adverbial root \emph{tu} \rede{uphill}. In analogy to the stress behavior, my default assumption is that the \isi{reduplication} is a prefixation, although the \isi{voicing} facts would support either option. The stem allomorph \emph{pum} is reduplicated to /pum-pum/ (the stem \emph{pups} surfaces only before vowels) and, subsequently, the stem undergoes \isi{voicing}, which is then spread to the first \isi{syllable} to preserve the identity between the base and the reduplicated morpheme. In contrast to this, in \emph{tutu} \rede{far up}, the intervocalic environment that  results from the \isi{reduplication} does not trigger \isi{voicing}.  
 
As stated in the beginning of this section, \isi{voicing} does not apply to aspirated plosives, at least not in the Tumok dialect (see \Next). Exceptions are found only in a handful of lexemes, mostly in ideophonic adverbs (see \sectref{sec-ideophone}). However, aspirated plosives (and the affricate) get voiced when they occur in \isi{function verb}s,\footnote{Function verbs are grammaticalized verbs, glossed as \rede{V2} (see Chapter \ref{verb-verb}.)} i.e., in word-medial position (see \NNext). These complex predicates also constitute one domain for \isi{stress assignment}, in contrast, for instance, to the southern neighbour language \ili{Chintang}, where each verbal stem in a \isi{complex predicate} constitutes a stress domain on its own \citep[57]{Bickeletal2007Free}. 

\ex.\a.\glll ŋkhyanna.\\
/N-khy-a-n=na/\\
{\sc neg-}go{\sc [3sg]-pst-neg=nmlz.sg}\\
\rede{He did not go.}
\b.\glll memphaʔle.\\
/meN-phat-le/\\
{\sc neg-}help{\sc -cvb}\\
\rede{without helping}


\ex.\a.\glll kam cog-a-ghond-a-ga=i.\\
/kam cok-a-khond-a-ka=i/\\
work do{\sc -imp-V2.roam-imp-2=emph}\\
\rede{Go on working.}
\b.\glll hab-a-bhoks-a=na.\\
/hap-a-phoks-a=na/\\
cry{\sc -pst-V2.split[3sg]-pst=nmlz.sg}\\
\rede{She broke out in tears.}

 

Yakkha has a class of composite predicates that consist of a noun and a verb. They show varying degrees of morphosyntactic freedom, but they are generally not as tightly fused as the  verb-verb predicates. This is also reflected by stress: noun and verb each have their own stress, even if this results in adjacent stress. Voicing, too, treats both components as separate items  (see \Next).\footnote{These predicates form a lexical unit though, and the nouns do not enjoy the syntactic freedom that is expected of full-fledged arguments. These predicates are best understood as idiomatic phrases (cf. Chapter \ref{noun-verb}).}

\ex.\a.\glll ˈsa.ya  ˈpok.ma\\
/saya pok-ma/\\
head.soul raise{\sc -inf}\\
\rede{to raise the head soul} (a ritual)
\b.\glll  ˈluŋ.ma ˈtuk.ma\\
/luŋma tuk-ma/\\
liver pour{\sc -inf}\\
\rede{to love}
\b.\glll  ˈsak ˈtu.ga.nai?\\
/sak tug-a=na=i/\\
hunger ache{\sc [3sg]-pst=nmlz.sg=q}\\
\rede{Are you hungry?/ Is he hungry?/ Are they hungry?}


Between vowels, voiced stops may further assimilate to their surrounding material and become continuants, as several alternations between intervocalic [b] and [w] show. Thus, \emph{kamnibak} \rede{friend} may also be pronounced [kamniwak], or the \isi{imperative} of \emph{apma} \rede{to come (from a visible distance on the same level)} can alternate between [aba] and [awa]. Like in Belhare \citep{Bickel1998Rhythm}, intervocalic /t/ may also become a continuant /r/, as some historical stem changes (e.g., \emph{*thut} → \emph{thur}) and in \isi{function verb}s show, e.g., the \isi{function verb} \emph{ris} that originates in the lexical stem \emph{tis} \rede{apply, invest}, or \emph{raʔ} originating in the lexical stem \emph{taʔ} \rede{bring (from further away)}. 


The suffix \emph{-ci} does not get voiced, neither in verbal nor in nominal inflection, as example \ref{hongma} has already shown. This exceptional behavior might point towards a more complex historical form of this suffix. The only instance of a voiced marker \emph{-ci} is in the second person dual pronoun \emph{njiŋda} (you), which is complex at least from a historical perspective. 

The affricate /tsʰ/ (written <c>) behaves exceptionally in other contexts, too. In the \isi{function verb} \emph{ca} \rede{eat} it does not undergo \isi{voicing} (see \Next[a]),\footnote{This function verb is the only one with initial /c/.} for which there is no neat explanation yet. Example \Next[b] shows that \isi{voicing} does apply to plosives in function verbs, and as example \NNext shows, stem-initial /c/ does get voiced in other environments. In some morphemes, the affricate shows free variation, as in the additive focus \isi{clitic} \emph{=ca}. It is found both voiced and unvoiced, neither related to individual nor to dialectal differences. 
% \ili{Limbu} has \emph{-si/-chi} for dual and \emph{-si} for nonsingular patient \citep[75ff]{Driem1987A-grammar}.} and in the nonsingular \isi{clitic} \emph{=ci} for noun phrases, the affricate is never voiced. 
 
 \ex.\a. \glll incama\\
 /in-ca-ma/\\
 trade{\sc -V2.eat-inf}\\
 \rede{to sell}
 \b. \glll  hambiʔma\\
 /ham-piʔ-ma/\\
 distribute{\sc -V2.give-inf}\\
 \rede{to distribute (among people)}
 
 
 \ex.\a. \glll njogwana.\\
 /n-cok-wa=na/\\
 {\sc 3pl.A-}do{\sc -npst=nmlz.sg}\\
 \rede{They will do it.}
 \b. \glll men-ja-le\\
 /men-ca-le/\\
 {\sc neg-}eat{\sc -cvb}\\
 \rede{without eating}
 

Another exception to the \isi{voicing} rule has to be mentioned, as shown in \Next[a] and \Next[b]. Stem-final /t/ remains voiceless between vowels. If the stem ends with a nasal and /t/, \isi{voicing} applies, as in \Next[c], and stem-initial /t/ undergoes \isi{voicing} as well. The absence of \isi{voicing} at the end of stems can be explained with the history of the  /-t/ final stems. Comparison with \ili{Chintang} and Belhare \citep{Bickel2003Belhare, Bickeletal2007Free} shows that there must have been geminated /tt/, resulting from a CVt stem to which the augment \emph{-t} was added (discussed in \sectref{stem}). Voicing does not apply when there is more than one underlying consonant between the vowels. 
			
			\ex.\a.\glll mituna.\\
			/mit-u=na/\\
			remember{\sc [pst]-3.P=nmlz.sg}\\
			\rede{He remembered it.} 
			\b.\glll  phatuci!\\
			/phat-u-ci/\\
			help{\sc -3.P[imp]-nsg.P}\\
			\rede{Help themǃ}
			\b.\glll chem endugana?\\
			/chem ent-a-u-ka=na/\\
			song apply{\sc -pst-3.P-2.A=nmlz.sg}\\
			\rede{Did you put on music?} 	
	 
	 
\subsection{The prefixation of underspecified nasals}\label{nas-pref}		 
		 
Yakkha has several nasal prefixes that do not constitute  syllables of their own, but result in onsets that consist of prenasalized consonants. The prefixes are underspecified for the place of articulation, and thus they always assimilate to the place of articulation of the following consonant. The nasal prefixes also trigger \isi{voicing} stem-initially, as could already be seen in \sectref{voicing} above. These nasal prefixes have several morphemic values, already mentioned in \sectref{syllable}, and repeated here for convenience: they index third person plural S and A arguments on verbs \Next[a] and verbal \isi{negation} \Next[b]. The nasal prefixes also encode second person singular possessors on nouns \Next[c], and in adverbs, they encode a distal relation (see \Next[d]). If the nasal prefix is attached to a nasal-initial stem, it yields an initial nasal geminate (see \NNext).
 
			
\ex.\ag.	m-by-a-ga-n=ha.\\
			{\sc 3pl.A-}give{\sc -pst-2.P-neg=nmlz.nsg}\\
			\rede{They gave it to you.} 
			\bg. ŋ-khy-a-n=na.\\
			{\sc neg-}go{\sc [3sg]-pst-neg=nmlz.sg}\\
			\rede{He did not go.}	
			\bg.	m-baŋ\\
			{\sc 2sg.poss-}house\\
			\rede{your house}
			\bg.ŋ-khaʔla\\
			{\sc dist-}like\_this\\
			\rede{like that}
			
			
			\ex.\ag.m-ma\\
			{\sc 2sg.poss-}mother\\
			\rede{your mother}
			\bg. n-nhe\\
			{\sc dist-}here\\
			\rede{there}

			
If the stem begins with a vowel or /w/, the nasal is realized as a velar nasal (see \Next). This might lead us to the conclusion that actually /ŋ/ is the underyling form and gets assimilated. This would, however, be the only instance of a morphophonological change from a velar nasal to [m] or [n] in Yakkha, and, thus, this option seems unlikely to me.


\ex. \ag. ŋ-og-wa-ci=ha.\\
			{\sc 3pl.A-}peck{\sc -npst-3nsg.P=nmlz.nsg}\\
			\rede{They (the roosters) peck them (the chicks).} 
	\bg. ŋ-ikt-haks-u-ci.\\
			{\sc 3pl.A-}chase{\sc -V2.send-3.P[pst]-3nsg.P}\\
			\rede{They chased them away.}
	\bg. kham ŋ-wapt-u=ha.\\
			soil  {\sc 3pl.A-}scratch{\sc -3.P[pst]=nmlz.nsg}\\
			\rede{They (the chicken) scratched the ground (they scrabbled about on the ground).}


A \isi{syllable} with a nasal before the consonant is marked in terms of the sonority hierarchy \citep{Jespersen1904_Lehrbuch, Selkirk1984_SyllableTheory, Hall2000Phonologie}. Therefore, the following  process can be noticed: if the preceding word (in the same clause) ends with a vowel, the nasal will resyllabify as the coda of the preceding word (see \Next), just as in Belhare \citep[547]{Bickel2003Belhare}. I have shown above that the domains for stress and for \isi{voicing} are not identical. This process adds a third domain of phonological rules to the picture, encompassing two words in terms of \isi{stress assignment}, as each of the words carries its own stress. Even though the nasal belongs to the preceding word in terms of \isi{syllable structure}, the choice of the nasal is determined by the following consonant, which also undergoes \isi{voicing} due to the nasal. This suggests a sequence of morphophonological processes, of which this resyllabification is the last to apply.    


\ex.\a.\glll liŋkhaci namnuŋ bagari\textbf{n} jog-a.\\
/liŋkha=ci     nam=nuŋ      bagari {\bf N}-cok-a/\\
Linkha{\sc =nsg} sun{\sc =com} bet {\sc 3pl-}do{\sc -pst}		\\
\rede{The Linkha clan had a bet with the sun.} \source{11\_nrr\_01.003}
\b.\glll  chuʔma\textbf{ŋ} gaksanoŋ, ...\\
/chuʔ-ma  {\bf N}-kaks-a-n=hoŋ/\\
tie{\sc -inf} {\sc neg-}agree{\sc [3sg]-pst-neg=seq}\\
\rede{It (the cow) was not okay with being tied, ...} \source{11\_nrr\_01.011} 
\b.\glll  nna\textbf{m} borakhyamanna.\\
 /nna {\bf N}-por-a-khy-a-ma-n=na/\\
that {\sc neg-}fall{\sc -pst-V2.go[3sg]-pst-neg=nmlz.sg}\\
\rede{That (stele) did not topple over.} \source{18\_nrr\_03.026}
\b.\glll  ka heʔniŋca\textbf{m} mandiʔŋanna.\\
/ka heʔniŋ=ca {\bf N}-mandiʔ-ŋa-n=na/\\
{\sc 1sg} when{\sc =add} {\sc neg-}get\_lost{\sc -1sg-neg=nmlz.sg}\\
\rede{I would never get lost.}\source{18\_nrr\_03.015}
					
			
			
\subsection{Vowel harmony}\label{vow-har}

Vowel harmony in Yakkha applies only to one prefix, namely to the possessive prefix  \emph{u-} for third person. It has an allomorph \emph{o-} that is triggered when the stressed \isi{syllable} of the stem contains the mid vowels /e/ or /o/, as illustrated by \tabref{vowelhar}. Suffixes do not  undergo \isi{vowel harmony} in Yakkha, and neither do other prefixes. 

One  exceptional \isi{case} has to be mentioned, the inflected form \emph{khohetu} \rede{he/she carried it off}. This is a complex verb that consists of the two verbal stems \emph{khuʔ} \rede{carry (on back)} and \emph{het} (a V2, indicating caused \isi{motion} away from a reference point). Apparently, the V2 makes the vowel in the first stem change to [o]. However, this is the only instance of \isi{vowel harmony} that has been encountered beyond the domain defined above.



\begin{table}[htp]
\begin{tabular}{llcll} 
 \lsptoprule
\multicolumn{2}{l}{{\textsc{before} /e/ \textsc{and} /o/}} &&\multicolumn{2}{l}{{\textsc{before} /u/, /i/, /a/}}\\
 \midrule
  \emph{o-heksaŋbe}  &\rede{behind her/him} &&\emph{u-paŋ}  &\rede{her/his house}\\
  \emph{o-hop}  &\rede{her/his nest} &&\emph{u-hiŋgilik}  &\rede{alive}\\
  \emph{o-tokhumak}  &\rede{alone} &&\emph{u-ʈukhruk}  &\rede{her/his body}\\
  \emph{o-senkhwak}  &\rede{her/his bone} &&\emph{u-mik}  &\rede{her/his eye}\\
  \emph{o-yok}  &\rede{her/his place/spot} &&\emph{u-tiŋgibhak}  &\rede{its thorn}\\
  \emph{o-poŋgalik}  &\rede{(its) bud} &&\emph{u-ʈaŋ}  &\rede{its horn}\\
  \emph{o-phok}  &\rede{her/his belly} &&\emph{u-muk}  &\rede{her/his hand}\\
  \emph{o-ʈesraŋ}  &\rede{reverse} &&\emph{u-nabhuk}  &\rede{her/his nose}\\
 \lspbottomrule
\end{tabular}
\caption{Vowel harmony}\label{vowelhar}
\end{table}



\subsection{Operations to avoid adjacent vowels}\label{strat-vow}

The processes that avoid vowel hiatus apply to adjacent vowels as well as to vowels that are separated by a glottal stop.\footnote{Diachronically, stems ending in a glottal stop used to be  CVt stems, and the /t/ got reduced to a glottal stop. Synchronically, stems ending in glottal stop often behave identical to stems that end in a vowel, in terms of morphophonological rules.} They are found in the verbal domain, since there are no suffixes or clitics beginning with a vowel  in the nominal domain.

\subsubsection{Vowel deletion}
 
The suffixes \emph{-a} and \emph{-u} can get deleted when they are adjacent to another vowel. In sequences of /-a-u/, for instance, /a/ gets deleted (see \Next[a]). This rule, however, also interacts with the morphology. While the past (and \isi{imperative}) suffix \emph{-a} is deleted when it is followed by the third person patient marker \emph{-u}, the same sequence, when it results from the nonpast marker \emph{-wa}, results in the deletion of \emph{-u} (see \Next[b]).   

\ex.\a.\glll tunduŋna.\\
/tund-\textbf{a}-u-ŋ=na/\\
understand{\sc -pst-3.P-1sg.A=nmlz.sg}\\
\rede{I understood her/him.}
\b.\glll  tundwaŋna.\\
/tund-wa-\textbf{u}-ŋ=na/\\
understand{\sc -npst-3.P-1sg.A=nmlz.sg}\\
\rede{I understand her/him.}


Suffix sequences of the underlying form /-a-i/ also result in the deletion of the suffix \emph{-a} (see \Next). When /a/ is part of the stem, however, nothing gets deleted (see \Next[c]).  Note also that intervocalic /h/ may become /y/, as in \Next[a].

\ex.\a.\glll kheiya.\\
/kheʔ-a-i=ha/\\
go{\sc -pst-1pl=nmlz.nsg}\\
\rede{We went.}
\b.\glll  tundigha.\\
/tund-a-i-ka=ha/\\
understand{\sc [3.A]-pst-2pl-2=nmlz.nsg}\\
\rede{They understood you (plural).}
\b.\glll hakokŋa caiwa.\\
/hakok=ŋa ca-i-wa/\\
later{\sc =ins} eat{\sc -1pl-npst}\\
\rede{We will eat later.}

Underlying sequences of three vowels are possible with open (CV and CVʔ) stems, in past and \isi{imperative} forms with a third person patient. In these verb forms, both suffixes are deleted. 

\ex.\a.\glll piŋ.na\\
/piʔ-a-u-ŋ=na/\\
give{\sc -pst-3.P-1sg.A=nmlz.sg}\\
\rede{I gave it to him.}
\b.\glll  soŋ.na\\
/soʔ-a-u-ŋ=na/\\
look{\sc -pst-3.P-1sg.A=nmlz.sg}\\
\rede{I looked at it.}
\b.\glll  haǃ\\
/haʔ-a-u/\\
bite{\sc -imp-3.P}\\
\rede{Bite (into) itǃ}
\b.\glll  cam.na\\
/ca-a-u-m=na/\\
eat{\sc -pst-3.P-1pl.A=nmlz.sg}\\
\rede{We ate it.}


\subsubsection{Ablaut} 

Ablaut is found only in one verb, in \emph{cama} \rede{eat}. Ablaut in some verbs is not unusual in Kiranti. The stem \emph{ca} has an allomorph \emph{co} that is not predictable from the phonological environment. It occurs when followed by other vowels, but not in all environments that would predict such a change if this was the condition.  Its distribution over the paradigm is shown in Chapter \ref{verbalmorph}, on page \pageref{par-cama-pst}.



\subsubsection{Insertion of glides}

If the back vowels (/a/, /o/ and /u/) belong to a verbal stem and are followed by the suffix \emph{-a}, the glide /y/ is inserted to avoid vowel hiatus. The morphological environment for these vowel sequences is provided by intransitive verbs, as well as in  in transitive verb forms with first or second person patients (see \Next). A similar process can  be encountered with stems that end in /ʔ/, with /ʔ/ being replaced by /y/, as in  \Next[d].

\ex.\a.\glll mima uhoŋbe uyana.\\
/mima u-hoŋ=pe u-a=na/\\
mouse {\sc 3sg.poss-}hole{\sc =loc}  enter{\sc [3sg]-pst=nmlz.sg}\\
\rede{The mouse entered her mousehole.}
\b.\glll  nam ayana.\\
/nam a-a=na/\\
sun descend{\sc [3sg]-pst=nmlz.sg}\\
\rede{The sun went down.}
\b.\glll  tayana.\\
/ta-a=na/\\
come{\sc [3sg]-pst=nmlz.sg}\\
\rede{He came.}% (from a place beyond sighting distance)
\b.\glll  soyaŋgana.\\
/soʔ-a-ŋ-ka=na/\\
look{\sc -pst-1sg.P-2.A=nmlz.sg}\\
\rede{You looked at me.}


\subsubsection{Gliding}

Front vowels of verbal stems may also be reduced to \isi{glides} when they are adjacent to /a/. The \isi{syllable} nucleus of the stem becomes part of the onset, and the word is again reduced by one \isi{syllable}, which is obvious because of the stress pattern. Examples \Next[a] and \Next[b] illustrate this for stems  ending in glottal stops and \Next[c] shows the same process with an open stem.


\ex.\a.\glll ˈkhyaŋ.na\\
/kheʔ-a-ŋ=na/\\
go{\sc -pst-1sg=nmlz.sg}\\
\rede{I went.}
\b.\glll ˈpyaŋ.na\\
/piʔ-a-ŋ=na/\\
go{\sc [3sg.A]-pst-1sg.P=nmlz.sg}\\
\rede{He gave it to me.}
\b.\glll  ˈsya.na\\
/si-a=na/\\
die{\sc [3sg]-pst=nmlz.sg}\\
\rede{He/she died.}

This may also happen when the stem has a back vowel. So far, this was only encountered for the verb \emph{luʔma} (see \Next). Other verbs, e.g., \emph{chuʔma} \rede{tie} appear in the expected form, e.g.,  \emph{chuyaŋna} \rede{he tied me (to something)}.

\ex.\a.\glll ˈlyaŋ.na\\
/luʔ-a-ŋ=na/\\
tell{\sc [3sg.A]-pst-1sg.P=nmlz.sg}\\
\rede{He told me.}
\b.\glll ˈlya.ha\\
/luʔ-a=ha/\\
tell{\sc [3sg.A;1.P]-pst=nmlz.nsg}\\
\rede{He told us.}


\subsection{Consonants in sonorous environment}\label{h-w-m}

\subsubsection{Intervocalic /h/ and /w/}
	
Intervocalic /h/ and /w/ also trigger vowel deletion. If the two vowels surrounding /w/ or /h/ have the same quality, the preceding vowel is deleted, even if this is the stem vowel. The deletion leads to new consonant clusters, i.e., to consonants followed by /w/ (see \Next[a]), or to aspirated voiced plosives (see \Next[b]). 

\ex.\a.\glll njwan.na\\
/n-ca-wa-n=na/\\
{\sc neg-}eat{\sc [3sg.A]-npst-neg=nmlz.sg}\\
\rede{He/she does not eat it.}
\b.\glll  tun.di.wa.gha\\
/tund-i-wa-ka=ha/\\
understand{\sc [3.A]-2pl.P-npst-2=nmlz.nsg}\\
\rede{He/they understand you (pl).}

If the vowels do not have the same quality and there is a transition from a close to an open vowel, intervocalic /h/ may also change to /y/  (see \Next).

\ex.\a.\glll   tun.dwa.ci.ya\\
/tund-wa-ci=ha/\\
understand{\sc [3sg.A]-npst-3nsg.P=nmlz.nsg}\\
\rede{He/she understands them.}
\b.\glll   ci.ya maŋ.cwa\\
/ci=ha maŋcwa/\\
get\_cold{\sc =nmlz.nsg} water\\
\rede{cold water}

The change of vowels to \isi{glides} and the realization of underlying /h/ as aspiration can even cross stem boundaries, as the following \isi{complex predicate}, consisting of three verbal stems, shows \Next. The underlying stems /piʔ/ and /heks/ fuse into [bhyeks].\footnote{The V2 \emph{-piʔ} indicates that a participant (the speaker, the subject or even someone else) is affected by the event, and the V2 \emph{-heks} specifies the temporal reference of the event as \isi{immediate prospective}. In pronunciation, they get fused to [bhyeks].} 


\ex.\glll a.cya tu.ga.bhyek.sana\\
/a-cya tuk-a-piʔ-heks-a=na/\\
{\sc 1sg.poss}-child  get\_ill{\sc [3sg]-pst-V2.give-V2.cut-pst=nmlz.sg}\\
\rede{My child is about to get ill.}


\subsubsection{Nasals in sonorous environment}\label{nas-son}

Nasals in sonorous environments are prone to phonological alternations. Nasal vowels are not part of the phoneme set of Yakkha. They may be generated, however, in  intervocalic environments at morpheme boundaries, or when a nasal occurs between a vowel and a liquid or a glide. This happens when the \isi{negative converb} (marked by prefix and suffix: \emph{meN}-Σ-\emph{le}) attaches to an open stem, or to a stem with initial /w/, /y/ or /l/. The nasal in \emph{meN}-Σ-\emph{le} is not specified. If it attaches to stems that have initial consonants, it assimilates to their place of articulation. Examples are provided in \tabref{nasal-son}. 

Another process producing nasal vowels was noticed in allegro forms of complex predicates such as \emph{ŋonsipma} \rede{feel shy} and \emph{thensipma} \rede{fit, suit}, which were pronounced \emph{ŋoĩsipma}  and \emph{theĩsipma} in fast speech.


\begin{table}[htp]
\begin{center}
\begin{tabular}{lll}%das gleiche mit longtable für mehrseitige tabellen
\lsptoprule
{\sc stem}&{\sc citation form}&{\sc negative converb}\\
\midrule
/waʔ/&\emph{waʔma} \rede{wear, put on}&\emph{mẽ.waʔ.le}  \rede{without wearing}\\
/a/&\emph{ama} \rede{descend}&\emph{mẽ.a.le}   \rede{without descending}\\
/u/&\emph{uma}  \rede{enter}&\emph{mẽ.u.le}  \rede{without wearing}\\
/lap/&\emph{lapma} \rede{seize, catch}&\emph{mẽ.lap.le}   \rede{without wearing}\\
/yok/&\emph{yokma} \rede{search}&\emph{mẽ.yok.le}  \rede{without wearing}\\
\lspbottomrule
\end{tabular}
\caption{Nasals in sonorous environment}\label{nasal-son}
\end{center}
\end{table}
 

\subsection{Assimilations}\label{ass-to-obs}

Syllable-final coronals assimilate to coronal fricatives, yielding a geminated fricative [sː] (written <ss>) (see \Next). This \isi{assimilation} is connected to stress. In unstressed syllabes, no \isi{assimilation} occurs, and the stem-final /t/ is simply deleted before fricatives (see \Next[c]). Occasionally, stem-final glottal stops can also undergo this \isi{assimilation}, but this is subject to free variation.


\ex. \a. \glll	es.se\\
			/et-se/\\
			apply{\sc -sup}\\
			\rede{in order to apply}
			\b.\glll	mis.saŋ\\
			/mit-saŋ/\\
			remember{\sc -sim}\\
			\rede{remembering}
			\b.\glll	ki.si.saŋ\\
			/kisit-saŋ/\\
			be\_afraid{\sc -sim}\\
			\rede{being afraid}
			

The following examples show that this process does not apply to the other plosives /k/ and /p/. Stems ending in a glottal stop are treated like open stems, illustrated by \Next[c].	Stems that have a coronal augment yield an underlying sequence of three consonants when followed by /s/. In this case, nothing gets assimilated. The general rule for augmented stems followed by consonants applies, i.e.,  the augment is simply omitted, as illustrated in \NNext. 
			
		\ex.\a.\glll ap.se\\
		/ap-se/\\
shoot{\sc -sup}\\
\rede{in order to shoot}
\b.\glll cok.se\\
/cok-se/\\
do{\sc -sup}\\
\rede{in order to do}	
	\b.\glll so.se\\
/soʔ-se/\\
look{\sc -sup}\\
\rede{in order to look}			
			
			
\ex.\a.\glll un.se\\
		/und-se/\\
		pull{\sc -sup}\\
		\rede{in order to pull}
		\b.\glll  chep.se\\	
			/chept-se/\\
			write{\sc -sup}\\
			\rede{in order to write}
			
			
Furthermore, stems ending in a coronal stop, and occasionally also stems ending in a glottal stop, show regressive  \isi{assimilation} to a velar place of articulation, as shown in \Next.
			
			\ex.\a.\glll phak.khuba\\
		/phat-khuba/\\
		help{\sc -nmlz}\\
		\rede{helper}
		\b.\glll khek.khuba\\
		/khet-khuba/\\
		carry\_off{\sc -nmlz}\\
		\rede{the one who carries it off}
		\b.\glll sok.khuba\\
		/soʔ-khuba/\\
		look{\sc -nmlz}\\
		\rede{the one who looks}
	
	
An optional regressive \isi{assimilation}, conditioned by fast speech, can be found in underlying sequences of \isi{nasals} followed by a palatal glide or a lateral approximant (/y/ or /l/), both stem-initially and stem-finally. In such environments, the nasal assimilates further, giving up its feature of nasality (see \Next). 

\ex.\a.\glll lleŋmenna.\\
		/N-leks-meʔ-n=na/\\
		{\sc neg-}become{\sc [3sg]-npst-neg=nmlz.sg}\\
		\rede{It will not happen./It is not alright.}
		\b.\glll mẽyelle.\\
		/meN-yen-le/\\
		{\sc neg-}obey{\sc -cvb}\\
		\rede{without listening/obeying}
		\b.\glll yyupmaci.\\
		/N-yupma=ci/\\
		{\sc 2sg.poss-}tiredness{\sc =nsg}\\
		\rede{your tiredness}\footnote{Some nouns are obligatorily marked for nonsingular, especially in experiential expressions.}


\subsection{Operations involving nasals}\label{nas-strat}			

\subsubsection{Nasality  assimilation}
		
The nasal consonants themselves also trigger several regressive \isi{assimilation} processes. The affected consonants may change in nasality or in place of articulation. Coronals and the glottal stop are particularly prone to assimilations, while the velar and the bilabial stop are less inclined to assimilate. Stem-final /t/ and /ʔ/ will assimilate completely if they are followed by stressed syllables starting in /m/ (see \Next[a]). Under the same condition, stems ending in velar stops (both plain and augmented) undergo nasal \isi{assimilation},  with the place of articulation being retained (see \Next[b] and \Next[c]).

\ex.\a.\glll pham.ˈmeŋ.na\\
/phat-me-ŋ=na/\\
help{\sc [3sg.A]-npst-1sg.P=nmlz.sg}\\
\rede{He/she helps me.}
\b.\glll  peŋ.ˈmeʔ.na\\
/pek-meʔ=na/\\
break{\sc [3sg]-npst=nmlz.sg}\\
\rede{It breaks.}
\b.\glll  naŋ.ˈmeʔ.na\\
/nakt-meʔ=na/\\
ask{\sc [3sg]-npst=nmlz.sg}\\
\rede{He asks.}

In stems that end in /n/ or /nd/ (with augmented /t/), the coda completely assimilates to [m]. In contrast to the \isi{assimilation} discussed above, this \isi{assimilation} is not sensitive to stress. For instance, stems like \emph{tund} \rede{understand} and \emph{yen} \rede{obey} have the infinitival forms \emph{tumma} and \emph{yemma}, respectively, with the stress falling on the first \isi{syllable}. Stems ending in a velar stop or in a bilabial stop never assimilate completely; their place of articulation is retained. Compare, e.g., \emph{pekma} \rede{break} (stem: \emph{pek}) with \Last[b] above. Following a general rule in Yakkha, augmented stems (ending in two consonants) block \isi{assimilation} and also other morphophonological processes, e.g., \emph{chepma} \rede{write} (stem: \emph{chept}). Furthermore, velar and bilabial \isi{nasals} never assimilate to other \isi{nasals}, in contrast to languages like Athpare and  Belhare \citep{Ebert1997A-grammar, Bickel2003Belhare}.


\subsubsection{Nasalization of codas}\label{nas-cod}

Nasalization of obstruents does not only happen as \isi{assimilation} to nasal material. When obstruents are adjacent in complex predicates, the first obstruent, i.e., the stem-final consonant of the first stem, becomes a nasal in order to avoid a marked structure. Examples are provided in  \tabref{nasalobs}.\footnote{The V2 \emph{-piʔ} has a suppletive form \emph{-diʔ}, which cannot be explained by phonological operations. It occurs only in intransitive uses of \emph{-piʔ \ti -diʔ} \rede{give} as a \isi{function verb}. The inflected forms show that the underlying stem is \emph{-piʔ}.} Within complex predicates this process is most frequently found in infinitival forms. In the inflected forms, morphological material (suffixes with vowel quality) gets inserted between the verbal stems, thus resolving the marked sequences of adjacent obstruents.

The nasal often retains the place of articulation of the underlying obstruent, but some assimilations are possible too, e.g., /sos-kheʔ-ma/ becoming \emph{soŋkheʔma} \rede{slide off} (slide-go). If the underlying obstruent is a glottal stop, the place of articulation of the nasal is always conditioned by the following consonant, e.g., \emph{han-cama} /haʔ-cama/ \rede{devour} (bite-eat). 

As \tabref{nasalobs} shows, both simple (CVC) and augmented stems (CVC-s and CVC-t) are subject to this change from obstruent to nasal. The same change can be observed in reduplicated adverbs and \isi{adjectives}, e.g., in  \emph{sonson} \rede{slanted} (derived from the verbal stem /sos/) or \emph{simsim} \rede{squinting, blinking} (derived from the verbal stem /sips/).

This process is also sensitive to stress. The last example of \tabref{nasalobs}, \emph{um.ˈkheʔ.ma}, with the stress on the second \isi{syllable}, can be contrasted with the nominalized \emph{ˈup.khu.ba} \rede{something that collapes}, with the stress on the first \isi{syllable}.  Here, the stem appears in the general form of \emph{t}-augmented stems that are followed by consonants: the augment is simply omitted.  


\begin{table}[htp]
\begin{center}
\begin{tabular}{lll} 
 \lsptoprule
\multicolumn{2}{l}{{\sc citation forms}} &{\sc stems}\\
 \midrule
  \emph{yuncama}  &\rede{laugh, smile} &/yut/ + /ca/\\
  \emph{suncama}  &\rede{itch} &/sus/ + /ca/\\
  \emph{incama}  &\rede{play} &/is/ + /ca/\\
  \emph{hancama}  &\rede{devour} &/haʔ/ + /ca/\\  
  \emph{sendiʔma}  &\rede{get stale} &/ses/ + /piʔ/\\
  \emph{mandiʔma} &\rede{get lost} &/mas/ + /piʔ/\\
  \emph{pendiʔma}  &\rede{get wet} &/pet/ + /piʔ/\\
  \emph{phomdiʔma}  &\rede{spill} &/phopt/ + /piʔ/\\ 
  \emph{sonsiʔma}  &\rede{slide, slip} &/sos/ + /siʔ/\\
  \emph{tomsiʔma}  &\rede{get confused} &/tops/ + /siʔ/\\
  \emph{yaŋsiʔma}  &\rede{get exhausted} &/yak/ + /siʔ/\\
  \emph{homkheʔma}  &\rede{get damaged} &/hop/ + /kheʔ/\\
  \emph{soŋkheʔma}  &\rede{slide off} &/sos/ + /kheʔ/\\
  \emph{umkheʔma}  &\rede{collapse} &/upt/ + /kheʔ/\\ 
 \lspbottomrule
\end{tabular}
\caption{Nasalization of obstruents stem-finally}\label{nasalobs}
\end{center}
\end{table}


\subsubsection{Insertion of nasals}

In addition to the nasalization of obstruents, \isi{nasals} can be inserted in complex predication, if the following condition is met: if the V2 in a \isi{complex predicate} starts with a vowel or in /h/, either the preceding consonants (the complete coda or only the augment of the first verbal stem) will become \isi{nasals}, or, when the first stem has CV or CVʔ shape, the default nasal /n/ will be inserted between the two stems. \tabref{nasalins} provides examples of citation forms of complex predicates with inserted \isi{nasals}, and their underlying stems. 


The process is not a blind insertion of phonetic material, i.e., it is not simply epenthesis. Remarkably, it is triggered by the phonological quality of non-adjacent morphological material: the change of stops to \isi{nasals} or the insertion of \isi{nasals} is conditioned by the availability of \isi{nasals} in the morphology that attaches to the stem. The suffixes containing \isi{nasals} have to attach directly to the complex stem in order to trigger the insertion of \isi{nasals}. Compare the examples in \Next. In \Next[a] and \Next[b], the sequence /pt/ becomes [mn], and the following /h/ is realized  as the aspiration of [n]. In \Next[c], the inflection does not immediately contain a nasal, and thus the phonological material of the stem remains as it is. It gets resyllabified, however, and the /h/ is realized as aspiration of the preceding consonant. Example \NNext, with the verb \emph{leʔnemma} \rede{let go, drop} illustrates the insertion of /n/ when a CV-stem (or CVʔ) and a vowel-initial stem are adjacent in complex predication. The same condition as in \Next can be observed. Only nasal material in the suffix string licenses the insertion of /n/ between the two verbal stems. 


\ex.\a. \glll lem.nhaŋ.ma\\
/lept-haks-ma/\\
throw{\sc -V2.send-inf}\\
\rede{to throw away/out}
\b. \glll lem.nhaŋ.nen?\\
/lept-haks-nen/\\
throw{\sc -V2.send-1>2}\\
\rede{Shall I throw you out?}
\b. \glll lep.thak.suŋ.na\\
/lept-haks-u-ŋ=na/\\
throw{\sc -V2.send-3.P[pst]-1sg.A=nmlz.sg}\\
\rede{I threw her/him out.}

\ex.\a. \glll leʔ.nen.saŋ\\
/leʔ-end-saŋ/\\
drop{\sc -V2.insert-sim}\\
\rede{stretching down}
\b. \glll u.laŋ le.ʔen.du.ci.ya\\
/u-laŋ leʔ-end-a-u-ci=ha/\\
{\sc 3sg.poss-}leg drop{\sc -V2.insert-pst-3.P-nsg.P=nmlz.nsg}\\
\rede{It (the aeroplane) lowered its landing gear.}



\begin{table}[htp]
\resizebox*{\textwidth}{!}{
\small
\begin{tabular}{ll} 
 \lsptoprule
{\sc citation forms}&{\sc stems}\\
 \midrule
\emph{hu.nhaŋ.ma}  \rede{burn down} &/huʔ/ + /haks/\\
 \emph{lem.nhaŋ.ma}   \rede{throw away/out} &/lept/  + /haks/\\
 \emph{khu.nhaŋ.ma}   \rede{rescue} &/khus/  + /haks/\\
 \emph{iŋ.nhaŋ.ma}   \rede{chase off} &/ikt/  + /haks/\\
 \midrule
 \emph{pheʔ.na.ma}   \rede{drop, leave at some place} &/phes/ + /a/\\
 \emph{et.na.ma}   \rede{enroll, install somewhere (and come back)}& /et/ + /a/\\
 \emph{tik.na.ma}   \rede{take along}& /tikt/ + /a/\\
 \emph{tiʔ.na.ma}   \rede{deliver (and come back), bring} &/tis/ + /a/\\
 \emph{yuk.na.ma}   \rede{put for s.b. and leave} &/yuks/ + /a/\\
 \midrule
 \emph{leʔ.nem.ma} \rede{drop}	&/leʔ/ + /end/\\
 \emph{hak.nem.ma} \rede{send down}&	/hakt/ + /end/\\
 \emph{aʔ.nem.ma} \rede{wrestle down}&	/a/ + /end/\\
 \emph{ak.nem.ma} \rede{kick down}	&/ak/ + /end/\\
 \emph{leʔ.nem.ma} \rede{drop}	&/leʔ/ + /end/\\
 \emph{lep.nem.ma} \rede{throw down}	&/lept/ + /end/\\
 \lspbottomrule
\end{tabular}
}
\caption{The insertion of \isi{nasals} in complex predication}\label{nasalins}
\end{table}


The insertion of /n/ can affect the coda of the first stem, too. Stems ending in /s/ may  change to CV-ʔ when followed by a vowel-initial stem, as in \emph{tiʔnama} \rede{deliver} (/tis + a/). This  again suggests a sequence of processes, i.e., the insertion of /n/, followed by the change of /s/ to [ʔ]. It is not clear, however, why these citation forms do not simply resyllabify, e.g., to [tisama] instead of [tiʔnama], because this resyllabification is exactly what happens in the corresponding inflected forms. Apparently, speakers prefer to keep morpheme boundaries and \isi{syllable} boundaries congruent in citation forms. Note that V2s starting in /h/ behave differently from V2s starting in a vowel, because a \isi{complex predicate} consisting of /khus/  + /haks/ does not become [khuʔ.nhaŋ.ma] but \emph{khu.nhaŋ.ma}.

\tabref{nas-sum} summarizes the processes of the preceding two sections, with examples for each process. To sum up, the insertion of \isi{nasals} and the transformation of obstruents to \isi{nasals} are employed to avoid marked structures such as adjacent vowels, adjacent obstruents, and impossible \isi{syllable} codas, while also maintaining the identity of  morpheme boundaries and \isi{syllable} boundaries. This stands in contrast to inflected forms, where resyllabification is unproblematic.

\begin{table}[htp]
\resizebox*{\textwidth}{!}{
\small
\begin{tabular}{lll} 
 \lsptoprule
{\sc operation}&{\sc citation form }&{\sc V.lex + V2}\\
 \midrule
/C{\tiny[1]}+C/ → {\bf N{\tiny[1]}}.C&\emph{hom.kheʔ.ma} \rede{get damaged} &/hop/ + /kheʔ/\\
/C{\tiny[1]}C{\tiny[2]}+V/ → C{\tiny[1]}.{\bf n}V&\emph{mak.ni.ma}   \rede{surprise}& /maks/ + /i/\\
/C{\tiny[1]}C{\tiny[2]}+hV/ → {\bf N{\tiny[1]}.n}hV&\emph{lem.nhaŋ.ma}   \rede{throw away/out} &/lept/  + /haks/\\
/s+hV/ → .{\bf n}hV&\emph{khu.nhaŋ.ma}   \rede{rescue} &/khus/  + /haks/\\
/s+V/ → ʔ.{\bf n}V&\emph{maʔ.ni.ma}   \rede{lose}& /mas/  + /i/\\
/V+V/ → Vʔ.{\bf n}V&\emph{aʔ.nem.ma} \rede{wrestle down}&	/a/ + /end/\\
 \lspbottomrule
\end{tabular}
}
\caption{Repair operations in complex predicates involving nasals}\label{nas-sum}

\end{table}



\subsection{Nasal copying}\label{sec-nasalcop}

In the verbal inflection of Kiranti languages, nasal morphemes can be realized up to three times in the suffix string, a process that was termed \rede{affix copying} or \rede{nasal copying}, e.g., in \citet{Driem1987A-grammar, Doornenbal2009A-grammar, Ebert2003Kiranti, Bickel2003Belhare}. Alternative analyses have been proposed to explain this process: \isi{recursive inflection} in \citet{Bickeletal2007Free} and radically underspecified segments  in \citet{Zimmermann2012_Affix}. 

Yakkha \isi{nasal copying} is illustrated by \Next. Suffixes that consist of \isi{nasals} or that contain \isi{nasals} occur more than once under certain conditions, and without any semantic consequences. There are no contrasting forms that lack the copied suffixes. It is  morphologically most economical to assume regressive copying, with the last nasal suffix serving as base. A comparison of the inflected forms in \Next below supports this reasoning, because the slots after the suffixes \emph{-meʔ} and \emph{-u} are filled with varying material.\footnote{Note that the glosses \rede{1sg.A} and \rede{\textsc{excl}} refer to the same morpheme, if the structure of the whole paradigm is taken into account. It is defined by the property [non-\isi{inclusive}]. This collapse of markers is also found  in the intransitive forms of the Belhare verbal inflection \citep{Bickel1995In-the-vestibule}. For the sake of the readability of the glosses, the morphological analysis as well as the \isi{alignment} patterns of particular morphemes are kept out of the glosses as far as possible.} What is remarkable about the \isi{nasal copying} is that the value of the underspecified nasal is determined by non-adjacent segments.


\ex.\a.\glll piŋ.ciŋ.ha\\
		/piʔ-a-u-N-ci-ŋ=ha/\\
		give\textsc{-pst-3.P-[N]-3nsg.P-1sg.A=nmlz.nsg}\\
	\rede{I gave it to them.} 
 	\b.\glll tun.dum.cim.ŋha\\
	/tund-a-u-N-ci-m-ŋ=ha/\\
	understand\textsc{-pst-3.P-[N]-3nsg.P-1pl.A-excl=nmlz.nsg}		\\
	\rede{We understand them.}
	\b.\glll ndum.men.cun.ci.ga.nha\\
	/n-tund-meʔ-N-ci-u-N-ci-ga-n=ha/\\
	\textsc{neg-}Σ\textsc{-npst-[N]-du.A-3.P-[N]-3nsg.P-2.A-neg=nmlz.nsg}\\
	\rede{You (dual) do not understand them.}

	
The motivation for this copying process might be a phonological repair operation to yield closed syllables.\footnote{Cf. \citet[22]{Schikowski2012_Morphology} for the same explanation on \ili{Chintang} suffix copying, although on p. 25 he points out that this explanation is not watertight, since some copying processes may even create open syllables.} Repair operations involving \isi{nasals} would not be uncommon for Yakkha, as I have pointed out in \sectref{nas-strat}. An obvious shortcoming of this explanation is that \isi{nasals} are not copied to all syllables that one would expect in light of a purely phonological condition (compare \Next[a] and \Next[b]). 


\ex.\ag.ŋ-khy-a-ma-ga-n=na (*ŋkhyanmanganna).\\
{\sc neg-}go\textsc{-pst-prf-2-neg=nmlz.sg}\\
\rede{You have not come.} 
\bg.ŋ-khy-a-ma-n-ci-ga-n=ha.\\
{\sc neg-}go\textsc{-pst-prf-[N]-du-2-neg=nmlz.nsg}\\
\rede{You (dual) have not come.}


An alternative analysis has been proposed by \citet{Zimmermann2012_Affix}, resulting from a \isi{comparison} of several Kiranti languages. In her approach, the copying is a morpheme-specific process, happening only in the vicinity of certain suffixes. In line with her observations, all instances of copied \isi{nasals} in Yakkha directly precede the suffix  \emph{-ci} (with the two morphological values \rede{dual} and \rede{{\sc 3nsg.P}}, see the \isi{paradigm tables} in \sectref{paradigmtables}). Hence, it is the suffix \emph{-ci} that licenses the \isi{nasal copying} in Yakkha. The process as such and the phonological content of the copies are morphologically informed; they are based upon the presence of certain morphological markers. In the absence of \emph{-ci} nothing gets copied, and the same holds for inflectional forms in which no \isi{nasals} are available to serve as base. Hence, \isi{nasal copying} is not just the blind fulfillment of a phonological constraint, as epenthesizing any nasal material would be. On the other hand, since no semantic content is added by the nasal copies, the operation is not purely morphological either, but located at the boundary between phonology and morphology.


Another observation made is that the nasal suffixes compete with regard to which suffix will serve as base for the copying. If we compare \Next[a] and \Next[b], we can see that here, the preferred choice is /n/, instantiated by  the \isi{negation} marker, although the closest available base in \Next[b] would be the velar nasal from the suffix \emph{-ŋ}. This shows that the choice is not determined by the linear succession of the available \isi{nasals}. The \isi{negation} is the only morphological contrast between the two verb forms, and the nasal that is copied changes from /ŋ/ to /n/, compared to \Next[a].  In \Next[c], there is a competition between /n/ and /m/ as bases, which is won by /m/. This selection principle holds throughout the inflectional paradigm, so that the hierarchy for the choice of the base must be /m/ > /n/ > /ŋ/.


\ex. \a.\glll tum.meŋ.cuŋ.ci.ŋha\\
/tund-meʔ-N-ci-u-N-ci-ŋ=ha/\\
		Σ\textsc{-npst-[N]-du.A-3.P-[N]-3nsg.P-excl=nmlz.nsg}\\
		\rede{We (dual, excl.) understand them.}
	\b.\glll ndum.men.cun.ci.ŋa.nha\\
	/n-tund-meʔ-N-ci-u-N-ci-ŋ(a)-n=ha/\\
		{\sc neg-}Σ\textsc{-npst-[N]-du.A-3.P-[N]-3nsg.P-excl-neg=nmlz.nsg}\\
		\rede{We (dual, excl.) do not understand them.}
	\b.\glll ndun.dwam.cim.ŋa.nha\\
	/n-tund-wa-u-N-ci-m-ŋ(a)-n=ha/\\
		{\sc neg-}Σ\textsc{-npst-3.P-[N]-3nsg.P-1pl.A-excl-neg=nmlz.nsg}\\
		\rede{We (plural) do not understand them.}
		



