\chapter{Adjectives and adverbs} \label{adj-adv}

Adjectives are lexical items specifying some property of a referent, while adverbs specify characteristics of an event such as cause, \isi{degree} and manner, and ground it in space and time. They are treated in one chapter because they are often derived from the same roots, mostly of verbal origin.  

The number of lexical \isi{adjectives} and adverbs, i.e., those that cannot be traced back to verbal stems, is rather small. Nevertheless, \isi{adjectives} and adverbs show some characteristics that motivate a separate lexical class. Most prominently, these are ideophonic patterns and the morphological processes of \isi{reduplication} and \isi{triplication}, which are highly  pro\-duc\-tive in this class, but only marginally found in other word classes. The derivational morphology attached to the mostly verbal bases deter\-mines which structural position in the clause they will occupy, and hence, whether they have adjectival or adverbial function. 

This chapter is structured as follows:  \isi{adjectives} are treated in \sectref{adj}. Comparative and equative constructions and the expression of \isi{degree} are treated in \sectref{sec-compar}.  The  derivations leading to the various types of adverbs are the topic of in \sectref{adv}. Reduplication, \isi{triplication} and ideophonic patterns are so rich that they deserve their own section (\sectref{redup}). 

Adjectives and adverbs that are employed for  spatial orientation, involving a topography-based  orientation system, will be discussed in Chapter \ref{ch-geomorphic}. 

\section{Adjectives}\label{adj}

\subsection{Kinds of adjectives}\label{adj-kinds}

The function of \isi{adjectives} is the modification of nouns, either inside the \isi{noun phrase} or as predicates of copular clauses. Many \isi{adjectives} are based on verbal stems historically, but not all of these stems behave like full-fledged verbs synchronically, for  instance in not showing the full range of inflectional possibilities that are known from verbs. 

The major strategy for the derivation of \isi{adjectives} is attaching the nominalizers \emph{=na} (when the head noun has singular \isi{number}) and \emph{=ha \ti =ya} (when the head noun has nonsingular \isi{number} or non-countable reference) to verbal roots, which results in a minimal relative clause (see \Next[a] and \Next[b], and Chapter \ref{ch-nmlz}). The bases of \isi{adjectives} are not necessarily verbs, however. These nominalizers can link any modifying material to a noun, regardless of its word class (see \Next[c]). This example also shows that \isi{adjectives} may head noun phrases, like minimal headless relative clauses. Such headless relative clauses are different from \isi{lexical nouns}; \isi{case} and \isi{number} marking are allowed on them, but possessive marking is restricted to \isi{lexical nouns}. 

\ex. \ag.ci=ha maŋcwa\\
get\_cold{\sc =nmlz.nc} water\\
\rede{cold water} (verbal: \emph{maŋcwa cisabhya} \rede{The water got cold.})
	\bg.haŋ=ha macchi\\
	be\_spicy{\sc =nmlz.nc} pickles\\
	\rede{hot pickles}
 	\bg.nna cancan*(=na)=bhaŋ\\
		that tall{\sc *(=nmlz.sg)=abl}\\
	\rede{from that tall one}  (referring to a rock)\source{38\_nrr\_07.040}
	
Some \isi{adjectives} look like  lexicalized inflected transitive verbs, like \emph{cattuna}, meaning  \rede{fat/strong} (no verbal root of this form attested) or the adjective in \Next.\footnote{The example also illustrates how the lexical meaning of \rede{be awake} has been extended metaphorically to mean \rede{witty, sprightly}.} 
	
	\exg.cend-u=na a-na \\
  wake\_up{\sc -3.P[pst]=nmlz.nsg} {\sc 1sg.poss-}eZ\\
  \rede{my witty elder sister} \source{40\_leg\_08.057}
  

Not all adjectival bases are synchronically found as verbs, though they show the typical augmented structure of a verbal stem. Some \isi{adjectives} show hybrid behavior, illustrating their verbal origin. For ins\-tance, \emph{khumdu} \rede{tasty} does not have a corresponding verb with the cita\-tion form \emph{khumma}. Yet, the adjective can be inflected for \isi{number} and \isi{negation} like a verb. Person and \textsc{tam} marking are not possible though.\footnote{The affirmative forms always display \emph{-u}, while the negative forms always display \emph{-i}.} The same behavior is found for \emph{ŋgolemninna} \rede{not smooth},  \emph{llininna} \rede{not heavy}.

\ex. \ag.khumdu=ha caleppa\\
tasty{\sc =nmlz.nc} bread\\
\rede{tasty bread}
\bg.kha cuwa ŋ-khumdi-n=ha!\\
this beer {\sc neg-}tasty{\sc -neg=nmlz.nc}\\
\rede{This beer is not tasty!}

  
Some bases with  unclear origin are \emph{heko} \rede{other}, \emph{ucun} \rede{good} and \emph{mam} \rede{big}.

\ex.\ag. mi=na khesup\\
small{\sc =nmlz.sg} bag\\
\rede{a/the small bag}
\bg. mãɖa luŋkhwak\\
	huge{\sc [nmlz.sg]} stone\\
	\rede{a/the huge rock}\footnote{This adjective has undergone a sound change: the nonsingular form is \emph{mamha}, but the singular form became \emph{mãɖa}, as a result of former *\emph{mamna}.}

	
There are only very few \isi{adjectives} that do not take the nominalizers \emph{=na} and \emph{=ha}. Another \isi{nominalizer}, \emph{-pa}, is found lexicalized in \emph{ulippa} \rede{old}. Other \isi{adjectives} appearing without prior nominalization are \mbox{\emph{maŋdu}} \rede{far} and \emph{upuŋge} \rede{free}. Lexemes with initial \emph{u-} are occasionally found among \isi{adjectives}, but more frequently so in adverbs. They originate from obligatorily possessed nouns (see \sectref{adv}).


	Many roots can serve as adjectival or as adverbial bases (see also  \sectref{adv}). A common marker for adverbial derivation is the \isi{comitative} \emph{=nuŋ} (also functioning as nominal \isi{case} marker). Compare the use of \emph{cattu} in \Next[a] and \Next[b].
	
	\ex. \ag.cattu=na pik apt-u!\\
	strong{\sc =nmlz.sg} cow bring\_across{\sc -3.P[imp]}\\
	\rede{Bring a fat/strong cow!}
	\bg.ka       tondaŋ    um-meʔ-nen,       nda cattu=nuŋ    lab-u-g=hoŋ                tokhaʔla ky-a.\\
	{\sc 1sg[erg]} from\_above pull{\sc -npst-1>2} {\sc 2sg} strong{\sc =com} hold{\sc -3.P[imp]-2.A=seq} upwards come\_up{\sc -imp}\\
	\rede{I will pull you up, grab it firmly and come up!} \source{01\_leg\_07.329}

	
\subsection{Color terms}\label{sec-color}

The system of Yakkha \isi{color terms}\footnote{The following discussion of \isi{color terms} relies on the natural stimuli in the environment, and on my observations of natural speech.} is worth mentioning because it only has four basic \isi{color terms}, with a privative distinction of \emph{phamna} \rede{red} and \emph{phimna} \rede{non-red}, in addition to the two terms at both ends of a monochrome lightness-scale, \emph{makhurna} \rede{black} and \emph{phuna} \rede{white}. Such an economical system is rather rare crosslinguistically, but the prominence of red conforms to the distributional restrictions discovered in the seminal study of \citet[2--3]{Berlinetal1969Basic}: 

\begin{itemize}
\item All languages contain terms for white and black.
\item If a language contains three terms, then it contains a term for red. 
\item If a language contains four terms, then it contains a term for yellow or green, but not both.
\end{itemize}

Via several derivations and combinations of the terms for red and non-red with the terms for black and white, one arrives at eleven \isi{color terms}, shown in \tabref{color} (in their singular forms with \emph{=na}). The term \emph{phamna} comprises  red, brown red and orange, and the term \emph{phimna} covers everything non-red, from yellow over green to blue. There is another word \emph{phiriryaŋna} for \rede{yellow}, but it is only used for food items, and could be derived from the same root as \emph{phimna}. Nowadays, a \ili{Nepali} loan has entered the language, replacing \emph{phimna} in as a term for \rede{yellow}: \emph{besareʔna}, derived from \ili{Nepali} \emph{besār} \rede{turmeric}, which is, however, not used as a color term in Nepali.  The monochrome terms can be used to specify the \isi{color terms} with regard to their brightness or darkness, e.g., \emph{maklup-maklupna phimna} for \rede{dark blue, dark green}, or \emph{maklup-maklupna phamna} for \rede{dark red, bordeaux red}. 



\begin{table}[htp]
\begin{centering}
\begin{tabular}{ll}
\lsptoprule
{\sc stem} & {\sc gloss} \\
\midrule
\emph{phamna}&\rede{red}\\
\emph{phimna}&\rede{(yellow), green, blue}\\
\emph{phuna}&\rede{white}\\
\emph{makhurna}&\rede{black}\\
\emph{phalik-phalikna}&\rede{reddish, pink, violet (dark and light shades)}\\
\emph{phiʔlik-phiʔliŋna}&\rede{greenish, blueish (sky blue, petrol, light green)}\\ 
\emph{phiriryaŋna}&\rede{yellow (food)}\\ 
\emph{besareʔna} [\textsc{nep}]&\rede{yellow}\\ 
\emph{phutiŋgirik}&\rede{bright white}\\
\emph{phutlek-phutlekna}&\rede{light grey, light yellow, light pink, beige}\\
\emph{maklup-maklupna}&\rede{dark brown/grey/blue/green/red}\\ 
\lspbottomrule
\end{tabular}
\caption{Color terms}\label{color}
\end{centering}
\end{table}
 
In order to distinguish the colors on the large scale of what is covered by \emph{phimna}, further modifications or comparisons can be made (see \Next).

\ex. \ag.sumphak loʔa=na phim=na\\
		leaf like{\sc =nmlz.sg} non-red{\sc =nmlz.sg}	\\
	\rede{as green as a leaf} 
 	\bg.besar loʔa=na phim=na\\
	turmeric like{\sc =nmlz.sg}	non-red{\sc =nmlz.sg}\\
	\rede{as yellow as turmeric}  
	 \bg.massi loʔa=na phim=na\\ 
	ink like{\sc =nmlz.sg} non-red{\sc =nmlz.sg}		\\ 
	\rede{as blue as ink}  

	
It is very likely that the bases of the \isi{color terms} are also verbs historically. \citep[292]{Doornenbal2009A-grammar} mentions a verb \emph{makma} \rede{be dark}  for \ili{Bantawa}, which must be cognate to \emph{makhurna} \rede{black} in Yakkha. Yakkha has a verbal stem \emph{phut} referring to the process of becoming white, which has only been found in connection with hair so far. The syllables \emph{-lik} and \emph{-lek} occuring in the derivations are also known as lexical \isi{diminutives} and from the derivation of adverbs. In addition to \isi{color terms}, there are the lexemes \emph{om(na)} \rede{bright, light}, \emph{kuyum(na)} \rede{dark} and \emph{chyaŋchyaŋ(na)} \rede{transparent}.

	
\subsection{Adjectives in attributive and in predicative function}\label{sec-adj-str}

In attributive function, the \isi{adjectives} always appear in their nominalized form (i.e., as relative clauses), apart from  the few exceptions mentioned above. 

\ex. \ag. su=ha cuwa\\
	be\_sour{\sc =nmlz.nc} beer\\
	\rede{sour beer}
\bg.lag=ha nasa=ci\\
		be\_salty{\sc =nmlz.nsg} fish{\sc =nsg}	\\
	\rede{the salty fish}
	
In \isi{predicative function} in copular clauses, some \isi{adjectives} may appear simply in non-nominalized  form. Compare the adnominal and predicative functions of \emph{cancan} \rede{high} and  \emph{ucun} \rede{good/nice} in \Next. 

\ex. \ag.nna   cancan=na luŋkhwak\\
		that high{\sc =nmlz.sg} rock	\\
	\rede{this high rock} \source{38\_nrr\_07.044}
 	\bg.nna  dewan-ɖhuŋga baŋna luŋkhwak sahro cancan sa-ma=na.\\
	that Dewan-stone called rock very high {\sc cop.pst[3sg]-prf=nmlz.sg}	\\
	\rede{That rock called Dewan stone was very high.} \source{38\_nrr\_07.039}
\bg. ucun=na paŋ\\ 
	good{\sc =nmlz.sg} house\\ 
	\rede{a nice house} 
	 \bg.purba patti dailo yuŋ-ma=niŋ  ucun n-leŋ-me-n.\\
	east side door put-{\sc inf=ctmp} good {\sc neg-}become{\sc [3sg]-npst-neg}\\
	\rede{If they (the Linkha clan members) put the door to the east, it will not be good.} \source{11\_nrr\_01.016}
	
Other \isi{adjectives} have to appear in nominalized form in the copular predicate, too. The nominalizers  cannot  be omitted in \Next. While the base \emph{mi} from \Next[a] is attested independently as a \isi{degree} particle \rede{a little}, the base \emph{heko} is not attested independently.

	\ex. \ag.nhaŋ=go        lambu=ca=le                 mi=na  leŋ-d-eʔ=na!\\
	and\_then{\sc =top} road{\sc =add=ctr} small{\sc =nmlz.sg}  become{\sc -V2.give-npst[3sg]=nmlz.sg}\\
	\rede{And then, the road, too, becomes narrow (unexpectedly)!} \source{28\_cvs\_04.011 }
	\bg.kaniŋ haksaŋ   heko=na         om.\\
	{\sc 1pl} {\sc compar} other{\sc =nmlz.sg} {\sc cop}\\
	\rede{He is different from us.} \source{21\_nrr\_04.009}
\bg.  uŋci=be=ca           niŋwa heko=na         leks-a=ha.\\
{\sc 3nsg=loc=add} mind other{\sc =nmlz.sg} become{\sc [3sg]-pst=nmlz.nc}\\
\rede{They also changed their mind.} \source{41\_leg\_09.068}

\section{Comparison, equation and degree}\label{sec-compar}
\subsection{Degree}

Adjectives can be modified by \isi{degree} adverbs like \emph{tuknuŋ} \rede{completely}, \emph{pyak} \rede{a lot}, \emph{mi/mimik/miyaŋ} \rede{a little}, a deictic series of \emph{khiŋ}, \emph{ŋkhiŋ} and \emph{hoŋkhiŋ} (\rede{this much}, \rede{that much}, \rede{as much as mentioned before}). Most of them are not restricted to \isi{adjectives}, but may also be used with nouns or verbs (see \sectref{sec-quant} for an overview). Furthermore, there are some \ili{Nepali} loans like \emph{sahro} or \emph{ekdam}, both best rendered as \rede{very}. In \Next[a], the interrogative \emph{ikhiŋ} \rede{how much} is used in an \isi{exclamative} utterance. 

\ex. \ag.pyak cancan, ikhiŋ   cancan!\\
		very high, how\_much high\\
	\rede{(It was) very high, how highǃ} \source{38\_nrr\_07.039}
 	\bg.uŋ=ci=go  miyaŋ mam=ha n-sa=ba.\\
	{\sc 3nsg=top} a\_little big{\sc =nmlz.nsg} {\sc 3pl-cop.pst=emph}\\
	\rede{They were a little older (than me).} \source{13\_cvs\_02.051}

	
There is no grammatical means to mark the excessive in Yakkha, which means that there is no regular way of stating that some property is beyond a certain tolerable measure, as expressed by the English particle \emph{too}. Excessiveness is expressed by the \isi{quantifiers} \emph{pyak} \rede{(very) much} or 
\emph{tuknuŋ} \rede{completely}, \emph{ibebe} \rede{(very/too) much} and consequently it is not possible in Yakkha to contrast \rede{very much} and \rede{too much}. Some \isi{adjectives} have lexicalized the notion of excessiveness, all from the  domain of taste: \emph{khikcok} \rede{quite bitter}, \emph{lakcok} \rede{quite salty}, \emph{limcok} \rede{quite sweet}. Although it is  always the same morpheme \emph{-cok} that is involved, it is restricted to a very small semantic domain (at least according to the current data set), and thus it lacks the productivity that would be expected of a grammatical marker.


\subsection{The equative}\label{sec-equ}

Equation is expressed by attaching the equative \isi{case} \emph{loʔa} \rede{like} to the standard of \isi{comparison} (see \Next). The marker \emph{-lo \ti lok \ti loʔ} is also known from Belhare as a \isi{comitative} and an adverbial \isi{clause linkage} marker \citep{Bickel1993Belhare} and as \rede{manner suffix} (deriving manner adverbs) from \ili{Bantawa}  \citep[299]{Doornenbal2009A-grammar}. In Yakkha, these functions are covered by the \isi{comitative} marker \emph{=nuŋ}. The \isi{equative } \emph{loʔa}  may also be employed in complement clauses and equative clauses (\rede{seem like [proposition]}, \rede{do as told/do as if [proposition]}). 

 \ex. \ag. gumthali loʔa\\ 
 swallow like\\
  \rede{like a swallow}
  \bg. anar loʔa et-u-ŋ=ha.\\ 
  pomegranate like perceive-{\sc 3.P.pst-1sg=nmlz.nsg}\\
  \rede{It seemed like pomegranate to me.} \source{19\_pea\_01.011}
  
If properties are compared, the same structure is employed (see \Next and \sectref{sec-color} for examples). The comparee may additionally be marked by an \isi{additive focus} marker.
  
  \ex. \ag.na loʔa nna=ca mãɖa.\\
this like that{\sc =add} big\\
  \rede{That one is as big as this one.}
  \bg.phuama chalumma loʔa keŋgeʔ=na.\\
  last\_born\_girl  second\_born\_girl  like tall{\sc =nmlz.sg}\\
  \rede{Phuama is as tall as Chalumma.}\footnote{Terms based on birth rank are commonly used to adress/refer to people, also outside the family context.}

 The following example shows that the resulting postpositional phrase may also be nominalized, yielding a headed relative clause in \Next[a], and a headless relative clause in \Next[b]. 
 
  \ex.\ag. lupluŋ loʔa=na       luŋdhaŋ=be\\
  den like{\sc =nmlz.sg} cave{\sc =loc}\\
  \rede{in a cave like a den} \source{22\_nrr\_05.095}
\bg.  u-ma              loʔa=na        sa=na=i.\\
{\sc 3sg.poss-}mother like{\sc =nmlz.sg} {\sc cop.pst[3sg]=nmlz.sg=emph}\\
\rede{It was like a female.} \source{19\_pea\_01.079}

 The comparee is often omitted in natural discourse. The following two examples, with the comparees expressed by demonstratives were found in a narrative \Next.  Since the comparees have a strong tendency to be  topical, they precede the standard of \isi{comparison}.

\ex. \ag.hau,  kha=go,      eŋ=ga              yapmi  loʔa=ha=ci=ca.\\
{\sc excla} these{\sc =top} {\sc 1incl.poss=gen} person like{\sc =nmlz.nsg=nsg=add}\\
\rede{Oh, these guys, they look like our people, too.} \source{22\_nrr\_05.044}
\bg.ŋkha=ci=go  kaniŋ=nuŋ   sahro toŋ-khuba   loʔa men=ha=ci.\\
  those{\sc =nsg=top} {\sc 1pl=com} very fit-{\sc nmlz} like {\sc neg.cop=nmlz.nsg=nsg}\\
  \rede{As for those (guys), they do not seem particularly similar to us!} \source{22\_nrr\_07.046}


\subsection{The comparative and the superlative}

The comparative and the \isi{superlative} are covered by a construction in which either \emph{haʔniŋ} or \emph{haksaŋ} have to be attached to the standard of \isi{comparison}, which is a noun or a pronoun in the majority of cases (see \Next). Both comparative markers can be used interchangeably. The parameter of \isi{comparison} does not receive any comparative marking; it appears in its basic form. Both markers have their origin in a converbal form (see also \sectref{postpos}).

\ex. \ag. heko=ha=ci=ga haʔniŋ  pharak\\
other{\sc =nmlz.nsg=nsg=gen} {\sc compar} different\\
\rede{different from the others people's (language)}
\bg. heko=ha nwak=ci haksaŋ miyaŋ alag [...] sa=na=bu.\\
	other{\sc nmlz.nsg} bird{\sc =nsg} {\sc compar}  a\_little different [...] {\sc cop.pst[3sg]=nmlz.sg=rep}\\
	\rede{He was a bit different from the other birds, they say.} \source{21\_nrr\_04.002}
%			\ex. \ag. ka nda haʔniŋ keŋgeʔ-ŋa=na\\
	%{\sc 1sg} {\sc 2sg} {\sc compar} tall{\sc -1sg=nmlz.sg}	\\
	%\rede{I am taller than you.}
	
Often, the parameter of \isi{comparison} is not expressed by an adjective, but by an inflected verb (see \Next). Not only stative or ingressive-stative verbs are possible, as \Next[b] with an embedded clause clearly shows.

\ex.\ag. ka uŋ haʔniŋ  tum-ŋa=na.\\
			{\sc 1sg} {\sc 3sg}  {\sc compar} be\_ripe-{\sc 1sg=nmlz.sg}	\\
	\rede{I am older than he is.}
	 \bg.  ka nda haʔniŋ lam-ma ya-me-ŋ=na.\\
			{\sc 1sg} {\sc 2sg}  {\sc compar}  walk{\sc -inf} be\_able{\sc -npst-1sg=nmlz.sg}\\
	\rede{I can walk (better/more) than you.} (Lit.: \rede{Compared to you, I can walk.})
	
The standard of \isi{comparison} may also be an \isi{adverb}, as in \Next. 

\exg. u-laŋ=ci encho haʔniŋ n-sas-a-ma.\\
		{\sc 3sg.poss}-leg{\sc =nsg} some\_time\_ago {\sc compar} {\sc 3pl-cop.pst-pst-prf}\\
	\rede{Her legs got stronger than last time.} (Lit.: \rede{They became (something), compared to the last time.})

In the \isi{superlative}, the standard of \isi{comparison} is always the exhaustive quantifier \emph{ghak} \rede{all} \Next. 

\ex. \ag. ghak haʔniŋ mi=na  mima\\
	all {\sc compar} small{\sc =nmlz.sg} mouse\\
\rede{the smallest mouse} \source{01\_leg\_07.003}
\bg. ghak haksaŋ tum=na  paŋ\\
all  {\sc compar} old{\sc =nmlz.sg} house\\
\rede{the oldest house} \source{27\_nrr\_06.039}


\section{Adverbs}\label{adv}

Adverbs cover a wide range of functions, from grounding an event in time and space to specifying its manner, intensity, cause and other characteristics of an event. Adverbs in Yakkha can be grouped as follows:

\begin{itemize}
\item manner adverbs derived by the \isi{comitative} \emph{=nuŋ}
\item temporal adverbs, mostly derived by the \isi{clause linkage} marker  \emph{=niŋ} 
\item adverbs originating from obligatorily possessed nouns
\item adverbs derived by \emph{-lik \ti -lek \ti -rik}
\item marginal derivations by \emph{-lleŋ} and \emph{-ci(k)}
\item non-derived adverbs
\item adverbs based on \isi{reduplication}, \isi{triplication} and ideophones (\sectref{redup})
\item adverbs used in spatial orientation, most of them embedded in a system of topography-based orientation (see \sectref{geodeixis})
\end{itemize}

The most common base for these derivations are verbal roots (most of them attested synchronically), but other bases, such  as \isi{demonstratives}, are possible as well. Some bases do not exist as independent words, so  that their word class and independent semantics cannot be reliably established. 

\subsection{Manner adverbs derived by the comitative \emph{=nuŋ}}
 
The major strategy to derive manner adverbs is attaching the \isi{comitative} \isi{case} \isi{clitic} \emph{=nuŋ} to roots of verbs with stative or ingressive-stative semantics (commonly both, which is evident from their interaction with tense-\isi{aspect} morphology). The functions of the \isi{comitative} marker range from nominal \isi{case} marking to marking subordinate clauses, so that this type of \isi{adverb} is strictly speaking a minimal adverbial clause.

\tabref{adv-nung} provides some examples of this adverbial derivation. The same roots can be turned  into \isi{adjectives} via the nominalizers \emph{=na} and \emph{=ha} (see \Next, further examples in \sectref{adj-kinds}).\footnote{Other Kiranti languages, e.g., \ili{Bantawa}, Athpare, Chamling and Belhare, use the manner suffix \emph{-loʔ} for the derivation of manner adverbs, which is also known as \isi{comitative} \isi{case} marker in some of them, e.g., in Belhare \citep[549]{Bickel2003Belhare} and in Athpare \citep[81]{Ebert1994The-structure}. The cognate form in Yakkha has developed into an equative postposition. The only \isi{adverb} derived by \emph{loʔa} in Yakkha is \emph{pyakloʔa} \rede{usually}, etymologically \rede{like many/like much}.}  One \isi{adverb} that was derived by the \isi{comitative}, namely \emph{tuknuŋ} (hurt={\sc com}) has further developed into a \isi{degree} marker with the meaning \rede{completely}. 
 
 
\begin{table}[htp]
\begin{centering}
\begin{tabular}{ll}
\lsptoprule
{\sc verbal root}&{\sc adverb} \\
\midrule
\emph{chak} \rede{be/get hard/difficult}&\emph{chaknuŋ} \rede{hard, difficult}\\
\emph{cis} \rede{be/get cold}&\emph{cinuŋ} \rede{feeling cold}\\
\emph{khikt} \rede{be/get bitter}&\emph{khiknuŋ} \rede{tasting bitter}\\
\emph{li} \rede{be/get heavy}&\emph{linuŋ} \rede{heavily}\\
\emph{limd} \rede{be/get sweet}&\emph{limnuŋ} \rede{tasting sweet}\\
\emph{lakt} \rede{be/get salty}&\emph{laknuŋ} \rede{tasting salty}\\
\emph{nek} \rede{be/get soft}&\emph{neknuŋ} \rede{softly, gently}\\
\emph{nu} \rede{be/get well}&\emph{nunuŋ} \rede{well, healthy}\\
\emph{tuk} \rede{hurt}&\emph{tuknuŋ} \rede{painfully} \ti\\
& \rede{completely}\\
\lspbottomrule
\end{tabular}
\caption{Manner adverbs derived by \emph{=nuŋ}}\label{adv-nung}
\end{centering}
\end{table}


\ex. \ag.khuŋ-kheʔ-ma=niŋa               li-nuŋ=ca        n-leŋ-me-n.\\
		 carry\_on\_back{\sc -V2.carry.off-inf=ctmp} be\_heavy{\sc =com=add} {\sc neg-}become{\sc [3sg]-npst-neg}\\
	\rede{It will not get heavy when we carry it, too.} \source{01\_leg\_07.044}
	\bg. li=na babu\\
	be\_heavy{\sc =nmlz.sg} boy\\
	\rede{a/the heavy boy}

\subsection{Temporal adverbs}

Many of the temporal adverbs, including  the interrogative  \emph{hetniŋ \ti heʔniŋ} \rede{when} involve the particle \emph{=niŋ}, which is also found as a \isi{clause linkage} marker for contemporal events. In contrast to the manner adverbs, the base for temporal adverbs is not verbal. Some roots are adverbs by themselves, some are \isi{demonstratives}. The deictic roots \emph{nam}, \emph{chim} and \emph{khop}, denoting distances counted in years (with the utterance context as zero point), do not occur independently. In these adverbs, \emph{=niŋ} is employed for past reference, while for future, the same roots end in  \emph{-ma}, e.g., \emph{namma} \rede{next year}, \emph{chimma} \rede{two years later}. \tabref{adv-ning} provides an overview of the temporal adverbs.
 
%The marker\emph{=niŋ} also derives the caritive postposition \emph{maʔniŋ} \rede{without} from the negative \isi{copula} \emph{man}, literally meaning \rede{while not existing}. 

\begin{table} 
\begin{centering}
\begin{tabular}{ll}
\lsptoprule
{\sc adverb}&{\sc gloss} \\
\midrule
\emph{heʔniŋ} &\rede{when}\\
\emph{asenniŋ} &\rede{(during) yesterday} \\
\emph{enchoʔniŋ} &\rede{on the day before yesterday}\\
	&	 \rede{recently}\\
\emph{onchoʔniŋ} &\rede{long time ago}\\
\emph{khaʔniŋ} &\rede{this time}\\
\emph{ŋkhaʔniŋ} &\rede{that time}\\
\emph{hoŋkhaʔniŋ} &\rede{right at that time}\\
\emph{heniŋ} &\rede{(during) this year}\\
\emph{namniŋ} &\rede{last year}\\
\emph{chimniŋ} &\rede{two years ago}\\
\emph{khopniŋ} &\rede{three years ago}\\
\emph{namniŋ-chimniŋ} &\rede{some years ago}\\
\lspbottomrule
\end{tabular}
\caption{Temporal adverbs derived by \emph{=niŋ}}\label{adv-ning}
\end{centering}
\end{table}

Other temporal  adverbs count the days before(i.e., in the past)  or ahead (i.e., in the future) of the point of speaking. They are listed in \tabref{adv-days} below, together with further temporal adverbs. Note that not  all of them necessarily have the time of speaking as their point of reference. For instance, \emph{wandikŋa} can mean \rede{tomorrow} or \rede{next day}. Two temporal adverbs can be compounded, yielding terms with less specific reference.

\begin{table} 
\begin{centering}
\begin{tabular}{ll}
\lsptoprule
{\sc adverb}&{\sc gloss} \\
\midrule
\emph{wandik-ucumphak} &\rede{some days/time ahead}\\
\emph{okomphak} &\rede{two days after tomorrow}\\
\emph{ucumphak} &\rede{the day after tomorrow}\\
\emph{wandikŋa} &\rede{tomorrow, next day}\\
\emph{hen-wandik} &\rede{these days}\\
\emph{hensen} &\rede{nowadays}\\
\emph{hen} &\rede{today}\\
\emph{wandik} &\rede{later}\\
\emph{lop} &\rede{now}\\
\emph{khem} &\rede{shortly before}\\
\emph{asen} &\rede{yesterday}\\
\emph{encho} \ti &\rede{day before yesterday}\\
\emph{achupalen} &\\
\emph{asenlek} &\rede{some days ago}\\
\emph{asen-encho} &\rede{some time ago}\\
\lspbottomrule
\end{tabular}
\caption{Further temporal adverbs}\label{adv-days}
\end{centering}
\end{table}


\subsection{Adverbs based on obligatorily possessed nouns}

A completely different etymological source for adverbs (and a few \isi{adjectives}) are obligatorily possessed nouns. The possessive prefix can show agreement with the subject of the verb that is modified by the \isi{adverb}, as in \Next, but mostly, the third person form is used. The shift from a noun to an \isi{adverb} is evident from the fact that these words do not have any nominal properties other than taking the possessive prefix. Further nominal modification or \isi{case} and \isi{number} marking, for instance, are not possible, and they are not arguments of the verbs; one would expect agreement morphology if this was the \isi{case}. \tabref{adv-poss} shows some examples. To my knowledge, similar lexicalizations have not been described for other Kiranti languages, except for a few examples from  Belhare mentioned by \citet[563]{Bickel2003Belhare}, who e.g., provides  a cognate to \emph{ochoŋna} \rede{new}. In \emph{uhiŋgilik} \rede{alive}, not a noun, but a  verb \emph{hiŋma} \rede{survive} was the base for the derivation process, and the possessive prefix was probably added later, in analogy to the other adverbs.

 \ex. \ag. a-tokhumak  yep-ma n-ya-me-ŋa-n=na.\\
 {\sc 1sg.poss}-alone    stand-{\sc inf} {\sc neg-}be\_able-{\sc npst-1sg-neg=nmlz.sg}\\
 \rede{I cannot stand alone.}\source{27\_nrr\_06.017}
 \bg. o-tokhumak nin-ca-meʔ=na.\\
  {\sc 3sg.poss}-alone    cook-{\sc V2.eat-npst[3sg]=nmlz.sg}\\
 \rede{He cooks and eats alone.} 
 \bg.eh,    na   nniŋ=ga piccha=go     u-hiŋgilik wet=na, haku=ca        tups-wa-m-ga=na.\\
 oh this {\sc 2pl=gen} child{\sc =top} {\sc 3sg.poss-}alive exist{\sc [3sg]=nmlz.sg} now{\sc =add} meet{\sc-npst-2pl.A-2=nmlz.sg}\\
 \rede{Oh, your child is alive, you will meet her again.} \source{22\_nrr\_05.087}
 \bg.  lambu o-tesraŋ ikt-wa-m=na.\\
road  {\sc 3sg.poss-}opposite chase{\sc -npst-1pl.A=nmlz.sg} \\
 \rede{We follow the road in the opposite direction (i.e., we run in the wrong direction).} \source{28\_cvs\_04.024}
 
 

\begin{table}
\begin{centering}
\begin{tabular}{ll}
\lsptoprule
{\sc \isi{adverb}/adjective}& {\sc gloss}\\
\midrule
\emph{uhiŋgilik} &\rede{alive}\\
\emph{ollobak}& \rede{almost}\\ % ? diff. stress
\emph{otokhumak} &\rede{alone}\\
\emph{ohoppalik}& \rede{empty}\\ 
%\emph{upuŋge}& \rede{freely}\\ 
\emph{ochoŋ} & \rede{new}\\
\emph{ulippa}& \rede{old}\\
\emph{oleʔwa}& \rede{raw, unripe}\\
\emph{otesraŋ}& \rede{reversed}\\
\emph{uimalaŋ} &\rede{steeply down}\\
\emph{uthamalaŋ}& \rede{steeply up}\\
%\emph{otheklup} &\rede{half}\\ 
\lspbottomrule
\end{tabular}
\end{centering}
\caption{Adverbs and \isi{adjectives} originating in obligatorily possessed nouns}\label{adv-poss}
\end{table}

    
\subsection{Adverbs derived by  \emph{-lik \ti -lek}}

Another marker that is frequently found  in adverbs (and in some \isi{adjectives}) is the lexical diminutive \emph{-lik \ti lek} (occasionally also \emph{-rik \ti \mbox{-rek}}), as shown in \tabref{lik}. It is also used in the derivation of \isi{lexical nouns} that are characterized by their small size (see \sectref{lex-noun-2}). Cognates of this marker exist in other Kiranti languages, e.g., \emph{-let} in Athpare \cite{Ebert1997A-grammar} and \emph{-cilet} in Belhare  \citep{Bickel2003Belhare}. All of these adverbs have verbal stems as their base, and often the resulting adverbs occur with just these verbs, thus merely adding  emphasis to the result of the verbal action, such as \emph{iplik} \rede{(properly) twisted}. Some forms in the table may also occur reduplicated. One ideophonic \isi{adverb} ending in \emph{-lek} was found, too: \emph{piciŋgelek}, imitating a high-pitched voice, like the calls of eagles or owls. Some examples can be found below in \Next.

%Athpare and \ili{Puma} have a diminutive \emph{let}, which supports this assumption (cf. \cite{Ebert1997A-grammar, Bickeletal2006The-Chintang}). 


\begin{table}
\begin{tabularx}{\textwidth}{llX}
\lsptoprule
{\sc adverb} & {\sc gloss}&{\sc verbal root}  \\%[1.5em]
\midrule
\emph{cicaŋgalik{\sc -redup}} &\rede{tumbling, overturning}&\emph{caks}   \rede{overturn}\\%[1.5em] %\newline (in somersaults, bulky objects)
\emph{hiklik} &\rede{turned around, upside down}&\emph{hiks}  \rede{turn}\\%[1.5em] 
%\emph{hiŋ} \rede{survive}&\emph{uhiŋgilik} &\rede{alive} \\[1.5em] 
\emph{iplik{\sc -redup}} &\rede{properly [twisted]}&\emph{ipt}  \rede{twist, wring}\\%[1.5em] 
\emph{kakkulik{\sc -redup}} &\rede{tumbling or {rolling down}}&\emph{kaks}   \rede{fall}\\%[1.5em] % \newline (round objects, smooth movement) 
\emph{pektuŋgulik} &\rede{[folded] properly, many times}&\emph{pekt}   \rede{fold}\\%[1.5em] 
\emph{phoplek} &\rede{[pouring out]  at once} &\emph{phopt}   \rede{spill, pour}\\%[1.5em] 
\emph{siklik} &\rede{[dying] at once}&\emph{si}   \rede{die}\\%[1.5em] 
\emph{sontrik} &\rede{[manner of] sliding, falling}&\emph{sos}   \rede{lie slanted}\\%[1.5em] 
\emph{wakurik} &\rede{bent, crooked}&\emph{wakt}   \rede{bend forcefully} \\%[1.5em] 
\emph{hobrek} &\rede{[rotten] completely}&\emph{hop}  \rede{rot} \\%[1.5em] 
\lspbottomrule
\end{tabularx}
\caption{Adverbs derived by \emph{-lik} (and allomorphs)}\label{lik}
\end{table}

\ex. \ag. maŋcwa phoplek  lept-haks-u.\\
		water at\_once throw{\sc -V2.send-3.P[imp]}\\
	\rede{Pour out the water at once.}
 	\bg.pektuŋgulik pekt-u=hoŋ u-lum=be kaici=ŋa yub-haks-u=na.\\
	properly\_folded fold{\sc -3.P[pst]=seq} {\sc 3sg.poss-}middle{\sc =loc} scissors{\sc =ins} cut{\sc -V2.send-3.P[pst]=nmlz.sg}		\\
	\rede{He folded it properly and cut it through in the middle with scissors.} \source{Cut and Break Clips \citep{Bohnemeyeretal2010_cut}}
	

\subsection{Marginal derivations}

Two further derivations were found, but each only with a handful of lexemes. One derivation creates adverbs based on verbal roots and a suffix \emph{-ci(k)},\footnote{Closing open syllables by /k/ is common in Yakkha and also known from the treatment of \ili{Nepali} loans see \sectref{loansphon}.} and a \isi{reduplication} of this complex of root and suffix. Three such adverbs were found, all from the semantic domain of experience:   \emph{hapcik-hapcik} \rede{whinily, weepily}, \emph{chemci-chemci} \rede{jokingly, tea\-sing\-ly}, \emph{yunci-yunci} \rede{smilingly}.

Another morpheme that is occasionally found  in adverbs is \emph{-lleŋ}. The currently known forms are: \emph{cilleŋ} \rede{lying on back}, \emph{walleŋ} \rede{lying on the front}, and \emph{cilleŋ-kholleŋ} \rede{rocking, swaying} (like a bus on a bad road or a boat in a storm).
There is a  directional \isi{case} marker \emph{-leŋ} in Belhare (\citealt{Bickel2003Belhare}; the notion expressed by \emph{khaʔla} in Yakkha), and thus it is very likely that this derivation has the same source, although such a marker does not exist in Yakkha synchronically.

\subsection{Non-derived adverbs}

Finally, there are also a few adverbs that have no transparent etymology, such as \emph{hani} \rede{fast}, \emph{swak} \rede{secretly}, \emph{tamba} \rede{slowly},\footnote{The final \isi{syllable} \emph{-ba} is a \isi{nominalizer}, but the origin of the stem \emph{tam} is not known.} \emph{pakha} \rede{outside} and \emph{sori} \rede{together}. Interestingly, these adverbs cannot be turned into \isi{adjectives} by nominalizing them; one could, for instance, not say *\emph{soriha yapmici} \rede{the people who are together}.

\section{Reduplication, triplication and ideophones}\label{redup}

Rhyming patterns as well as ideophones are very common in Yakkha adverbs and \isi{adjectives}, and often both are combined. Since they are exceedingly rare in the other word classes, they can be taken as an indicator (albeit rather statistic than categorical) for adverb-hood or adjective-hood. The bases for \isi{reduplication} can be of verbal, adverbial or ideophonic nature. As always, there are some  bases with obscure origin, too. The bases for \isi{triplication} are always monosyllabic and lack independent meaning. Ideophonic adverbs are based on a similarity relation between their phonetic form and the concept they express. This is not necessarily a relation based on acoustic similarities (as in onomatopoeia); other senses such as sight, taste or smell can as well be involved in ideophonic expressions. Hence, the relation between signifier and signified is more iconic than in “core” lexemes, where the semantics and the phonological form are in an arbitrary relationship.

The phonological behavior of reduplicated/triplicated forms and that of ideophones often shows deviations from the core lexicon, such as peculiar stress patterns or unusual segments that do not occur in nouns or verbs of the language (such as /gh/ or /bh/ in Yakkha). This has already been noted for \ili{Bantawa} by \citet{Raietal1997Triplicated}, who label them paralexemes, relating the exceptional behavior of such forms to their emphatic or expressive function (expressing feelings or the attitude of the speaker). 

Reduplicated \isi{adjectives} and adverbs are always stressed on the second \isi{syllable} (\emph{can.ˈcan}). This suggests an analysis of \isi{reduplication} as a prefixation. Bisyllabic words are generally stressed on the first \isi{syllable} in Yakkha (cf. Chapter \ref{phon}), but since prefixes are not part of the stress domain in Yakkha, words consisting of a prefix and a monosyllabic stem are stressed on the second \isi{syllable}. Triplicated forms are always stressed on the last \isi{syllable}, which is exceptional for Yakkha \isi{stress assignment}. 

\subsection{Reduplication in adjectives}

The reduplicated \isi{adjectives} mostly relate to physical features like size, form or texture. Another group are \isi{adjectives}  based on \isi{experiencer} verbs. The above-mentioned pattern of nominalization to indicate attributive or nominal usage (cf. \sectref{adj-kinds}) also holds for \isi{adjectives} derived by \isi{reduplication} (see \Next). 

\ex. \ag.u-yabuluʔa ikhiŋ jonjon=naǃ\\
		{\sc 3sg.poss}-lips how\_much elevated{\sc =nmlz.sg}\\
	\rede{How bulging his lips are!}
 	\bg.chainpur cancan=na=be waiʔ=na.\\
	Chainpur high{\sc =nmlz.sg=loc} exist{\sc [3sg]=nmlz.sg}		\\
	\rede{Chainpur is in a high (place).}  
		\bg.a-phok gaŋgaŋ leks-a=na.\\
	{\sc 1sg.poss}-stomach burstingly\_full  become{\sc [3sg]-pst=nmlz.sg}		\\
	\rede{My stomach is now full as a tick.}  

 \tabref{adj-red} shows the verbal roots serving as bases (as far as they can be reconstructed) and the corresponding reduplicated adjectival forms. Generally, post-nasal \isi{voicing} of unaspirated consonants applies, and is copied to the first \isi{syllable} to yield maximal identity between base and reduplicated \isi{syllable}. Thus, forms like \emph{bumbum} or \emph{jonjon} emerge, which are unusual from the perspective of Yakkha phonological rules, because they display voiced initial obstruents in a language that has largely lost the contrast between voiced and unvoiced obstruents. The only exception is \emph{cancan}, which retains its unvoiced obstruents, but the affricate behaves exceptional also in other lexemes with respect to the \isi{voicing} rule. With regard to the verbal bases, augmented stems (i.e., with a \textsc{cvc}-t structure) omit the augment /-t/ before reduplicating. Stems alternating between a \textsc{cvc}-s and a \textsc{cvn} structure (such as \emph{caks \ti caŋ}), generally choose the \textsc{cvn} stem form as base for the \isi{reduplication} (see  \sectref{stem} for \isi{stem formation}). If the base has \textsc{cvc} structure and the consonants have the same place of articulation, this does not result in gemination in the reduplicated form. Rather, the coda consonant is omitted in the first \isi{syllable} (e.g., \emph{pha.ˈphap}). Some of these \isi{adjectives} can be combined to yield further meanings, e.g., \emph{chekchek-boŋboŋ} (low-elevated) \rede{zig-zag, uneven}. 

\begin{table}
\begin{centering}
{\small
\begin{tabular}{lll}
\lsptoprule
{\sc verbal base}&{\sc adjective}& {\sc gloss}\\
\midrule
\emph{cand} \rede{rise up}&\emph{cancan} &\rede{tall, high} \\  
\emph{chekt} \rede{close}&\emph{chekchek} &\rede{deep, low, narrow}\\  
-- &\emph{chenchen} &\rede{lying}, \rede{sidesleeping}\\  
\emph{chiks \ti chiŋ} \rede{tighten, tie off}&\emph{chiŋchiŋ} &\rede{tight} \\  
\emph{chuks \ti chuŋ} \rede{be wrinkled}&\emph{chuŋchuŋ} &\rede{wrinkled}\\  
\emph{cos} \rede{push} &\emph{jonjon} &\rede{sticking out, bulging} \\  
-- &\emph{gaŋgaŋ} &\rede{[belly] full as a tick} \\  
\emph{hupt} \rede{tighten, unite}&\emph{hubhub} &\rede{buxom, compact}\\  
%\emph{keks \ti keŋ} \rede{ripen}&\emph{keŋgeʔ}&\rede{long, tall} \\  
\emph{kept} \rede{stick, glue}&\emph{kepkep} &\rede{concave, sticking to} \\  
%\emph{khopt} \rede{fit around}&\emph{khopkhop} &\rede{round, in wiggly lines} \\  
-- &\emph{lenlen} &\rede{horizontally huge, lying} \\  
%\emph{lukt} \rede{run, stumble}&\emph{lukluk} &\rede{short} \\ 
\emph{mopt} \rede{cover, close}&\emph{mopmop} &\rede{covered} \\  
-- &\emph{nepnep} &\rede{short in height} \\  
-- &\emph{pakpak} &\rede{hollow, bowl-shaped} \\  
\emph{pekt} \rede{fold}&\emph{pekpek} &\rede{flat, thin, folded} \\  
\emph{phaps \ti pham} \rede{entangle}&\emph{phaphap} &\rede{[hair] entangled, scraggy} \\  
\emph{phopt} \rede{spill, turn over}&\emph{phophop} &\rede{face-down, overturned} \\  
\emph{pok} \rede{get up, rise}&\emph{pokpok} &\rede{in heaps, sticking out} \\  
\emph{poks \ti poŋ} \rede{explode}&\emph{boŋboŋ} &\rede{elevated, convex} \\  
\emph{pups \ti pum} \rede{tuck up, roll in fist}&\emph{bumbum} &\rede{[plastering a house] thickly}/ \\  
&&\rede{[body parts] swollen}/\\
 & &\rede{[teeth] sticky} \\  
\emph{pur} \rede{cut off, break off}&\emph{pupup} &\rede{chubby, short and fat} \\  
%- &\emph{sepsep} &\rede{thin, not healthy} \\  
\emph{sos} \rede{lie slanted}&\emph{sonson} &\rede{[sliding] horizontally}\\  
\emph{yok} \rede{search, look for}&\emph{yokyok} &\rede{carefully, balancing}\\  
\lspbottomrule
\end{tabular}
}
\caption{Adjectives derived by reduplication}\label{adj-red}
\end{centering}
\end{table}


Some \isi{adjectives} derived from experiential verbs  are shown in \tabref{adj-exp}. They always have causative semantics, as shown in \Next. Their bases are from those experiential verbs that code the \isi{experiencer} as possessor (cf. \sectref{exp}). These verbs consist of a noun (denoting a sensation or a body part) and a verb, often a \isi{motion} verb. The \isi{reduplication} only involves  the verbal stem of these compounds. In attributive position, they host the usual nominalizers \emph{=na} or \emph{=ha}. Since the stem \emph{keʔ} \rede{come up}, that is involved in many of these compounds, ends in a glottal stop, which never occurs word-finally in Yakkha, it is replaced by /k/ at the end of the word.

\begin{table}
\begin{tabular}{lll}
\lsptoprule
{\sc verbal base}&{\sc adjective}& {\sc gloss} \\
\midrule
\emph{lok-khot} \rede{get furious}&\emph{lok-khokhok} &\rede{causing fury} \\  
\emph{chik-ek} \rede{get angry/hateful}&\emph{chik-ekek} &\rede{causing anger/hate} \\  
\emph{hakamba-keʔ} \rede{yawn}&\emph{hakamba-kekek} &\rede{causing to yawn} \\  
\emph{luŋma-tukt} \rede{love}&\emph{luŋma-tuktuk} &\rede{loveable, pitiable} \\  
\emph{pomma-keʔ} \rede{get lazy}&\emph{pomma-kekek} &\rede{making lazy} \\  
\emph{yuncama-keʔ} \rede{have to laugh}&\emph{yuncama-kekek} &\rede{funny, ridiculous} \\  
\emph{chippa-keʔ} \rede{be disgusted}&\emph{chippa-kekek} &\rede{disgusting}\\  
\lspbottomrule
\end{tabular}
\caption{Adjectives derived from experiential verbs}\label{adj-exp}
\end{table}

\ex. \ag.batti chik-ʔekek leks-a=naǃ\\
electricity causing\_hate become{\sc [3sg]-pst=nmlz.sg}\\
\rede{The power cuts drive me mad already!}
\bg.hakamba-kekeʔ=na ceʔya\\
making\_yawn{\sc =nmlz.sg} matter\\
\rede{talk that makes me sleepy}

\subsection{Reduplication in adverbs}

\tabref{adv-red} shows adverbs derived by \isi{reduplication}. Their number is far lower than that of reduplicated \isi{adjectives}. The verbs that provide the base for the adverbs may occur together with the adverbs that are derived out of them, see e.g., \Next[a]. In such cases, it is hard to say what the semantic contribution made by the adverbs is, apart from emphasis. In the same example the \isi{adverb} also serves as base for a rhyme \emph{miŋmiŋ}, which adds further emphasis. For \emph{lumlum} \rede{loudly}, it is not quite clear whether it may also have an onomatopoeic component.

\begin{table}
\begin{tabular}{lll}
\lsptoprule
{\sc verbal base}&{\sc adverb}& {\sc gloss}\\
\midrule
\emph{cend} \rede{wake up}&\emph{cencen} &\rede{[sleeping] lightly}\\  
\emph{chups} \rede{gather}&\emph{chumchum} &\rede{gathered, economically, sparing}\\  
\emph{chuŋ} \rede{wrap, pack}&\emph{chuŋchuŋ} &\rede{sadly, sunken}\\  
\emph{lus} \rede{roar, deafen}&\emph{lumlum} &\rede{loudly, powerfully}\\
\emph{maks} \rede{wonder}&\emph{maŋmaŋ} &\rede{wondering}\\  
\emph{sips} \rede{twinkle, squint}&\emph{simsim} &\rede{squinting, blinking}\\  
%\emph{thak} \rede{open mouth/eyes}&\emph{thaŋthaŋ} &\rede{staring stupidly} (Dandagaun dialect)\\ % - Dandagaun
\lspbottomrule
\end{tabular}
\caption{Adverbs derived by \isi{reduplication} of verbal roots}\label{adv-red}
\end{table}


\ex.\ag.maŋmaŋ-miŋmiŋ m-maks-a-by-a-ma.\\
	wondering-{\sc rhyme} {\sc 3pl}-be\_surprised-{\sc pst-V2.give-pst-prf}\\
	\rede{They were utterly surprised.} \source{22\_nrr\_05.028}
 	\bg.lumlum  mokt-u-ga=iǃ\\
	loudly beat{\sc -imp[3.P]-2.A=foc}\\
	\rede{Beat (the drum) loudlyǃ}
	
	
Reduplication of independent adverbs (and \isi{adjectives}) is also possible, expressing intensity or iterativity (see \Next).\footnote{See \citet[304]{Doornenbal2009A-grammar} for a similar point on \ili{Bantawa} triplicated adverbs.} 
	
	\ex.\ag.sakhi iblik-iblik ipt-a=na.\\
	thread twisted{\sc -redup} twist{\sc -pst[3sg]=nmlz.sg}\\
	\rede{The thread is properly twisted.} 
	\bg.batti simik-simik hand-u=na.\\
	light blinking{\sc -redup} burn{\sc -3.P[pst]=nmlz.sg}\\
	\rede{The (electric) torch is blinking.}
	
	
	Some of the reduplicated adverbs add /e-/ to each component, without further change of meaning (see \tabref{adv-tab-2}). This is attested for Belhare, too, analyzed as marking extension \citep{Bickel1997Dictionary}.
	
\begin{table}
\begin{centering}
\begin{tabular}{ll}
\lsptoprule
{\sc verbal base}&{\sc adverb}\\
\midrule
\emph{ipt} \rede{twist, wring}&\emph{iblik-iblik} \rede{twisted}\\  
\emph{sips} \rede{close [eyes]}&\emph{simik-simik} \rede{blinking}\\  
\emph{khik} \rede{be bitter}&\emph{ekhik-ekhik} \rede{tasting bitter}\\
\emph{khumdu} \rede{tasty}&\emph{ekhumdu-ekhumdu} \rede{tasting good}\\
\emph{maŋdu} \rede{far}&\emph{emaŋdu-emaŋdu} \rede{far away}\\
-- &\emph{esap-esap} \rede{swiftly}\\
-- &\emph{elok-elok} \rede{from far away}\\
\lspbottomrule
\end{tabular}
\caption{Reduplication of adverbs}\label{adv-tab-2}
\end{centering}
\end{table}


\subsection{Triplication}\label{sec-trip}

Triplication patterns, similar to those found in \ili{Bantawa} and \ili{Chintang} (cf. \citealt{Rai1984A-descriptive, Raietal1997Triplicated, Raietal2005Triplication}) were also found in Yakkha (see \tabref{trip}). Triplicated forms in Yakkha differ from those in the two  languages mentioned above in three ways: 

\begin{itemize}
\item  they are not derived from stems that have an arbitrary, lexical, non-iconic meaning; most of them have an ideophonic component (i.e., an iconic relationship between the concept expressed and the phonological form)
\item  they never host the suffix \emph{-wa} (which is a property of \ili{Chintang} and \ili{Bantawa} triplicated adverbs)\footnote{The suffix is an adverbializer in these languages.}
\item they always change the initial consonant in the syllables of the rhyme, i.e., only the vowel of the base is retained
\end{itemize}


The \isi{triplication} pattern in Yakkha involves a \isi{syllable} CV (occasionally CV-ŋ) functioning as the base, and two suffixed syllables building a rhyme, changing the initial consonant to /r/, /l/, or (rarely) to /t/, /c/, /k/ or /b/. Occasionally, the syllables building the rhyme are closed by a velar stop or nasal, as in \emph{seleŋleŋ} or \emph{siliklik}. The vowel remains the same in all three syllables. This process has to be analyzed as \isi{triplication} and not simply as recursive \isi{reduplication}, because bisyllabic words such as \emph{huru} or \emph{phili} do not exist.\footnote{The same was found in \ili{Chintang} \citep{Raietal2005Triplication}, while in  \ili{Bantawa}, some forms may also appear with just one repeated \isi{syllable}, suggesting an analysis of \isi{triplication} as recursive \isi{reduplication} with the function of emphasis in \ili{Bantawa} \citep[304]{Doornenbal2009A-grammar}.} Triplicated adverbs show a divergent stress pattern; it is always the last \isi{syllable} that is stressed.


\begin{table}
\begin{tabular}{ll}
\lsptoprule
{\sc adverb}&{\sc gloss}\\
\midrule
\emph{bhututu} &\rede{farting sound}\\ 
\emph{gururu} &\rede{[coming] in flocks, continuously (e.g., at festivals)}\\ 
\emph{haŋcaŋcaŋ} &\rede{dangling}\\ 
\emph{hibibi} &\rede{[wind] blowing gently}\\ 
\emph{hururu} &\rede{[wind] blowing strongly} (also in \textsc{nep})\\ 
\emph{khiriri} &\rede{spinning, revolving}\\ 
\emph{lututu} &\rede{[dough, soup] being too thin} \\ 
\emph{pelele} &(i) \rede{pulling something heavy or blocked}\\ 
&(ii) \rede{[shawl, clothes] come undone}\\ 
\emph{phelele} &\rede{[bird flying] up high}\\ 
\emph{philili} &\rede{[butterfly] jittering}\\ 
\emph{phururu} &\rede{[manner of] strewing, dispersing}\\ 
\emph{pololo} &\rede{[bamboo, construction materials] being too long to handle}\\ 
\emph{pururu} &\rede{[flowing] in streams}\\ 
\emph{seleŋleŋ} &\rede{[wind] blowing strongly such that leaves start to rustle}\\ 
\emph{siliŋliŋ} &\rede{shaking}\\ 
\emph{siliklik} &\rede{fuming with anger}\\ 
\emph{serere} &\rede{[drizzling] thinly, [morning sunbeams] thinly}\\ 
\emph{sototo} &\rede{[walking, moving] one after the other}\\ 
\emph{thokokok} &\rede{shaking heavily [from fever, earthquake]}\\ 
\emph{tholoklok} &\rede{[boiling] vigorously}\\ 
\emph{tururu} &\rede{[blood, tears] flowing, dripping}\\ 
\emph{walaŋlaŋ} & \rede{bursting out in laughter}\\  	
\emph{yororo} & \rede{[fire wood heap, rice terrace] falling and tearing along}\\  	
\lspbottomrule
\end{tabular}
\caption{Adverbs involving triplication}\label{trip}
\end{table}

Some examples of triplicated adverbs are provided in \Next. As \Next[b] illustrates, \isi{adjectives} may be derived from these adverbs via the nominalizers \emph{=na} and \emph{=ha}.  

 \ex. \ag.o-heli tururu lond=ha.\\	
	{\sc 3sg.poss}-blood flowing	come\_out{\sc [pst]=nmlz.nsg}	\\
	\rede{He was bleeding profusely.}
	 \bg.hiwiwi=na hiʔwa\\
	blowing\_gently{\sc =nmlz.sg}	wind\\
	\rede{a gentle wind} 
	 \bg.ka caram=be khiriri is-a-ŋ=na.\\
		{\sc 1sg}	yard{\sc =loc} spinning 	revolve{\sc -pst-1sg=nmlz.sg}\\
	\rede{I was spinning around in the yard.} 
	 \bg. heko=na         whak=pe      a-tek              het-u=hoŋ              ka  haŋcaŋcaŋ chu-ya-ŋ.\\
	other{\sc =nmlz.sg} twig{\sc =loc} {\sc 1sg.poss-}clothes get\_stuck{\sc -3.P[pst]=seq} {\sc 1sg} dangling hang{\sc -pst-1sg}\\
	 \rede{My clothes got caught on another branch, and then I was dangling there.} \source{42\_leg\_10.032 }
	
\subsection{Ideophonic adverbs}\label{sec-ideophone}

Several adverbs have ideophonic quality, i.e., there exists an iconic relationship between their form and some \isi{aspect} of their meaning. The similarity relation may be based on sound as in onomatopoeia, but it may also be based on the visual, olfactory or haptic senses \citep{Caughley1997_Vowel}. \tabref{onomat} provides an overview; some examples from natural language are shown in  \Next. The adverbs that modify processes or activities have a reduplicated structure; only those that modify \isi{punctual} events do not occur in reduplicated form. The bases for the \isi{reduplication} can consist of up to three syllables. Ideophones often show some deviating behavior with regard to the general phonological outlook of a language. The same can be said about Yakkha ideophones. Initials such as /gʰ/ or /jʰ/ are not found beyond ideophones, and voiced initials like /b/ are rare, too.

%Some onomatopoeic adverbs are not based on reduplicated forms but on a doubled rhyme of a monosyllabic base. They are known under the term \rede{triplication}, and have also been described in other Kiranti languages such as \ili{Bantawa} and \ili{Chintang}  \citep{Raietal1997Triplicated, Raietal2005Triplication}. Although some of the triplicated adverbs have an ideophonic base, too, they are discussed in a separate section (\sectref{sec-trip}). 

\begin{table}
\begin{centering}
\resizebox{\linewidth}{!}{
\begin{tabular}{ll}
\lsptoprule
{\sc adverb}&{\sc gloss}\\
\midrule
\emph{boʔle-boʔle} & \rede{[manner of] stuttering, stammering}\\
\emph{chok} & \rede{suddenly [piercing]}\\
\emph{ebbebe} & \rede{trembling}\\
\emph{ghok-ghok} &\rede{pig grunts}\\
\emph{ghwa-ghwa} & \rede{bawling}\\
\emph{hesok-hesok} & \rede{[manner of] breathing with difficulty}\\
\emph{hobrok} & \rede{[falling, dropping] at once}\\  
\emph{hoŋghak-hoŋghak} & \rede{[walking] with sudden steps (like drunken people)}\\  
\emph{jhellek} & \rede{flashing}\\  
\emph{kai-kai} & \rede{[sound of] weeping}\\  
\emph{kerek-kerek} & \rede{chewing hard things (like bones)}\\  
\emph{khobak-khobak} & \rede{[manner of] crawling}\\  
\emph{khoblek} & \rede{[manner of] finishing the plate}\\  
\emph{khoʔluk-khoʔluk} & \rede{[sound of] coughing}\\  
\emph{kurum-kurum} & \rede{chewing hard, crunchy things (like chocolate)}\\  
\emph{kyaŋ-kyaŋ} & \rede{barking lightly}\\  
\emph{lak} & \rede{being dropped}\\  
\emph{oenk-oenk} & \rede{buffalo grunts}\\  
\emph{phorop-phorop} & \rede{[sound of] slurping (e.g., tea, soup)}\\  
\emph{phutruk-phutruk} & \rede{[manner of] jumping around}\\  
\emph{syaŋ} & \rede{[flying] like a rocket, by being thrown or shot}\\  
\emph{sukluk} & \rede{dozing off for a short moment (like in a boring meeting)}\\  
\emph{taŋpharaŋ-taŋpharaŋ} & \rede{staggering}\\  
\emph{thaʔyaŋ-thaʔyaŋ} & \rede{[manner of] walking with difficulty}\\  
\emph{thulum-thulum} & \rede{wobbling (like fat or breasts)}\\  
\emph{ʈhek} & \rede{[manner of] hitting lightly}\\  
\emph{ʈhwaŋ} & \rede{sudden bad smell}\\  
\emph{ʈuk-ʈuk} & \rede{[sitting] squatted, crouching}\\  
\emph{whaŋ-whaŋ} & \rede{[barking] loudly}\\
\emph{wop} & \rede{[manner of] slapping with full hand}\\
& \rede{(producing a deep, loud sound)}\\
\emph{yakcik-yakcik} & \rede{[sound of] squeezing, chewing (e.g., chewing gum)}\\  
\emph{yakpuruk-yakpuruk} & \rede{[sound of] squeezing (e.g., millet mash for beer)}\\  
\emph{yaŋgaŋ-yaŋgaŋ} & \rede{[manner of] toppling over (humans and objects)}\\  
\lspbottomrule
\end{tabular}
\caption{Ideophonic adverbs}\label{onomat}
}
\end{centering}
\end{table}


\ex. \ag.na picha khoʔluk-khoʔluk hot-a-s-heks-a=na.\\
		this child coughing-{\sc redup} cough-{\sc pst-V2.die-V2.cut-pst[3sg]=nmlz.sg}	\\
	\rede{This child is about to die, having a coughing fit.} 
 	\bg.u-laŋ men-da-le=na picha khobak-khobak lam-meʔ=na.\\
	{\sc 3sg.poss}-leg {\sc neg}-come-{\sc neg=nmlz.sg} child  crawling-{\sc redup} walk{\sc -npst[3sg]=nmlz.sg}		\\
	\rede{The child that cannot walk (yet) moves crawling.} 
 \bg. boʔle-boʔle ceŋ-meʔ=na.\\
 stammering talk{\sc -npst[3sg]=nmlz.sg}\\
 \rede{He is stammering.}
\bg.sukluk ips-a-khy-a=na.\\
dozing\_off sleep{\sc [3sg]-pst-V2.go-pst=nmlz.sg}\\
\rede{She dozed off.}
\bg. ka  ebbebe   kisit-a-ŋ  khoŋ    ghwa-ghwa     hab-a-ŋ.\\
{\sc 1sg} trembling be\_afraid{\sc -pst-1sg} so\_that bawling cry{\sc -pst-1sg}\\
\rede{I was scared, so that I bawled out loudly.} \source{42\_leg\_10.047}
\bg. uŋci=ga    sokma  ʈhwaŋ                 nam-ma.\\
 {\sc 3nsg=gen} breath smelling\_awfully smell{\sc [3]-prf}\\
\rede{Their breath smelled awfully.} \source{41\_leg\_09.045}



