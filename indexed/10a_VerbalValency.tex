%\begin{titlepage}
%\centering
%\vspace*{5em}
%\textbf{\huge A Grammar of Yakkha\\[1em]Volume II\\[2em]\Large Diana Schackow}
%\end{titlepage}

\chapter{Transitivity}\label{verb-val}

This chapter deals with \isi{argument structure}, \isi{valency alternations} and \isi{transitivity operations} in Yakkha. The term \isi{argument structure} is understood here as “the configuration of arguments that are governed by a particular lexical item" \citep[1130]{Haspelmath2004_Valency}. In \sectref{frames}, the  different verb frames of \isi{argument realization} are identified by the number of possible arguments and their \isi{case} and agreement properties. Several verbs occur in more than one frame; their alternations are treated in \sectref{valclass}.\footnote{Only predicates with nominal arguments are discussed in this chapter. For predicates taking clausal complements see Chapter \ref{compl}.} Apart from those alternations that result in straightforward classes, there are also \isi{transitivity operations} that are more productive and not related to certain verb classes (see \sectref{trans-op}). One has to distinguish between operations that change the \isi{argument structure} by adding or removing argument roles, and those that merely change the \isi{argument realization} by changing the \isi{case} or \isi{person marking} properties for an argument. 

Many markers of \isi{transitivity operations} have been verbs historically. They are also found as parts of lexically complex predicates, and the grammatical functions are often related to the lexical meanings of these markers. This multiplicity of functions can be viewed as a result of simultaneous \isi{grammaticalization} and \isi{lexicalization} processes of certain verbs (see also Chapter \ref{verb-verb}). 

\section{Frames of argument realization}\label{frames}

\subsection{Theoretical preliminaries}

Before starting with the description of the argument frames, some methodological and terminological remarks are in order. The argument frames are identified by three parameters:  agreement,  \isi{case} and the question of how the verbal semantics interact with these two formal means. For this purpose, generalized semantic roles (\textsc{gsr}s) are identified for each predicate. Following \citet{Bickel2010_Grammatical}, these roles are labelled as follows: \rede{A} stands for the most agent-like argument of a transitive predicate,\footnote{Yakkha does not exhibit differences between A arguments of two-argument and three-argument predicates, so that they do not have to be distinguished.} \rede{P} stands for the most patient-like argument of transitive predicates, \rede{S} stands for the sole argument of intransitive predicates. For \isi{three-argument verbs}, the most theme-like argument has the label \rede{T} and the most goal-like argument has the label \rede{G}.  By relying on generalized semantic roles to identify the arguments of a predicate, one does not imply that these roles build coherent  classes of arguments that are characterized by some common semantic or formal property. For example, locative-marked arguments can have the semantic role of a goal, a recipient, a location, a source or a stimulus. Crucially, \textsc{gsr}s make sense only in relation to the particular predicates or predicate classes. No further morphosyntactic consequences, e.g., pivots in some constructions, can be inferred from these terms, as different types of pivots may occur in Yakkha syntax (see Chapters \ref{ch-nmlz}, \ref{adv-cl} and \ref{compl}). The \isi{argument realization} does not always match with the semantic \isi{transitivity}, e.g., in transimpersonal verbs (see \sectref{tr-imp}). Nevertheless, a “standard” intransitive and a “standard” transitive frame could be identified, which are the most common frames of \isi{argument realization}. As arguments are frequently dropped in Yakkha, many examples in this chapter come from elicited data.

In the following, I will outline the two parameters of \isi{argument realization} in Yakkha, which are \isi{person marking} and \isi{case} marking. 
As for \isi{person marking}, Yakkha distinguishes intransitive and transitive inflectional paradigms (compare the marking of the verbs with regard to the role of the  argument \emph{kaniŋ} in example \Next). Thus, there are three possible values: arguments may trigger intransitive (subject) agreement, transitive subject agreement or object agreement (the latter two being indicated as A or P in the glosses).\footnote{For the function of the frequently occurring main clause nominalization see \sectref{nmlz-uni-3}. For the conditions of the nominative-\isi{ergative} syncretism see \sectref{case-erg}.} 

\ex.\ag.\label{ex-gointr}kaniŋ khe-i=ha.\\
{\sc 1pl} go{\sc -1pl[pst]=nmlz.nsg}\\
\rede{We went.} (S)
\bg.kaniŋ kei kheps-u-m=na.\\
{\sc 1pl[erg]} drum hear{\sc -3.P[pst]-1pl.A=nmlz.sg}\\
\rede{We heard the drum.} (A)
\bg.uŋci=ŋa kaniŋ kheps-a=ha.\\
{\sc 3nsg=erg} {\sc 1pl}  hear{\sc -pst[1.P]=nmlz.nsg}\\
\rede{They heard us.} (P)


The agreement markers are not uniformly aligned, so that, for the purposes of this chapter, \isi{person marking} is  presented as tripartite, i.e., agreement with S is different from agreement with A and also different from agreement with P.\footnote{The \isi{person marking} on the Yakkha verb combines accusative, \isi{ergative}, neutral and hierarchical \isi{alignment}, see Chapter \ref{verb-infl} on the verbal morphology.} The reader should bear in mind that the indications in the glosses (\rede{A, P}) refer only to the type of \isi{person marking}, following a common labelling tradition for languages where the verbs show agreement with more than one argument. These labels facilitate reading  the glosses, but they should not be conflated with the semantic roles of the verbal arguments, which can be \rede{S, A, P, T, G}.\footnote{To illustrate this with an example: the verb \emph{cimma} \rede{learn} is semantically transitive; it has an A argument (the learner) and a P argument (the thing learned, the knowledge acquired). The verb  is, however, inflected with intransitive morphology (triggered by the semantic A argument, the learner), thus behaving like the verb in  \ref{ex-gointr} with respect to \isi{person marking}.} 

Arguments can be marked with an \isi{ergative}, a \isi{nominative}, a \isi{genitive}, a \isi{locative}, an \isi{instrumental} and, albeit less commonly, with a \isi{comitative} or an \isi{ablative}. Yakkha has an \isi{ergative}/\isi{instrumental} syncretism; the \isi{case} marking does not distinguish between agent and instrument, but subsumes both roles under the umbrella category \rede{effector} \citep{VanValinetal1996The-case}. In the following sections, schematic diagrams will illustrate the mapping of the semantic roles to the \isi{case} and agreement properties for each argument frame. Altogether, 22 verb frames can be established. They can roughly be divided into intransitively inflected verbs, transitively inflected verbs, \isi{three-argument verbs}, experiencer-as-possessor predicates, copular verbs and light verbs.

In the schematic diagrams of the frames of \isi{argument realization},  capital letters stand for the respective \textsc{gsr}s of a predicate. Labels like \rede{{\sc erg}} indicate the \isi{case} marking. The agreement is indicated with \rede{s}, \rede{a} and \rede{o} (standing for intransitive subject person marking, transitive subject person marking and transitive object person marking, respectively), with the corresponding \textsc{gsr} following in square brackets.\footnote{The same notational convention is employed  e.g., in \citet{Schikowski2014_Flexible}.} 
 
\subsection{The standard intransitive frame}\label{stand-itr}

\noindent {\bf \{S-{\sc nom} V-s[S]\}}

\noindent 
In the standard \isi{intransitive frame}, the subject is in the unmarked \isi{nominative} \isi{case} (not written in the glosses) and triggers agreement on the verb. Verbs such as \emph{imma} \rede{sleep},  \emph{posiʔma} \rede{vomit} and \emph{numa} \rede{get well, recover} belong to this frame.

\ex. \ag. ka posit-a-ŋ=na.\\
		 {\sc 1sg} vomit-{\sc pst-1sg=nmlz.sg}	\\
	\rede{I vomited.}
 	\bg. nda nu-ga=na?\\
	{\sc 2sg}  get\_well{\sc [pst]-2=nmlz.sg}	\\
	\rede{Are you fine?}
	\bg. n-yag-a-sy-a-ga-n=na=i?\\ 
{\sc neg}-feel\_exhausted{\sc -pst-mddl-pst-2-neg=nmlz.sg=q}\\
\rede{Are you not exhausted?}

\subsection{The intransitive experiencer frame}\label{itr-exp}

\noindent {\bf \{A-{\sc nom} P-{\sc loc/ins/abl/com} V-s[A]\}}

\noindent 
Some \isi{experiencer} verbs allow the expression of overt stimulus arguments, despite being identical to  the standard \isi{intransitive frame} with respect to its person marking morphology. The stimulus can be marked by various peripheral cases like the \isi{ablative}, the \isi{locative}, the \isi{instrumental} and the \isi{comitative}, as illustrated by example \Next[a] and \Next[b].  These \isi{experiencer} verbs are typically etymologically complex (both Noun-Verb and Verb-Verb compounds), as they often have bisyllabic stems, and Kiranti languages, following a broader tendency in Southeast Asian languages, are typically characterized by monosyllabic morphemes \citep{Matisoff1990_Bulging}. Some verbs of this frame have metaphorical meaning: \rede{to be hungry} is expressed as in \Next[c], without the intention to exaggerate or to be ironic. 

\ex. \ag.  ka nda=bhaŋ/nda=nuŋ kisit-a-ŋ=na.\\
	{\sc 1sg} {\sc 2sg=abl/2sg=com} be\_afraid-{\sc pst-1sg=nmlz.sg}\\
	\rede{I was afraid of you.}
\bg. ka cokleʈ=pe kam-di-me-ŋ=na.\\
	{\sc 1sg} sweets{\sc =loc} pine\_over-{\sc V2.give-npst-1sg=nmlz.sg}\\
	\rede{I pine over sweets.} 	
\bg. sak=ŋa n-sy-a-ma-ŋa-n=na.\\
			hunger{\sc =ins} {\sc neg-}die{\sc -pst-prf-1sg-neg=nmlz.sg}\\
		\rede{I am not hungry.}
				
\subsection{The motion verb frame}\label{itr-motion}
\noindent {\bf \{A-{\sc nom} P-{\sc loc} V-s[A]\}}
 
\noindent 
Motion verbs are intransitively inflected, but they have two arguments, as they entail a mover (A) and the location or goal of the movement (P) in their conceptualization. This is also borne out by the natural language data: most of the \isi{motion verbs} express the location overtly, marked by a \isi{locative}. In a language that has generally more covert than overtly realized arguments, this can be counted as a strong indicator for the entailment of the \isi{locative} argument in the verbal semantics. The location or goal can be expressed by an \isi{adverb}, as in \Next[a], or by a \isi{noun phrase} (see \Next[b] - \Next[d]). 

\ex. \ag. kucuma heʔne khy-a=na?\\
		dog where go{\sc [3sg]-pst=nmlz.sg} 	\\
	\rede{Where did the dog go?}	
	\bg. koŋgu=be thaŋ-a=na.\\
		hill{\sc =loc} climb{\sc [3sg]-pst=nmlz.sg}\\
	\rede{He climbed on the hill.}
	\bg.saŋgoŋ=be yuŋ-a=na.\\
	 mat{\sc =loc} sit\_down{\sc [3sg]-pst=nmlz.sg}\\
	\rede{He sat down on the mat.}	
	\bg.  taŋkheŋ=be  pes-a-khy-a=niŋ, ...\\
	 sky{\sc =loc} fly{\sc [3sg]-pst-V2.go-pst=ctmp}		\\
	\rede{When he (the bird) flew into the sky, ...} \source{21\_nrr\_04.031}	


Under certain circumstances the \isi{locative} on the goal argument can be omitted, e.g., when the location is a specific place with a name, or if it is a place that one typically moves to, such as villages, countries, the school, the work place and the like (see \Next and Chapter \ref{case-nom}).\footnote{Example (a) refers to a marriage custom called \emph{bagdata}, see Chapter \ref{social}.}  Only unmodified nouns can appear without the \isi{locative}; if the reference of the noun is narrowed down and made definite, e.g., by a possessive or demonstrative pronoun, it has to take the \isi{locative} \isi{case} (see \Next[c]).   

\ex.\ag. liŋkha=ci=ga teʔma  bagdata nak-se mamliŋ ta-ya-ma.\\
	a\_clan{\sc =nsg=gen} clan\_sister finalization\_of\_marriage ask\_for{\sc -sup} Mamling  come{\sc [3sg]-pst-prf}\\
	\rede{A Linkha clan sister came to Mamling to ask for her \emph{bagdata} (ritual).} \source{37\_nrr\_07.002}
\bg. hiʔwa pes-a=na.\\
wind fly{\sc [3sg]-pst=nmlz.sg}\\\
\rede{He flew (up) into the air.} \source{21\_nrr\_004.051}
\bg. nna*(=be)=go imin thaŋ-ma?\\
that{\sc *(=loc)=top} how climb{\sc -inf} \\
\rede{But how to climb to that (place)?} \source{22\_nrr\_05.098}



\subsection{The standard monotransitive frame}\label{stand-tr}

\noindent {\bf \{A-{\sc erg} P-{\sc nom} V-a[A].o[P]\}}

\noindent
This frame characterizes the majority of the monotransitive verbs, such as \emph{nima} \rede{see} and  \emph{mokma} \rede{beat}. The verb shows agreement with both A and P. The A argument is  marked by an \isi{ergative} \isi{case} \emph{=ŋa} (see \Next[a]), except for first and second person pronouns, which exhibit an \isi{ergative}/\isi{nominative} syncretism (see also Chapter \ref{case-erg}).\footnote{Note that nouns with first and second person reference are possible in Yakkha (as if saying \rede{An old woman AM tired}; see also \sectref{flex-agr}). If they are A arguments of transitive verbs, they are marked by an \isi{ergative}, other than the first and second person pronouns.} The condition for the \isi{ergative}/\isi{nominative} syncretism is identical to this frame throughout all the transitively inflected frames. The P arguments are in the \isi{nominative} \isi{case}. 

 \ex. \ag.isa=ŋa chemha tuks-u=ha?\\
		who{\sc =erg} liquor spill{\sc -3.P[pst]=nmlz.nc}\\
		\rede{Who spilled the liquor?}
 \bg. ka iya=ca ŋ-kheps-u-ŋa-n=ha.\\
  {\sc 1sg[erg]} what{\sc =nmlz.nc=add}  {\sc neg}-hear-{\sc 3.P[pst]-1sg.A-neg=nmlz.nc}\\
 \rede{I did not hear anything.}
 
\subsection{The experiencer-as-object frame}\label{tr-objex}

\noindent {\bf \{A-{\sc nom} P-{\sc erg} V-a[P].o[A]\}}

\noindent
Experiential events often show deviations from the standard marking patterns of argument encoding (see, e.g.,  \citealt{Bhaskararao2004_Nonnominative} and  \citealt{Malchukov2008Split}). There is one frame in Yakkha that is identical to the standard \isi{monotransitive frame}, but the marking of A and P is reversed;  the \isi{experiencer} triggers object agreement on the verb, while the stimulus triggers subject agreement (zero for third person singular) and hosts the \isi{ergative} \isi{case} \isi{clitic}. Notwithstanding the non-canonical agreement and \isi{case} properties, the preferred \isi{constituent order} is A-P-verb, and constructions with an S/A pivot, for instance, select the \isi{experiencer}. The majority of the verbs belonging to this frame are related  to the ingestion of food or to the consumption of other supplies, illustrated in \Next.

\ex. \ag. ka macchi=ŋa haŋd-a-ŋ=na.\\
		{\sc 1sg} pickles{\sc =erg} taste\_spicy{\sc -pst-1sg.P=nmlz.sg}\\
		\rede{The pickles tasted hot to me.}\footnote{The Maithili loanword \emph{macchi} has developed several meanings in Yakkha, namely \rede{chili plant}, \rede{chili powder} and \rede{hot pickle or sauce}.}
	\bg. ka haŋha=ŋa khot-a-ŋ=na.\\	
		{\sc 1sg} hot\_spices{\sc =erg} have\_enough{\sc -pst-1sg.P=nmlz.sg}\\
		\rede{I have enough spice (in my food).}
	\bg. nasa=ga ŋai=ŋa khikt-a-ŋ=na.\\ 
		fish{\sc =gen} stomach{\sc =erg} taste\_bitter{\sc -pst-1sg.P=nmlz.sg}\\
	\rede{The fish stomach tasted bitter to me.}
	
	
Verbs that refer to being affected by natural or supernatural powers also follow the object-\isi{experiencer} frame, e.g., \emph{teʔnima} \rede{be possessed, suffer from evil spirit} in  \Next[a].  The verb \rede{be drunk} is expressed as shown in \Next[b]. The stem \emph{sis} literally also means \rede{kill} with an animate, intentional A argument, but as the example shows, metaphorical meanings are possible as well. Notably, in this predicate, the stimulus is often omitted; \emph{sis} has undergone a metaphorical extension towards the meaning of \rede{being drunk}.

\ex.\ag. puŋdaraŋma=ŋa  teps-y-uks-u=na.\\
forest\_goddess{\sc =erg} be\_possessed{\sc -prf-3.P[pst]=nmlz.sg}\\
\rede{He is possessed by the forest goddess.} 
\bg. (raksi=ŋa)  sis-a-ga=na=i?\\
		liquor{\sc =erg} kill{\sc -pst-2.P=nmlz.sg=q}\\
		\rede{Are you drunk?}


 
\subsection{The transimpersonal frame}\label{tr-imp}

\noindent {\bf \{S-{\sc nom} V-a[3].o[S]\}}

\noindent
The \isi{transimpersonal frame}  is similar to the object-\isi{experiencer} frame. The verbs inflect transitively, but there is no overt A argument, the verbs show default third person singular subject agreement (zero). The sole argument is in the \isi{nominative} and triggers  object agreement on the verb. Diachronically there probably was  an overt A, but the only remnant found synchronically is the agreement; all attempts at producing an overt A were regarded as ungrammatical. \citet{Malchukov2008Split} notes that such constructions tend to be \isi{experiencer} constructions crosslinguistically. In Yakkha, however, transimpersonal verbs are not \isi{experiencer} verbs, as the subjects of these verbs are not typically animate, sentient beings. Verbs belonging to this frame often have change-of-state semantics, e.g., \emph{cikma} \rede{ripen}, \emph{lokma} \rede{boil}, \emph{homma} \rede{swell}, \emph{huʔma} \rede{be blocked}, \emph{ŋomma} \rede{remain}, shown in \Next. 
 
\ex.\ag. a-nabhuk hut-u=na.\\
 {\sc 1sg.poss-}nose be\_blocked{\sc -3.P[pst]=nmlz.sg} \\
\rede{I have a blocked nose.} 
\bg. a-laŋ=ci homd-u-ci=ha.\\
 {\sc 1sg.poss-}leg{\sc =nsg} swell{\sc -3.P[pst]-nsg.P=nmlz.nsg} \\
\rede{My legs are swollen.}
 \bg. cama ŋond-u=ha.\\
rice remain{\sc -3.P[pst]=nmlz.nc}\\
 \rede{(Some) rice remained.}
\bg. cuwa cikt-u=ha.\\
beer ripen{\sc -3.P[pst]=nmlz.nc}\\
 \rede{The beer is well-fermented.}
 
 An agent or cause can only be expressed indirectly, via adverbial clauses such as in \Next. A transitive structure, with an overt A argument can be achieved by a causative derivation, as shown in \NNext[b] (see \Next[a] for the same verbal stem without a causative derivation). 
 
 \exg.tumbuk poks-a=niŋa ten lus-u=na.\\
gun explode{\sc [3sg]-pst=ctmp} village  deafen{\sc -3.P[pst]=nmlz.sg}\\
\rede{When the gun exploded, the village was deafened (by the noise).}


 
\ex.\ag. maŋcwa lokt-u=ha.\\
water boil{\sc -3.P[pst]=nmlz.nc}\\
\rede{The water boiled.} 
\bg. kamala=ŋa maŋcwa lok-met-wa=ha.\\
Kamala{\sc =erg} water boil{\sc -caus-npst[3.P]=nmlz.nc}\\
\rede{Kamala boils water.}


Transimpersonal verbs are a solid class in Kiranti languages, found e.g., in \ili{Limbu} \citep[451]{Driem1987A-grammar}, in \ili{Thulung} \citep[42]{Allen1975Sketch} and in \ili{Bantawa} \citep[222]{Doornenbal2009A-grammar}. In Yakkha, 29 transimpersonal verbs have been found so far.\footnote{The following transimpersonal verbs have been found so far in Yakkha: \emph{chamma} \rede{spread, increase}, \emph{cemma} \rede{get well, recover}, \emph{choma} \rede{tingle}, \emph{cikma} \rede{ripen},  \emph{cipma} \rede{rise} (only for water),  \emph{hekma} \rede{get stuck, choke}, \emph{heʔma} \rede{be entangled, hang, snag}, \emph{homma} \rede{swell}(stem: \emph{homd}), \emph{homma} \rede{fit into} (stem: \emph{hond}), \emph{huʔma} \rede{be blocked}, \emph{keŋma} \rede{bear fruit}, \emph{khakma} \rede{freeze}, \emph{khekt} \rede{freeze, harden}, \emph{khopma} \rede{fit around something}, \emph{leʔma} \rede{flourish, be prosperous},   \emph{lokma} \rede{boil},\emph{mopma} \rede{be clouded, be dull}, \emph{ŋomma} \rede{remain},   \emph{oʔma} \rede{hatch},  \emph{pheʔma} \rede{bloom}, \emph{phiŋma} \rede{get clear}, \emph{puʔma} \rede{spill, overboil},   \emph{sipma} \rede{evaporate}, 
 \emph{suncama} \rede{itch}, \emph{tapma} \rede{last long}, \emph{wemma} \rede{get intoxicated, be insolent}, \emph{yeŋma} \rede{be strong, be tough}.} 


%other grammars checked, found nothing in yamphu, athpare, dumi, belhare, kiranti-intro in STlang, structure of kiranti languages

\subsection{Marginally occurring frames}\label{tr-marg}
\subsubsection{The locative object frame}\label{tr-loc}

\noindent {\bf \{A-{\sc erg} P-{\sc loc} V-a[A].o[3]\}}

\noindent
One verb,  \emph{tama} \rede{arrive (at)}, differs from the standard \isi{monotransitive frame} in  marking the P argument with the \isi{locative} \isi{case}.  The object agreement slot is always filled by default third person object agreement: speech-act participants cannot be the objects of this verb. Rather, one would express  such content as \rede{arrive at your place}, with a third person object agreement.

\exg. lalubaŋ=nuŋ   phalubaŋ=ŋa   mamliŋ=be   tas-a-ma-c-u.\\
Lalubang{\sc =com} Phalubang{\sc =erg} Mamling{\sc =loc} arrive{\sc -pst-prf-du-3.P}\\
\rede{Lalubang and Phalubang have arrived in Mamling.} \source{22\_nrr\_05.041}
  
 

\subsubsection{The semi-transitive frame}\label{tr-semi}

\noindent {\bf \{S-{\sc erg} V-a[S].o[3]\}}

\noindent
In the \isi{semi-transitive frame}, the verb is transitively inflected and the sole argument receives \isi{ergative} marking, but overt objects are suppres\-sed. The verb shows default third person singular object agreement, as in \Next. The expression of the P (the excreted substance) is not just considered redundant, but unacceptable.\footnote{See  \citet[1480]{Li2007Splitergativity} for a similar class of verbs in \ili{Nepali}.} This frame is like the mirror-image of the trans-impersonal frame discussed above. So far, however, the verb \emph{oma} \rede{vomit} in \Next is the only member of this frame.

\exg. tug-a-by-a=na yapmi=ŋa os-u=ha.\\
 get\_sick{\sc -pst-V2.give-pst[3sg]=nmlz.sg} person{\sc =erg} vomit{\sc -3.P=nmlz.nc}\\
\rede{The sick person vomited.}  


\subsubsection{The double nominative frame}\label{itr-teach}

\noindent {\bf \{A-{\sc nom} P-{\sc nom} V-s[A]\}}

\noindent
This frame was found only for one verb, but it is listed for the sake of completeness. The verb \emph{cimma} \rede{learn} is inflected intransitively, although it takes two arguments. Both A and P are in the \isi{nominative}, and  A triggers the verbal \isi{person marking} (see  \Next[a]). With transitive agreement morphology the verb becomes the ditransitive verb \rede{teach} (see \Next[b]). Except for the additional argument, this alternation is identical to the labile alternation discussed in \sectref{labile}. 

\ex. \ag. hari iŋlis cind-a=na.\\
		Hari English learn{\sc [3sg]-pst=nmlz.sg}	\\
	\rede{Hari learned English.} 
 	\bg. kamala=ŋa hari iŋlis cind-u=na.\\
	Kamala{\sc =erg} Hari English teach{\sc -3.P[pst]=nmlz.sg}		\\
	\rede{Kamala taught Hari English.} 

\subsection{Three-argument verbs}\label{three-arg-frame}

The \isi{case} and agreement properties of the subjects of \isi{three-argument verbs} are not different from those of monotransitive verbs. The \isi{argument realization} of the T and G arguments, however, deserves a closer look. It is determined  by both semantic roles and the referential properties of the arguments.  The choice of the agreement triggering argument for the nominalizing clitics \emph{=na} and \emph{=ha} need not be the same as for the verbal agreement. The nominalizers are partly aligned according to the referential properties of the arguments and partly according to their role. This is discussed in detail in \sectref{verb-infl} and \sectref{nmlz-uni-3} and will not figure prominently in the following treatment of three-argument frames.
 
\subsubsection{The double object frame}

\noindent {\bf \{A-{\sc erg} G-{\sc nom} T-{\sc nom}  V-a[A].o[G]\}}

\noindent
In the \isi{double object frame}, both T and G arguments are in the \isi{nominative} \isi{case}. The verb agrees with the A and usually with the G argument, except for some pragmatically marked scenarios where T becomes the agreement trigger (see \sectref{three-arg}). The choice of the \isi{nominalizer} on the finite verb depends on T when T has third person reference: singular T triggers \emph{=na}, and nonsingular or non-countable T triggers \emph{=ha} (compare \Next[b] and \Next[c]). The verbs belonging to this frame are typically verbs of caused possession and benefactives (both derived and underived), and thus, the G arguments are typically animate in this frame.
 
\ex. \ag. ka  a-ni mendhwak hakt-wa-ŋ=na.\\
{\sc 1sg[erg]} {\sc 1sg.poss}-aunt goat send{\sc -npst[3.P]-1sg.A=nmlz.sg}\\
\rede{I send my aunt a goat.}
\bg. ka nda eko cokleʈ piʔ-nen=na.\\
		\sc{1sg[erg]} \sc{2sg} one sweet give\sc{[pst]-1>2=nmlz.sg}\\
		\rede{I gave you a sweet.}
	\bg. ka nda pyak cokleʈ piʔ-nen=ha.\\
	\sc{1sg[erg]} \sc{2sg} many sweet give\sc{[pst]-1>2=nmlz.nsg}\\
	\rede{I gave you many sweets.}

\subsubsection{The indirective frame}

\noindent {\bf \{A-{\sc erg}  G-{\sc loc/abl/com} T-{\sc nom} V-a[A].o[T]\}}

\noindent
The \isi{indirective frame} is more frequent than the \isi{double object frame}, i.e., there are more verbs that follow this frame. The G argument may have goal or source role and is marked by a \isi{locative} (see \Next) or, occasionally, by an \isi{ablative} or \isi{comitative} \isi{case} (see \NNext), while the T argument is in the \isi{nominative} and triggers object agreement on the verb (including the nominalizers). Mostly, caused \isi{motion} is  expressed by verbs of this frame. 

\ex. \ag. ka a-cya=ci iskul=be paks-wa-ŋ-ci-ŋ=ha.\\
		\sc{1sg[erg]} \textsc{1sg.poss}-child\sc{=nsg} school\sc{=loc} send-\sc{npst-1sg.A-nsg.P-1sg.A=nmlz.nsg}\\
	\rede{I send my children to school.}
		\bg. ak=ka khorek cula=ga u-yum=be yuks-uks-u-ŋ=na.\\
	\sc{1sg.poss=gen} bowl hearth\sc{=gen} \textsc{3sg.poss}-side\sc{=loc} put-\sc{prf-3.P[pst]-1sg.A=nmlz.sg}\\
	\rede{I have put my bowl next to the hearth.}
\bg.ama=ŋa a-nuncha netham=be nes-u=na.\\
mother{\sc =erg} \textsc{1sg.poss}-younger\_sibling  bed\sc{=loc} lay\sc{-3.P[pst]=nmlz.sg}\\
\rede{Mother laid my younger sister on the bed.}
	
 
\ex.\ag.khaʔniŋgo tuʔkhi leŋ-meʔ=niŋa  heko=ha=ci=nuŋ  yaŋ  naŋ-ca-ma ucun men.\\
but        trouble happen{\sc [3sg]-npst=ctmp} other{\sc=nmlz.nsg=nsg=com}  money beg{\sc-V2.eat-inf} nice {\sc neg.cop}\\
\rede{But in difficult times, it is not good to ask others for money.} \source{01\_leg\_07.257}
		\bg. haku nhaŋto    m-ba=nuŋ                nasa ŋ-in-wa-n-ci-ŋa-n=ha.\\
		now {\sc temp.abl} {\sc 2sg.poss-}father{\sc =com} fish {\sc neg-}buy{\sc -npst-neg-3nsg.P-excl-neg=nmlz.nsg}\\
		\rede{From now on I will not buy fish from your father.} \source{01\_leg\_07.208}

The \isi{locative} can also mark adjuncts, yielding clauses that look superficially identical to the \isi{indirective frame}. However, the adjuncts have to be distinguished from locative-marked arguments. In \Next, for instance, it is straightforward that the locative-marked noun phrases refer to  circumstances  such as time, place, manner, quantity \citep[108]{Tesniere1959Elements} and are thus adjuncts. The decision whether a participant is an argument or an adjunct is, however, not that trivial for all the predicates.

\ex.\ag. a-ppa=ŋa ka omphu=be nis-a-ŋ=na.\\
{\sc 1sg.poss-}father{\sc =erg} \sc{1sg} verandah{\sc =loc} see{\sc -pst-1sg.P=nmlz.sg}\\
\rede{Father saw me on the verandah.}
\bg. a-ma=ŋa tan=be tek akt-u=na.\\
{\sc 1sg.poss-}mother{\sc =erg} loom{\sc =loc} fabric weave{\sc -3.P[pst]=nmlz.sg}\\
\rede{Mother wove (a piece of) fabric on the loom.}
 
\subsubsection{The secundative frame}

\noindent {\bf \{A-{\sc erg}  G-{\sc nom} T-{\sc ins} V-a[A].o[G]\}}

\noindent
The verbs of the \isi{secundative frame} denote events of throwing, hitting, covering, applying, exchanging, events of creative or destructive impact. The T argument is marked by an \isi{instrumental} \isi{case}, but it is not always an instrument in the classical sense of “used by the agent to act on the patient” \citep[140]{Andrews1985The-major}, as \Next[a] shows. The G argument is  in the unmarked \isi{nominative} and triggers agreement on the verb (including the nominalizers). Some verbs of this frame may alternate with the \isi{indirective frame} (the \rede{spray-load alternation}, see \sectref{three-arg}).

\ex. \ag. ka cabak=ŋa paŋge lend-u-ŋ=ha.\\
		1sg rice{\sc =ins} millet exchange-{\sc 3.P[pst]-1sg.A=nmlz.nsg}	\\
	\rede{I exchanged rice for millet.}
	\bg. u-ppa=ŋa hammana=ŋa picha ept-u=na.\\
	{\sc 3sg.poss-}father{\sc =erg}  blanket{\sc =ins} child cover{\sc -3.P[pst]=nmlz.sg}\\
	\rede{The father covered his child with a blanket.}
\bg.eko phiswak=ŋa sum=ci ʈukra yub-u-ci=ha.\\
one knife{\sc =ins} three{\sc =nsg} piece cut{\sc -3.P[pst]-3nsg.P=nmlz.nsg}\\
\rede{He cut it into three pieces with a small knife.} (Cut-and-break clips, \citealt{Bohnemeyeretal2010_cut})

\subsection{The experiencer-as-possessor frame}\label{exp}

\noindent {\bf \{S-{\sc gen/nom}  {\sc poss-}N V-s[3]\} \ti }

\noindent {\bf \{A-{\sc gen/erg}  P-{\sc nom} {\sc poss-}N V-a[A].o[3]\}}

\noindent
Experiential predicates are characterized by the core participant being emotionally or sensationally affected by the event. This makes the thematic role \rede{experiencer} less agent-like, which is often reflected in the treatment of \isi{experiencer} arguments as non-prominent (“downgrading” in \citealt{Bickel2004The-syntax}), e.g., by non-canonical \isi{case} marking or by deviating agreement patterns (\citealt[22]{Levinetal2005_Argument}, \citealt[185]{Naess2007_Prototypical}). We have already seen a class of \isi{experiential predicates} in Yakkha that code their A arguments like standard objects. However, downgrading of an argument in one part of grammar, for instance, in \isi{case} marking, does not necessarily imply downgrading in other domains, for instance access to pivothood or reflexivization \citep[77]{Bickel2004The-syntax}.

Most experiential events in Yakkha, and generally in Kiranti languages, are expressed by complex predicates consisting of a noun and a verb, and the 
\isi{experiencer} (i.e., the A argument) is coded as the possessor of the noun (see \sectref{nv-comp-poss}).\footnote{See \citet{Bickel1997The-possessive} for Belhare.} 
The nouns that belong to such predicates denote sensations,  feelings,  character traits, moral qualities or affected body parts (hence, the term \emph{psych-noun}). 
Noun-verb compounds for the expression of experiential events are not unique to Yakkha or Kiranti languages; they belong to a broader Southeast Asian 
pattern \citep{Matisoff1986Hearts}. 

Morphosyntactically, the psych-noun hosts a possessive prefix that refers to the \isi{experiencer}. The noun may also trigger agreement on the verb (see \Next for examples). Some psych-nouns are conceptualized as nonsingular and thus trigger nonsingular verbal agreement. The predicates can be grouped into intransitively and transitively inflected verbs. Some verbs show alternations. The two schematic diagrams above only show the most common frames  of the experiencer-as-possessor predicates, corresponding to \Next[a] and \Next[b], respectively. In \Next[c], the stimulus triggers object agreement. For a  detailed description of the subframes and alternations see \sectref{poss-e1}.

\ex.\ag. ŋ=ga yupma(=ci) n-yus-a(=ci)?\\	
		{\sc 2sg.poss=gen} sleepiness{\sc (=nsg)} 	{\sc 3pl}-be\_full{\sc -pst(=nsg)}\\
		\rede{Are you tired?}
		\bg. ŋ-khaep cips-u-ga=na=i?\\
	{\sc 2sg.poss-}interest/wish complete{\sc -3.P[pst]-2.A=nmlz.nsg=q}\\
	\rede{Are you satisfied?} 
	\bg. nda ka ijaŋ n-lok khot-a-ŋ-ga=na?\\
	{\sc 2sg[erg]} {\sc 1sg} why {\sc 2sg.poss}-anger have\_enough{\sc -pst-1sg.P-2sg.A=nmlz.sg}\\
	\rede{Why are you mad at me?}


\subsection{Copular and light verb frames}\label{cop}

Copular clauses are different from the other clauses insofar as the predicate is not a verb but a nominal, adjectival or \isi{locative} constituent \citep[225]{Dryer2007Clause}. The constituents in copular clauses do not have semantic roles. Yakkha has two copular frames which can roughly be characterized as the identificational and the existential frame.\footnote{Such a two-way distinction in the copular frames is common in languages of the Himalayan region (see, e.g., \citealt{Genetti2007_Newari} on \ili{Newari}, and \citealt{Matthews1984Course} on \ili{Nepali}), but the exact distribution of the copular verbs probably differs from language to language.} While the equational frame is expressed by a copular verb (that is lacking an infinitival form) or by a copular particle \emph{om} that is not found elsewhere in simple clauses, the existential frame is expressed by two standard intransitive verbs: \emph{wama} \rede{be, live, exist} and \emph{leŋma} \rede{become, happen, come into being}. 

\subsubsection{Frame (a): Identification, equation,  class inclusion}

Two different forms participate in Frame (a): a copular verb that shows the expectable inflectional categories of person, \isi{number}, \isi{tense}/\isi{aspect} and \isi{polarity}, and a copular particle \emph{om} that can also refer to any person, but is not inflected, apart from the nonsingular marker \emph{=ci} (see \Next[c]). They are used to equate or identify two entities, and to state class inclusion (see \Next).  The forms of the copular verb are suppletive in the nonpast; in the past forms it has a stem \emph{sa} (see \sectref{cop-infl} on the morphology of the copulas). The copular verb and the particle are optional, and thus they are often omitted. The particle is also used as affirmative interjection \emph{om} \rede{yes}.  The domains of these two copular devices overlap, and in one instance they were found combined, too (see \Next[d], where this combination seems to yield emphasis).
 
 \ex.\ag. ka khasi ŋan.\\
	\sc{1sg} castrated\_goat  {\sc cop.1sg} \\
	\rede{I am a castrated goat.} \source{31\_mat\_01.074}
\bg. ka isa om?\\
{\sc 1sg} who {\sc cop}\\
\rede{Who am I?}
\bg.susma=nuŋ    suman  na           nuncha           om=ci.\\
Susma{\sc =com} Suman eZ yB {\sc cop=nsg}\\
\rede{Susma and Suman are (elder) sister and (younger) brother.} \source{01\_leg\_07.035}
\bg.ka  na   puŋda=ga    khuncakhuwa ŋan    om!\\
{\sc 1sg} this jungle{\sc =gen} thief  {\sc cop.1sg} {\sc cop}\\
\rede{I am the thief of this jungle!} \source{01\_leg\_07.335}


\subsubsection{Frame (b): Existence, attribution, location, possession}

Two verbs occur in Frame (b): the verb  \emph{wama}  \rede{be, live, exist} is a stative verb expressing existence.  Its stem shows irregular behavior: the basic stem form is \emph{waiʔ \ti waeʔ \ti weʔ}, and additionally \emph{ma} can be found in some negated forms (see \sectref{cop-infl} for the inflection of the copulas). This verb can occur in the \isi{motion verb frame}, expressing location or possession (see \Next[a]). It can also be used to express a property of the copular topic, with an adjective as the predicate (see \Next[b] and \Next[c]).

	\ex. \ag. ibebe    pyak encho         paŋ=ci    m-ma-ya-nin=ha.\\
somewhere much long\_time\_ago house{\sc =nsg} {\sc neg-}exist{\sc -pst-pl.neg=nmlz.nsg}\\
  \rede{Once upon a time there were no houses.}\footnote{The adverbial phrase \emph{ibebe} is a fixed expression that originates in  \emph{ibe-ibe} \rede{somewhere-somewhere}.} \source{27\_nrr\_06.001}
  \bg. bani man=na.\\
  habit exist{\sc .neg.npst[3sg]=nmlz.sg}\\
	\rede{There is no (such) habit.} 
	\bg.nna  cuʔlumphi haku=ca   ceŋgaceŋ       waeʔ=na.\\
	that stele now{\sc =add} straight\_upright be{\sc [3sg;npst]=nmlz.sg}\\
\rede{This stele stands straight upright even now.} \source{18\_nrr\_03.030}
\bg. piccha=go  uhiŋgilik weʔ=na.\\ 
child{\sc=top} alive exist{\sc npst[3sg]=nmlz.sg}	\\ 
\rede{But the child is alive.} \source{22\_nrr\_05.087}	


The second verb of frame (b) is the ingressive-\isi{phasal} verb \emph{leŋma} (stem: \emph{leks}) \rede{become, come into being, happen}, shown in \Next. Apart from this meaning it is also used to express non-permanent properties, as in \Next[c].\footnote{In (c), the property as such is of course permanent, but the subordinate clause puts the property in the perspective of a specific time.}

\ex.\ag. na=ga  suru   imin leks-a=na    baŋniŋ.\\
		this{\sc =gen} beginning how become{\sc [3sg]-pst=nmlz.sg} as.for\\
	\rede{As for how she came into being...} \source{14\_nrr\_02.002}
	\bg.  hoŋkaʔla  leks-a=hoŋ, ...\\
	like\_that become{\sc [3sg]-pst=seq}\\
\rede{As it became like that, ...} \source{11\_nrr\_01.019}
\bg.limlim lim=nuŋ leŋ-me.     khuŋ-kheʔ-ma=niŋa          li=nuŋ=ca    n-leŋ-me-n.\\
sweet sweet{\sc =com} become{\sc [3sg]-npst} carry{\sc -V2.carry.off-inf=ctmp} heavy{\sc =com=add} {\sc neg-}become{\sc [3sg]-npst-neg}\\
\rede{The sweet will be tasty. While carrying, it also will not be heavy.}\source{01\_leg\_07.044}


Example \Next from a pear story shows a nice minimal pair between the identificational and the existential \isi{copula} (\emph{sana} and \emph{waisa}). The \isi{identificational copula} only takes nominal predicates. The question word \emph{imin} \rede{how} is nominalized, and combined with the \isi{identificational copula}. The existence of the snow is, however, expressed by the existential  verb \emph{wama}.

\exg.i=na, la, ʈhoŋ=ca       imin=na       sa=na,        hiuŋ=le      wai-sa,  i=ya?\\
what{\sc =nmlz.sg}, {\sc filler}, place{\sc =add} how{\sc =nmlz.sg} {\sc cop.pst[3sg]=nmlz.sg} snow{\sc =ctr} exist{\sc -pst[3sg]} what{\sc =nmlz.nc}\\
\rede{What, well, what kind of place was it, there was snow, what was it?} (Context: the speaker is unsure, because she is trying to understand what happens in the pear story film. Her interpretation of the distorted quality of the footage is that it must be a snowy place.)	\source{19\_pea\_01.002}

\subsubsection{Light verbs}

 The \isi{light verb} strategy is commonly used to introduce \ili{Nepali} verbs or \isi{light verb} constructions into Yakkha. The construction is parallel to the \ili{Nepali} source construction, but the \ili{Nepali} light verbs are replaced by the Yakkha lexemes \emph{wama} \rede{exist, be} and \emph{cokma} \rede{do}. 
 
 The resulting structure also gets formally adjusted to the Yakkha morphosyntax. In \ili{Nepali}, some S/A arguments  (e.g., of knowledge and \isi{experiential predicates}) are marked by the dative \emph{-lāī} (the \ili{Nepali} translation of \Next[a] would be \emph{ma-lāī ali-ali thāhā cha}), and the verb shows third person agreement with the noun in the \isi{nominative}. But as there is no dative \isi{case} in Yakkha, the result of the calquing is a nominative-marked subject and a \isi{light verb} that triggers third person agreement. In \isi{light verb} constructions which are not calques from \isi{experiential predicates}, the verb agrees with the subject (S or A, see \Next[b]). Although overt P arguments are possible, as \emph{nam}  \rede{sun} in \Next[b], the light verbs found so far are always inflected intransitively, and A arguments in the \isi{ergative} \isi{case} were not found.

\ex. \ag. ka mimik   thaha       waeʔ=na.\\
		{\sc 1sg} a\_bit knowledge exist{\sc [3sg]=nmlz.sg}\\
	\rede{I know a little bit.} \source{13\_cvs\_02.022} 
	\bg. liŋkha=ci     nam=nuŋ      bʌgʌri n-jog-a.\\
	a\_clan{\sc =nsg} sun{\sc =com} bet {\sc 3pl-}do{\sc -pst}\\
	\rede{The Linkhas had a bet with the sun.} \source{11\_nrr\_01.003}

	
The same strategy  is also used for borrowing \ili{Nepali} verbal stems into Yakkha (see \Next, with the \ili{Nepali} verb \emph{haraunu} \rede{lose}). The \ili{Nepali} stems are integrated into Yakkha by means of the suffix \emph{-a} (also found in related languages, e.g., \emph{-ap} in Belhare, \citealt[559]{Bickel2003Belhare}). The resulting lexeme \emph{hara} is then treated like any other noun by the \isi{light verb}.

\exg. ŋkhoŋ    liŋkha   baji=be    har-a cog-a-khy-a.\\
and\_then a\_clan bet{\sc =loc} lose{\sc -nativ} do{\sc [3sg]-pst-V2.go-pst}    \\
\rede{And then the Linkha man lost the bet.}\source{11\_nrr\_01.012}

