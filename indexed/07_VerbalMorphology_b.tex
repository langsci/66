\subsection{Tense/aspect paradigm tables}\label{paradigmtables}

\begin{table}[hb]
\begin{centering}
%\begin{tabular}{l|p{3.5cm}|p{3.5cm}}
\begin{tabular}{lll}
\lsptoprule
&\emph{kheʔma} \rede{go}&\emph{apma} \rede{come}\\ 
\midrule
 {\sc 1sg} & \emph{khemeŋna}&\emph{ammeŋna}\\
 &\emph{ŋkhemeʔŋanna}&\emph{ŋammeŋanna}\\
 {\sc 1du.excl} &\emph{khemeŋciŋha} &\emph{ammeŋciŋha}\\
 & \emph{ŋkhemenciŋanha}&\emph{ŋammenciŋanha}\\
 {\sc 1pl.excl} &\emph{kheiwaŋha}&\emph{abiwaŋha}\\
  & \emph{ŋkheiwaŋanha}& \emph{ŋabiwaŋanha}\\
 {\sc 1du.incl} &\emph{khemeciya }&\emph{ammeciya}\\
 & \emph{ŋkhemencinha}&\emph{ŋammencinha}\\
 {\sc 1pl.incl} &\emph{kheiwha }&\emph{abiwha}\\
 &\emph{ŋkheiwanha}& \emph{ŋabiwanha}\\
 \midrule
 {\sc 2sg }&	\emph{khemekana }&\emph{ammekana} \\
 &\emph{ŋkhemekanna}&\emph{ŋammekanna} \\
 {\sc 2du}	& \emph{khemecigha}&\emph{ammecigha} \\
 & \emph{ŋkhemenciganha}&\emph{ŋamenciganha}\\
 {\sc 2pl}	&\emph{kheiwagha}&\emph{abiwagha} \\
 & \emph{ŋkheiwaganha}&\emph{ŋabiwaganha}\\
 \midrule
 {\sc 3sg} &\emph{khemeʔna}&\emph{ammeʔna} \\
 & \emph{ŋkhemenna}&\emph{ŋamenna}\\
 {\sc 3du} & \emph{khemeciha}&\emph{ammeʔciya}\\
 &\emph{ŋkhemencinha}& \emph{ŋamencinah}\\
 {\sc 3pl} &\emph{ŋkheme(haci)} &\emph{ŋamme(haci) }\\
 & \emph{ŋkhemen(haci)}&\emph{ŋamen(haci)}\\
\lspbottomrule
\end{tabular}
\caption{Nonpast paradigm of \emph{kheʔma} \rede{go} and   \emph{apma} \rede{come} (affirmative and negative)}\label{par-npst-intr}
\end{centering}
\end{table}

\begin{landscape}

\begin{table}[p] 
{\small
% \resizebox*{!}{.6\textheight}{ % When in landscape mode, we can use \textwidth for vertical size aligning ;) - It works WAY better than \resizebox*{!}{n\textheight}
\resizebox{\textwidth}{!}{
\begin{tabular}{l|p{2.4cm}|p{2.0cm}|p{2.6cm}|p{2.8cm}|p{2.6cm}|p{3.2cm}|p{3.2cm}}
%\begin{tabular}{|l|l|l|l|l|l|l|l|l|}
\lsptoprule
		&	{\sc 1sg}  &	 {\sc 1nsg}  & {\sc 2sg}	 & {\sc 2du} & {\sc 2pl} & {\sc 3sg} & {\sc 3nsg} \\
\hline
{\sc 1sg }		&\multicolumn{2}{c|}{\cellcolor[gray]{.8}} & \emph{tummeʔnenna}& & & \emph{tundwaŋna} &\emph{tundwaŋciŋha}\\
		& \multicolumn{2}{c|}{ \cellcolor[gray]{.8}} &\emph{ndummeʔnenna}& & & \emph{ndundwaŋanna}&\emph{ndundwaŋciŋanha}\\
 \cline{1-1} \cline{4-4}  \cline{7-8} 			
{\sc 1du.excl} 	& \multicolumn{2}{c|}{\cellcolor[gray]{.8}} & \multicolumn{2}{r|}{\emph{tummeʔnencina}}&&\emph{tummeŋcuŋna}&\emph{tummeŋcuŋciŋha}\\
		& \multicolumn{2}{c|}{\cellcolor[gray]{.8} } &\multicolumn{2}{r|}{\emph{ndummeʔnencinha}}&&\emph{ndummencuŋanna}&\emph{ndummencunciŋanha}\\			
 \cline{1-1} \cline{4-5}  \cline{7-8}			
{\sc 1pl.excl}		&\multicolumn{2}{c|}{ \cellcolor[gray]{.8}}& \multicolumn{3}{r|}{\emph{tummeʔneninha}} &\emph{tundwamŋana} &\emph{tundwamcimŋha}   \\
		&\multicolumn{2}{c|}{ \cellcolor[gray]{.8}}& \multicolumn{3}{r|}{\emph{ndummeʔneninha}}&\emph{ndundwamŋanna} &\emph{ndundwamcimŋanha}\\ 
 \cline{1-1} \cline{4-8}  		
 {\sc 1du.incl} 	&\multicolumn{2}{c|}{\cellcolor[gray]{.8} }&\multicolumn{3}{c|}{\cellcolor[gray]{.8}}&\emph{ tummecuna}  &   \emph{tummecuciha}\\
		& \multicolumn{2}{c|}{ \cellcolor[gray]{.8}}&\multicolumn{3}{c|}{\cellcolor[gray]{.8}}&\emph{ndummencunna}&\emph{ndummencuncinha}\\
 \cline{1-1} \cline{7-8}			
{\sc 1pl.incl} 	&	\multicolumn{2}{c|}{ \cellcolor[gray]{.8}}&\multicolumn{3}{c|}{\cellcolor[gray]{.8}} &\emph{ tundwamna}  & \emph{tundwamcimha}  \\
		& \multicolumn{2}{c|}{\cellcolor[gray]{.8} }&\multicolumn{3}{c|}{ \cellcolor[gray]{.8}}&\emph{ndundwamninna} & \emph{ndundwamcimninha}\\
\hline				
{\sc 2sg} 		&	\emph{tummeŋgana}  &	    & \multicolumn{3}{c|}{ \cellcolor[gray]{.8}}& \emph{tundwagana}  &   \emph{tundwacigha}\\
		&  \emph{ndummeŋganna} &  & \multicolumn{3}{c|}{ \cellcolor[gray]{.8}}&\emph{ ndundwaganna} &\emph{ndundwanciganha}  \\
 \cline{1-2} \cline{7-8}			
{\sc 2du}		&	\multicolumn{2}{c|}{}     & \multicolumn{3}{c|}{\cellcolor[gray]{.8}} & \emph{tummecugana}  &  \emph{tummecucigha} \\
		&  	\multicolumn{2}{c|}{\emph{tummekaha}} & \multicolumn{3}{c|}{\cellcolor[gray]{.8} }&\emph{ndummencuganna}  & \emph{ndummencunciganha} \\
 \cline{1-1} \cline{7-8}			
{\sc 2pl}		&	\multicolumn{2}{c|}{\emph{ndummekanha}} & \multicolumn{3}{c|}{ \cellcolor[gray]{.8}}& \emph{tundwamgana}& \emph{tundwamcimgha}  \\
		&	\multicolumn{2}{c|}{ }& \multicolumn{3}{c|}{\cellcolor[gray]{.8}}&\emph{ ndundwamganna} & \emph{ndundwamcimganha} \\
\hline			
{\sc 3sg} 		&\emph{tummeŋna}	  &	      	&  			&    &    	&\emph{tundwana}&\emph{tundwaciya}\\
		& \emph{ndummeŋanna}   &   	& \emph{tummekana}	&  &   &\emph{ndundwanna}&\emph{ndundwancinha}\\
  \cline{1-2}  \cline{7-8}					
{\sc 3du}& \multicolumn{2}{c|}{}&\emph{ndummekanna}& \emph{tummecigha}&\emph{tundiwagha}&  \emph{tummecuna} & \emph{ tummecuciha}\\
	& \multicolumn{2}{c|}{\emph{tummeha}}& &\emph{ndummenciganha}& \emph{ndundiwaganha} & \emph{ndummencunna} & \emph{ndummencuncinha}\\
 \cline{1-1} \cline{4-4} \cline{7-8}	
{\sc 3pl}&\multicolumn{2}{c|}{\emph{ndummenha}}	&\emph{ndummekana}& & &\emph{ndundwana}&\emph{ndundwaciha}\\	
	&\multicolumn{2}{c|}{ }&\emph{ndummekaninna}&& &\emph{ndundwaninna}&\emph{ndundwancininha} \\
\lspbottomrule
\end{tabular}
}
}
\caption{Nonpast paradigm of \emph{tumma} \rede{understand} (affirmative and negative)}\label{par-tund-npst}
\end{table}



\begin{table}[p]
% \resizebox*{!}{.5\textheight}{
\resizebox{\textwidth}{!}{
\begin{tabular}{lllllll}
\lsptoprule
 &{\sc pst}&{\sc pst.neg}&{\sc prf}&{\sc prf.neg}&{\sc pst.prf}&{\sc pst.prf.neg}\\
\hline
{\sc 1sg}& \it abaŋna & \it ŋabaŋanna& \it abamaŋna& \it ŋabamaŋanna& \it abamasaŋna& \it ŋabamasaŋanna \\
{\sc 1du.excl}& \it  abaŋciŋha& \it ŋabanciŋanha& \it abamaŋciŋha& \it ŋabamanciŋanha& \it abamasaŋciŋha& \it ŋabamasanciŋanha\\ 
{\sc 1pl.excl}& \it  abiŋha& \it ŋabiŋanha& \it abimiŋha& \it ŋabimiŋanha& \it abimisiŋha& \it ŋabimisiŋanha\\
{\sc 1du.incl}& \it abaciha& \it ŋabancinha& \it abamaciha& \it ŋabamancinha& \it abamasaciha& \it ŋabamasancinha\\
{\sc 1pl.incl}& \it  abiha& \it ŋabinha& \it abimiha& \it ŋabiminha& \it abimisiha& \it ŋabimisinha\\
\hline
{\sc 2sg}& \it  abagana& \it ŋabaganna& \it abamagana& \it ŋabamaganna& \it abamasagana& \it ŋabamasaganna\\
{\sc 2du}& \it  abacigha& \it ŋabanciganha& \it abamacigha& \it ŋabamanciganha& \it abamasacigha& \it ŋabamasanciganha\\
{\sc 2pl}& \it  abigha& \it ŋabiganha& \it abimigha& \it ŋabimiganha& \it abimisigha& \it ŋabimisiganha\\
\hline
{\sc 3sg}& \it  abana& \it ŋabanna& \it abamana& \it ŋabamanna& \it abamasana& \it ŋabamasanna\\
{\sc 3du}& \it  abaciha& \it ŋabancinha& \it abamaciha& \it ŋabamancinha& \it abamasaciya& \it ŋabamasancinha\\
{\sc 3pl}& \it  ŋabahaci& \it ŋabanhaci& \it ŋabamhaci& \it ŋabamanhaci& \it ŋabamashaci& \it ŋabamasanhaci\\
\lspbottomrule
\end{tabular}
}
\caption{Simple past, perfect and past perfect paradigm of \emph{apma} \rede{come}}\label{par-apma-pst}
\end{table}

\begin{table}[p] 
% \resizebox*{!}{.6\textheight}{
\resizebox{\textwidth}{!}{
\small
\begin{tabular}{l|p{2.4cm}|p{1.5cm}|p{2.4cm}|p{2.4cm}|p{2.4cm}|p{3.2cm}|p{3.2cm}}
\lsptoprule
		&	{\sc 1sg}  &	 {\sc 1nsg}  & {\sc 2sg}	 & {\sc 2du} & {\sc 2pl} & {\sc 3sg} & {\sc 3nsg} \\
%\midrule
\hline
{\sc 1sg }		&  \multicolumn{2}{c|}{ \cellcolor[gray]{.8}} & \it  piʔnenna& \it  & \it  & \it  piŋna & \it piŋciŋha \\
		&   \multicolumn{2}{c|}{\cellcolor[gray]{.8} } & \it mbiʔnenna& \it  & \it  & \it  mbiŋanna& \it mbiŋciŋanha\\
 \cline{1-1} \cline{4-4}	 \cline{7-8}					
{\sc 1du.excl} &  \multicolumn{2}{c|}{\cellcolor[gray]{.8}} &   \multicolumn{2}{r|}{\it piʔnencina}& \it  & \it pyaŋcuŋna& \it pyaŋcuŋciŋha\\
&  \multicolumn{2}{c|}{ \cellcolor[gray]{.8}} &  \multicolumn{2}{r|}{\it mbiʔnencina}& \it  & \it mbyancuŋanna& \it mbyancuŋciŋanha\\	
 \cline{1-1} \cline{4-5}	 \cline{7-8}						
{\sc 1pl.excl}	& \multicolumn{2}{c|}{ \cellcolor[gray]{.8}}&  \multicolumn{3}{r|}{\it piʔnenina} & \it pimŋana & \it pimcimŋha   \\
		& \multicolumn{2}{c|}{ \cellcolor[gray]{.8}}& \multicolumn{3}{r|}{\it mbiʔnenina}& \it mbimŋanna & \it mbimcimŋanha\\
 \cline{1-1} \cline{4-8}				
{\sc 1du.incl}	& \multicolumn{2}{c|}{ \cellcolor[gray]{.8}}&  \multicolumn{3}{c|}{ \cellcolor[gray]{.8} }& \it  pyacuna  & \it    pyacuciha\\
		& \multicolumn{2}{c|}{ \cellcolor[gray]{.8}}&  \multicolumn{3}{c|}{ \cellcolor[gray]{.8}}& \it mbyancunnaha& \it mbyancuncinha\\
 \cline{1-1} \cline{7-8}				
{\sc 1pl.incl} 	& 	\multicolumn{2}{c|}{\cellcolor[gray]{.8} }&  \multicolumn{3}{c|}{\cellcolor[gray]{.8} } & \it  pimna  & \it  pimcimha   \\
		&  \multicolumn{2}{c|}{ \cellcolor[gray]{.8}}& \multicolumn{3}{c|}{\cellcolor[gray]{.8} }& \it mbimninna & \it  mbimcimninha\\
%\cline{1-1} \cline{4-8}			
\hline
{\sc 2sg}		& \it 	pyaŋgana& \it 	    &   \multicolumn{3}{c|}{\cellcolor[gray]{.8} }& \it  pigana  & \it    picigha\\
		& \it mbyaŋganna   & \it   &  \multicolumn{3}{c|}{\cellcolor[gray]{.8}}& \it  mbiganna & \it mbinciganha  \\
 \cline{1-2} \cline{7-8}			
{\sc 2du}		& 	\multicolumn{2}{c|}{}     &  \multicolumn{3}{c|}{\cellcolor[gray]{.8}} & \it  pyacugana  & \it   pyacucigha \\
		&  	\multicolumn{2}{c|}{\it  pyagha} &   \multicolumn{3}{c|}{\cellcolor[gray]{.8}}& \it mbyancuganna  & \it  mbyancunciganha \\
 \cline{1-1} \cline{7-8}			
{\sc 2pl}	& 	\multicolumn{2}{c|}{\it mbyaganha} &   \multicolumn{3}{c|}{ \cellcolor[gray]{.8}}& \it  pimgana& \it  pimcimgha  \\
		&  	\multicolumn{2}{c|}{ }&   \multicolumn{3}{c|}{\cellcolor[gray]{.8} }& \it  mbimganna & \it  mbimcimganha \\
%\midrule			
\hline
{\sc 3sg}	& \it pyaŋna	  & \it 	      	& \it   			& \it     & \it     	& \it pina& \it piciya\\
		& \it  mbyaŋanna   & \it    	& \it  pyagana& \it   & \it    & \it mbinna& \it mbincinha\\
  \cline{1-2}  \cline{7-8}					
{\sc 3du}		&   \multicolumn{2}{c|}{}& \it mbyaganna& \it pyacigha & \it    piigha& \it   pyacuna & \it   pyacuciya\\
		&   \multicolumn{2}{c|}{\it piya}& \it   		& \it 		mbyanciganha	& \it  mbiiganha & \it  mbyancunna & \it  mbyancuncinha\\
 \cline{1-1} \cline{4-4} \cline{7-8}	
{\sc 3pl}		&\multicolumn{2}{c|}{\it mbyanha}& \it mbyaganha& \it & \it  & \it mbina& \it mbiciya\\	
	& \multicolumn{2}{c|}{ }& \it mbyaganinna& \it  & \it  & \it mbininna& \it mbincininha \\
\lspbottomrule
\end{tabular}
}
\caption{Simple past paradigm of \emph{piʔma} \rede{give} (affirmative and negative, with singular T argument)}\label{par-pipma-pst}
\end{table}



\begin{table}[p]  
\hskip-2cm
% \resizebox*{!}{.6\textheight}{
\resizebox{\textwidth}{!}{
\small
\begin{tabular}{l|p{2.4cm}|p{2.0cm}|p{2.6cm}|p{2.6cm}|p{2.6cm}|p{3.2cm}|p{3.2cm}}
\lsptoprule
		&	{\sc 1sg}  &	 {\sc 1nsg}  & {\sc 2sg}	 & {\sc 2du} & {\sc 2pl} & {\sc 3sg} & {\sc 3nsg} \\
\hline
{\sc 1sg}	& \multicolumn{2}{c|}{ \cellcolor[gray]{.8}} & \it chimnenna& \it  & \it  & \it  chimduŋna & \it chimduŋciŋha\\
		&   \multicolumn{2}{c|}{\cellcolor[gray]{.8} } & \it nchimnenna& \it  & \it  & \it  nchimduŋanna& \it nchimduŋciŋanha\\
 \cline{1-1} \cline{4-4}  \cline{7-8} 			
{\sc 1du.excl}	&  \multicolumn{2}{c|}{\cellcolor[gray]{.8}} &  \multicolumn{2}{r|}{\it chimnencinha}& \it & \it chimdaŋcuŋna& \it chimdaŋcuŋciŋha\\
		&   \multicolumn{2}{c|}{\cellcolor[gray]{.8} } &  \multicolumn{2}{r|}{\it nchimnencinha}& \it & \it nchimdancuŋanna& \it nchimdancunciŋanha\\			
 \cline{1-1} \cline{4-5}  \cline{7-8}			
{\sc 1pl.excl}		&  \multicolumn{2}{c|}{ \cellcolor[gray]{.8}}&   \multicolumn{3}{r|}{\it chimneninha} & \it chimdumŋana & \it chimdumcimŋha   \\
		&  \multicolumn{2}{c|}{ \cellcolor[gray]{.8}}&  \multicolumn{3}{r|}{\it nchimneninha}& \it nchimdumŋanna & \it nchimdumcimŋanha\\ 
 \cline{1-1} \cline{4-8}  		
 {\sc 1du.incl} 	&  \multicolumn{2}{c|}{ \cellcolor[gray]{.8}}&  \multicolumn{3}{c|}{\cellcolor[gray]{.8}}& \it  chimdacuna  & \it    chimdacuciha\\
		&  \multicolumn{2}{c|}{ \cellcolor[gray]{.8}}& \multicolumn{3}{c|}{\cellcolor[gray]{.8}}& \it nchimdancunna& \it nchimdancuncinha\\
 \cline{1-1} \cline{7-8}			
{\sc 1pl.incl}	&	\multicolumn{2}{c|}{ \cellcolor[gray]{.8}}&\multicolumn{3}{c|}{\cellcolor[gray]{.8}} & \it  chimdumna  & \it  chimdumcimha  \\
		&  \multicolumn{2}{c|}{ \cellcolor[gray]{.8}}&  \multicolumn{3}{c|}{ \cellcolor[gray]{.8}}& \it nchimdumninna & \it  nchimdumcimninha\\
\hline				
{\sc 2sg}		& \it 	chimdaŋgana  & \it 	    &  \multicolumn{3}{c|}{ \cellcolor[gray]{.8}}& \it  chimdugana  & \it    chimducigha\\
		& \it   nchimdaŋganna & \it   & \multicolumn{3}{c|}{ \cellcolor[gray]{.8}}& \it  nchimduganna & \it nchimdunciganha  \\
 \cline{1-2} \cline{7-8}			
{\sc 2du}		& \multicolumn{2}{c|}{}     &  \multicolumn{3}{c|}{\cellcolor[gray]{.8}} & \it  chimdacugana  & \it   chimdacucigha \\
		&  	\multicolumn{2}{c|}{\it chimdagha} & \multicolumn{3}{c|}{ \cellcolor[gray]{.8}}& \it nchimdancuganna  & \it  nchimdancunciganha \\
 \cline{1-1} \cline{7-8}			
{\sc 2pl}		& 	\multicolumn{2}{c|}{\it nchimdaganha} &  \multicolumn{3}{c|}{ \cellcolor[gray]{.8}}& \it  chimdumgana& \it  chimdumcimgha  \\
		& \multicolumn{2}{c|}{ }&  \multicolumn{3}{c|}{\cellcolor[gray]{.8} }& \it  nchimdumganna & \it  nchimdumcimganha \\
\hline			
{\sc 3sg} 		& \it chimdaŋna	  & \it 	      	& \it   			& \it     & \it     	& \it chimduna& \it chimduciya\\
		& \it  nchimdaŋanna   & \it    	& \it  chimdagana	& \it   & \it    & \it nchimdunna& \it nchimduncinha\\
  \cline{1-2}  \cline{7-8}					
{\sc 3du}&   \multicolumn{2}{c|}{}& \it nchimdaganna& \it  chimdacigha& \it chimdigha& \it   chimdacuna & \it   chimdacuciha\\
	& \multicolumn{2}{c|}{\it chimdaha}& \it  & \it nchimdanciganha& \it  nchimdiganha & \it  nchimdancunna & \it  nchimdancuncinha\\
 \cline{1-1} \cline{4-4} \cline{7-8}	
{\sc 3pl} &  \multicolumn{2}{c|}{\it nchimdanha}	& \it nchimdagana& \it  & \it  & \it nchimduna& \it nchimduciha\\	
	& \multicolumn{2}{c|}{ }& \it nchimdaganinna& \it & \it  & \it nchimduninna& \it nchimduncininha \\
\lspbottomrule
\end{tabular}
}
\caption{Simple past paradigm of \emph{chimma} \rede{ask} (affirmative and negative)}\label{par-chimd-pst}
\end{table}



\begin{table}[p]  
% \resizebox*{!}{.6\textheight}{
\resizebox{\textwidth}{!}{
\small
\begin{tabular}{l|p{2.4cm}|p{1.5cm}|p{2.4cm}|p{2.4cm}|p{2.4cm}|p{3.2cm}|p{3.2cm}}
\lsptoprule
		&	{\sc 1sg}  &	 {\sc 1nsg}  & {\sc 2sg}	 & {\sc 2du} & {\sc 2pl} & {\sc 3sg} & {\sc 3nsg} \\
\hline
{\sc 1sg}		&\multicolumn{2}{c|}{\cellcolor[gray]{.8} } & \it  caʔnenna& \it  & \it  & \it  caŋna & \it caŋciŋha \\
		& \multicolumn{2}{c|}{\cellcolor[gray]{.8} } & \it njaʔnenna& \it  & \it  & \it  njaŋanna& \it njanciŋanha\\
 \cline{1-1} \cline{4-4}	 \cline{7-8}					
{\sc 1du.excl} &  \multicolumn{2}{c|}{\cellcolor[gray]{.8}} &  \multicolumn{2}{r|}{\it caʔnencina}& \it  & \it cayaŋcuŋna& \it cayaŋcuŋciŋha\\
&  \multicolumn{2}{c|}{ \cellcolor[gray]{.8}} & \multicolumn{2}{r|}{\it njaʔnencina}& \it  & \it njayancuŋanna& \it njayancuŋciŋanha\\	
 \cline{1-1} \cline{4-5}	 \cline{7-8}						
{\sc 1pl.excl}	&  \multicolumn{2}{c|}{\cellcolor[gray]{.8} }&   \multicolumn{3}{r|}{\it caʔnenina} & \it camŋana & \it camcimŋha   \\
		&  \multicolumn{2}{c|}{\cellcolor[gray]{.8} }&  \multicolumn{3}{r|}{\it njaʔnenina}& \it njamŋanna & \it njamcimŋanha\\
 \cline{1-1} \cline{4-8}				
{\sc 1du.incl}	&  \multicolumn{2}{c|}{ \cellcolor[gray]{.8}}&  \multicolumn{3}{c|}{  \cellcolor[gray]{.8}}& \it  cayacuna  & \it    cayacuciha\\
		&  \multicolumn{2}{c|}{ \cellcolor[gray]{.8}}& \multicolumn{3}{c|}{\cellcolor[gray]{.8} }& \it njayancunna& \it njayancuncinha\\
 \cline{1-1} \cline{7-8}				
{\sc 1pl.incl} 	& \multicolumn{2}{c|}{\cellcolor[gray]{.8} }&  \multicolumn{3}{c|}{ \cellcolor[gray]{.8}} & \it  camna  & \it  camcimha   \\
		&  \multicolumn{2}{c|}{\cellcolor[gray]{.8} }& \multicolumn{3}{c|}{\cellcolor[gray]{.8} }& \it njamninna & \it  njamcimninha\\
\cline{1-1} \cline{4-8}			
\hline				
{\sc 2sg}		& \it 	cayaŋgana& \it 	    & \multicolumn{3}{c|}{\cellcolor[gray]{.8} }& \it  cogana  & \it   coːcigha\\
		& \it njayaŋgana   & \it   &\multicolumn{3}{c|}{ \cellcolor[gray]{.8}}& \it  njoganna & \it njonciganha  \\
 \cline{1-2} \cline{7-8}			
{\sc 2du}		& \multicolumn{2}{c|}{}     & \multicolumn{3}{c|}{\cellcolor[gray]{.8}} & \it  cayacugana  & \it   cayacucigha \\
		& \multicolumn{2}{c|}{\it cayagha} & \multicolumn{3}{c|}{ \cellcolor[gray]{.8}}& \it njayancuganna  & \it  njayancunciganha \\
 \cline{1-1} \cline{7-8}			
{\sc 2pl}	& \multicolumn{2}{c|}{\it njayaganha} & \multicolumn{3}{c|}{\cellcolor[gray]{.8} }& \it  camgana& \it  camcimgha  \\
		& 	\multicolumn{2}{c|}{ }& \multicolumn{3}{c|}{\cellcolor[gray]{.8} }& \it  njamganna & \it  njamcimganha \\
\hline			
{\sc 3sg}	& \it coyaŋna	\ti  & \it 	 coya   \ti  	& \it   	coyagana		\ti & \it   coʔyacigha \ti  & \it   coigha \ti 	& \it cona & \it cociya \\
					& \it cayaŋna	  & \it 	      caya	& \it  cayagana 			& \it  cayacigha   & \it   caigha	& \it  & \it  \\
					& \it  njayaŋanna   & \it   njayanha 	& \it  njayaganna& \it  njayanciganha & \it  njaigha  & \it njonna& \it njoncinha\\
\hline
		% \cline{1-2}  \cline{7-8}					
{\sc 3du }		&  \multicolumn{2}{c|}{}& \it cayagana& \it cayacigha & \it    caigha& \it   cayacuna & \it   cayacuciya\\
		& \multicolumn{2}{c|}{}& \it   	njayaganna	& \it njayanciganha	& \it  njaiganha& \it  njayancunna & \it njayancuncinha \\
 \cline{1-1} \cline{4-4} \cline{7-8}	
{\sc 3pl}		& \multicolumn{2}{c|}{\it caya}& \it njayagana& \it & \it  & \it njona& \it njociya\\	
	& \multicolumn{2}{c|}{\it  njayanha}			& \it njayaganinna& \it  & \it  & \it njoninha& \it njoncininha \\
\lspbottomrule
\end{tabular}
}

\caption{Simple past  paradigm of \emph{cama} \rede{eat} (affirmative and negative)}\label{par-cama-pst}
\end{table}


\begin{table}[p] 
\small
\hskip-2cm
% \resizebox*{!}{.6\textheight}{
\resizebox{\textwidth}{!}{
\begin{tabular}{l|p{2.8cm}|p{1.6cm}|p{2.8cm}|p{3.0cm}|p{2.8cm}|p{3.2cm}|p{3.6cm}}
\lsptoprule
		&	{\sc 1sg}  &	 {\sc 1nsg}  & {\sc 2sg}	 & {\sc 2du} & {\sc 2pl} & {\sc 3sg} & {\sc 3nsg} \\
\hline
{\sc 1sg 	}	&\multicolumn{2}{c|}{ \cellcolor[gray]{.8}} & \it chimnuŋnenna& \it  & \it  & \it  chimduksuŋna & \it chimduksuŋciŋha\\
		&  \multicolumn{2}{c|}{\cellcolor[gray]{.8} } & \it nchimnuŋnenna& \it  & \it  & \it  nchimduksuŋanna& \it nchimduksuŋciŋanha\\
 \cline{1-1} \cline{4-4}  \cline{7-8} 			
{\sc 1du.excl}	& \multicolumn{2}{c|}{\cellcolor[gray]{.8}} &\multicolumn{2}{r|}{\it chimnuŋnencinha}& \it & \it chimdamaŋcuŋna& \it chimdamaŋcuŋciŋha\\
		&  \multicolumn{2}{c|}{\cellcolor[gray]{.8} } & \multicolumn{2}{r|}{\it nchimnuŋnencinha}& \it & \it nchimdamancuŋanna& \it nchimdamancunciŋanha\\			
 \cline{1-1} \cline{4-5}  \cline{7-8}			
{\sc 1pl.excl}		&  \multicolumn{2}{c|}{\cellcolor[gray]{.8} }&  \multicolumn{3}{r|}{\it chimnuŋneninha} & \it chimduksumŋana & \it chimduksumcimŋha   \\
		& \multicolumn{2}{c|}{\cellcolor[gray]{.8} }& \multicolumn{3}{r|}{\it nchimnuŋneninha}& \it nchimduksumŋanna & \it nchimduksumcimŋanha\\ 
 \cline{1-1} \cline{4-8}  		
 {\sc 1du.incl} 	& \multicolumn{2}{c|}{ \cellcolor[gray]{.8}}&  \multicolumn{3}{c|}{\cellcolor[gray]{.8}}& \it  chimdamacuna  & \it    chimdamacuciha\\
		& \multicolumn{2}{c|}{ \cellcolor[gray]{.8}}& \multicolumn{3}{c|}{\cellcolor[gray]{.8}}& \it nchimdamancunna& \it nchimdamancuncinha\\
 \cline{1-1} \cline{7-8}			
{\sc 1pl.incl}	& 	\multicolumn{2}{c|}{\cellcolor[gray]{.8} }& \multicolumn{3}{c|}{\cellcolor[gray]{.8}} & \it  chimduksumna  & \it  chimduksumcimha  \\
		&  \multicolumn{2}{c|}{\cellcolor[gray]{.8} }& \multicolumn{3}{c|}{ \cellcolor[gray]{.8}}& \it nchimduksumninna & \it  nchimduksumcimninha\\
\hline				
{\sc 2sg}		& \it 	chimdamaŋgana  & \it 	    &\multicolumn{3}{c|}{ \cellcolor[gray]{.8}}& \it  chimduksugana  & \it    chimduksucigha\\
		& \it   nchimdamaŋganna & \it   & \multicolumn{3}{c|}{\cellcolor[gray]{.8} }& \it  nchimduksuganna & \it nchimduksunciganha  \\
 \cline{1-2} \cline{7-8}			
{\sc 2du}		& \multicolumn{2}{c|}{}     & \multicolumn{3}{c|}{\cellcolor[gray]{.8}} & \it  chimdamacugana  & \it   chimdamacucigha \\
		& \multicolumn{2}{c|}{\it chimdamagha} & \multicolumn{3}{c|}{\cellcolor[gray]{.8} }& \it nchimdamancuganna  & \it  nchimdamancunciganha \\
 \cline{1-1} \cline{7-8}			
{\sc 2pl}		& \multicolumn{2}{c|}{\it nchimdamaganha} & \multicolumn{3}{c|}{\cellcolor[gray]{.8} }& \it  chimduksumgana& \it  chimduksumcimgha  \\
		& \multicolumn{2}{c|}{ }& \multicolumn{3}{c|}{ \cellcolor[gray]{.8}}& \it  nchimduksumganna & \it  nchimduksumcimganha \\
\hline			
{\sc 3sg} 		& \it chimdamaŋna	  & \it 	      	& \it   			& \it     & \it     	& \it chimduksuna& \it chimduksuciya\\
		& \it  nchimdamaŋanna   & \it    	& \it  chimdamagana	& \it   & \it    & \it nchimduksunna& \it nchimduksuncinha\\
  \cline{1-2}  \cline{7-8}					
{\sc 3du}&  \multicolumn{2}{c|}{}& \it nchimdamaganna& \it  chimdamacigha& \it chimdimigha& \it   chimdamacuna & \it   chimdamacuciha\\
	& \multicolumn{2}{c|}{\it chimdamha}& \it  & \it nchimdamanciganha& \it  nchimdimiganha & \it  nchimdamancunna & \it  nchimdamancuncinha\\
 \cline{1-1} \cline{4-4} \cline{7-8}	
{\sc 3pl} &  \multicolumn{2}{c|}{\it nchimdamanha}	& \it nchimdamagana& \it  & \it  & \it nchimduksuna& \it nchimduksuciha\\	
	&\multicolumn{2}{c|}{ }& \it nchimdamaganinna& \it & \it  & \it nchimduksuninna& \it nchimduksuncininha \\
\lspbottomrule
\end{tabular}
}
\caption{Perfect paradigm of \emph{chimma} \rede{ask} (affirmative and negative)}\label{par-chimd-prf}
\end{table}

\end{landscape}

%\pagestyle{scrheadings}

\section{Mood}\label{mood}

Apart form the indicative, Yakkha distinguishes subjunctive, \isi{optative} and  \isi{imperative} \isi{mood}. These \isi{mood} inflections generally do not allow the nominalizing clitics \emph{=na} and \emph{=ha}, since they are functionally connected to assertions and questions (but see below for an exception).

\subsection{The subjunctive}

The subjunctive \isi{mood} is crosscut by what is best described as a \isi{tense} distinction (nonpast/past). The nonpast subjunctive does not have a dedicated marker, but is simply characterized by the absence of \isi{tense} marking. It is used for hypothetical statements, hortatives, warnings, threats, and \isi{permissive} questions (\rede{May I ...?}). The past subjunctive is found in counterfactual statements, but also in adverbial clauses, especially in conditionals, when the speaker assesses the chances for the condition to come true as rather low (see \sectref{adv-cl-cond} and \sectref{adv-cl-count}). The past subjunctive is marked by \emph{-a}, and hence, the forms of the past subjunctive paradigm are, in most cases, identical to the past indicative forms, without the clitics \emph{=na} and \emph{=ha}, however.\footnote{Alternatively, one could propose that Yakkha has no \isi{mood} distinction in the past, but the clitics  \emph{=na} and \emph{=ha} instead, which overtly mark the indicative. However, first of all, these clitics are optional also in the indicative, as they fulfill a discourse function. Secondly, a few forms in the inflectional paradigms of past indicative and past subjunctive indeed look different from each other (see right below in the text).} In third person plural forms of intransitive verbs, the \isi{negation} in the past subjunctive looks different from the past indicative (e.g., \emph{ŋkhyanhaci} \rede{they did not go} vs. \emph{ŋkhyanin} \rede{they might not go}). 


The negated forms are built in analogy to the indicative negated forms, i.e., either with \emph{N-...-n} or \emph{N-...-nin}. Here, one can note a slight extension of the domain of \emph{-nin}: the form \emph{chim}, for third person acting on first, has the negative counterpart \emph{nchimnin}, since monosyllabic  \emph{*n-chim-n } would not be a well-formed \isi{syllable} in Yakkha. Surprisingly, some negated forms of the nonpast subjunctive are marked by \emph{=na} in the intransitive paradigm, which is the only exception to the rule that the nominalizing clitics do not occur in the \isi{mood} paradigms.\footnote{An ad-hoc explanation is that negations are more assertive than affirmative forms and thus allow this marker. The question why only the present negated forms take \emph{=na} can be answered similarly. The present forms are more \rede{real} and thus more assertive. They denote rather likely events, while the past subjunctive denotes events that are more detached from the speech situation, such as highly hypothetical and counterfactual events. The syncretism of past and irrealis forms is not unusual crosslinguistically and can be attributed to a semantic feature of \rede{dissociativeness} that they have in common (see, e.g., \citet[88]{Bickel1996Aspect} for the same point on the Belhare system).} It is unusual, however, that the singular form \emph{=na} occurs invariably, and never \emph{=ha}. Alternatively, this marker \emph{=na} could be analyzed as a dedicated marker for nonpast subjunctive negative forms.

Intransitive  subjunctive paradigms are provided in \tabref{par-sbjv-intr}, exemplified by \emph{kheʔma} \rede{go},  with a few forms not attested. Since the suffix \emph{-a} is deleted in the presence of the suffix \emph{-i}, the forms with first and second person plural are identical in the nonpast subjunctive and the past subjunctive. A transitive paradigm for the nonpast subjunctive is shown in \tabref{par-chimd-sbjv}.  In the  form for {\sc 2sg>1sg}, \emph{chimdaŋga}, an /a/ gets epenthesized to resolve the impossible sequence of consonants in the underlying form /chimd-ŋ-ga/, making this form  identical to  the past subjunctive form.

\begin{table}[htp]
\resizebox{\textwidth}{!}{
\begin{tabular}{lllll}
\lsptoprule
&\multicolumn{2}{c}{{\sc nonpast subjunctive}}&\multicolumn{2}{c}{{\sc past subjunctive}}\\
		&{\sc affirmative} & {\sc negative} & {\sc affirmative} & {\sc negative}  \\
\midrule
		{\sc 1sg}& \it kheʔŋa& \it ŋkheʔŋanna& \it khyaŋ& \it ŋkhyaŋan \\
		{\sc 1du.excl}& \it kheciŋ& \it ŋkheciŋanna& \it khyaŋciŋ& \it ŋkhyanciŋan \\
		{\sc 1pl.excl}& \it kheiŋ& \it ŋkheiŋanna& \it kheiŋ& \it ŋkheiŋan \\
		{\sc 1du.incl}& \it kheci& \it ŋkhecinna& \it khyaci& \it ŋkhyancin \\
		{\sc1pl.incl}& \it khei& \it ŋkheinna& \it khei& \it ŋkhein \\
\midrule
		{\sc 2sg}& \it kheka& \it ŋkhekanna& \it khyaga& \it ŋkhyagan \\
		{\sc 2du}& \it kheciga& \it ŋkheciganna& \it khyaciga& \it ŋkhyancigan \\
		{\sc2pl}& \it kheiga& \it ŋkheiganna& \it kheiga& \it ŋkheigan \\
\midrule
		{\sc 3sg}& \it khe& — & \it khya& \it ŋkhyan \\
		{\sc 3du}& \it kheci&   — & \it khyaci& \it ŋkhyancin \\
		{\sc 3pl}& \it ŋkhe(ci)&   — & \it ŋkhya(ci)& \it ŋkhyanin \\
\lspbottomrule	
\end{tabular}
}
\caption{Subjunctive paradigm of \emph{kheʔma} \rede{go}}\label{par-sbjv-intr}
\end{table}


  Some typical examples of the nonpast subjunctive are provided in \Next: questions, hortatives, warnings,  threats and \isi{permissive} questions.   

\ex. \ag. heʔne khe-i?\\
		where go{\sc -1pl[sbjv]}	\\
	\rede{Where should we go?}
	 \bg.imin cog-u-m?\\
 how do{\sc -3.P-1pl.A[sbjv]}\\
  \rede{How should we do it?}
	 \bg. ciya hops-u-m?\\
		tea sip{\sc -3.P-1pl.A[sbjv]}	\\
	\rede{Shall we have tea?}
	 \bg. sori khe-ci-ŋ?\\
	together go{\sc -1du-excl[sbjv]}\\
	\rede{May we go together?} (asking someone who is not coming with the speaker)
	\bg.lem-nhaŋ-nen?\\
  throw{\sc -V2.send-1>2[sbjv]}\\
  \rede{Shall I throw you out?}
  \bg.kaŋ-khe-kaǃ\\
 fall{\sc -V2.go-2[sbjv]}\\
  \rede{You might fall downǃ}
  \bg.uŋ-u-mǃ\\
  drink{\sc -3.P-1pl.A[sbjv]}\\
  \rede{Let's drinkǃ}
  \bg.haku im-ci\\
  now sleep{\sc -du[sbjv]}\\
  \rede{Let's sleep now (dual).}
  
Some examples of the Past  Subjunctive are provided in \Next. Mostly, they are found in counterfactual clauses (with the irrealis \isi{clitic} \emph{=pi \ti =bi}), but also in vague statements about the future, i.e., when the realis status cannot be confirmed yet. The overwhelming majority of \isi{conditional clauses} are in the past subjunctive.
  
  \ex. \ag.manuŋ=go heco=bi.\\
  otherwise{\sc =top} win{\sc [3sg;sbjv]=irr}\\
  \rede{If not, he would have won!} 
  \bg.casowa           n-jog-a-n     bhoŋ puŋdaraŋma=ga      o-lok     khom-me.\\
  Casowa\_festival {\sc neg-}do{\sc [3sg]-sbjv-neg} {\sc cond} forest\_goddess{\sc =gen} {\sc 3sg.poss-}anger scratch{\sc [3sg]-npst}\\
  \rede{If Casowa is not celebrated, the forest goddess will get angry.} \source{01\_leg\_07.119}
  
A very nice contrastive  example of  nonpast and  past subjunctive is shown in \Next. The speaker considers doing something and first is unsure, using the past subjunctive. Then, as she is more determined, she uses the nonpast subjunctive.

\ex. \ag.ki   nhe=le           ips-a-ŋ?\\
or here{\sc =ctr} sleep{\sc -sbjv-1sg}\\
\rede{Or should I sleep right here?} \source{36\_cvs\_06.220}
\bg.nhe  im-ŋa              haʔlo.\\
here sleep{\sc -1sg[sbjv]} {\sc excla}\\
\rede{I could sleep here anyway.} \source{36\_cvs\_06.221}
	

%

\begin{landscape}
\begin{table} 
\hskip-2cm
{\small
% \resizebox*{!}{.65\textheight}{
\resizebox{\textwidth}{!}{
\begin{tabular}{l|p{2.4cm}|p{2.0cm}|p{2.6cm}|p{2.6cm}|p{2.6cm}|p{3.2cm}|p{3.2cm}}
\lsptoprule
		&	{\sc 1sg}  &	 {\sc 1nsg}  & {\sc 2sg}	 & {\sc 2du} & {\sc 2pl} & {\sc 3sg} & {\sc 3nsg} \\
\hline
{\sc 1sg 	}	& \multicolumn{2}{c|}{ \cellcolor[gray]{.8}} & \it chimnen& \it  & \it  & \it  chimduŋ & \it chimduŋciŋ\\
		& \multicolumn{2}{c|}{\cellcolor[gray]{.8} } & \it nchimnen& \it  & \it  & \it  nchimduŋan& \it nchimduŋciŋan\\
 \cline{1-1} \cline{4-4}  \cline{7-8} 			
{\sc 1du.excl }	&  \multicolumn{2}{c|}{\cellcolor[gray]{.8}} &  \multicolumn{2}{r|}{\it chimnencin}& \it & \it chimcuŋ& \it chimcuŋciŋ\\
		& \multicolumn{2}{c|}{ \cellcolor[gray]{.8}} &  \multicolumn{2}{r|}{\it nchimnencin}& \it & \it nchimcuŋan& \it nchimcunciŋan\\			
 \cline{1-1} \cline{4-5}  \cline{7-8}			
{\sc 1pl.excl}		& \multicolumn{2}{c|}{ \cellcolor[gray]{.8}}&  \multicolumn{3}{r|}{\it chimnenin} & \it chimdumŋa & \it chimdumcimŋa   \\
		&  \multicolumn{2}{c|}{\cellcolor[gray]{.8} }&\multicolumn{3}{r|}{\it nchimnenin}& \it nchimdumŋan & \it nchimdumcimŋan\\ 
 \cline{1-1} \cline{4-8}  		
 {\sc 1du.incl} 	& \multicolumn{2}{c|}{ \cellcolor[gray]{.8}}& \multicolumn{3}{c|}{\cellcolor[gray]{.8}}& \it  chimcu  & \it    chimcuci\\
		&  \multicolumn{2}{c|}{\cellcolor[gray]{.8} }& \multicolumn{3}{c|}{\cellcolor[gray]{.8}}& \it nchimcun& \it nchimcuncin\\
 \cline{1-1} \cline{7-8}			
{\sc 1pl.incl }	& \multicolumn{2}{c|}{ \cellcolor[gray]{.8}}& \multicolumn{3}{c|}{\cellcolor[gray]{.8}} & \it  chimdum  & \it  chimdumcim  \\
		&  \multicolumn{2}{c|}{\cellcolor[gray]{.8} }& \multicolumn{3}{c|}{\cellcolor[gray]{.8} }& \it nchimdumnin & \it  nchimdumcimnin\\
\hline				
{\sc 2sg }		& \it 	chimdaŋga  & \it 	    &\multicolumn{3}{c|}{ \cellcolor[gray]{.8}}& \it  chimduga & \it    chimduciga\\
		& \it   nchimdaŋgan & \it   & \multicolumn{3}{c|}{\cellcolor[gray]{.8} }& \it  nchimdugan & \it nchimduncigan  \\
 \cline{1-2} \cline{7-8}			
{\sc 2du}		&	\multicolumn{2}{c|}{}     &  \multicolumn{3}{c|}{\cellcolor[gray]{.8}} & \it  chimcuga  & \it   chimcuciga \\
		& 	\multicolumn{2}{c|}{\it chimga} & \multicolumn{3}{c|}{\cellcolor[gray]{.8} }& \it nchimcugan  & \it  nchimcuncigan \\
 \cline{1-1} \cline{7-8}			
{\sc 2pl}		& 	\multicolumn{2}{c|}{\it nchimgan} &  \multicolumn{3}{c|}{ \cellcolor[gray]{.8}}& \it  chimdumga& \it  chimdumcimga  \\
		& 	\multicolumn{2}{c|}{ }& \multicolumn{3}{c|}{\cellcolor[gray]{.8} }& \it  nchimdumgan & \it  nchimdumcimgan \\
\hline			
{\sc 3sg} 		& \it chimŋa	  & \it 	      	& \it   chimga	 	& 	  &     	& \it chimdu& \it chimduci\\
		& \it  nchimŋan   & \it    	& \it nchimgan& \it  & \it    								& \it nchimdun& \it nchimduncin\\
  \cline{1-2}  \cline{4-4} \cline{7-8}					
{\sc 3du}& \multicolumn{2}{c|}{}&\it chimciga&\it chimciga& \it chimdiga& \it   chimcu & \it   chimcuci\\
	&  \multicolumn{2}{c|}{\it chim}	& \it nchimcigan		&   \it nchimcigan & \it  nchimdigan & \it  nchimcun & \it  nchimcuncin\\
 \cline{1-1} \cline{4-4} \cline{7-8}	
{\sc 3pl} & \multicolumn{2}{c|}{\it nchimnin}	& \it nchimga& \it  & \it  & \it nchimdu& \it nchimduci\\	
	& \multicolumn{2}{c|}{ }				& \it nchimganin& \it & \it  						& \it nchimdunin& \it nchimduncinin \\
\lspbottomrule
\end{tabular}
}
}
\caption{Nonpast subjunctive paradigm of \emph{chimma} \rede{ask} (affirmative and negative), becomes \isi{optative} by addition of \emph{-ni}}\label{par-chimd-sbjv}
\end{table}
\end{landscape}
 
%\pagestyle{scrheadings}
 
	
\subsection{The Optative}

The optative is morphologically marked by the suffix \emph{-ni} , which is attached to  the nonpast subjunctive forms described above. It expresses the speaker's wish for an event  to be realized, while its realization is beyond the speaker's reach, as in the examples in \Next. The expression \emph{leŋni}, the third person singular \isi{optative} of \emph{leŋma} \rede{be, become} is also used to state agreement on the side of the speaker when he is asked or encouraged to do something.
 
\ex. \ag. o-chom=be tas-u-ni.\\
		{\sc 3sg.poss-}summit{\sc =loc} reach{\sc -3.P-opt}	\\
		\rede{May she reach the top./ May she be successful.}
	 \bg. ucun leŋ-ni.\\ 
		good become{\sc [3sg]-opt}	\\
	\rede{May it (your work) turn out nicely.}
	 \bg.miʔ-ŋa-ni.\\
  think{\sc -1sg.P-opt}\\
  \rede{May he remember me.}
   \bg.mit-aŋ-ga-ni.\\
  think{\sc -1sg.P-2.A-opt}\\
  \rede{May you remember me.}
  
The \isi{optative} is also found in purposive adverbial clauses with the purposive/conditional conjunction \emph{bhoŋ}, discussed in \sectref{adv-cl-fin-purp} (see also \Next). 

\exg.ap-ŋa-ni bhoŋ ka-ya-ŋ=na.\\
come{\sc -1sg-opt} {\sc cond} call{\sc -pst-1sg=sg}\\
\rede{She called me, so that I would  come.}

Negation is marked by \emph{N-...-n}, and by \emph{N-...-nin} when there is no vowel preceding the suffix. \tabref{par-opt-intr} illustrates this by means of the third person intransitive forms.

\begin{table}[htp]
\begin{center}
\begin{tabular}{lll}
\lsptoprule
&\multicolumn{2}{c}{{\sc optative}}\\
\midrule
		{\sc 3sg} & \it kheʔni& \it ŋkheʔninni\\
 		{\sc 3du} & \it khecini& \it ŋkhecinni\\
		{\sc 3pl} & \it ŋkheʔni& \it ŋkheʔninni\\
\lspbottomrule	
\end{tabular}
\end{center}
\caption{Optative, third \isi{person} (intransitive)}\label{par-opt-intr}
\end{table}


\subsection{The Imperative}

The Imperative expresses orders and requests. It is coded by the morpheme \emph{-a}, like the simple past and the past subjunctive. The conflation of past and \isi{imperative} morphology is also known from other Kiranti languages \citep{Bickel2003Belhare, Ebert2003Camling} and apart from that it is crosslinguistically common, too. Thus, \isi{imperative} forms are almost identical to the past forms, except for a new plural morpheme \emph{-ni}. 

The negated \isi{imperative} expresses negative requests and negative orders (i.e., prohibitions). It is also used in implorations like e.g., \emph{nsisaŋan!} \rede{Do not kill me!}. Paradigms can be found in \tabref{par-imp-intr} for intransitive and in \tabref{par-imp-tr} for transitive verbs.
 
Imperatives directed at more than one person show dual or plural morphology (see \Next).  Imperatives can be intensified by adding person inflection and the emphatic marker \emph{=i} (see \Next[b] - \Next[d]). 
 
 \ex. \ag.ab-a.\\
 come{\sc -imp}\\
 \rede{Come.}
 \bg.ab-a-ga=iǃ\\
come{\sc -imp-2=emph}\\
 \rede{Comeǃ}
 \bg.ab-a-ci-ga=iǃ\\
come{\sc -imp-du-2=emph}\\
 \rede{Comeǃ} (dual)
 \bg.ab-a-ni-ga=i, nakhe omphu=be yuŋ-a-ni-ga=iǃ\\
come{\sc -imp-pl-2=emph}, hither verandah{\sc =loc} sit{\sc -imp-pl-2=emph}\\
 \rede{Come, and sit down here on the verandahǃ} (plural)
 
The imperatives show a second register, increasing the  politeness of the order or request. The marker \emph{=eba} (historically probably a combination of the two  emphatic markers  \emph{=i} and \emph{=pa}) can be added to the \isi{imperative} forms to make them more polite, similar to the function of the  particle \emph{na} in \ili{Nepali}  \Next. This politeness, however, can be countered ironically by adding \emph{=ʔlo} to these polite imperatives, an \isi{exclamative} particle which usually signals that  the patience of the speaker is getting low (see \Next[c], \emph{ca} has ablaut).\footnote{Combinations of emphatic particles and information-structural clitics with \emph{=ʔlo} result in a word with regard to stress (e.g., [ˈco.e ˈba.ʔlo] in \ref{coebalo}). According to the \isi{voicing} rule,  this complex of two stress domains is still one word, however. Other examples of this phonological fusion of particles are \emph{ˈleʔ.lo}, \emph{ˈhaʔ.lo}, \emph{ˈcaʔ.lo} (see Chapter \ref{particles}).}

\ex. \ag.ab-a=eba.\\
		 come{\sc -imp=pol.imp}\\
	\rede{Please come.}
 	\bg.ŋ-ab-a-n=eba.\\
		{\sc neg-}come{\sc -imp-neg=pol.imp}\\
	\rede{Please do not come.}
	\bg.\label{coebalo}co=eba=ʔloǃ \\
	eat{\sc [imp]=pol.imp=excla}\\
	\rede{Eat alreadyǃ} 


\begin{table}[htp]
\begin{centering}
\begin{tabular}{lllll}
\lsptoprule
		&{\sc imperative} & {\sc negated} & {\sc imperative} & {\sc negated}  \\
&\multicolumn{2}{c}{\emph{kheʔma} \rede{go}}&\multicolumn{2}{c}{\emph{apma} \rede{come}}\\
\midrule
		{\sc 2sg}& \it khya& \it ŋkhyan & \it aba & \it ŋaban \\
		{\sc 2du}& \it khyaci& \it ŋkhyancin & \it abaci & \it ŋabancin\\
		{\sc 2pl}& \it khyani& \it ŋkhyanin & \it abani & \it ŋabanin \\
\lspbottomrule
\end{tabular}
\caption{Imperative paradigm, intransitive verbs}\label{par-imp-intr}
\end{centering}
\end{table}

 

\begin{table}[htp]
\resizebox*{\textwidth}{!}{
\small
\begin{tabular}{l|l|l|l|l|l}
\lsptoprule
	&{\sc 1sg}&{\sc 1nsg}&{\sc 3sg}&{\sc 3nsg}&{\sc detrans} \\
\hline
\multicolumn{6}{l}{}\\
\multicolumn{6}{c}{\emph{piʔma} \rede{give}}\\
\hline
{\sc 2sg}	& \it pyaŋ & \it & \it  pi & \it pici& \it pya\\
	& \it mbyaŋan& \it & \it mbin& \it mbincin& \it mbyan\\
 \cline{2-2} %\cline{4-6}
{\sc 2du}	& \multicolumn{2}{c|}{} & \it  pyacu& \it pyacuci& \it pyaci\\
	&  \multicolumn{2}{c|}{\it pya}& \it  mbyancun& \it mbyancuncin& \it mbyancin \\
% \cline{1-1} \cline{4-6}
{\sc 2pl}	&  \multicolumn{2}{c|}{\it mbyan}& \it  pyanum& \it pyanumcim& \it pyani\\
	&  \multicolumn{2}{c|}{}& \it  mbyanumnin& \it mbyanumcimnin& \it mbyanin\\
\hline
\multicolumn{6}{l}{}\\
\multicolumn{6}{c}{\emph{chimma} \rede{ask}}\\
\hline
{\sc 2sg}	& \it   chimdaŋ& \it & \it  chimdu & \it chimduci& \it chimda\\
	& \it  nchimdaŋa& \it & \it nchimdun & \it  nchimduncin& \it nchimdan\\
 \cline{2-2}  %\cline{4-6}
{\sc 2du}	&  \multicolumn{2}{c|}{} & \it   chimdacu& \it chimdacuci & \it chimdaci\\
	& \multicolumn{2}{c|}{\it chimda}& \it  nchimdancun & \it nchimdancuncin & \it nchimdancin\\
% \cline{1-1} \cline{4-6}
{\sc 2pl}& \multicolumn{2}{c|}{\it nchimdan}& \it chimdanum & \it chimdanumcim& \it chimdani\\
	& \multicolumn{2}{c|}{}& \it  nchimdanumnin& \it nchimdanumcimnin& \it nchimdanin\\
\lspbottomrule
\end{tabular}
}
\caption{Imperative paradigm, transitive verbs }\label{par-imp-tr}
\end{table}



\section{Periphrastic honorific inflection}\label{honorific}

Honorific inflection in indicatives is not found in the Tumok dialect, but during a short stay in Dandagaun  village I noticed a \isi{honorific} construction which is similar to the \ili{Nepali} honorific construction  in its form and function. The construction uses an infinitival form of the lexical verb and a copular auxiliary. The function of the auxiliary is carried out up by the verb \emph{leŋma} \rede{be, become}. It is inflected intransitively  and shows agreement with the subject (S or A) of the semantic head.  The \ili{Nepali} source construction is built by adding an inflected form of a \isi{copula} (always third person, \emph{huncha/hunna/bhayo/bhaena} \rede{is/is not/was/was not}) to the infinitival form of the semantic head, which is used for both addressing people and talking about people. For instance, \emph{garnuhuncha} is the honorific way of saying both \rede{he does} and \rede{you do} in \ili{Nepali}. In Yakkha, it is not a fixed third person form of \emph{leŋma} that is added to the \isi{infinitive}, but the verb is inflected for second person, too, showing agreement with the S argument (see \Next, I have no data for transitive forms). Naturally, the first person is  impossible with honorifics. 

\ex. \ag. heʔne kheʔ-ma leks-a-ga=na?\\
where go-{\sc inf} be-{\sc pst-2sg=sg}\\
\rede{Where did you go?}
\bg. heʔnaŋ ta-ma leks-a-ga=na?\\
where.from come-{\sc inf} be-{\sc pst-2sg=nmlz.sg}\\
\rede{Where do you come from?}

This construction also has a corresponding \isi{imperative}, again analogous to the \ili{Nepali} construction, with the \isi{infinitive} and the third person \isi{optative} form of the \isi{auxiliary verb}  \emph{leŋma}, which is  \emph{leŋni} in Yakkha (see \Next), and \emph{hos} in \ili{Nepali}.\footnote{\emph{Toŋba} is millet beer that is served in a small wooden or nowadays aluminum barrel, with a lid and a pipe, hence the verb \rede{suck}.}

\ex.\ag. toŋba piʔ-ma leŋ-ni=ba.\\
beer{\sc } suck-{\sc inf} be{\sc [3sg]-opt=emph}\\
\rede{Please drink the beer.}
\bg. kinama ca-ma leŋ-ni=ba.\\
fermented\_soybean\_dish eat-{\sc inf} be{\sc [3sg]-opt=emph}\\
\rede{Please eat the \emph{kinama}.}

The functional domain of the honorific inflection in Yakkha slightly differs from the source language. While in \ili{Nepali} the honorific pronouns and verb forms are also used to address elders within the family and other respected, but very close people like the husband (not the wife), the Yakkha  honorific inflection rather signals respectful behavior that is connected to social distance (as far as could be told after my short stay in Dandagaun).


\section{The inflection of the copulas}\label{cop-infl}
 
In this section, the inflection of two copular verbs will be discussed. The inflectional categories are similar to those in the regular verbal inflection, i.e., person, \isi{polarity} and \textsc{tam}, but they show some formal and functional peculiarities. For instance, two prefix slots can be found in the copular inflection. Furthermore, some forms make a nonpast/future distinction, which is not found in the regular verbal inflection. As for the semantics of the inflectional forms presented here, I can only present tentative conclusions. Further examples of the use of the copulas are shown in \sectref{cop}. 

\subsection{The identificational copula (with a zero \isi{infinitive})}

The \isi{identificational copula} is used to express identification, \isi{equation} and class inclusion (see \Next). It does not have an infinitival form. The stem of this \isi{copula} is zero in the present \isi{tense} and \emph{sa} in the \isi{past tense}s. In the affirmative present forms, the \isi{copula} has overt forms only for speech-act participants, and even there it is optional. In the other tenses and in negated clauses, the \isi{copula} is obligatory. 


\ex. \ag. ka, ka khaʔla    ŋan=na=ba!      \\
\sc{1sg}  \sc{1sg}  like\_this \sc{cop.1sg.npst=sg=emph }\\
\rede{I, I am just like this!}\source{21\_nrr\_04.006} 
\bg. nda isa=ga u-cya gan?\\
{\sc 2sg} who{\sc =gen} {\sc 3sg.poss}-child  \sc{cop.2sg.npst}\\
\rede{Whose child are you?}
\bg. ka=ca chalumma ŋan.\\
\sc{1sg=add} second\_born\_daughter \sc{cop.1sg.npst} \\
\rede{I am also a second-born daughter.} 


An overview of the person and \isi{tense}/\isi{aspect} inflection of the \isi{identificational copula} is provided in \tabref{par-cop-ident}. In the present \isi{tense}, the copular inflection consists of suppletive forms that resemble the person markers. Deviations from  the verbal \isi{person marking} are, however, the dual forms starting in \emph{nci-} instead of \emph{-ci}, the plural forms starting in  \emph{si-} instead of  \emph{-i} and of course the complete zero marking for the third person in the affirmative. No stem could be identified in these forms; it probably had little phonological weight and got lost.\footnote{The development of identificational or equational copulas out of inflectional material is not unknown in \isi{Tibeto-Burman}; it is also found e.g., in Northern Chin \cite[9]{DeLancey2011_Notes}.}  A further idiosyncrasy of all present forms (affirmative and negative) is that they end in /n/, which does not seem to carry any semantic load. It is unlikely that this is a stem, because the person markers usually come as suffixes; at least one would have to explain why the order of stem and suffixes is reversed here.  Note that, due to the absence of specific markers, the dual forms of the third person and the first person \isi{inclusive} are identical.

Negation in the present forms is marked by the prefix \emph{me(N)-}, which is also found as \isi{negation} marker in nonfinite forms like infinitives and converbs. In the past, \isi{negation} is marked as in the regular verbal inflection, by the combinations of prefix and suffix \emph{N-...-n} or \emph{N-...-nin}. The \isi{nasal copying} known from the verbal inflection is found in the copular inflection too, with the same constraints applying as described in \sectref{sec-nasalcop}. The third person singular nonpast form \emph{menna} also functions as interjection \rede{No}. 


\begin{landscape}
\begin{table}[htp]
\begin{centering}
% \resizebox*{!}{.5\textheight}{
\resizebox{\textwidth}{!}{
\begin{tabular}{lllllll}
\lsptoprule
  & {\sc npst.aff} & {\sc npst.neg} & {\sc pst.I.aff} & {\sc pst.I.neg} & {\sc pst.II.aff} &  {\sc pst.II.neg} \\
\midrule
{\sc 1sg} & \it  ŋan  & \it  meʔ-ŋan & \it  sa-ŋ=na & \it  n-sa-ŋa-n=na & \it  sa-ya-ŋ=na & \it   n-sa-ya-ŋa-n=na\\
{\sc 1du.excl} & \it   nciŋan  & \it  me-nciŋan & \it  sa-ŋ-ci-ŋ=ha & \it  n-sa-n-ci-ŋa-n=na & \it  sa-ya-ŋ-ci-ŋ=ha & \it   n-sa-ya-n-ci-ŋa-n=na\\
{\sc 1pl.excl} & \it   siŋan  & \it  me-siŋan & \it  s-i-ŋ=ha & \it  n-s-i-ŋa-n=ha & \it  sa-i-ŋa-n=ha & \it  n-sa-i-ŋa-n=ha  \\
{\sc 1du.incl} & \it  ncin  & \it  me-ncin & \it  sa-ci=ha & \it  n-sa-n-ci-n=ha & \it  sa-ya-ci=ha & \it   n-sa-ya-n-cin=ha\\
{\sc 1pl.incl} & \it   sin  & \it  me-sin & \it  s-i=ha  & \it  n-s-i-n=ha & \it  sa-i=ha & \it  n-sa-i-n=ha\\
\midrule
{\sc 2sg} & \it   kan & \it  me-kan & \it  sa-ga=na 	& \it  n-sa-ga-n=na 			& \it  sa-ya-ga=na  & \it  n-sa-ya-gan=na\\
{\sc 2du} & \it   ncigan & \it  me-cigan & \it  sa-cig=ha 	& \it   n-sa-n-ci-ga-n=ha	& \it  sa-ya-ci-g=ha & \it  n-sa-ya-n-ci-gan=ha\\
{\sc 2pl} & \it   sigan & \it  me-sigan & \it   s-i-g=ha 	& \it   	n-s-i-ga-n=ha		& \it  sa-i-g=ha	& \it n-sa-i-ga-n=ha  \\
\midrule
{\sc 3sg} &   —   & \it  men  & \it  sa=na & \it  n-sa-n=na & \it  sa-ya=na & \it  n-sa-ya-n=na \\
{\sc 3du} &   — & \it  mencin  & \it  sa-ci=ha & \it  n-sa-n-ci-n=ha & \it   sa-ya-ci=ha & \it  n-sa-ya-n-ci-n=ha\\%mencinha was constructed in analogy
{\sc 3pl} & —   & \it  men(=ha=ci)  & \it  n-sa=ha=ci & \it  n-sa-nin=ha & \it   n-sa-ya=ha & \it  n-sa-ya-nin=ha \\
\lspbottomrule
\end{tabular}
}
\end{centering}
\caption{Person and \isi{tense}/\isi{aspect} inflection of the identificational copula}\label{par-cop-ident}
\end{table}
\end{landscape}



The \isi{person marking} in the past \isi{tense} forms is more regular than in the nonpast, since they have a stem to which the regular person markers can attach. The \isi{identificational copula} distinguishes five inflectional series in the \isi{past tense}s. There is the the simple past (\rede{Past I}), expressed by the past stem \emph{sa} and the regular person inflection (\emph{sa} is reduced to \emph{s} when followed by \emph{-i} for {\sc 1/2pl}). This past stem can also host the past marker \emph{-a \ti -ya}, and it is not clear yet what the semantic effect of this is, hence this category is simply called \rede{Past II} here (see \tabref{par-cop-ident}). Furthermore there is the perfect (not included in the table), which is marked regularly by the already familiar perfect marker \emph{-ma \ti -mi} preceding the person inflection of the Past II forms (e.g., \emph{sayamaŋna}  for first person singular). What is  different from the regular intransitive marking is the occurrence of the \isi{negation} marker \emph{-nin} in third person negated past forms, which is otherwise only found in the transitive paradigms. The fourth and fifth inflectional series are only attested in negated forms; they are discussed below.  


The simple past forms are the most frequently used \isi{tense} forms of the \isi{identificational copula} in the current corpus. Unfortunately, the  analysis of the \isi{past tense}s cannot be corroborated by much natural data, so that the precise answer to the question of their application has to be left for a later stage (but see \Next for a few examples).

\ex. \ag.uŋci=go     miyaŋ mam=ha n-sa.\\
{\sc 3nsg=top} a\_little big{\sc =nmlz.nsg} {\sc 3pl-cop.pst}\\
\rede{They were a bit older.} 
\bg.uŋ tuknuŋ    luŋmatuktuk=na        sa-ya-ma.\\
{\sc 3sg} completely loving{\sc =nmlz.sg} {\sc cop.pst[3sg]-pst-prf}\\
\rede{She was a very loving person.}\footnote{This example also shows the narrative function of the Perfect \isi{tense}; it frequently occurs in stories (see \sectref{prf}).}  \source{01\_leg\_07.061 }
\bg.pyak encho         ka  miya   sa-ya-ŋ=niŋa.\\
much long\_ago {\sc 1sg} small {\sc cop.pst-pst-1sg=ctmp}\\
\rede{Long ago, when I was a child, ...} \source{42\_leg\_10.002}
 
 
A further negated inflectional paradigm can be found for the present \isi{tense} (only {\sc 1/2pl} forms are attested, see \Next) and for Past I and II (see \tabref{meti}). These forms are marked by attaching a prefix \emph{ta- \ti ti-} to the copular stem, to the right of the \isi{negation} prefix, which is  \emph{me-}, (not \emph{N-} as in the past forms shown above). This is the only instance of a second prefixal slot in the entire verbal inflection.\footnote{A further puzzle is that the attested nonpast forms are identical to the corresponding forms in Past I, but it can be explained by the absence of a dedicated nonpast marker and the deletion of {\sc pst} \emph{-a} due to {\sc 1/2pl} \emph{-i}.}  Corresponding affirmative forms are not attested. Superficially, the semantics of these forms are equivalent to the forms shown in \tabref{par-cop-ident}, see \Next[a]; so the tentative conclusion is that this prefix only has an emphatic function.

\ex. \ag.kaniŋ hironi me-ti-siŋan?\\
{\sc 1sg} bollywood\_heroine {\sc neg-emph-cop.npst.1pl}\\
\rede{Aren't we Bollywood heroines?} (same: \emph{mesiŋan})
\bg.elaba=ci=ŋa n-lu-ks-u-ci:     nniŋda yakkhaba    me-ti-sigan=ha.\\
a\_clan{\sc =nsg=erg} {\sc 3pl.A-}tell{\sc -prf-3.P-3nsg.P}  {\sc 2pl} Yakkha\_\isi{person} {\sc neg-emph-cop.npst.2pl=sg}\\
\rede{The Elabas told them: you are not Yakkhas.} \source{39\_nrr\_08.07 }

\begin{table}[htp]
\begin{centering}
\begin{tabular}{lll}
\lsptoprule
  & {\sc neg.pst.I.emph}  & {\sc neg.pst.II.emph} \\
\midrule
{\sc 1sg} & \it  me-ta-sa-ŋa-n=na  & \it  me-ta-sa-ya-ŋa-n=na\\
{\sc 1du.excl} & \it  me-ta-sa-n-ci-ŋa-n=ha  & \it  me-ta-sa-ya-n-ci-ŋa-n=ha\\
{\sc 1pl.excl} & \it   me-ti-s-i-ŋa-n=ha & \it  me-ta-sa-i-ŋa-n=ha\\
{\sc 1du.incl} & \it  me-ta-sa-n-ci=ha & \it  me-ta-sa-ya-n-ci-n=ha\\
{\sc 1pl.incl} & \it  me-ti-s-i-n=ha & \it  me-ta-sa-i-n=ha\\
\midrule
{\sc 2sg} & \it  me-ta-sa-ga-n=na & \it  me-ta-sa-ya-ga-n=na\\
{\sc 2du} & \it  me-ta-sa-n-ci-ga-n=ha & \it  me-ta-sa-ya-n-ci-ga-n=ha\\
{\sc 2pl} & \it   me-ti-s-i-ga-n=ha & \it  me-ta-sa-i-ga-n=ha\\
\midrule
{\sc 3sg} & \it  me-ta-sa-n=na & \it  me-ta-sa-ya-n=na\\
{\sc 3du} & \it  me-ta-sa-n-ci-n=ha & \it  me-ta-sa-ya-n-ci-n=ha\\
{\sc 3pl} & \it  me-ta-sa-nin=ha & \it  me-ta-sa-ya-nin=ha\\
\lspbottomrule
\end{tabular}
\end{centering}
\caption{Alternative past and \isi{negation} inflection of the copula}\label{meti}
\end{table}



\subsection{The existential verb \emph{wama}}

The \isi{existential verb} \emph{wama}  is probably the only verb that has a purely static temporal profile, i.e., without containing the notion of the inception of the state (exactly as in Belhare, see \citealt[212]{Bickel1996Aspect}). It may translate as \rede{be, exist, live, stay} and is found in copular frames expressing location, existence and also in clauses with adjectival predicates (see \Next for examples). In many other Kiranti languages, a cognate of the verb \emph{yuŋma} \rede{sit, live, exist} has this function. In Yakkha,  \emph{yuŋma} is restricted to the meaning \rede{sit (down)}. A paradigm of  various \isi{tense} inflections of \emph{wama} is provided in \tabref{par-wa-ma} on page \pageref{par-wa-ma}.

\ex. \ag.uhiŋgilik wɛʔ=na.\\
alive be{\sc [npst;3sg]=nmlz.sg}\\ 
\rede{She is alive.}
\bg.n-na n-nuncha ŋ-waiʔ=ya=ci?\\
{\sc 2sg.poss-}elder\_sister  {\sc 2sg.poss-}younger\_sibling {\sc 3pl-}exist{\sc [npst]=nmlz.nsg=nsg}\\
\rede{Do you have sisters?}


The stem \emph{wa} of this verb has several allomorphs. There are the nonpast allomorphs \emph{wai(ʔ) \ti wɛʔ}, which have resulted from a contraction of the stem and the nonpast marker (/wa-meʔ/), diachronically. Such processes are also found in other verbs; take e.g., the underlying form /leŋdimeʔna/ which is also found as [leŋdeʔna] in fast speech. In the verb of existence, however, this contraction is lexicalized, since it may also host the nonpast marker. Marked by nonpast \emph{-meʔ}, these forms have \isi{continuative} or future semantics, extending the state beyond the time of the utterance context, as shown in \Next. Furthermore, the verb has an allomorph \emph{wai} (bisyllabic) in the past forms. 

\exg.nna  tas-wa=na=be  yog-a-n-u-m,      nna=be,    nniŋ=ga       mamu wa-meʔ=na.\\
that reach{\sc -npst=nmlz.sg=loc} search{\sc -imp-pl-3.P-2pl.A} that{\sc =loc} {\sc 2pl.poss=gen} girl be{\sc [3sg]-npst=sg}\\
\rede{Search there where it lands (a clew of thread), your girl will (still) be there.} \source{22\_nrr\_05.095}


Since with the absence of \emph{-meʔ} in the nonpast the {\sc 1/2.pl} forms (marked by the suffix \emph{-i}) became identical in the present and in the simple past, they have received further marking: instead of the expected \emph{wa-i=ha} for {\sc 1pl.incl.npst} or \emph{wa-i-g=ha} for {\sc 2pl.npst} one finds \emph{wa-i-niti-ha} and \emph{wa-i-niti-gha}, respectively. The marker \emph{-niti} is not attested elsewhere in the verbal morphology.\footnote{There is a second person plural suffix \emph{-ni} in the \isi{imperative} paradigm, and we have seen above that there is a prefix \emph{ta- \ti ti-} in some forms of the past inflection of the \isi{identificational copula}. It is possible that there is an etymological link between \emph{-niti} and these affixes, but any claims in this regard would be highly speculative.}

In the \isi{negation} paradigm, forms with a suppletive stem  \emph{ma} exist alternatively to forms with \emph{wai}, throughout all \isi{tense} forms (\emph{mai} in the plupast, in analogy to affirmative \emph{wai}). The most commonly heard form is the third \isi{person} \emph{manna/manhaci}, stating the absence of something (see \Next[a]).\footnote{Since the other person forms show an initial geminate, I assume that the third person underwent formal reduction due to frequent use. The form \emph{ma(n)} is also the base for \isi{postpositions} and conjunctions. Combined with the adverbial \isi{clause linkage} marker \emph{=niŋ} it has  developed into the privative \isi{case} \emph{maʔniŋ} \rede{without}, and combined with the clause linkage marker  \emph{=hoŋ} it yields  the clause-initial conjunction \emph{manhoŋ} \rede{otherwise}.} As in the affirmative forms, attaching the nonpast marker results in a future reading (see  \Next[c]). I have one contrastive example suggesting that \emph{ma}-forms are not interchangeable with \emph{wa} (and its allomorphs), and thus that \emph{ma} is not simply an allomorph of \emph{wa} (compare \Next[b] and (c)). Unfortunately, the current data set is not sufficient to determine the exact difference between these two negated stems. The \emph{ma}-series is more frequent in my Yakkha corpus. 

\ex. \ag.sambakhi=ci ma-n=ha=ci.\\
potato{\sc =nsg} exist{\sc .neg-neg=nmlz.nsg=nsg}\\
\rede{There are no potatoes.}
\bg.ŋ-wa-meʔ-ŋa-n.\\
{\sc neg-}stay{\sc -npst-1sg-neg}\\
\rede{I will not stay.}
\bg.      nda ta-me-ka=niŋ ka m-ma-me-ŋa-n.\\
{\sc 2sg} come{\sc -npst-2=ctmp} {\sc 1sg} {\sc neg-}be{\sc -npst-1sg-neg}\\
\rede{I will not be here when you come.}


In addition to the nonpast inflections, three past series were found, formed similarly to the regular verbal inflection: a simple past formed by \emph{-a},  a perfect formed by adding \emph{-ma \ti -mi} to the simple past, and another past \isi{tense} (\rede{Past II}, yet unanalyzed) formed by adding \emph{-sa \ti -si} to the simple past forms. This Past II  has no parallel forms in the regular verbal inflectional paradigm.  The simple past again seems to be the default choice (see \Next[a]), and the other two are more specific. The perfect is found in narratives, relating to events that have some relevance for the story, i.e., in sentences setting the stage for further information to come (see \Next[b] and \Next[c], from the beginning of a narrative and from childhood memories, respectively). The Past II forms refer to events that preceded another salient event in the past. In \Next[d], the speaker  refers to the time when people came to propose a marriage to her daughter, but the conversation  takes place already after the wedding, which was the main topic of the conversation. 

\ex. \ag.     ŋkhaʔniŋ eko paŋ=ca         m-ma-ya-n=niŋa       tumhaŋ=ŋa   paŋ  cog-uks-u.\\
that\_time one house{\sc =add} {\sc neg-}exist{\sc [3sg]-pst-neg=ctmp} Tumhang{\sc =erg} house make{\sc -prf-3.P}\\
\rede{Back then, when there was not a single house, Tumhang made a house.} \source{27\_nrr\_06.038 }
\bg.uŋci=nuŋ   pyak yaŋ  m-ma-ya-ma-n.\\
{\sc 3nsg=com} much money {\sc neg-}exist{\sc [3sg]-pst-prf-neg}\\
\rede{They did not have much money.} \source{01\_leg\_07.304}
\bg.buŋga-bic=pe      wa-i-mi-ŋ.\\
Bunga-beach{\sc =loc} live{\sc -1pl-prf-excl}\\
\rede{We lived at Bunga Beach (a place in Singapore).} \source{13\_cvs\_02.062}
\bg. njiŋda m-mai-sa-n-ci-ga-n=ha.\\
{\sc 2du} {\sc neg-}exist{\sc -pst-neg-du-2-neg=nsg}\\
\rede{The two of you had not been here (when they came).} \source{36\_cvs\_06.306}        


Subjunctive forms can be found as well, both in the nonpast \isi{subjunctive }\Next[a], e.g., for hortatives, and in the past \isi{subjunctive }\Next[b], e.g., for irrealis clauses. The past subjunctive is identical to the simple past.  \tabref{par-wa-ma} shows the  inflections with a few forms marked by \rede{(?)}, which means that they  have been reconstructed, but not attested.

\ex. \ag.kaniŋ wa-i.\\
{\sc 1pl} be{\sc -1pl[sbjv]}\\
\rede{Let us live/stay (here).}
\bg.a-ma=nuŋ a-na=ŋa y-yog-a-n=niŋ=bi ka hensen ŋ-wa-ya-ŋa-n=bi.\\
{\sc 1sg.poss-}mother{\sc =com} {\sc 1sg.poss-}eZ{\sc =erg} {\sc neg-}search{\sc -sbjv-neg=ctmp=irr} {\sc 1sg} nowadays {\sc neg-}exist{\sc -sbjv-1sg-neg=irr} \\
\rede{If my mother and sister had not looked (for me), I would not be alive now.} \source{42\_leg\_10.052}

%

\begin{sidewaystable}
\resizebox{16cm}{!}{
\begin{tabular}{lllllll}
\lsptoprule
  & {\sc npst.aff} & {\sc npst.neg} & {\sc pst.aff} & {\sc pst.neg} & {\sc pst.II.aff}  &  {\sc pst.II.neg}\\
\midrule
{\sc 1sg} & \it  waiʔ-ŋa=na & \it  ŋ-waiʔ-ŋa-n=na & \it  wa-ya-ŋ=na & \it  ŋ-wa-ya-ŋa-n=na & \it  wai-sa-ŋ=na & \it  ŋ-wai-sa-ŋa-n=na \\
 & \it   & \it  m-ma-ŋa-n=na & \it   & \it  m-ma-ya-ŋa-n=na & \it   & \it  m-mai-sa-ŋa-nna\\
{\sc 1du.excl} & \it   wai-ŋ-ci-ŋ=ha	 & \it  ŋ-wai-n-ci-ŋa-n=ha & \it  wa-ya-ŋ-ci-ŋ=ha	 & \it  ŋ-wa-ya-n-ci-ŋa-n=ha & \it  wai-sa-ŋ-ci-ŋ=ha & \it   ŋ-wai-sa-n-ci-ŋa-n=ha\\
 & \it   & \it  m-ma-n-ci-ŋa-n=ha & \it   & \it  m-ma-ya-n-ci-ŋa-n=ha & \it   & \it  m-mai-sa-n-ci-ŋa-n=ha\\
{\sc 1pl.excl} & \it   wai-niti-ŋ=ha & \it  ŋ-wai-niti-ŋa-n=ha (?) & \it  wa-i-ŋha	 & \it  ŋ-wa-i-ŋa-n=ha & \it  wa-i-si-ŋ=ha & \it  ŋ-wa-i-si-ŋa-n=ha \\
 & \it   & \it  m-ma-i-ŋa-n=ha & \it   & \it  m-ma-i-ŋa-n=ha & \it   & \it  m-ma-i-si-ŋa-n=ha\\
{\sc 1du.incl} & \it  wai-ci=ha		 & \it  ŋ-wai-n-ci-n=ha			 & \it  wa-ya-ci=ha		 & \it  ŋ-wa-ya-n-ci-n=ha & \it  wai-sa-ci=ha & \it   ŋ-wai-sa-n-ci-n=ha\\
 & \it   & \it  m-ma-n-ci-n=ha & \it   & \it  m-ma-ya-n-ci-n=ha & \it   & \it  m-mai-sa-n-ci-n=ha\\
{\sc 1pl.incl} & \it  wai-niti=ya & \it  ŋ-wai-niti-n=ha (?) & \it  wa-i=ya & \it  ŋ-wa-i-n=ha & \it  wai-s-i=ha & \it  ŋ-wa-i-si-n=ha\\
 & \it   & \it  m-ma-i-n=ha & \it   & \it  m-ma-i-n=ha (?) & \it   & \it  m-mai-s-i-n=ha\\
\midrule
{\sc 2sg} & \it   wai-ka=na & \it  ŋ-wai-ka-n=na & \it  wa-ya-ga=na & \it  ŋ-wa-ya-ga-n=na & \it  wai-sa-ga=na & \it  ŋ-wai-sa-ga-n=na\\
 & \it   & \it  m-ma-ga-n=na & \it   & \it  m-ma-ya-ga-n=na & \it   & \it  m-mai-sa-ga-n=na\\
{\sc 2du} & \it  wai-ci-g=ha & \it  ŋ-wai-n-ci-ga-n=ha & \it  wa-ya-ci-g=ha & \it  ŋ-wa-ya-n-ci-ga-n=ha & \it  wai-sa-ci-g=ha & \it   ŋ-wai-sa-n-ci-ga-n=ha\\
 & \it   & \it  m-mai-n-ci-ga-n=ha & \it   & \it  m-ma-ya-n-ci-ga-n=ha & \it   & \it  m-mai-sa-n-ci-ga-n=ha\\
{\sc 2pl} & \it  wai-niti-g=ha & \it  ŋ-wai-niti-ga-n=ha (?)& \it  wa-i-g=ha	 & \it  ŋ-wa-i-ga-n=ha & \it  wai-s-i-g=ha & \it   ŋ-wai-s-i-ga-n=ha\\
 & \it   & \it  m-ma-i-ga-n=ha & \it   & \it  m-ma-i-ga-n=ha (?) & \it   & \it  m-mai-s-i-ga-n=ha\\
\midrule
{\sc 3sg} & \it   waiʔ=na	 & \it  ŋ-wai-n=na & \it  wa-ya=na	& \it  ŋ-wa-ya-n=na & \it  wai-sa=na & \it   ŋ-wai-sa-n=na\\
 & \it   & \it  m-ma-n=na & \it   & \it  m-ma-ya-n=na & \it   & \it  m-mai-sa-n=na\\
{\sc 3du} & \it  wai-ci=ha & \it  ŋ-wai-n-ci-n=ha	 & \it  wa-ya-ci=ha & \it  ŋ-wa-ya-n-ci-n=ha & \it  wai-sa-ci=ha & \it  ŋ-wai-sa-n-ci-n=ha\\
 & \it   & \it  m-ma-n-ci-n=ha & \it   & \it  m-ma-ya-n-ci-n=ha & \it   & \it  m-mai-sa-n-ci-n=ha\\
{\sc 3pl} & \it   ŋ-waiʔ=ha=ci & \it  ŋ-wai-nin=ha & \it  ŋ-wa-ya-ci & \it  ŋ-wa-ya-n=ha=ci & \it  ŋ-wai-s=ha=ci & \it  ŋ-wai-sa-n=ha=ci \\
 & \it   & \it  m-ma-nin=ha & \it   & \it  m-ma-ya-n=ha=ci & \it   & \it  m-mai-sa-nin=ha\\
\lspbottomrule
\end{tabular}
}
\caption{Person and \isi{tense}/\isi{aspect} inflection of  \emph{wama} \rede{be, exist, live}}\label{par-wa-ma}
\end{sidewaystable}

%\pagestyle{scrheadings}



\section{Further markers}\label{furtherverbal}

Two further markers that do not fit in the previous sections have to be mentioned. First, there is a suffix \emph{-a}, attached to \ili{Nepali} verbal roots when they occur as loans with Yakkha light verbs, as shown in \Next.

\ex. \ag.khic-a cog-u!\\
press{\sc -nativ} do{\sc -3.P[imp]}\\
\rede{Record it!}
\bg.i=ha=ca im-ma por-a n-joŋ-me-ŋa-n.\\
what{\sc =nmlz.nc=add} buy{\sc -inf} have\_to{\sc -nativ} {\sc neg-}do{\sc -npst-1sg-neg}\\
\rede{I do not have to buy anything.} \source{28\_cvs\_04.187}

Another marker functions like a complement verb, despite being  a bound morpheme. The marker \emph{-les} states that the subject has knowledge or skills and is able to perform the activity denoted by the lexical verb. As the suffix has the typical structure of a verbal stem, it has probably developed out of a verb. However, there is no verb with the stem \emph{les} synchronically.

\ex. \ag.phuama=ŋa=ca        cek-les-wa,                hau!\\
last\_born\_girl{\sc =erg=add} speak-know{\sc -npst[3A;3.P]} {\sc excla}\\
\rede{Phuama also knows how to speak, ha!} \source{36\_cvs\_06.503 }
\bg.        pheŋ-les-wa-m-ci-m-ŋa.\\
plough-know{\sc -npst-1pl.A-3nsg.P-1pl.A-excl}\\
\rede{We know how to plough (with oxen).} \source{28\_cvs\_04.152}

\section{Non-finite forms}\label{nonfiniteforms}

Non-finite forms in Yakkha include the \isi{infinitive} marked by \emph{-ma} and occasionally \emph{-sa} (attested only in negated complement constructions with \emph{yama} \rede{be able to}), a nominalization in \emph{-khuba} (which constructs nouns and participles with S or A role) and several converbal forms, all attached directly to the stem: the \isi{supine converb} in \emph{-se}, the \isi{simultaneous converb} in \emph{-saŋ} and the \isi{negative converb} marked by \emph{meN-...-le} (discussed in Chapter \ref{adv-cl}).

The \isi{infinitive} occurs in infinitival complement constructions and in the deontic construction (all discussed in Chapter \ref{compl}). The latter allows for some further marking such as \isi{negation} by \emph{men-}, the nominalizing clitics \emph{=na} and \emph{=ha}, and the nonsingular marker \emph{=ci} to indicate nonsingular objects (cf. \sectref{obl}). From a functional perspective, the \isi{infinitive} with a deontic reading is finite, as it can stand independently as an utterance and does not rely upon another syntactic unit. 

Occasionally, the \isi{infinitive} is also found in infinitival adverbial clauses and in adverbial clauses that usually contain inflected verbs, i.e., clauses marked in \emph{=niŋ(a)} (cotemporal events), \emph{=hoŋ} (sequential events) and \emph{bhoŋ} (\isi{conditional clauses}). This is the \isi{case} when the reference of the arguments is not specified, i.e., in general statements, best rendered with \rede{when/ if one does X, ...} (see Chapter \ref{adv-cl}). 

Note that there is a special \isi{negation} marker \emph{me(N-)} which is only found in nonfinite forms and in the inflection of the \isi{copula}, i.e., in  infinitives, the nominalization in \emph{-khuba} and  the \isi{negative converb}. Except for this \isi{negation} marker and the clitics on the deontic infinitives, no \isi{tense}/\isi{aspect}, \isi{mood} or \isi{person marking} is found on \isi{non-finite forms}.
