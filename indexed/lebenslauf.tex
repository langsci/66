
%\chapter*{Lebenslauf}\addcontentsline{toc}{chapter}{Lebenslauf}\markboth{Lebenslauf}{Lebenslauf}
\chapter*{Lebenslauf}\markboth{Lebenslauf}{Lebenslauf}
\thispagestyle{empty}
\pagestyle{empty}

\parindent0pt
%\renewcommand{\tabularxcolumn}[1]{>{\raggedright\arraybackslash}p{#1}}


\begin{tabular}{@{}ll}

Geburtsdatum und -ort:&		17.09.1979\\
&							Frankfurt (Oder)\\
%Wohnort: &Kiel, Deutschland\\
%Staatsangehörigkeit:&		deutsch\\
E-Mail:&					post@dianaschackow.de\\
Website:&					http://www.dianaschackow.de\\
\end{tabular}


\section*{Ausbildung}

\begin{tabularx}{\textwidth}{@{}p{9em}X}
2013 – 2014& Doktoratsstudium in Allgemeiner Sprachwissenschaft; Universität Zürich (Prädikat: \emph{summa cum laude})\\
2009 –  2013&	Doktoratsstudium in Allgemeiner Sprachwissenschaft; Universität Leipzig\\
2004 – 2008&	Magister der Sprachwissenschaft, Tibetologie und Indologie; Universität Leipzig\\
2000 – 2004&	Studium der Linguistik und Zentralasienstudien; Humboldt-Universität zu Berlin und Technische Universität Berlin (Zwischenprüfung)\\
1999&				Allgemeine Hochschulreife; Frankfurt (Oder) \\
\end{tabularx}


\section*{Berufserfahrung}

\begin{tabularx}{\textwidth}{@{}p{9em}X}
2012&		Unterricht: Strukturkurs Yakkha, Universität Leipzig\\
2011& Unterricht: Crashkurs \ili{Nepali}, Universität Leipzig\\
2008 – 2009&		Wissenschaftliche Hilfskraft im \ili{Chintang} and \ili{Puma} Documentation Project (Universität Leipzig)\\
2005 – 2008&	Studentische Hilfskraft im \ili{Chintang} and \ili{Puma} Documentation Project (Universität Leipzig)\\
2005 –  2007&		Studentische Hilfskraft für diverse Projekte am \textsc{mpi} für Evolutionäre Anthropologie, Leipzig\\
 2003&			Forschungspraktikum am \textsc{mpi} für Kognitions- und Neurowissenschaften, Leipzig (Betr.: Ina Bornkessel)\\
\end{tabularx}


\section*{Studien- und Arbeitsaufenthalte im Ausland}

\begin{tabularx}{\textwidth}{@{}p{9em}X}
2009, 2010, 2011, 2012&		Feldforschung zum Yakkha, \isi{Nepal}\\
2006, 2008&					Feldforschung zum \ili{Puma}, \isi{Nepal}; im \ili{Chintang} and \ili{Puma} Documentation Project \\
2003&						Intensivkurs Tibetisch, Lotsawa Rinchen Zangpo Translator Programme, Indien\\
\end{tabularx}


\section*{Publikationen}

\begin{tabularx}{\textwidth}{@{}p{9em}X}
2013&				Yakkha-Nepali-English digital dictionary (\url{http://dianaschackow.de/?nav=dictionary}) \\
2012 (Hauptautor)&	Morphosyntactic properties and scope behavior of ‘subordinate’ clauses in \ili{Puma} (Kiranti). In: Gast, V. \& H. Diessel (Eds.), Clause-combining in cross-linguistic perspective. Berlin: Mouton de Gruyter. (mit B. Bickel, V.S. Rai, M. Gaenszle, N. Sharma, A. Rai, S.K. Rai)\\
2012&				Referential hierarchy effects in three-argument constructions in Yakkha. In: Linguistic Discovery 10.3.\\
2011 (Ko-Autor)&Binomials and the Noun-to-Verb Ratio in \ili{Puma} Rai Ritual Speech. Anthropological Linguistics 53.4 (M. Gaenszle (Hauptautor), B. Bickel, N. P. Sharma, J. Pettigrew, A. Rai, S. K. Rai)\\
2009 (Ko-Autor)&	Puma-Nepali-English dictionary
(mit B. Bickel, V.S. Rai, M. Gaenszle, N. Sharma, A. Rai, S.K. Rai, S. Günther). Kirat \ili{Puma} Rai Tupkhabangkhala, Lalitpur, Kathmandu.\\
2009 (Ko-Autor)& Audiovisual \ili{Puma} corpus. (mit B. Bickel, V.S. Rai, S. K. Rai, N. P. Gautam (Sharma), A. Rai, M. Gaenszle) Elektronische Datenbank. DoBeS-Archiv.\\
\end{tabularx}


\section*{Stipendien}

\begin{tabularx}{\textwidth}{@{}p{9em}X}
2013&			Reisestipendium; National Science Foundation und International Conference for Language Documentation and Conservation, Universität von Hawai'i\\
2012 – 2013&		Individual Studentship Grant (No. IGS 154), Endangered Languages Documentation Programme; SOAS, London)\\
2009 – 2012&		Graduiertenstipendium des Landes Sachsen\\
2006, 2008, 2011 & Reisestipendien des DAAD für ein Praktikum und zwei Forschungsreisen nach \isi{Nepal}\\
\end{tabularx}

