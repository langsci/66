\subsection{The reflexive construction}\label{refl}

Yakkha does not have reflexive pronouns. The reflexive is constructed by a \isi{complex predicate} with the V2 \emph{-ca} \rede{eat}. It indicates that the A and P argument of the predicate have identical reference.  The resulting verb gets detransitivized with regard to \isi{case} and \isi{person marking}, as shown in \Next. This construction can only express complete coreferentiality, so that propositions like \rede{I saved us} can neither be expressed by the verbal morphology nor by the reflexive derivation  in Yakkha.
The various other functions of this V2 are treated in \sectref{V2-eat}. Apart from reflexive constructions, it also occurs in many lexicalized predicates with classical middle semantics, such as grooming and social interactions. Some of the \isi{labile verbs} that are discussed in \sectref{labile} also show reflexive semantics when they are inflected intransitively, without attaching the reflexive marker. 

\exg. nda (aphai) moŋ-ca-me-ka=na.\\
\sc{2sg} (self) beat-\sc{V2.eat-npst-2=nmlz.sg}\\
\rede{You beat yourself.}
 
The examples below show that the reflexive can also apply to a quantified \isi{noun phrase} \Next[a], to a question pronoun \Next[b], and to negated propositions \Next[c]. There are no dedicated negative pronouns in Yakkha; \isi{negation} is constructed by a question pronoun with the \isi{additive focus} \isi{clitic} \emph{=ca} and verbal \isi{negation} markers. In irrealis (question and \isi{negation}) contexts, both singular and nonsingular inflection is possible, depending on the potential referents that the speaker has in mind (or what the speaker assumes the addressee has in mind).

\ex. \ag.ghak sar n-so-ca-ya.\\
		all teacher {\sc 3pl}-look-{\sc V2.eat-pst}\\
	\rede{All teachers looked at themselves.}
 	\bg.\label{uchik}isa u-chik ekt-a-ca-ya=na.\\
	who {\sc 3sg.poss}-hate make\_break{\sc [3sg]-pst-V2.eat-pst=nmlz.sg}\\
	\rede{Who is angry at himself?}
	 \bg.isa=ca n-so-ca-ya-n=ha=ci.\\
	who{\sc =add}  {\sc neg}-look-{\sc V2.eat-pst-neg=nmlz.nsg=nsg}\\
	\rede{No one looked at themselves.}


In \isi{three-argument verbs}, there are two potential candidates for coreference with A. Whether A controls G or T is a matter of the original frame of the verb. For double object verbs, coreference with T is ungrammatical (see \Next[a]), while coreference with G is fine (see \Next[b]).\footnote{It seems crosslinguistically unexpected that the coreference of A and G is accepted, while the coreference of A and T is ungrammtical. Kazenin states the implicational universal that \rede{[...] if a language allows verbal marking of indirect reflexives, it allows verbal marking of direct reflexives as well.} \citep[918]{Kazenin2001_Verbal}. } It is not possible for T and G to be coreferential in the reflexive derivation, i.e., to express propositions like \rede{I showed him to himself (in the mirror)}.

\ex. \ag. *ka ama (aphai phoʈo=be)  soʔmen-ca-me-ŋ=na.\\
\sc{1sg[erg]} mother (self photo\sc{=loc}) show-\sc{V2.eat-npst-1sg=nmlz.sg} \\
Intended: \rede{I show myself to mother (on the photo).}
	\bg. ka (aphai) cokleʈ pin-ca-me-ŋ=na.\\
		\sc{1sg[erg]} (self) sweet give-\sc{V2.eat-npst-1sg=nmlz.sg}\\
\rede{I give myself a sweet.}


As for \isi{three-argument verbs} of the \isi{indirective frame}, A can be coreferential with the argument in the \isi{nominative}. This is illustrated by the verb \emph{thumma} \rede{tie to}.  In \Next[a] the  frame is shown for \isi{comparison}. Example \Next[b] shows the reflexive, where the \isi{locative} G argument is retained and the \isi{nominative} T argument is coreferential with A, and thus unexpressed. The A argument changes its \isi{case} marking from \isi{ergative} to \isi{nominative}.  

\ex. \ag.a-ppa=ŋa on siŋ=be thund-u=na.\\
		{\sc 1sg.poss}-father{\sc =erg} horse tree{\sc =loc} tie{\sc -3.P[pst]=nmlz.nsg}\\
	\rede{My father tied the horse to the tree.}
 	\bg.a-nuncha siŋ=be thun-ca-meʔ=na.\\
{\sc 1sg.poss}-younger\_sibling tree\sc{=loc} tie-\sc{V2.eat[3sg]-npst=nmlz.sg}\\
	\rede{My brother ties himself to a tree.}


Reflexivization is possible with verbs of the experiencer-as-pos\-ses\-sor frame (cf. \sectref{exp} and \sectref{nv-comp-poss}), too, as shown above in example \ref{uchik}.


In complex sentences, e.g., in embedded complement clauses, the reflexive V2 can mark the main verb, although the reflexive semantics actually apply to the predicate in the embedded clause (see \Next).\footnote{The complement-taking predicate \emph{miʔma} means \rede{want} when it takes infinitival complements; its meaning in other constructions is \rede{hope, like, think} (see also Chapter \ref{compl}).} This is, however, only possible in the type of complement construction that embeds infinitives, where the embedded and the main clause  S/A argument are necessarily coreferential.\footnote{Note that the \isi{ergative} on A is retained in this construction, which suggests that the A argument belongs to the embedded transitive clause (a \isi{case} of backward control).} 


\exg. uŋ=ŋa phoʈo cok-ma min-ca-meʔ=na.\\
		{\sc 3sg=erg} photo make{\sc -inf} want-{\sc V2.eat[3sg]-npst=nmlz.sg}	\\
	\rede{She wants to take a photo of herself.}
 	



\subsection{The reciprocal construction}\label{refl3}

The reciprocal is constructed by attaching the suffix \emph{-khusa} to the stem of the lexical verb and employing the verb \emph{cama} \rede{eat} as auxiliary (see \Next[a]). The lexical verb and the auxiliary have to be adjacent, but the \isi{degree} of morphological fusion is lower than in the \isi{reflexive construction} and complex predication in general. Inflectional prefixes attach to the auxiliary, not to the lexical verb. As the reciprocal expresses mutual actions, it is characterized by at least two participants that both simultaneously have the role of actor and undergoer. The reciprocal participants are fused into one \isi{noun phrase}. The construction only applies to transitive verbs, and it always formally detransitivizes the predicate, by assigning the \isi{nominative} \isi{case} to the A arguments and by inflecting the auxiliary intransitively, even when the lexical verb is a three-argument verb, as in  \Next[b]: here, the G argument  is coreferential with A and hence it is omitted, while the T remains on the surface, retaining the \isi{case} marking of its frame of \isi{argument realization} (unmarked \isi{nominative} in the \isi{double object frame}). Contexts where reciprocals of double object verbs have coreferential A and T arguments are hard to imagine, and those proposed were rejected (see ungrammatical \Next[c]). 
 
Inherently reciprocal verbs such as \emph{tupma} \rede{meet}, \emph{tuma} \rede{fight}, \emph{khima} \rede{quarrel} and \emph{cuŋma} \rede{wrestle} are intransitive in Yakkha; they do not permit the reciprocal operation. 
 

\ex. \ag. kanciŋ [...] sok-khusa=se ca-ya-ŋ-ci-ŋ.\\
\sc{1du}  [...] look-\sc{recip=restr} eat\sc{.aux-pst-excl-du-excl}\\
\rede{We (dual, excl) just looked at each other.} (A=P) \source{40\_leg\_08.070}
\bg. kanciŋ phuŋ pi-khusa ca-me-ci=ha.\\
\sc{1du} flower give-\sc{recip} eat\sc{.aux-npst-1du=nmlz.nsg}\\
\rede{We (dual, incl) give flowers to each other.} (A=G)
\bg. *kanciŋ ama(=be) soʔmek-khusa ca-me-ci=ha.\\
\sc{1du} mother(\sc{=loc}) show-\sc{recip} eat\sc{.aux-npst-[1]du=nmlz.nsg}\\
Intended: \rede{We showed each other to mother (e.g., on a photo).} (*A=T)


The antecedent of the coreferential argument always has to be the agent, as with the reciprocal of \emph{nis} \rede{see, know}, yielding \rede{introduce, get to see/know each other} in \Next[a]. Coreferential T and G are possible, however, when the causative marker \emph{-met} is attached to the auxiliary, so that the \isi{reciprocal construction} serves as input to a \isi{causative construction} (see \Next[b]). The arguments that are fused into one \isi{noun phrase} are the A and P arguments of the \isi{reciprocal construction}, and simultaneously they are T and G arguments of the \isi{causative construction} \emph{nikhusa cameʔma} \rede{introduce to each other}, which shows transitive \isi{person marking} and \isi{ergative} \isi{case} marking on A. The causative verb \emph{nimeʔma}, without the reciprocal, also exists; it is a three-argument verb with the meaning  \rede{introduce (X to Y)}.

\ex. \ag. kanciŋ ni-khusa  ca-me-ci=ha.\\
			  \sc{1du} see/know-{\sc recip}  eat{\sc .aux-npst-[1]du=nmlz.ns}\\
			\rede{We will get to see/know each other.} (A=P)
	\bg. uŋ=ŋa uŋci ni-khusa ca-met-u-ci=ha.\\
	\sc{3sg=erg} \sc{3nsg} see/know-\sc{recip} eat\sc{.aux-caus-3.P[pst]-nsg.P=nmlz.nsg}\\
	\rede{He introduced them (to each other).} ([[A=P.{\sc recip}], G=T.{\sc caus}])	
	

In the \isi{indirective frame} (characterized by \isi{locative} or \isi{ablative} marking on the G argument, see \sectref{three-arg-frame}), the \isi{reciprocal construction} can express coreference of A and T or A and G, regardless of the \isi{case} and agreement properties of the arguments in the corresponding  non-reci\-pro\-cal predicate. The possibilities are restricted only by the verbal semantics, i.e., whether the T or the G argument is animate/human and thus eligible for being coreferential with A.  In \Next[a], the A argument  is coreferential with T, while in \Next[b], A is coreferential with G.

\ex. \ag.	uŋci hoŋma=be luŋ-khusa ca-ya-ci=ha.\\
		\sc{3nsg} river{\sc =loc} drown-{\sc recip}   eat{\sc .aux-pst-[3]du=nmlz.nsg}\\
	\rede{They (dual) drowned each other in the river.} (A=T)
 	\bg.uŋci yaŋ khu-khusa ca-me-ci=ha.\\
	\sc{3nsg} money steal{\sc -recip} eat{\sc .aux-npst-[3]du=nmlz.nsg}\\
	\rede{They steal money from each other.} (A=G)
	
	 
In the \isi{secundative frame} (characterized by \isi{instrumental} marking on the T argument, see \sectref{three-arg-frame}), animate or human T arguments are hardly conceivable, and thus, only instances with coreferential A and G could be attested, as shown in \Next.
	
	
\exg.ibebe n-juŋ-a-ma,   ikhiŋ=ga tabek=ŋa ce-ŋkhusa  n-ja-ya=em,  barcha=ŋa  hok-khusa  n-ja-ya=em, luŋkhwak=ŋa lep-khusa n-ja-ya, ikhiŋ=ga bhuiʈar=ŋa ap-khusa n-ja-ya.\\
	anywhere {\sc 3pl-}fight{\sc -pst-prf} so\_big{\sc =gen}  khukuri\_knife{\sc =ins} cut{\sc -recip} {\sc 3pl-}eat{\sc .aux-pst=alt} spear{\sc =ins} pierce{\sc -recip} {\sc 3pl-}eat{\sc .aux-pst=alt} stone{\sc =ins} throw{\sc -recip} {\sc 3pl-}eat{\sc .aux-pst} so\_big{\sc =gen} catapult{\sc =ins} shoot{\sc -recip} {\sc 3pl-}eat{\sc .aux-pst}\\
	\rede{They fought so much, with knives so big, whether they cut each other with knives, whether they stabbed each other with lances, they threw stones at each other, they shot each other with a really big catapult.} \source{39\_nrr\_08.21--2}
	

Derived verbs can also serve as input to the \isi{reciprocal construction}, as shown for the \isi{benefactive} in \Next[a] and for the causative in \Next[b].

	\ex. \ag.kanciŋ ʈopi pham-bi-khusa ca-me-ci=ha.\\
	\sc{1du}  cap knit{\sc -V2.give-recip} eat{\sc .aux-npst-[1]du=nmlz.nsg}\\
	\rede{We knit caps for each other.}
	 \bg.kaniŋ cuwa=ŋa khoʔ-meʔ-khusa ca-i-wa.\\
	\sc{1pl} beer\sc{=ins} have\_enough\sc{-caus-recip} eat{\sc .aux-1pl-npst}\\
	\rede{We serve each other beer.} (Lit. \rede{We make each other have enough beer.})



\subsection{The middle construction}\label{middle}


Middle verbs are characterized by denoting an event that“affects the subject of the verb or his interests”, to take up the definition by \citet[373]{Lyons1969_Introduction}. Characteristic for a middle situation is the low elaboration of participants in an event \citep[3]{Kemmer1993_Middle}. Agent and patient have the same reference, just as in the reflexive. In the middle, however, agent and patient are less distinct conceptually, because many of the events do not presuppose a volitional agent. Volitionality is a crucial feature of a prototypical agent \citep{Hopperetal1980Transitivity, Foleyetal1984Functional}. Hence, the middle is semantically less transitive than a reflexive, but still more transitive than an intransitive verb  \citep[73]{Kemmer1993_Middle}. 

The Yakkha middle is marked by \emph{-siʔ} , which behaves like a \isi{function verb}, despite originating in a suffix, as \isi{comparison} with other \isi{Tibeto-Burman} languages shows (see \sectref{V2-mddl}). The distinctive semantic criterion of the middle marker \emph{-siʔ} in Yakkha is the low intentionality and volitionality  on part of  the subject. The middle derivation detransitivizes the verbs (compare \Next[a] and \Next[b]). With a few verbs, \emph{-siʔ} may indicate a reciprocal reading, but, crucially, only when the action was performed unintentionally (see \NNext).

\ex. \ag.ka bhitta=be kila likt-u-ŋ=na.\\
		{\sc 1sg[erg]} wall{\sc =loc} nail 	drive\_in-{\sc 3.P[pst]-1sg.A =nmlz.sg}	\\
	\rede{I drove a nail into the wall.}
 	\bg.ka  likt-a-sy-a-ŋ=na.\\
		{\sc 1sg}  drive\_in{\sc -pst-mddl-pst-1sg=nmlz.sg} 		\\
	\rede{I got stuck (in the mud, head first).}


\ex. \ag.ka hen=ca a-ʈukhruk dailo=be lukt-i-ŋ=na.\\
		{\sc 1sg[erg]} today{\sc =add} {\sc 1sg.poss}-head door{\sc =loc} knock{\sc -compl[3.P;pst]-1sg.A=nmlz.sg} \\
	\rede{I knocked my head at the door even today.}
 	\bg.lukt-a-sy-a-ŋ-ci-ŋ=ha.\\
	knock-{\sc pst-mddl-pst-excl-[1]du-excl=nmlz.nsg}	\\
	\rede{We (dual) bumped into each other.}


The semantics of verbs that take the middle marker cover the situation types commonly associated with the category of middle crosslinguistically: grooming and body care, \isi{motion}, change in body posture, reciprocal events, e\isi{motion}, cognition and spontaneous events. The middle marker \emph{-siʔ} encodes grammatical functions as well as lexicalized meanings, just as the reflexive/auto\isi{benefactive} V2 \emph{-ca} (see \sectref{V2-eat}). For more on  \emph{-siʔ} see \sectref{V2-mddl}.


\subsection{V2 stems signalling animate T arguments}\label{t-sap}

Certain scenarios in \isi{three-argument verbs} require additional marking in Yakkha. As the T argument of \isi{three-argument verbs} is typically less topic-worthy, salient or lower on a referential hierarchy than the G argument, one could expect an increase in morphological complexity in the verb when the T argument is higher on the referential hierarchy or when the G argument is lower than expected, i.e., \rede{the construction which is more marked in terms of the direction of information flow should also be more marked formally} \citep[128]{Comrie1989Language}. Such a marking is comparable to inverse marking for agent and patient, as found, e.g., in Algonquian languages \citep{Zuniga2007_From}. According to \citet[90]{Haspelmath2007Ditransitive}, such verbal marking has not been found  for the relation of T and G in \isi{three-argument verbs} yet. 

\largerpage
Yakkha, too, does not have one dedicated marker for \rede{inverse} scenarios of T and G. But there is a tendency for animate or human T (and P) arguments to require a serial verb construction, and thus more complexity in the verb. Several V2 stems  can be found in this function, most prominently \emph{-khet \ti -het} \rede{carry off}, \emph{-end} \rede{insert}, \emph{-raʔ} \rede{bring} and \emph{-haks} \rede{send}. As there are several V2s  with different semantics, it is not their only function to indicate referentially high T arguments. They can even be found with in\isi{animate T arguments}. The crucial point is that certain scenarios cannot be expressed without using them, as for instance in example \Next. The stealing of things is expressed by a simple verb stem (see \Next[a]), while stealing a person cannot be expressed with a simple verb. Instead, the complex construction with the V2 \emph{-het} \rede{carry off} is used, implying caused \isi{motion} away from a point of reference  (see \Next[b]). If, instead, the V2 \emph{-haks} \rede{send} is applied to the lexical stem \emph{khus} \rede{steal}, the meaning changes to \rede{rescue} (see \Next[c]). The simple stem \emph{khus}, however, cannot express events with human T arguments.


\ex. \ag. pasal=bhaŋ yaŋ khus-uks-u=ha.\\
		shop{\sc =abl} money steal{\sc -prf-3.P[pst]=nmlz.nc}	\\
	\rede{(He) has stolen money from the shop.}
 	\bg.  ka ijaŋ a-paŋ=bhaŋ khus-het-a-ŋ-ga=na?\\ 
	{\sc 1sg} why {\sc 1SG.poss-}house{\sc =abl} steal{\sc -V2.carry.off-pst-1.P-2.A=nmlz.sg}		\\
	\rede{Why did you steal me from my home?} 
	\bg.  kiba=bhaŋ khus-haks-a-ŋ-ga=na.\\ 
tiger{\sc =abl} steal{\sc -V2.send-pst-1.P-2.A=nmlz.sg}		\\
	\rede{You saved me from the tiger.} 
	

Some examples from natural texts are provided in \Next.	

\ex.\ag.nhaŋa   nnakha yapmi ta-khuwa=ci       ikt-haks-u=ci.\\
and\_then those person come{\sc -nmlz=nsg} chase{\sc -V2.send-3.P[pst]-nsg.P}\\
\rede{And then she chased away those people who were coming.} \source{14\_nrr\_02.034}
 	\bg. ak=ka  kamniwak=ci   hip-paŋ tikt-u-ra-wa-ŋ-ci-ŋ.\\
{\sc 1sg.poss=gen} friend{\sc =nsg} two{\sc -clf.hum} guide{\sc -3.P-V2.bring-npst-1sg.A-3nsg.P-1sg.A}\\
	\rede{I will bring along two of my friends.}  \source{14\_nrr\_02.023}
	
A very typical example is also the verb \emph{pinnhaŋma} \rede{send off} shown in \Next, which has already acquired a fixed meaning \rede{marry off (one's daughter)}.

\exg. m-ba=ŋa nda ka=be pin-nhaŋ-me-ŋ=na=bu=i?\\
{\sc 2sg.poss}-father{\sc =erg} {\sc 2sg} {\sc 1sg=loc}	give{\sc -V2.send-npst-1sg.P=nmlz.sg=rep=q}\\
	\rede{(Did they say that) your father will give you to me (in marriage)?} 

	
It can be concluded that the higher complexity and greater semantic specification of an event via serialization is necessary in, but not restricted to events with referentially high T (and occasionally also P) arguments.
	
\subsection{Historical excursus: Stem augments}\label{stemchange}
\largerpage
Yakkha verbal stems can be divided into unaugmented and augmented roots (see also \sectref{stem}). Both open ((C)V) and closed ((C)VC) stems can be extended by the coronal augments \emph{-s} and \emph{-t}. These augments can be related to transitivizing suffixes in Proto-\isi{Tibeto-Burman}, often with \emph{-s} coding a causative and \emph{-t} coding a directive or a \isi{benefactive} derivation (see \citealt[457]{Matisoff2003Handbook}, \citealt[160]{Driem1989_Reflexes}). Synchronically, however, the augmentation does not constitute a productive pattern. 

Some reflexes of this old system can, however, still be found in correspondences such as in \tabref{stem-aug}, albeit only for a small fraction of the verbal lexicon. Complete stem triads (consisting of an unaugmented, an \emph{-s}-augmented and a \emph{-t}-augmented root) are exceedingly rare, and synchronically, many intransitive verbs with augmented stems exist as well, which clearly shows that a regular correspondence between augmentation and transitivization is not given synchronically.\footnote{Comparing the stems in Yakkha with other Kiranti languages, the form and meaning of the augmented stems do not correspond across individual languages. Unaugmented stems in one Kiranti language may have augments in another language, and augments may differ for cognate roots, which adds support to the reconstruction of these augments as non-integral part of the verbal stem, i.e., as suffixes.} The stem alternations do not necessarily entail an increase in the number of arguments; sometimes just the properties of the arguments change, along with the \isi{case} and \isi{person marking}. For instance, \emph{haks} and \emph{hakt} both mean \rede{send}, but the goal of \emph{haks} is in the \isi{locative} \isi{case} and referentially unrestricted, while \emph{hakt} takes a human goal in the \isi{nominative}, which also points to the former use of the augment /-t / as a \isi{benefactive} marker. We have seen above in \sectref{benefactive} that there is also a suffix \emph{-t}  in the \isi{benefactive} derivation, which is probably also related to these old suffixes. It is only employed as a secondary marker, accompanying the primary \isi{benefactive} marker, the V2 \emph{-piʔ}. 


\begin{table}[t]
\begin{center}
\resizebox{\textwidth}{!}{
\begin{tabular}{lll} 
 \lsptoprule
(C)V(C) &(C)V(C)-s&(C)V(C)-t\\
 \midrule
 \emph{ap}  \rede{come} (same level, close) & &\emph{apt}  \rede{bring}\\
 & \emph{haks}   \rede{send somewhere}&  \emph{hakt}   \rede{send to someone}\\
  \emph{keʔ} \rede{come up}  & &\emph{ket}  \rede{bring up}\\
  \emph{kheʔ} \rede{go}  & &\emph{khet}  \rede{carry off}\\
  \emph{khuʔ} \rede{carry}  & \emph{khus} \rede{steal}&\emph{khut}  \rede{bring}\\
  \emph{luʔ} \rede{tell, say}  & \emph{lus} \rede{deafen, roar}&\emph{lut}  \rede{tell for someone}\\
  &\emph{maks}  \rede{wonder, look around}&\emph{makt} \rede{see in dream}  \\
  \emph{si} \rede{die}  &\emph{sis}  \rede{kill}&\\
  \emph{ta}  \rede{come} (general) &\emph{tas}  \rede{arrive at}&\emph{taʔ}  \rede{bring to}\\
 & \emph{uks}   \rede{come down}&  \emph{ukt}   \rede{bring down}\\
  \emph{yuŋ} \rede{sit} &\emph{yuks}  \rede{put}&\emph{yukt}  \rede{put for s.o.}\\
 \lspbottomrule
\end{tabular}
}
\caption{Stem augmentation and \isi{transitivity} correspondences}\label{stem-aug}
\end{center}
\end{table}

The stem \emph{tup} \rede{meet} also undergoes the stem alternation. While the unaugmented stem is inherently reciprocal and is thus inflected intransitively (and thus, necessarily, takes nonsingular arguments), the stem \emph{tups} is transitively  inflected and takes two arguments that cannot have identical reference.

	 \ex. \ag. kanciŋ tub-a-ŋ-ci-ŋ=ha.\\
 {\sc 1du} meet-{\sc pst-excl-du-excl=nmlz.nsg}\\
 \rede{We  met.}
	 \bg.ka ŋ-gamnibak n-dups-u-ŋa-n=na.\\
	 {\sc 1sg[erg]} {\sc 2sg.poss-}friend  {\sc neg-}meet{\sc -3.P[pst]-1sg.A-neg=nmlz.sg}\\
	\rede{I did not meet/find your friend.}









