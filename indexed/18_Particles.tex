\chapter{Discourse particles and interjections}\label{particles}

This chapter provides an overview of operators on the discourse level. Syntactically speaking they are optional, but of course omitting them does not make sense from a discourse perspective. Particles are invariant morphemes and not part of the inflectional paradigm of the verb or the noun. Some particles attach to phrases (potentially including adverbial clauses), others attach to the end of a sentence and accordingly, their scope properties have the potential to differ (see Table \ref{tab-particles} for an overview). It should be noted that the term \emph{particle} does not make any statements about their nature, apart from the fact that they are uninflected. The operators discussed here can be phonologically bound,  free, or have  variable phonological status. Some of them can also  be stacked to arrive at more specific discourse functions, which in some cases leads to new, phonologically unbound forms. 
The functions of the particles include indicating topicalized or focussed constituents of the sentence (cf. §\ref{ptcl-top} and §\ref{ptcl-foc}). Other particles rather have scope over a section of discourse that is bigger than one sentence, most prominently \emph{baŋniŋ}. Some particles indicate the source of information (evidentiality markers) and the assessment of the speaker regarding the likelihood or reliability of a piece of information (epistemic markers). They are discussed, together with a marker of mirativity, in §\ref{ptcl-evid}. Exclamative particles are the topic of §\ref{ptcl-excla}. Further particles discussed here are the marker of alternatives, the marker of truth-value questions, the vocative and the insistive particles in §\ref{ptcl-further}.  This chapter also contains an overview of the \isi{interjections} found in Yakkha (§\ref{interjections}). Two further markers with discourse function have been discussed at length in §\ref{nmlz-uni-3}: the nominalizing clitics \emph{=na} and \emph{=ha}. 

Rarely, markers from \ili{Nepali} are used as well, such as emphatic \emph{ni}, the initiative \emph{lu}, and the \isi{probability particle} \emph{hola} (paraphrasable with \rede{probably}). They are not discussed here. The only markers from \ili{Nepali} that are found frequently enough to be considered in this discussion are the \isi{contrastive topic} particle (see §\ref{ptcl-top}) and the \isi{mirative} marker  (see §\ref{ptcl-evid}).

This chapter has the character of a descriptive overview, providing merely impressionistic conclusions. It does not present a theory of discourse marking in Yakkha, as an in-depth discourse analysis  has not been undertaken for Yakkha  yet. 

\begin{table}[htp]
\center
\begin{tabular}{lll}
\lsptoprule
		{\sc particle} & {\sc function} & {\sc domain} \\
\midrule
\emph{=ko \ti =go}&topic&phrase, adv. clause\\
\emph{=chen}&\isi{contrastive topic}&phrase\\
\emph{baŋniŋ}&\isi{textual topic}, \isi{quotative}&adv. clause\\
\emph{=se}&\isi{restrictive focus}&phrase, adv. clause\\
\emph{=ca}&additive focus&phrase, adv. clause\\
\emph{=pa \ti =ba}& emphasis&sentence\\
\emph{=i}& emphasis &sentence\\
\emph{=le}&\isi{contrastive focus}&phrase\\
\emph{=maŋ}&emphasis&phrase\\
\emph{=pu \ti =bu}&\isi{reportative}& phrase, emb. clause\\
\emph{loppi}&probability/hypothetical &sentence\\
\emph{=pi \ti =bi}&irrealis &clause\\
\emph{rahecha}&\isi{mirative}&sentence\\
\emph{lai}&\isi{exclamative}&sentence\\
\emph{=ʔlo}&\isi{exclamative}&sentence\\
\emph{hau}&\isi{exclamative}&sentence\\
\emph{=em}&alternation&clause\\
\emph{i}&truth-value question&sentence\\
\emph{=u}&vocative&phrase\\
\emph{au}&insistive& sentence\\
\lspbottomrule	
\end{tabular}
\caption{Overview of discourse particles}\label{tab-particles}
\end{table}

\section{Topic}\label{ptcl-top}

\subsection{The particle \emph{=ko \ti =go}}

The topic particle \emph{=ko \ti =go} (alternation conditioned by the \isi{voicing} rule, cf. §\ref{voicing}) marks the constituent in the sentence about which a question or  assertion is made, and is thus only found once in a sentence. It  attaches to constituents of any kind (see \Next), including adverbial clauses (see \NNext). In \Next[a], \emph{=go} attaches to a possessive pronoun that refers to the protagonist of a narrative. The constituent marked for topic does not have to be given in a certain stretch of discourse; in \Next[b] for instance, the topic of the sentence is a newly introduced discourse topic. In \Next[c] the assertion is made about a certain time that is expressed by an adverbial. 

\ex. \ag. uk=ka=go  pik=ci wai-sa=bu.\\
{\sc 3sg.poss=gen=top} cow{\sc =nsg} exist{\sc -pst.prf=rep}\\
\rede{He had cows.} \source{11\_nrr\_01.005}
\bg. a-pum=go  nu=na hola ni?\\
		{\sc 1sg.poss-}grandfather{\sc =top} get\_well{\sc [3sg]=nmlz.sg} probably {\sc ptcl}\\
		\rede{The grandfather is fine, most probably?}\footnote{\emph{hola} and \emph{ni} are \ili{Nepali} particles and they do not occur frequently (yet), so they are not discussed further here.} \source{06\_cvs\_01.083}
		\bg.heʔnasen=go    n-lit-a-ma-n=ha.\\
		thesedays{\sc =top} {\sc neg-}plant{\sc [3sg]-pst-prf-neg=nmlz.nc}\\
		\rede{These days it has not been planted.} \source{36\_cvs\_06.094}
		
		\ex.\ag.nhe  nis-u-ŋ=niŋ=go                     pako         sa=na\\
		here see{\sc [pst]-3.P-1sg.A=ctmp=go} old  {\sc cop.pst[3sg]=nmlz.sg}\\
		\rede{When I had seen him here, he was (already) old.} \source{06\_cvs\_01.088}
		\bg.wa=ci         ŋ-ga-ya=hoŋ=go                  om     leks-a-khy-a\\
		chicken{\sc =nsg} {\sc 3pl-}call{\sc -pst=seq=top} bright become{\sc [3]-pst-V2.go-pst}\\
		\rede{The cocks crowed, and the day started.} \source{37\_nrr\_07.028}
		\bg.massina=ci    bhoŋ=go,    massina=ca        ŋ-und-wa=ci\\
		small{\sc =nsg} {\sc cond=top} small{\sc =add} {\sc 3pl.A-}pull{\sc -npst-3nsg.P}\\
		\rede{If they (the fish) are small, they also pull them out.} \source{13\_cvs\_02.021}
				

The constituent marked by \emph{=ko} comes sentence-initially (see \LLast and \Last), or in a detached position. Example \Next shows a right-dislocated topic in an afterthought which is marked by \emph{=ko} that was probably uttered to correct the plural marking of the verb to dual in the pronoun. The same is illustrated below by \NNext[b] and \NNext[c].

\exg.  haku i=ya  ka-m  haʔlo, kanciŋ=go?\\
	now what{\sc =nmlz.sg} say{\sc -1pl.A[sbjv]} {\sc excla} {\sc 1du=top} \\
	\rede{Now what shall we say, I mean, the two of us?} \source{13\_cvs\_02.54}		

	
The topic particle is also involved in a fixed construction, where the finite main verb is preceded by its infinitival form, which is marked by the topic particle. This grammatical construction implies that a restriction applies to the propositional content expressed in the sentence  (see the examples in \Next). In example \Next[a], people discuss where one could purchase fish, and that even though a certain household might have fish, they probably do not have enough to sell them to others. The question in \Next[c] implies that things could have been worse, as the talk is about an arranged marriage. The construction does not only occur with verbs, but also with other word classes in \isi{predicative function} (see \Next[d]). The same structure occurs in \ili{Nepali} with the \isi{infinitive} in \emph{-nu} and the topic marker \emph{ta}. 

\ex. \ag.yo    gumba=ci=ge=ca  wa-ma=go  wam-me=ha, \\
across Gumba{\sc =nsg=loc=add} exist{\sc -inf=top} exist{\sc [3sg]-npst=nmlz.nsg}\\
\rede{Gumba's family over there has some (fish), (but ...)} \source{13\_cvs\_02.057}
\bg.we=ppa,             wa-ma=go,          ca-ma khoblek. cama=ŋa   ŋ-khom-me-n.\\
exist{\sc [3sg;npst]=emph} exist{\sc -inf=top} food eating\_up\_at\_once food{\sc =erg} {\sc neg-}be\_enough{\sc [1.P]-npst-neg}\\
\rede{There is, there is food, (but) it is eaten up quickly. We do not have enough food.} \source{28\_cvs\_04.169-70}
\bg.nhaŋ    cekt-a-ci=ha,                cek-ma=go?\\
and\_then speak{\sc -pst-du=nmlz.nsg} speak{\sc -inf=top}\\
\rede{And then they talked, at least?} (bride and groom) \source{36\_cvs\_06.289}
\bg.khumdu=go khumdu, khatniŋgo haŋ=ha\\
tasty{\sc =top} tasty, but spicy{\sc =nmlz.nc} \\
\rede{It is tasty, but spicy.}


The topic particle is also part of the lexicalized adversative clause-initial connective \emph{khatniŋgo/ŋkhatniŋgo} \rede{but}, which historically is a combination of a temporal \isi{adverb} meaning \rede{at this/that time} and the topic marker.

\subsection{The \isi{contrastive topic} particle \emph{=chen} (from \ili{Nepali})}

In addition to the Yakkha topic marker \emph{=ko}, the particle \emph{=chen} (with a freely alternating allomorph  [cen]), a borrowed form of  the \ili{Nepali} particle \emph{cāhĩ}, can be employed (see §\ref{loansphon}).  It has a stronger reading, marking contrastive topics  in contexts where the speaker switches the topic or singles out one constituent (about which an assertion is made) against other constituents, as shown in \Next[a]. In \Next[b], the protagonists of the story are marked by \emph{=chen}, because the preceding content was about the people who chased them away.
In contrast to \emph{=ko},  \emph{=chen} is restricted to constituents of clauses; it is not found on adverbial clauses. 

\ex. \ag. imin lak-m=ha baŋniŋ, ka=chen  na doku=hoŋ=be        u-me-ŋ, [...] nniŋda lakt-a-ni\\
how dance{\sc -inf=nmlz.nsg} about {\sc 1sg=top} this basket{\sc =inside=loc} enter{\sc -npst-1sg} [...] {\sc 2pl} dance{\sc -imp-pl}\\
\rede{As for how to dance, I will climb into this basket, [...] and you (plural) will dance.}\source{14\_nrr\_02.28-9}
\bg.nhaŋ khaci     lalubaŋ=nuŋ   phalubaŋ=chen khali puŋda=e     kheʔ-m=ha\\
and\_then these Lalubang{\sc =com} Phalubang{\sc =top} only forest{\sc =loc} go{\sc -inf[deont]=nmlz.nsg}\\
\rede{And then, those (guys) Lalubang and Phalubang only had the option to go to the forest.}\source{22\_nrr\_05.045}
%\bg.nhaŋto phu=na=chen            seg-haks-u-ŋ=hoŋ:                        haku=ca chubuk=ka    caleppa py-a.\\
%and\_then white{\sc =nmlz.sg=top} sort{\sc -V2.send-3.P[pst]-1sg.A=seq} now{\sc =add} ashes{\sc =gen} bread give{\sc -pst[1.P]}\\
%\rede{Then, I sorted out the white bread: Give me the bread of ashes again.} \source{40\_leg\_08.060}
 
\subsection{The \isi{quotative} and \isi{textual topic} particle \emph{baŋniŋ}}
	
This particle marks the question or topic that a broader section of discourse is about (see \Next). The particle constitutes a stress domain by itself. The voiced initial could be a reflex of its formerly bound nature. Yakkha also has a complementizer \emph{baŋna}, embedding clauses to nouns, sometimes also translating as \rede{so-called}. Although the origin of \emph{baŋ} is not clear, it looks as if a root of an \isi{utterance predicate} has been  nominalized to yield \emph{baŋna}, and the cotemporal adverbial clause linkage marker  \emph{=niŋ} (see §\ref{sim-finite}) has been added to the same root to yield \emph{baŋniŋ}. In fact, the particle in \Next[c] has the function of marking direct speech in much the same function as \emph{bhoŋ} in purpose clauses and utterance predicates (see §\ref{adv-cl-fin-purp} and \ref{utterance-pred}). In clauses with  \emph{bhoŋ} as a \isi{quotative} marker, the subject of the main clause and the source of the utterance are coreferential. This is not the \isi{case} for speech marked by  \emph{baŋniŋ}, as \Next[c] clearly shows. This example also shows that it is possible for the topic marker \emph{=ko} to attach to  \emph{baŋniŋ}.

\ex.\ag.nna  ceŋ    waiʔ=na=ŋa      haku=ca   kaniŋ i  miʔ=m=ha baŋniŋ: (...)\\
that{\sc [nom]} upright be{\sc [3sg]=nmlz.sg=erg} now{\sc =add} {\sc 1pl[nom]} what think{\sc -inf[deont]=nmlz.nsg} about\\
\rede{As that (stele) stands upright, what we have to think even now, ...} \source{18\_nrr\_03.031}
\bg.\label{ex-bu}liŋkha=ŋa=bu         i=ya                i=ya                men-jok-ma              baŋniŋ:\\
a\_clan{\sc =erg=rep}  what{\sc =nmlz.nc} what{\sc =nmlz.nc} {\sc neg-}do{\sc -inf[deont]} about\\
\rede{As for what things the Linkha clan is not allowed to do, (...)} \source{11\_nrr\_01.022}
\bg.maŋcwa=le          apt-u     baŋniŋ=go   i=ha=le        kheps-uks-u=ha? \\
water{\sc =ctr} bring{\sc -imp} {\sc quot=top} what{\sc =nmlz.nc=ctr} hear{\sc -prf-3.P=nmlz.nc} \\
\rede{When we told her to fetch water, what the heck did she understand?} \source{42\_leg\_10.045}


\section{Focus and emphasis}\label{ptcl-foc}

\subsection{The \isi{restrictive focus} particle \emph{=se}}
		
Restrictive focus is expressed by the particle \emph{=se}. It is attached to the constituent it focusses on. A sentence with a constituent being focussed by the restrictive marker  expresses that out of a given  set only this constituent fulfills the necessary condition for the proposition to be true. The \isi{restrictive focus} marker thus has a semantic impact on the sentence, and does not merely add emphasis (see also \citet{Koenig1993_Focus}). 

\ex. \ag. ka=go na mamu hen=se nis-u-ŋ=na\\
	 	{\sc 1sg=top} this girl today{\sc=restr} know{\sc -3.P[pst]-1sg.A=nmlz.sg}\\
	 	\rede{But I saw this girl only today!}
	 	\bg.  hoŋkhiŋ=se\\
	 	that\_much{\sc=restr}\\
	 	\rede{That much only.}\source{11\_nrr\_01.42}
		\bg.yakkhaba=ga=se       cekt-uks-u-m\\
		Yakkha\_person{\sc =gen=restr} speak{\sc -prf-3.P[pst]-1pl.A}\\
		\rede{We have only talked in the language of the Yakkha people.} \source{ 36\_cvs\_06.609}
		

The \isi{restrictive focus}  marker can also be found on adverbial clauses (see \Next[a]), and also inside adverbial clauses, as the focus domain generally extends into subordinate clauses in Yakkha (see \Next[b]).

%ham-ne-ma=hoŋ=se                     ŋ-leŋ-me-n=na=i.\\
%weep{\sc -V2.lay-inf=seq=restr} {\sc neg-}be\_alright{\sc -npst-neg=nmlz.sg=foc}\\
 %\rede{One cannot keep crying all the time.} \source{36\_cvs\_06.265} 
 \ex. \ag. sumphak phophop     n-jog-uks-u=hoŋ=se                             camyoŋba m-by-uks-u-ci\\
 leaf upside\_down {\sc neg-}do{\sc -prf-3.P=seq=restr} food {\sc 3pl.A-}give{\sc -prf-3.P[pst]-3nsg.P}\\
 \rede{They gave the food to them only after turning the leaf-plates upside-down.}
 \bg.nna  mʌndata=se                     m-bi-wa-ci=nuŋ                    samma cahĩ,  [...]\\
 that marriage\_custom{\sc =restr} {\sc 3pl.A-}give{\sc -npst-3nsg.P=com.cl} until {\sc top}\\
 \rede{And as long as only the Mandata is given, ... (the wife does not belong to the husband's side).}  \source{26\_tra\_02.007}
 
 The marker \emph{=se} is also found in a construction that expresses very urgent requests or imperatives. In such constructions, an infinitival form, marked by the \isi{restrictive focus} particle, precedes the inflected verb, as shown in \Next.

 \exg. oe     chiŋdaŋ            oe=maŋ,       nda yep-ma=se               yeb-a-sy-a!\\
 {\sc addr} pillar {\sc addr=emph} {\sc 2sg} stand{\sc -inf=restr} stand{\sc -imp-mddl-imp}\\
 \rede{Oh, pillar, oh, if you only stood upright!} \source{27\_nrr\_06.023}
 
 
\subsection{The additive focus particle \emph{=ca}}\label{ptcl-additive}

 Additive focus is marked by the \isi{clitic} \emph{=ca}.\footnote{In related languages, the cognate of this marker has aspirated /ch/ initially (e.g. in  \ili{Bantawa} and Belhare \citep{Doornenbal2009A-grammar, Bickel2003Belhare}).} This marker expresses that content is added to some presupposed or previously activated  content, or that some participant is included to the presupposed set of participants, as illustrated by \Next. The marker attaches  to constituents of any type, including adverbial clauses, as will be shown below. 
 
 \ex.\ag. nda=nuŋ ka=ca khe-me-ŋ=na.\\
 {\sc 2sg=com}  {\sc 1sg=add}  go{\sc -npst-1sg=nmlz.sg}\\
 \rede{I will also go with you.}
 \bg.             kaniŋ=ca        yakkha          siŋan.\\
 {\sc 1pl=add} Yakkha {\sc cop.1pl.excl}\\
 \rede{We are also Yakkha people.} \source{39\_nrr\_08.08}
\bg.  encho=ca                ta-ya-ma=ga=na\\
some\_time\_ago{\sc =add} come{\sc -pst-2=nmlz.sg}\\
\rede{You have come before, too.} \source{28\_cvs\_04.164}
 
The additive focus particle  may also  be attached to question words,  yielding pro-forms that include all conceivable referents and thus have an exhaustive or ‘free-choice’\footnote{See \citet[980]{Koenig1993_Focus}.}  reading \rede{any} or \rede{ever},  as in \Next. In this function, the additive focus particle is often combined with the \isi{sequential clause linkage} marker \emph{=hoŋ} (see \Next[b] and (c) and the following paragraph). 

\ex.\ag.hetna=ca  tihar ta-meʔ=niŋa,  na uŋci=ga    yad=be khaʔla,   nniŋda lakt-a-ni.\\
which{\sc =add} a\_hindu\_festival come{\sc [3sg]-npst=ctmp} this {\sc 3nsg=gen} remembrance{\sc =loc} like\_this {\sc 2pl} dance{\sc -imp-pl.imp}\\
\rede{Whichever Tihar day comes, dance like this, in memory of them (goddess Sangdangrangma and her companions).}\footnote{Narrative 14 is about a Yakkha goddess called \emph{Saŋdaŋraŋma}, also called \emph{Dokeni}. Tihar is the Hindu festival of the goddess Laksmi, celebrated in October or November, and this celebration  has been transformed to a Yakkha celebration in the Yakkha cultural sphere. The Hindu goddess of fortune, wealth  and prosperity \emph{Lakṣmī } is identified with \emph{Saŋdaŋraŋma}.}\source{14\_nrr\_02.37}
\bg.na   makhur=na        caleppa hen   imin=hoŋ=ca          ca-ma=na.\\
this black{\sc =nmlz.sg} bread today how{\sc =seq=add} eat{\sc -inf[deont]=nmlz.sg}\\
\rede{This black bread has to be eaten today, by all means.} \source{40\_leg\_08.056}
\bg. i=ya=hoŋ=ca                       cok-ma     yas-wa-g=ha.\\
what{\sc =nmlz.nc=seq=add} do{\sc -inf} be\_able{\sc -npst-2=nmlz.nc}\\
\rede{You can achieve anything.} \source{01\_leg\_07.031}
 \bg.iʔbeniŋ=ca\\
 what\_time{\sc =add}\\
 \rede{at any time} \source{13\_cvs\_02.005}
  
 Scalar notions,  translatable with \rede{even}, can also be marked by \emph{=ca} (see the example in \Next), but constituents with a  scalar reading are rare in the corpus. Such notions are expressed in concessive clauses increasingly (see §\ref{adv-cl-conc}), marked by the related particle \emph{=hoŋca}, which is a combination of \isi{sequential clause linkage} marker and additive particle (see \Next[c]). In \Next[c] the scalar reading is combined with \isi{restrictive focus} \rede{even if only}. 
 
 \ex. \ag.      jammai, kha  yamuŋ=ca,       ghak heŋ-nhaŋ-ma. \\
 all this beard{\sc =add} all cut{\sc -V2.send-inf[deont]}\\
 \rede{All (hair), even this beard, one has to cut off all of it (while mourning).}  \source{29\_cvs\_05.058}\\
 \bg.ucun=na        yapmi=ca         ucun n-nis-wa-m-nin.\\
 nice{\sc =nmlz.sg} person{\sc =add} nice {\sc neg-}see{\sc -npst-1pl.A-3pl.neg}\\
 \rede{Even nice people will seem ugly to us (if we are forced to marry them).}  \source{36\_cvs\_06.330}
 \bg. liŋkha=ga      teʔmaŋa=se=hoŋca, ...\\
 a\_clan{\sc =gen} clan\_sister{\sc =restr=seq=add}\\
 \rede{Even if it is only a Linkha sister (as opposed to a Linkha sister who will marry a Limbukhim guy), ...} \source{37\_nrr\_07.095}
 
 
In a pattern that is common in the languages of \isi{South Asia} (and possibly beyond), the additive focus marker is employed together with verbal \isi{negation} to express exhaustive \isi{negation} (i.e. to the greatest conceivable extent),  which is often paraphrasable with  English \rede{any}. In this function, it is typically attached to interrogative words, as shown in \Next.
 
 \ex. \ag. ka hetniŋ=ca m-man-diʔ-ŋa-n=na.  \\
 {\sc 1sg} when{\sc =add} {\sc neg-}get\_lost{\sc -V2.give-1sg-neg=nmlz.sg}\\
 \rede{I would never get lost.}\source{18\_nrr\_03.015}
 \bg. hou,   ka       eko=ca        m-pham-me-ŋ-ga-nǃ                    i=na=le?\\
 {\sc excla} {\sc 1sg} one{\sc =add} {\sc neg-}help{\sc -npst-1sg.P-2.A-neg} what{\sc =nmlz.sg=ctr}\\
 \rede{Man, you do not help me even with one (word, line); what is going on?} (a complaint uttered while singing a song) \source{ 07\_sng\_01.16}
 


\subsection{The \isi{emphatic particle} \emph{=pa \ti =ba}}

The \isi{emphatic particle} \emph{=pa} is typically attached to the inflected verb or to other sentence-final elements, like the dislocated phrase in \Next[a]. The function of this marker is to indicate that the hearer should be aware of the propositional content already, or to emphasize its truth, paraphrasable with German \rede{doch}, or English \rede{of course}. The particle is found on assertions (affirmative and negated), imperatives, hortatives and \isi{permissive} questions, but never on content or truth-value questions (see \Next).  In assertions, it occus mutually \isi{exclusive} with the nominalizers \emph{=na} and \emph{=ha} in sentence-final position. Its function seems similar to these nominalizers, except for its occurrence on imperatives and hortatives, and its absence on the above-mentioned question types (see the ungrammatical examples in \NNext). Etymologically, the particle might have developed from the \isi{nominalizer} \emph{=pa}. 

\ex. \ag.a-ppa=ŋa et-u-ci=ba, samundra=be=pa\\
 {\sc 1sg.poss-}father{\sc =erg} apply{\sc [3sg.A;pst]-3.P-3nsg.P=emph} ocean{\sc =loc=emph}\\
 \rede{My father used them (the fishing rods), in the ocean.} \source{13\_cvs\_02.024}
\bg.nhe, uk=ka          u-cya=ci               mohan=ŋa      hiŋ-ma=ci=ba.\\
here {\sc 3sg.poss=gen} {\sc 3sg.poss-}child{\sc =nsg} Mohan{\sc =erg} support{\sc -inf[deont]=nsg=emph}\\
\rede{Here, her children, of course Mohan has to care for them.}\source{28\_cvs\_04.145}
 \bg. kucuma=ŋa co-i-ks-u=ha?  khem=ba!\\
dog{\sc =erg} eat{\sc -compl-prf-3.P=nmlz.nc} before{\sc =emph}\\
 A: \rede{Did the dog eat up?} B: \rede{Already before!} 
  
  \ex. \ag.*man=ba?\\
  {\sc exist.neg[3]=emph}\\
  Intended: \rede{There is none?}
  \bg.*khiŋ=ba?\\
  this\_much{\sc =emph}\\
   Intended: \rede{This much?}
    \bg.*heʔna khy-a=ba?\\
  where go{\sc [3sg]-pst=emph}\\
   Intended: \rede{Where did he go?}
  
  
Example \Next shows \emph{=pa \ti =ba} in \isi{permissive} questions. Here, it implies that the speaker expects a positive answer (see \Next).  In imperatives, the marker is employed to make commands and requests more polite, occurring in a fixed sequence of the focus marker \emph{=i} (often realized as [e]) and \emph{=ba} (see \NNext).

\ex. \ag.ca-ŋ-so-ŋ=ba?\\
 eat{\sc [3.P]-1sg.A-V2.look[3.P]-1sg.A[sbjv]=emph}\\
 \rede{May I try and eat it?} \source{17\_cvs\_03.301}
\bg. na=be yuŋ-ma leŋ-me=pa? \\
this{\sc =loc} sit{\sc -inf} be\_alright{\sc [3sg]-npst=emph}\\
 \rede{Is it allowed to sit here?}

 \ex. \ag.yokmet-a-ŋ=eba \\
 tell{\sc -imp-1sg.P=emph}\\
 \rede{Please tell me about it.}\source{19\_pea\_01.005}
 \bg.  co=eba\\
 eat{\sc [imp]=emph}\\
 \rede{Please eat it.}
 
 The \isi{emphatic particle}  \emph{=pa} also frequently combines with other particles. The particles \emph{=le} and \emph{=ba} are combined in a fixed expression that is shown in \Next[a]. It can attach to the \isi{emphatic particle} \emph{=maŋ} and to the \isi{restrictive focus} particle \emph{=se} when it attaches to constituents in \isi{predicative function}.
 
 \ex.\ag. i=na=le=ba\\
 	what{\sc =nmlz.sg=ctr=emph}\\ 
 	\rede{watchamacallit, what to say}
 \bg.mi=na=maŋ=ba!\\
small{\sc =nmlz.sg=emph=emph}\\
 \rede{It is so small!} \source{36\_cvs\_06.225}
 \bg.hen khiŋ se=ppa\\
 today this\_much {\sc restr=emph}\\
 \rede{Today it is this much only.} 
 
\subsection{The \isi{emphatic particle} \emph{=i}}
 
 The emphatic  particle \emph{=i} is frequently found  in assertions that emphasize the truth value of some propositional content (see \Next ), usually following the clause-final nominalizers \emph{=na} and \emph{=ha} (see §\ref{nmlz-uni-3}), but never attaching to clauses that are marked by \emph{=pa}. The focus particle is most likely related to the phonologically unbound question marker \emph{i} that occurs in truth-value questions (see below).

 
 \ex. \ag. khaʔla,  eŋ=ga              ceʔya=i   chak=ha=i, chak=ha!\\
	 like\_this {\sc 1pl.poss=gen} language{\sc =emph} difficult{\sc =nmlz.nc=emph} difficult{\sc =nmlz.nc}  \\
	 \rede{It is like this, our language is difficult, difficult.} \source{36\_cvs\_06.544}
	 	 \bg.menna=i, paip cok-se khe-me-ŋ=na=i\\
	 {\sc cop.neg=foc}, pipe do{\sc -sup} go{\sc -npst-1sg=nmlz.sg=emph}\\
	\rede{No, I will go to fix the pipe.}
	
	
This  particle is also found in a negative construction with an infinitival form preceding the finite verb. The construction expresses exhaustive \isi{negation} \rede{not at all}.
	
\ex. \ag.pheŋ-ma=i              m-pheks-a-ma-n=na.\\
plough{\sc -inf=emph} {\sc neg-}plough{\sc [3sg]-pst-prf-neg=nmlz.sg}\\
\rede{It is not ploughed at all.}\source{06\_cvs\_01.081}
\bg.ka  ni-ma=i          n-nis-wa-ŋa-n=na.\\
{\sc 1sg[erg]} know{\sc -inf=emph} {\sc neg-}know{\sc -npst-1sg.A[3.P]-neg=nmlz.sg}\\
\rede{I do not know (any songs) at all.}\source{06\_cvs\_01.104}	
\bg.ni-ma=i          men-ni-ma=na              yapmi=be?\\
know{\sc -inf=emph} {\sc neg-}know{\sc -inf=nmlz.sg} person{\sc =loc}\\
\rede{(Getting married) to a person one does not know at all?} \source{36\_cvs\_06.325}
 
\subsection{The \isi{contrastive focus} particle \emph{=le}}
   
The particle  \emph{=le} carries a strongly contrastive notion. It is functionally equal to the \ili{Nepali} particle \emph{po}, which indicates that some new information stands in a strong contrast to the expectations, or was not part of the presupposed knowledge (see \Next[a]). Thus, it expresses a certain amount of surprise on part of the speaker. The particle is often accompanied by the \isi{mirative} \emph{rahecha \ti raecha} (borrowed from \ili{Nepali}), and the functions of both markers are indeed related. The particle is not restricted to marking contrastive information from the perspective of the speaker. In \Next[b], the assertion stands in contrast to the adressee's expectations, as this assertion belongs to an argument about a particular patch of farming ground. 

\ex. \ag. are,   heʔne khy-a-ŋ=na=lai,        ka? lambu=go     naʔmo=le             sa=na\\
ohǃ?  where  go{\sc -pst-1sg=nmlz.sg=excla} {\sc 1sg} way{\sc =top} down\_here{\sc =ctr} {\sc cop.pst[3sg]=nmlz.sg}\\
\rede{Holy crackers, where did I go? The way was down here!}\source{28\_cvs\_04.027}
\bg. na=go       aniŋ=ga=le   kham!\\
this{\sc =top} {\sc 1pl.incl.poss=gen=ctr} ground\\
\rede{(But) this is our ground!} \source{22\_nrr\_05.007}
\bg. ka=go a-sap=le thakt-wa-ŋ=na\\
{\sc 1sg=top} {\sc 1sg.poss-[stem]=ctr} like{\sc -npst[3.P]-1sg.A=nmlz.sg}\\
\rede{But I like it (in contrast to the other people present).}

As one can see in the examples in \Next, the marker also occurs in questions, marking the constituent that is focussed on. The marker expresses that the speaker is particularly clueless about the possible answer, i.e. that nothing is presupposed. In \Next[a], the speaker mistook the image quality of the Pear Story film \citep{Chafe1980The-Pear} for snow. In \Next[b], the protagonists of the narrative arrived in some unknown place after fleeing from their enemies. 
 
\ex. \ag.hiuŋ=le wai-sa=em  i=le wai-sa=em?\\
show{\sc =ctr} exist{\sc [3sg]-pst=alt} what{\sc =ctr} exist{\sc [3sg]-pst=alt} \\
\rede{Was there snow, or what was it?} \source{19\_pea\_01.004}
\bg. heʔne=le   ta-i=ya   m-mit-a-ma=hoŋ  uŋci m-maks-a-by-a-ma=ca\\
where{\sc =ctr} come{\sc -1pl=nmlz.nsg} {\sc 3pl-}think{\sc -pst-prf=seq} {\sc 3nsg} {\sc 3pl-}be\_surprised{\sc -pst-V2.give-pst-prf=add}\\
\rede{They were surprised, wondering: Where in the world did we arrive?} \source{22\_nrr\_05.030}


\subsection{The \isi{emphatic particle} \emph{=maŋ}}

 The particle \emph{=maŋ} is another emphatic marker, as the examples in \Next illustrate. It attaches to constituents of any kind, whether arguments or adverbs or clause-initial conjunctions. It is paraphrasable with \rede{just} or \rede{right} in English, but its  function is not fully understood yet.
 
\ex.\ag. i cok-ma? yakkha ten=be khaʔla=maŋǃ\\
what do{\sc -inf} Yakkha village{\sc =loc} like\_this{\sc =emph}\\
\rede{What to do? After all, it is just like this in a Yakkha village!} 
  \bg.nhaŋ=ŋa=maŋ ŋ-ikt-haks-u, ...\\
  and\_then{\sc =ins=emph} {\sc 3pl.A-}chase{\sc -V2.send-3.P[pst]}, ...\\
  \rede{And right then, they chased him away, ...}\source{18\_nrr\_03.011}
 \bg.nna  jeppa=maŋ   eko mem-muʔ-ni-ma=na                         len sa-ya.\\
that really{\sc =emph} one {\sc neg-}forget{\sc -compl-inf=nmlz.sg} day {\sc cop.pst[3sg]}\\
\rede{That really was  an unforgettable day.} \source{41\_leg\_09.069}
 \bg.ma,     na=ci=ga                 niŋwa=maŋ   om \\
mother sister{\sc =nsg=gen} mind{\sc =emph} {\sc cop} \\
\rede{Of course, it is just the concern/love of my mother and my sister.}\footnote{Context: a child pondering about being scolded for doing dangerous things. This example is exceptional in not displaying the obligatory possessive marking that is usually found on \isi{kinship} terms. The reason might be that both participants are highly topical in the narrative, and had been mentioned several times before.}  \source{42\_leg\_10.048,}
 

	
\section{Epistemic, evidential and \isi{mirative} markers}\label{ptcl-evid}

The criterion for a marker to be classified as epistemic is that it expresses the commitment of the speaker to the reality of an event. Evidential markers, on , purely indicate the source of the information \citep{Cornillie2009_Evidentiality}. Yakkha distinguishes evidential and epistemic markers. Furthermore, the notion of mirativity has to be distinguished from both evidentiality and epistemic modality. It expresses the unexpected status of some information \citep{DeLancey1997Mirativity}.  

\subsection{The \isi{reportative} particle \emph{=pu \ti =bu}}\label{hearsay}
 
The \isi{reportative} particle conveys a purely evidential notion, namely that the source of the information is not the speaker but someone else. The source of information can be either unspecific (translating as \rede{It is said.}, see \Next) or a specific, quotable source, as in \NNext. The particle is frequently found in narrations of legends and myths, as they are a prime example of hearsay know\-ledge. It mostly occurs sentence-finally; it usually attaches to the finite verb, but it can also attach to any other constituent (including adverbial clauses, see \Next[c]), also more than once per sentence (see \NNext[b]). 
 
 \ex. \ag. nam wandik=ŋa       lom-meʔ=niŋa                kam  cok-ni-ma           sa=bu\\
 sun tomorrow{\sc =ins} come.out{\sc [3sg]-npst=ctmp} work do{\sc -compl-inf} {\sc cop.pst=rep}\\
 \rede{It is said that he had to have the work completed when the sun would rise on the next day.}\source{11\_nrr\_01.007}
 \bg.nna  puŋdaraŋma=cen,     eko maŋme leks-a-ma=na=bu\\
that forest\_goddess{\sc =top} one eagle become{\sc [3sg]-pst-prf=nmlz.sg=rep}\\
 \rede{That forest goddess, she became an eagle, it is said.}\source{22\_nrr\_05.108}
\bg. u-milak          khokt-a-by-a                                bhoŋ=bu,    ŋkha desan-masan  n-da-ya,            n-nis-wa-n-ci-n=ha=bu.\\
{\sc 3sg.poss-}tail  cut{\sc -sbjv[3sg]-V2.give-sbjv} {\sc cond=rep}  that  scary\_ghosts {\sc 3pl-}come{\sc -sbjv} {\sc neg-}see{\sc -npst-neg-3nsg.P-neg=nmlz.nsg=rep}\\
\rede{If one cuts their  (the dogs')  tails, they do not see the scary ghosts coming, it is said.} \source{28\_cvs\_04.213}

 \ex. \ag.\label{lu}lu,   abo, hamro yakkha,    eh, aniŋ=ga  ceʔya=ŋ=bu chem lum-biʔ-ma=na=lai\\
 {\sc init}, now our Yakkha, oh, {\sc 1pl.excl=gen} matter{\sc =ins=rep} song tell{\sc -V2.give-inf=nmlz.sg=excla}\\
 \rede{Alright, in our Yakkha, oh [switching to Yakkha], now we have to sing a song in our language, she said.}\source{ 06\_cvs\_01.102}
 \bg.a-na=bu      khe-meʔ=na=bu,      \\
{\sc 1sg.poss-}sister{\sc =rep} go{\sc [3sg]-npst=nmlz.sg=rep}\\
\rede{My sister has to go, she (my sister) says.} \source{36\_cvs\_06.558} 
 
In \Last[a], the constituent that is marked by \emph{=bu} is in focus. The speaker remembers that she was asked to sing a song in her language, and not in \ili{Nepali}, and hence she puts the \isi{reportative} marker on the focussed constituent \emph{aniŋga ceʔyaŋ(a)} \rede{in our language}.


Occasionally, the \isi{reportative} marker \emph{=bu} is followed by an \isi{utterance predicate}, as in \Next. Its occurrence is, however, not obligatory on complements of predicates that embed speech in Yakkha (cf. §\ref{utterance-pred}).
 
 \exg. hiʔwa=ga hiʔwa wait=na=bu  ŋ-gam-me\\
 wind{\sc =gen} wind exist{\sc [npst;3sg]=nmlz.sg=rep} {\sc 3pl-}say{\sc -npst}\\
 \rede{They say that he only lives in the air.}\source{21\_nrr\_04.052} 
 
%28\_cvs\_04.254ŋkhaʔlabu     leŋmeʔnabu.


\subsection{The \isi{probability particle} \emph{loppi}}

This particle expresses that the speaker assesses the content of the proposition to be likely, but not an established fact, very much like the English adverbs \rede{maybe}, \rede{probably}, \rede{possibly}. It generally comes sentence-finally. Etymologically, it is a combination of the marker \emph{lo}\footnote{Synchronically, it only exists  as a clause linkage marker in Yakkha, see §\ref{adv-cl-int}. In Belhare, there are a \isi{comitative} marker \emph{lok} and a focus marker \emph{(k)olo} that could be related to Yakkha \emph{lo} \citep{Bickel2003Belhare}.} and the \isi{irrealis particle} \emph{=pi} (see below). Examples are provided in \Next. This particle is also found in questions (see \Next[b]) and in counterfactual clauses (see §\ref{adv-cl-count}). 

\ex. \ag. kaʈhmandu=ko     men=na            loppi,  men=na.\\
Kathmandu{\sc =top} {\sc neg.cop[3sg]=nmlz.sg} probably  {\sc neg.cop[3sg]=nmlz.sg} \\
\rede{It was probably not in Kathmandu [pondering] - no.} \source{36\_cvs\_06.311}
\bg.bappura  isa=ga    u-cya            loppi?\\
poor\_thing who{\sc =gen} {\sc 3sg.poss-}child probably\\
\rede{Poor thing, whose child could it possibly be?} \source{01\_leg\_07.292}


\subsection{The \isi{irrealis particle} \emph{=pi \ti =bi}}

The marker \emph{=pi} (realized as [bi] after \isi{nasals} and after vowels) is  found mostly on counterfactual \isi{conditional clauses}, but also on hypothetical clauses. In counterfactual clauses, it attaches to the inflected verb of the main clause and to the clause linkage marker of the subordinate clause, either cotemporal \emph{=niŋ(a)} or sequential  \emph{=hoŋ}.  There  actually are sequences of clause linkage marker, topic particle \emph{=ko} and irrealis marker \emph{=pi}, in all examples in \Next. The verbs in counterfactual clauses are always marked for the Past Subjunctive (see §\ref{mood}).
	
	\ex. \ag. ɖiana=ca  piʔ-ma=hoŋ=go=bi   cond-a-sy-a=bi=baǃ khatniŋgo man-ninǃ\\
	Diana{\sc =add} give{\sc -inf[deont]=seq=top=irr} be\_happy{\sc [3sg]-sbjv-mddl-sbjv=irr=emph} but {\sc neg.cop-3pl}\\
	\rede{We would have had to give them (fish) to Diana, too, she would have been happyǃ But there aren't any.}\source{13\_cvs\_02.056}
	\bg.nniŋ=ga ten a-sap n-thakt-u-ŋ=niŋ=go=bi, ka n-da-ya-ŋa-n=bi.\\
	{\sc 2pl.poss=gen} village {\sc 1sg.poss-[stem]} {\sc neg-}like{\sc -3.P-1sg.A[sbjv]=ctmp=top=irr} {\sc 1sg} {\sc neg-}come{\sc -sbjv-1sg-neg=irr}\\
	\rede{If I had not liked your village, I would not have come.}
	\bg.ŋ-gind-a-by-a-masa-n=niŋ=go=bi ikhiŋ n-leks-a=biǃ\\
	{\sc neg-}rot{\sc -sbjv-V2.give-sbjv-pst.prf-neg=ctmp=top=irr} how\_many {\sc 3pl-}become{\sc -sbjv=irr}\\
	\rede{If they (the apples) were not rotten, how many would we have!}
	 
	
The particle \emph{=pi} does not only occur on counterfactual clauses. It can also mark clauses referring to hypothetical situations where it expresses the speaker's assessment about the unlikelihood of an event. In example \Next, the hypothetical situation that it will rain stands in opposition to the more likely, but also yet unrealized scenario of a hail storm, as judged by the speaker. In this example, the particle expresses the speaker's assessment of the likelihood of an event. The speaker is 99 percent sure that it will hail and thus uses a  \isi{counterfactual clause} for  the proposition that contains the event of raining. 

\exg.wasik=se ta-ya=hoŋ=go=bi, ucun leks-a sa=bi, khatniŋgo phom ta-meʔ=na loppi\\
 rain{\sc =restr} come{\sc [3sg]-sbjv=seq=top=irr} nice become{\sc [3sg]-sbjv} {\sc cop.sbjv=irr} but hail come{\sc [3sg]-npst=nmlz.sg} probably\\
\rede{If it just rained, it would have been nice, but it there will probably be hail.}


The interaction between the irrealis marker and the clause linkage markers it attaches to is nicely illustrated by the following clausal minimal pair of hypothetical and counterfactual information in \Next. Both sentences are inflected for the Past Subjunctive (which is in many forms homophonous with the past \isi{tense}), but the clause linkage marker \emph{=niŋ} in \Next[a] establishes a simultaneous relation between the clauses while  \emph{=hoŋ} \Next[b] establishes a sequential relation, indicating that the event must have occurred prior to the main clause, and thereby implying that it has  not been realized. 

\ex. \ag.makalu nis-u-ŋ=niŋ=go=bi cond-a-sy-a-ŋ=bi\\
Makalu see{\sc -3.P-1sg.A[sbjv]=ctmp=top=irr} be\_happy{\sc -sbjv-mddl-sbjv-1sg=irr}\\
\rede{If I could see Mt. Makalu, I would be happy.}
\bg.makalu nis-u-ŋ=hoŋ=go=bi cond-a-sy-a-ŋ=bi \\
Makalu see{\sc -3.P-1sg.A[sbjv]=seq=top=irr} be\_happy{\sc -sbjv-mddl-sbjv-1sg=irr}\\
\rede{If I had seen Mt. Makalu, I would have been happy.}


\subsection{The \isi{mirative} particle \emph{rahecha} (from \ili{Nepali})}   
 
 The \ili{Nepali} \isi{mirative} \emph{rʌhechʌ \ti raicha},\footnote{Diachronically, it is a perfective form of the verb \emph{rahanu} \rede{to remain}.}  was borrowed into Yakkha and is used as a sentence-final marker of surprise about the propositional content, for instance when the speaker shares newly discovered information. In \Next[a], the speaker discovers that something must be wrong with the water pipe. In \Next[b] the speaker remembers something that she had not been aware of at first, only after someone reminded her.
 
 \ex. \ag. maŋcwa mi=na  rahecha\\
 water small{\sc =nmlz.sg} {\sc mir}\\
 \rede{The water got less.}\source{13\_cvs\_02.071}
 \bg.ka  ŋkhaʔla bhoŋ tutunnhe  bhauju=ghe    wa-ya-masa-ŋ=na    raecha, tunnhe=ba.\\
 {\sc 1sg} like\_that {\sc cond} up\_there sister-in-law{\sc =loc} exist{\sc -pst-pst.prf-1sg=nmlz.sg} {\sc mir} up\_there{\sc =emph}\\
 \rede{If it is like that, I had been uphill, at my sister-in-law's place, I see, uphill.} \source{36\_cvs\_06.399}
 
 
Mirativity in Yakkha is not restricted to surprise at the time of speaking, and it does not just indicate the speaker's surprise. The \isi{mirative} particle is frequently found in narratives, where the notion of unexpectedness either relates to some point in the time line of the story or to the hearer's state of mind, as the one who tells the story can rarely be surprised of what he tells himself. The function or the \isi{mirative} marker in narratives rather is to draw the attention of the hearer to the plot than to signal surprise.  Further examples from narratives are provided in \Next. In \Next[c], the speaker reports about  the events she saw in the Pear Story film \citep{Chafe1980The-Pear}. 
 
 %22\_nrr\_05.005 nna  limbuci            ŋwayanaŋa                  limbuten                 nluksun                        rahecha
 
 \ex. \ag.cokcoki-netham=be,       eko maɖa oʈemma kham  wait=na        rahecha\\
 star-bed{\sc =loc} one big plain ground exist{\sc [3sg;npst]=nmlz.sg} {\sc mir}\\
 \rede{In the place called Bed of Stars, there is a huge plain area!}\source{37\_nrr\_07.049}
 \bg.pak=na  baŋna mi=na  phalubaŋ, na   huture    sa-ma      raecha.\\
 younger{\sc =nmlz.sg} called small{\sc =nmlz.sg} Phalubang this a\_clan {\sc cop.pst-prf} {\sc mir}\\
 \rede{The younger one, the smaller one, Phalubhang, he was a Huture.}\source{22\_nrr\_05.074}
 \bg.khus-het-u=ha  raicha.\\
	steal{\sc -V2.carry.off-3.P[pst]=nmlz.nsg} {\sc mir}\\
	\rede{He stole them and carried them off!}\source{20\_pea\_02.015}
 
 As these  examples have shown, the \isi{mirative} prefers nominalized sentences as hosts (see §\ref{nmlz-uni-3}). Another marker often found together with the \isi{mirative} is the focus marker \emph{=le} (see above) that signals contrast to presupposed content, or the absence of presupposed content.

 
\section{Exclamatives}\label{ptcl-excla}

\subsection{The \isi{exclamative} particle \emph{lai}}
 
The  function of the sentence-level particle \emph{lai} (always in sentence-final position) is to add a certain vigor and force to assertions and exclamations (see \Next), to rhetorical questions as in \NNext[a], and to deontic statements, as in \NNext[b]. It can in most cases be paraphrased with the \ili{Nepali} particle \emph{ni}, or with English \rede{just} and \rede{of course}. It bears its own stress, but it can also be cliticized in fast speech. 
  
 \ex. \ag. ikhiŋ chiʔ=na yapmi laiǃ \\
 		how\_much greedy{\sc =nmlz.sg} \isi{person }{\sc excla}\\
 		\rede{What a greedy personǃ}
 	\bg. om lai!\\
 	{\sc cop} {\sc emph}\\
 	\rede{Yes, of course!} 
	  \bg. koi  khaʔla lai!\\
   		some like\_this {\sc excla}\\
   \rede{Some are just like this!}\source{13\_cvs\_02.10}  %- kunai kunai esto ni!
	
 	\ex.\ag. A: hetne wei-ka=na?   - B: paŋ=be lai. tuʔkhi ca=se   i    kheʔ-ma lai?\\
 		A: where live{\sc [npst]-2=nmlz.sg} - B: house{\sc =loc} {\sc excla} trouble eat{\sc -sup}  what go{\sc =inf} {\sc excla}\\
 		\rede{A: Where do you live? - B: Just at home. Why should I go to suffer (in a marriage)?!}
 	\bg. n-da-ci bhoŋ=go     im-m=ha=ci lai,  ca-m=ha=ciǃ\\
 			{\sc 3pl.A-}bring{\sc -3nsg.P[sbjv]} {\sc cond=top} buy{\sc -inf[deont]=nmlz.nsg=nsg} {\sc excla} eat{\sc -inf[deont]=nmlz.nsg=nsg}\\
 	\rede{If they bring some (fish), we would definitely have to buy and eat them!} \source{13\_cvs\_02.056}
	

\subsection{The \isi{exclamative} particle \emph{=ʔlo}}
 
 The particle \emph{=ʔlo} is used frequently in colloquial speech, signalling a certain lack of patience on the side of the speaker, and possibly also a frustrative notion. It occurs in assertions, imperatives, and questions alike. The marker is always bound, attaching to the sentence-final constituent, which is often another discourse particle, such as \emph{=ha} (see  §\ref{nmlz-uni-3}), emphatic \emph{=ba} or contrastive \emph{=le} (see \Next). When \emph{=ʔlo} attaches to other particles, the resulting units become independent words regarding  stress (e.g. \emph{haʔlo}, \emph{baʔlo}, \emph{leʔlo/laʔlo}, which are always stressed on the first \isi{syllable}), but not with regard to the \isi{voicing} rule. The most commonly heard particle is \emph{haʔlo} (see \Next[b]).

\ex. \ag.i=ʔlo?\\
what{\sc =excla}\\
\rede{What (the heck)?} (also used as a filler, like \rede{watchamacallit})
\bg. wa-ni haʔlo. n-da-me-n=na bhoŋ, n-da-nin-ni haʔloǃ\\
			exist{\sc [3sg]-opt} {\sc excla} {\sc neg-}come{\sc [3sg]-npst-neg=nmlz.sg} {\sc cond} {\sc neg-}come{\sc [3sg]-neg-opt} {\sc excla}\\
	\rede{It is alright! If it (the electricity) does not come, may it not come!}
	\bg. pi-haks-a baʔlo!\\
	give{\sc -V2.send-pst[3A;1.P]} {\sc excla}\\
	\rede{They just gave me away in marriage(, so what)!?} \source{06\_cvs\_01.042}
\bg.pi-m-ci-m baʔlo, ŋ-ga-ya-ma=hoŋ\\
 give{\sc -1pl.A-3nsg.P-1pl.A[sbjv]} {\sc excla} {\sc 3pl-}say{\sc -pst-prf=seq}\\
 \rede{Then let us give it to them eventually, they said, ...}\source{22\_nrr\_05.131}
  \bg.ta-met-u-ŋ baʔloǃ  ka=go     jokor leʔlo.\\
come{\sc -caus-3.P[pst]-1sg.A} {\sc excla} {\sc 1sg=top} self-deciding\_\isi{person }{\sc excla}\\
\rede{Of course I brought her (the second wife)! I can decide for myself!} \source{06\_cvs\_01.077} 

There is a dialectal variety of \emph{=ʔlo}, \emph{=kho} , which can, for instance, be found in the dialect spoken in Ankhinbhuin village (see \Next).

\exg. hetnaŋ tae-ka=na lai kho?\\
			where\_from come{\sc [npst]-2=nmlz.sg} {\sc emph} {\sc excla}\\\
	\rede{Where on earth did you come from?}
	
	
As \Next shows, the complex \isi{exclamative} particles are also found in combination  with the \isi{mirative}.
 
  \exg.n-so-ks-u-n=na  rahecha baʔlo\\
  {\sc neg-}see{\sc -prf-3.P-neg=nmlz.sg} {\sc mir} {\sc excla}\\
  \rede{Oh, he has not seen it!}\source{34\_pea\_04.039}
	
\subsection{The \isi{exclamative} \emph{hau} \ti \emph{=(a)u}} 

Another \isi{exclamative} particle is \emph{hau}. It is generally found at the end of a sentence, turning assertions into exclamations (see \Next). Judging from the available examples, it also carries a \isi{mirative} notion, expressing that the speaker is emotionally involved by making a discovery. It generally bears its own stress, but it can also occur cliticized, thus reduced to mere [au] or even just [u]. 

Occasionally, the particle is also found at the beginning of sentences in order to draw the attention of the hearer to the propositional content, for instance when the topic is changed \NNext[a], or when an exciting discovery has been made, as in \NNext[b].

\ex. \ag.sal=go     mund-i-ŋ=na=i    hau!\\
year{\sc =top} forget{\sc -compl-1sg.A=nmlz.sg=emph} {\sc excla}\\
\rede{I just forgot the year (of my birth), ha!}\source{ 06\_cvs\_01.034}
\bg. taŋkheŋ ka-ya=na              hau!\\
sky call{\sc [3sg]-pst=nmlz.sg} {\sc excla}\\
\rede{It thundered!}\source{13\_cvs\_02.062}
\bg.mun-nhe    sombare  daju=ge            ŋ-waiʔ=ya-ci=bu hau,                   jeppa!  \\
down\_here Sombar brother{\sc =loc} {\sc 3pl-}exist{\sc =nmlz.nsg=nsg=rep} {\sc excla} really \\
\rede{Down here at brother Sombar's house they have them (fish) for sale, I have heard, really!}\source{13\_cvs\_02.057}

 \ex. \ag.hau,  salle=be        heʔne wei-ka=na?\\
 {\sc excla} Salle{\sc =loc} where exist{\sc -2=nmlz.sg}\\
 \rede{Hey, where in Salle do you live?}\source{06\_cvs\_01.090}
 \bg.hau,  kha=go, eŋ=ga  yapmi  loʔa=ci=ca\\
 {\sc excla} these{\sc =top} {\sc 1pl.incl.poss=gen} person like{\sc =nsg=add}\\
 \rede{Really, they are like our people, too!}\source{22\_nrr\_05.044}
 
\section{Further particles}\label{ptcl-further}
\subsection{The \isi{alternative particle} \emph{=em}}\label{ptcl-alt}

If the speaker relates two alternative propositional contents, two clauses are juxtaposed and equally marked by the particle \emph{=em}, as shown in \Next. The marker may be fused with the preceding material, e.g. with  \emph{=na} to [nem] (see \Next[c]) or with \emph{=le} to [lem] (see \NNext[a]).

\ex. \ag. lag=ha=em lim=ha=em?\\
salty{\sc =nmlz.nc=alt} sweet{\sc =nmlz.nc=alt}\\
\rede{Is it (the tea) salty or sweet?}
\bg. hetniŋ hetniŋ om=em men=em  mit-wa-m=ha.\\
when when	{\sc cop=alt} {\sc cop.neg=alt} think{\sc -npst[3.P]-1pl.A=nmlz.nc}\\
	\rede{Sometimes we think: is it true or not?}
	\bg.   khumdu=n=em           ŋkhumdi=n=em?\\
tasty{\sc =nmlz.sg=alt} not\_tasty{\sc =nmlz.sg=alt}	\\
	\rede{Is it tasty or not?} \source{36\_cvs\_06.244}
	\bg.cek-met-u-m-ci-m-ga=m,                                    n-jek-met-u-m-ci-m-ga-n=ha=m?\\
talk{\sc -caus-3.P[pst]-2pl.A-3nsg.P-2pl.A-2=alt}	{\sc neg-}talk{\sc -caus-3.P[pst]-2pl.A-3nsg.P-2pl.A-2-neg=nmlz.nsg=alt}	\\
	\rede{Did you make them (prospective bride and groom) talk or not?} \source{36\_cvs\_06.323}
 
 The  particle is thus not only found in alternation questions; it can also express uncertainty, or a lack of knowledge  (see \Next).
 
 \ex. \ag. na=le=m,              lambu, heʔne lambu? na!\\
 this{\sc =ctr=alt} way where way this\\
 \rede{Is this the road, where is the road? This one!}\source{36\_cvs\_06.216}
 \bg.  i    luʔ-ni-me=he=m?\\
 what tell{\sc -compl-npst=nmlz.nc=alt}\\
 \rede{What will he possibly tell (us)?} \source{36\_cvs\_06.343}
 \bg.dharan=be     waisa-ci-ga=em,             hetne  waisa-ci-ga           haʔlo?\\
 Dharan{\sc =loc} {\sc cop.pst-du-2=alt} where  {\sc cop.pst-du-2} {\sc excla}\\
 \rede{Or were you in Dharan; where were you, then?} \source{36\_cvs\_06.315}
 
 
\subsection{The \isi{question particle} \emph{i}}\label{ptcl-q}

The \isi{question particle} \emph{i} marks truth-value questions, as in \Next. In most cases, the clause it attaches to is nominalized by \emph{=na} or \emph{=ha}. The marker may carry its own stress, but it also occurs phonologically bound, the alternation probably just being conditioned by fast speech. If it occurs independently, a glottal stop is prothesized, as it typically is in vowel-initial words in Yakkha. The marker is most likely ety\-mo\-lo\-gi\-cally related to the (bound) emphatic  marker \emph{=i} that has been described above.

	\ex. \ag.  raksi=ŋa  sis-a-ga=na=i? \\
		liquor{\sc =erg} kill{\sc [3sg.A]-pst-2.P=nmlz.sg=q}\\
		\rede{Are you drunk?}
		\bg. ka i? \\
		{\sc 1sg} {\sc q}\\
		\rede{I?}
		\bg.yakthu i?\\
		enough {\sc q}\\
		\rede{Did you have enough?} \source{36\_cvs\_06.248 }
\bg. nis-u-ga=na=i?\\
	know{\sc -3.P[pst]-2.A=nmlz.sg=q}\\
	\rede{Do you know it?}
 
 
 
\subsection{The \isi{insistive particle} \emph{(a)u}}  

The \isi{insistive particle} is found on imperatives and on hortatives, adding force and emphasis to these speech acts (see \Next). 
Possibly related to this marker is a  vocative that is found occasionally, as shown in \NNext. 

\ex. \ag.ah,    so-se           ab-a-ci,              au?\\
yes look{\sc -sup} come{\sc -imp-du} {\sc insist}\\
\rede{Yes, please come to look (at the bride), will you?} \source{36\_cvs\_06.416}
\bg.lu,     haku=chen   pog-i,         au?\\
alright now{\sc =top} raise{\sc -1pl[sbjv]} {\sc insist}\\
\rede{Alright, now let us get up, will we?} \source{36\_cvs\_06.556}

 
 \exg.a-na=u!\\
 {\sc 1sg.poss-}eZ=voc\\
 \rede{He, sister!}



\section{Interjections}\label{interjections}
 
 
 Yakkha has a closed class of \isi{interjections}, such as \emph{hoʔi}, \emph{bhela} and \emph{yakthu} (all meaning \rede{Enough!}), \emph{om} \rede{yes}, \emph{menna/manna} \rede{no} (identificational and existential, respectively), and \emph{issaŋ}, the latter being pronounced with a rising intonation. It stands for  \rede{I do not know} or \rede{I have no idea}, just like the \ili{Nepali} interjection \emph{khoi}. Examples are provided in \Next. Interjections are usually employed as sentence replacements, but in Yakkha it is still possible to additionally  express a topic, as in \Next[a]. The affirmative and the negative \isi{interjections} also have copular function (see also §\ref{cop-infl} and \ref{cop}). Interjections, like sentences, can host further discourse particles such as in \Next[d]. In the Ankhinbhuin dialect, \emph{om} has an alternant \emph{ommi}. 
 
 \ex. \ag.ka=ca hoʔi.\\
 {\sc 1sg=add} enough\\
 \rede{I also had enough.}
 \bg.yakthu=i?\\
 enough{\sc =q}\\
 \rede{Did you have enough?}
 \bg.men=na, nna=maŋǃ\\
 {\sc cop.neg=nmlz.sg} that{\sc =emph}\\
 \rede{No, it is that oneǃ}
 \bg.heʔne khet-u haʔlo,                 issaŋ       laʔlo!\\
 where carry\_off{\sc -3.P[pst]}  {\sc excla} {\sc ign} {\sc excla}\\
 \rede{Where did he take it - I have no idea!} \source{20\_pea\_02.031}
 \bg.om=ba!\\
 yes{\sc =emph}\\
 \rede{Yes, of course!}
 
 
 
A particle \emph{lu} (borrowed from \ili{Nepali}) can be  found sentence-initially (see example \ref{lu}) or replacing sentences. It is employed to initiate an action, both in commands and in hortatives.

 There is no expression for \rede{thank you} in Yakkha. Common greetings are \emph{sewayo}, and \emph{semeʔnenna}, which is the verb \rede{greet} with 1>2 inflection \rede{I greet you}. 

Geomorphic \isi{interjections}, prompting the addressee to look in a particular direction (\emph{tu} for \rede{Look uphill!}, \emph{mu} for \rede{Look downhill!} and \emph{yu} for \rede{Look over there!}) have been  discussed in §\ref{geodeixis}.  
  

  
 

