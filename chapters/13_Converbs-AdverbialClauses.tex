

\chapter{Adverbial clause linkage}\label{adv-cl}


\section{Introduction}

This chapter deals with the types of adverbial clause linkage in Yakkha. Adverbial clause linkage is defined here in a broad sense, i.e. as any type of clause linkage that does not result from verbal subcategorization (i.e. complement clauses) and nominal modification (i.e. relative clauses). Adverbial clauses are always marked by clause-final morphemes, and generally display the same constituent order as in independent clauses. 
Despite being quite diverse formally, what they all have in common is one functional property: they modify the propositional content of a clause adverbially. Adverbial clauses lack an autonomous profile; the event they refer to has to be interpreted in the perspective of another event.\footnote{The lack of an autonomous profile is, of course, also true for complement clauses and relative clause, see e.g. the definition of subordination in \citet{Cristofaro2003Subordination}.}


The majority of clause linkage variables analyzed here  suggest that adverbial clause linkage in Yakkha falls into two basic types: (i) converbal clauses,\footnote{See e.g. \citet{Haspelmath1995The-converb}, who defines converbs as non-finite adverbial forms that modify verbs or clauses.} non-finite, with overall operator scope and (ii) finite\footnote{“Finiteness” is understood here as a property of the verb. Used in this sense, finiteness does not entail all the properties typically associated with main clauses.  Although the verb carries inflectional markers (for person, TAM and negation) and can thus be regarded as finite in some types of adverbial clauses, these clauses cannot, for instance, have a right-detached position. Likewise,  they cannot contain certain clause-level operators such as the mirative \emph{rahecha} (see also \citet[220]{Lehmann1988Towards} for a scale of desentialization in  clause linkage typology).}  adverbial clauses (containing an inflected verb), allowing not only overall operator scope, but also main clause operator scope. Table \ref{cl-twotypes} roughly summarizes their  characteristics (see also \citet{Bierkandtetal_Scope}).\footnote{The percentages given in this chapter rely on the corpus described in §\ref{corpus}, containing 3012 clauses and roughly 13.000 annotated words.} However, the individual clause linkage types show varying degrees of semantic integration and dependency on the main clause, and the morphosyntactic properties of these adverbial clauses do not always correlate in expected ways. Distinctions made by one variable, e.g. coreference of certain arguments, may be crosscut by other variables, e.g. finiteness or operator scope, adding support to notions of clause linkage as an essentially multidimensional phenomenon (which are at least as old as \citet{HaimanThompson1984_Subordination}). Some variables, like center-embedding (not embedding in the structural sense, only positional, see below), have to be assessed statistically, not categorically. Furthermore, a comparison of illocutionary operators and negation reveals that operator scope does not always behave in a uniform way. Thus, a definition of types of adverbial clause linkage (and also any other subordination type) necessitates a fine-grained ana\-lysis of a considerable number of variables, in order to see how the properties cluster in particular languages (cf. \citet{Bickel2010_Capturing} for a general approach and \citet{Schackowetal2012_Morphosyntactic} for a case study on Puma (also Kiranti)). 



\begin{table}[htp]
\begin{centering}
\begin{tabular}{lll}
\lsptoprule
{\sc variables}&{\sc converbal} & {\sc finite}\\
\midrule
finiteness&stem&inflected verb\\
shared S/A arguments&mostly&constraint-free\\
operator scope&overall &overall or main clause only\\
(polarity, illoc.)&&\\
center-embedding&between 15 and 33 \%&maximally 2 \%\\
\lspbottomrule
\end{tabular}
\caption{Some characteristics of converbal and finite clause linkage}\label{cl-twotypes}
\end{centering}
\end{table}



Table \ref{cl-overview} shows an overview of the individual  Yakkha clause linkage types and their markers, classified according to their semantics. Apart from the two major types just mentioned, other types such as infinitival adverbial clauses are possible. Some of the clause linkage markers participate in more than one clause linkage construction, e.g. \emph{=hoŋ} (sequential clauses, narrative clause-chaining) and \emph{=nuŋ} (circumstantial clauses and temporal clauses with the meaning \rede{as long as}).  Some markers may combine with information-structural particles to yield further types, such as concessive and counterfactual clauses. The forms also demonstrate another common Tibeto-Burman characteristic: the use of case markers for clause linkage (cf. case markers for genitive \emph{=ga}, ergative \emph{=ŋa} and comitative \emph{=nuŋ}). 

\begin{table}[htp]
\begin{centering}
\begin{tabular}{llp}
\lsptoprule
{\sc type}&{\sc semantics} & {\sc marker}\\
\midrule
converbal& supine, purpose of movement & \emph{-se}\\
infinitival&purpose &\emph{-ma=ga}\\
infinitival&causal&\emph{-ma=ŋa}\\
converbal& simultaneous, manner& \emph{-saŋ}\\
converbal&negation& \emph{meN-...-le}\\
infinitival& manner (\rede{...ly/in a way that}, \rede{as much as}) &\emph{=nuŋ}\\
finite ({\sc sbjv})   & temporal (\rede{as long as}) &\emph{=nuŋ}\\
finite/infinitival&conditional, temporal&\emph{bhoŋ}\\
finite ({\sc opt})&purpose&\emph{bhoŋ}\\
finite/infinitival&sequential&\emph{=hoŋ}\\
finite&narrative clause chaining&\emph{=hoŋ}\\
finite&concessive&\emph{=hoŋ=ca}\\
finite&cotemporal&\emph{=niŋ(a)}\\
finite&counterfactual&\emph{=niŋ(=go)=bi} \ti  \emph{=hoŋ(=go)bi}\\
finite&interruptive&\emph{=lo}\\
\lspbottomrule
\end{tabular}
\caption{Adverbial clause linkage types in Yakkha}\label{cl-overview}
\end{centering}
\end{table}


In the remainder of this chapter, the clause linkage constructions will be discussed with regard to their morphosyntactic and semantic properties. Important parameters are operator scope, focussing possibilities and the ability to occur center-embedded (a statistical rather than a categorical variable in Yakkha). An adverbial clause is center-embedded if it occurs inside the main clause, i.e. if it is both preceded and succeeded by material of the main clause (examples will be shown in the individual sections).

Among the scope properties, the main distinction is that between overall scope and main clause scope (as proposed in \citet{Bierkandtetal_Scope}). Overall scope implies that the operator has scope over the whole sentence, including the link between the two events, as in \Next: the negation has scope over the whole event, allowing for different interpretations, depending on the question where the focus is (e.g. on \rede{he}, on \rede{play}, on \rede{after}, on \rede{ate} and so on). This scope type indicates a rather close semantic link between the two clauses, and may lead to a phenomenon such as \emph{negative transport},  as we will see below (cf. \citealt[Ch. 5]{Horn1989A-natural}, and \citealt{Bickel1993Belhare} for Belhare). 
%\footnote{This scope behavior is reminiscent of the Macro-Event Property \citep{Bohnemeyeretal2007_Principles} that has been discussed in Chapter \ref{verb-verb}.}

\ex. \emph{He did not play after he ate.}\\

In the case of main clause scope, in contrast, the operator does not reach beyond the main clause, as exemplified by \Next. The value of the scope parameter is straightforward to establish for negation, but it is potentially problematic for illocutionary force (cf. below).

\ex. \emph{After reaching his home, he did not feel like eating any more.}

\section{The supine converb \emph{-se}}\label{sup}

The supine converb is marked by the suffix \emph{-se} attaching directly to the stem of the verb in the dependent clause.  The converbal clause expresses the purpose of a motion event or caused motion event, i.e. the verb in the main clause always has to have motion semantics. The moving participant of the main clause has to be coreferential with the S or A of the supine clause: if the main verb is intransitive, the subjects of both clauses have to be coreferential (see \Next[a]); if the main verb is transitive, the subject of the converbal clause is coreferential with the P of the main clause (see \Next[b]). 
		
\ex. \ag.	yakkha ceʔya cin-se ta-ya-ŋ=na\\
			Yakkha language learn{\sc -sup} come-{\sc pst-1sg=nmlz.sg}\\
			\rede{I came to learn the Yakkha language.}	
	\bg.\label{se-appa}a-ppa=ŋa  yaŋ  nak-se  paks-a-ŋ=na\\
		{\sc 1sg.poss-}father{\sc =erg} money ask{\sc -sup}  send-{\sc pst-1sg.P=nmlz.sg}\\
		\rede{Father sent me to ask for money.}\source{01\_leg\_07.202}
		
For one speaker, who has lived  in Kathmandu for many years, the constraint on coreference appeared to be less strict. The subject of \emph{uŋma} \rede{drink} in \Next is coreferential with the G argument of the main verb, not with the argument that is caused to move.\footnote{This might be due to influence from parallel constructions in Nepali that are less constrained with regard to coreference.}

		\exg.		raksi  uŋ-se          ŋgha=ci=ha            jammai jammai ŋ-haps-u-bi-wa-ci=ha\\
	liquor   drink{\sc -sup} those{\sc =nsg=nmlz.nsg} all all {\sc 3pl.A-}distribute{\sc -3.P-V2.give-npst-3nsg.P=nmlz.nsg}\\
		\rede{They distribute the liquor among all the people, in order to drink it.}\source{25\_tra\_01.130}
		
The subordinate clause can be center-embedded, exemplified here by \Next[a]:\footnote{See also example \ref{se-appa}.} the verb \emph{whapma} would license an ergative, but an ergative is ungrammatical here, because the overt argument belongs to the intransitive main clause. Center-embedding can also be determined by semantic factors, as in \Next[b]. The deictic adverb \emph{to} belongs to the main clause, as such adverbs generally come with motion verbs. It was found that 33.8\% of the sentences with supine converbs gave positive evidence for the converbal clause to be center-embedded, which is the highest number among the adverbial clauses.\footnote{All percentages are from \citep{Bierkandtetal_Scope}, who used the same corpus that serves as database for this grammar, containing 3012 clauses. The positive evidence established does not tell us anything about the percentage of clauses that are not center-embedded, because this information can only be established in clauses with a sufficient number of overt arguments.} Examples  \Last and  \Next also show that dependent and main verb do not have to be adjacent, but in most cases they are.
 		
\ex.\ag. maghyam(*=ŋa)   tek=ci        whap-se                hoŋma=be    khy-a-ma\\
old\_woman(*=erg) fabric{\sc =nsg} wash{\sc -sup} river{\sc =loc} go{\sc [3sg]-pst-prf}\\
\rede{The old woman went to the river to wash clothes.}\source{01\_leg\_07.286}	
\bg.ka  to  cin-se=ca    kheʔ-ma   ŋan\\
{\sc 1sg} up teach{\sc -sup=add} go{\sc -inf[deont]} {\sc cop.1sg}\\
\rede{I have to go up to teach, too.}\source{36\_cvs\_06.102}
		
Occasionally, the supine converbal marker \emph{-se} is found in combination with the conjunction \emph{bhoŋ}, which is used for conditional, complement and purpose clauses (see §\ref{adv-cl-fin-purp}), and which is less restricted semantically and in terms of coreference. In \Next[a], the reason for adding this conjunction is probably that the requirement of coreference between the A of the converbal clause and the P of the transitive motion verb is not met. The reason why the conjunction is used in \Next[b] is not clear, however.  

\ex.\ag.ŋkha u-in=ci    taŋ-se    bhoŋ a-muk           end-u-ŋ=niŋa=go...\\
those {\sc 3sg.poss-}egg{\sc =nsg} take\_out{\sc -sup} {\sc purp} {\sc 1sg.poss-}hand   insert{\sc -3.P[pst]-1sg.A=ctmp=top}\\ 
\rede{When I put my hand inside (the hole in the tree) to take out its eggs, ...}\source{42\_leg\_10.028}
\bg. maŋdu ten=bhaŋ nniŋda tup-se bhoŋ ta-i-mi-ŋ=ha i, aŋoʈeŋba=ci\\
distant village{\sc =abl} {\sc 2pl} meet{\sc -sup} {\sc purp} come{\sc -1pl[pst]-prf-excl=nmlz.nsg} {\sc foc} brother\_in-law{\sc =nsg}\\
\rede{We came from far away from (our) villages to meet you, brothers-in-law.} \source{41\_leg\_09.028}


A supine clause is tightly linked to the main clause; any operator which can be attached to the main verb, whether of negation or of illocutionary force, will have scope over the whole complex event of movement with a purpose, illustrated for deontic modality in \LLast[b] above and for imperative, hortative and interrogative mood in \Next below.  

\ex.\ag.ah,    so-se           ab-a-ci,      au?\\
yes look{\sc -sup} come\_over{\sc -imp-du} {\sc insist}\\
\rede{Yes, please come over to look (at the newlyweds), will you?} \source{36\_cvs\_06.416}
\bg. khi khon-se         puŋda=be    khe-ci\\
yam dig{\sc -sup} forest{\sc =loc} go{\sc -du[hort]}\\
\rede{Let us go into the forest to dig yams.} \source{40\_leg\_08.006}
\bg.  wa=ci         ijaŋ ca  ca-se    n-da-me-n=ha=ci?\\
chicken{\sc  =nsg} why food eat{\sc -sup} {\sc neg-}come{\sc -npst-neg=nmlz.nsg=nsg}\\
\rede{Why do the chicken not come to eat?} \source{40\_leg\_08.069}

It is, however, possible to focus on the event denoted by the supine clause, as the next example illustrates by means of negation. The motion verb in the main clause contains the presupposed information and the supine clause contains the asserted information. In fact, this is the case for the majority of supine converbal constructions.\footnote{As also noted by \citet[12-7]{Haspelmath1995The-converb}, a subordinate clause (which converbal clauses are) narrows down the reference of the main clause.} In \Next[a], the speaker corrects another speaker's claim that the researcher came to do sightseeing. Hence, the purpose of coming is the controversial information, not the fact that she came. The purpose clause which contains the previous claim hosts the topic marker \emph{=go}, while the new, corrected  purpose clause receives the contrastive focus marker \emph{=le}, which is also frequently found in mirative contexts. Both operators are otherwise found on constituents, not at the end of clauses, which indicates that the converbal clause is not clause-like, but occupies the structural position of an adverbial in the sentence. The negated copula in \Next[a] emphasizes the contrast between purpose 1 and purpose 2. The same clause could also be paraphrased by \Next[b], with the negation marked on the main verb, and the converbal clause attracting focus.\footnote{In such a clause, the defautl reading is the one where the converbal clause attracts focus, but it could also entail that the whole event of \rede{coming in order to look} did not take place at all.} 
 	
\ex.\ag. so-se=go   men=na,     por-a         cok-se=le ta-ya-ma=na\\
look{\sc -sup=top} {\sc neg.cop.3sg=nmlz.sg} study{\sc -nativ} do{\sc -sup=ctr} come{\sc [3sg]-pst-prf=nmlz.sg}\\
\rede{Not to look (at our village), she came to STUDY!}\source{28\_cvs\_04.165}
\bg. so-se n-da-ya-ma-n=na\\
look{\sc -sup} {\sc neg-}come{\sc [3sg]-pst-prf-neg=nmlz.sg}\\
\rede{She did not come to look.}


Example \Next illustrates the same point for the scope and focus of questions. While the scope is over the whole event, i.e. over the connection between dependent and main clause, the focus of the question targets the converbal clause. This sentence was uttered when two people met at a water tap, and hence, the main verb \rede{come} must be part of the presupposition.\footnote{Questions relating to what one is doing or where one is going are a common way of greeting someone in the colloquial register.}

\exg. tek whap-se ta-ya-ga=na?\\
		clothes   wash{\sc -sup} come{\sc -pst-2=nmlz.sg}\\
		\rede{Did you come to wash clothes?}

Example \Next  and \NNext illustrate that the focus inherent in the negation (and the illocutionary force in \Next) may either be on the supine clause or on the whole event, but not on the main clause alone. Regardless of the focus options, both \Next[a] and \Next[b] can be circumscribed with \rede{It is not the case that you should go to watch the bride},\footnote{Two interpretations are possible: the absence of obligation, or a prohibition.} and \LLast[a] can be circumscribed with \rede{It is not the case that she came to study}. Thus, negating the supine clause necessarily results in the negation of the whole event, just as negating the main event necessarily entails the negation of the purposive event, because it is always interpreted in the perspective of the main clause. 

Negating the main event without negating the purposive event is not possible with \emph{-se} (see unacceptable \NNext[a]).  In order to achieve such an interpretation, a different strategy has to be used, namely the less restrictive purpose construction with the infinitive and \emph{bhoŋ} \NNext[b].  

\ex.\ag.beuli so-se ŋ-khy-a-n, cama ca-se seppa!\\
bride   look{\sc -sup}   {\sc neg-}go{\sc -imp-neg} food   eat{\sc -sup} {\sc restr}\\
\rede{Do not go to look at the bride, only to eat the food.}
\bg. beuli so-se ŋ-khy-a-n, ka=nuŋ wa-ya!\\
bride   look{\sc -sup} {\sc neg-}go{\sc -imp-neg} {\sc 1sg=com} stay{\sc -imp}\\
\rede{Do not go to look at the bride, stay with me!}

\ex.\ag.*im-se bhya=be ŋ-khy-a-ŋa-n=na\\
sleep{\sc -sup} wedding{\sc =loc} {\sc neg-}go{\sc -pst-1sg-neg=nmlz.sg}\\
Intended: \rede{In order to sleep, I skipped the wedding.} \\
(possible, but implausible reading: \rede{I did not go to the wedding to sleep.})
\bg. im-ma bhoŋ bhya=be ŋ-khy-a-ŋa-n=na\\
sleep{\sc -inf} {\sc purp} wedding{\sc =loc} {\sc neg-}go{\sc -pst-1sg-neg=nmlz.sg}\\
\rede{In order to sleep, I skipped the wedding.}


It has been  mentioned above that discourse particles targeting constituents, such as the topic marker \emph{=go} and the focus marker \emph{=le}, can be attached to supine clauses. Other constituent focus markers found on supine clauses are the additive focus particle \emph{=ca}  (see \Next[a]) and the restrictive focus particle \emph{=se} (see \Next[b]). In \Next[a], the additive focus marker has scope over the whole event; the event of going is not presupposed here. However, as main clauses cannot host constituent focus markers, \emph{=ca} has to be attached to the converbal clause. The situation is different in \Next[b], where the restrictive focus marker \emph{=se} targets only  the event denoted by the converbal clause.

		
	\ex.\ag.      kaniŋ piknik ca-se=ca                khe-i-mi-ŋ=ha\\
		{\sc 1pl} picnic eat{\sc -sup=add} go{\sc -1pl[pst]-prf-excl=nmlz.nsg}\\
		\rede{We also went for a picpic.}  \source{01\_leg\_07.268}
	\bg. chemha uŋ-se=se ta-ya=na, eŋ=ga ceʔya cin-se n-da-ya-n=na!\\
	liquor   drink{\sc -sup=restr} come{\sc [3sg]-pst=nmlz.sg} {\sc 1pl.incl=gen} language   learn{\sc -sup} {\sc neg-}come{\sc [3sg]-pst-neg=nmlz.sg}\\
	\rede{He just came to drink liquor, not to study our language.}
	
As I have shown above, the converbal clause and the main clause may have intervening constituents between them, although the corpus does not contain many instances of this. In \Next, the question word  constitutes the focussed information and is thus found in the preverbal position, the preferred  position for focussed constituents.

\exg.   tukkhi ca-se      i    kheʔ-ma=lai?\\
		 pain  eat{\sc -sup} what go{\sc -inf[deont]=excla}\\
		 \rede{Why should I go (and marry) to earn troubles?} \source{06\_cvs\_01.052}
	

\section{Infinitival purpose clauses in \emph{-ma=ga}}\label{maga}

Another way of expressing purpose involves an infinitive which is marked by the genitive marker (cf. §\ref{case-gen}). The purposive use of the genitive seems to be interchangeable with purpose clauses in \emph{bhoŋ} (discussed below), but \emph{bhoŋ} is not restricted to infinitives and thus more frequent. This construction is not only found with motion events; any event happening for the sake of another event can be expressed like in \Next. The clause linkage is less tight here. There is no constraint on the coreferentiality of the arguments. In \Next[a], one could argue that the constituent marked by the genitive is actually modifying the noun (\emph{kuʈuni}), but one also finds plenty of examples like \Next[b] and (c), where there is no noun. Argument marking remains as in simple clauses, as is evidenced by (c). This example also shows that the infinitive of purpose can be negated independently.

\ex.\ag.khoŋ nak-se kheʔ-ma=ga eko kuʈuni n-yog-wa\\
afterwards ask{\sc -sup}  go{\sc -inf=gen} one matchmaker {\sc 3pl.A-}search{\sc -npst[3.P]}\\
\rede{In order to go and ask (for the bride), they look for a matchmaker.} \source{25\_tra\_01.04}
\bg. kaĩci kob-u=hoŋ hek-ma=ga thag-u=na\\
scissors   pick\_up{\sc -3.P=seq} cut{\sc -inf=gen} open{\sc -3.P[pst]=nmlz.sg}\\
\rede{He picked up and opened the scissors in order to cut (something).}  \source{Cut-and-break clips, \citet{Bohnemeyeretal2010_cut}}
\bg.uŋci=ŋa men-ni-ma=ga cum-i\\
{\sc 3nsg=erg} {\sc neg-}see{\sc -inf=gen} hide{\sc -1pl[pst]}\\
\rede{We hid, so that they would not see us.} 
		
		
\section{Infinitival causal clauses in \emph{-ma=ŋa}}\label{manga}

Infinitives carrying an ergative marker are interpreted as finite causal adverbial clauses, as in \Next. Causal interpretations may, however, also obtain in temporal clauses marked by \emph{=niŋ} and \emph{=hoŋ}. The causal infinitives require coreference between the dependent and main clause S and A arguments, while the finite clause linkage types show no constraints in this respect.

\ex.\ag.\label{cungchen}cuŋ=chen   pyak  tuŋ-me=hoŋ=ca yoniŋ-kheniŋ  koʔ-ma=ŋa, cameŋwa  ca-saŋ  lak-ma      puk-ma=ŋa  ina=ca   thaha l-leks-a-n=na.\\
cold{\sc =top} very hurt{\sc -npst=seq=add} thither-hither walk{\sc -inf=erg.cl} food eat{\sc -sim} dance{\sc -inf} jump{\sc -inf=erg.cl} what{\sc =add} knowledge {\sc neg-}become{\sc -pst-neg=nmlz.sg}\\
\rede{Even though it was very cold, because of walking around, eating, dancing and jumping, we did not notice anything.}  \source{01\_leg\_07.270}
\bg.leʔnamcuk     puŋda koʔ-ma=ŋa         sak    tug-a-by-a.\\
whole\_day jungle walk{\sc -inf=erg} hunger hurt{\sc [3sg]-pst-V2.give-pst}\\
\rede{Having wandered around in the jungle the whole day, we got hungry.} \source{40\_leg\_08.016}

		
\section{The simultaneous converb \emph{-saŋ}}\label{sim}

The simultaneous converb, marked by the suffix \emph{-saŋ} attaching directly to the verbal stem, connects two events that happen at the same time or during the same period (see \Next[a]). The verb in the converbal clause cannot host any inflectional morphology; the converbal clause is dependent on the main clause regarding its TAM interpretation and the reference of its arguments. The converbal clause may also express the manner of how the main activity is done (see \Next[b], (c) and \NNext).  The S and A arguments of both clauses have to be coreferential. The construction cannot be used to refer to events that start during another event, i.e. for propositions like \rede{While walking, I slipped and fell}. Both events have to be ongoing at the point of reference. Punctual verbs like \rede{cough} and \rede{jump} receive an iterative reading in a converbal clause in \emph{-saŋ} (\Next[b] illustrates this effect).

		 \ex.\ag. yapmi  paŋ-paŋ=be nak-saŋ kheʔ-ma\\
		people   house-house{\sc =loc} ask{\sc -sim} go{\sc -inf[deont]}\\
	\rede{The people have to go from house to house, asking (for food).} \source{14\_nrr\_02.30}
	\bg.	khi=ga u-thap yok-saŋ maŋcusiŋcu khond-a-ŋ-c-u-ŋ\\
			yam{\sc =gen} {\sc 3sg.poss-}plant   poke-{\sc sim} so-and-so dig\_out{\sc -pst-excl-du-3.P-excl}\\
			\rede{Poking the yam plants, we somehow dug around (without satisfying results).}  \source{40\_leg\_08.012}
	\bg. sondu=ŋa    kisi-saŋ             luks-u:	\\
	Sondu{\sc =erg} be\_afraid{\sc -sim} tell{\sc -3.P[pst]}\\
	\rede{Frightened, Sondu told him: ...}\source{01\_leg\_07.200}

The case marking of the subjects in example \Last[a] and \Last[c] provides evidence for the converbal clause occurring center-embedded, which could be affirmed for 15\% of the simultaneous converb clauses in the present corpus. The nominative case in \Last[a] and the ergative case in  \Last[c] undoubtedly belong to the respective  main verbs, as the verbs in the converbal clauses would license different case marking (see §\ref{frames}). The opposite scenario can also be found, where the overt argument belongs to the converbal clause, as in  \Next, where the main verb would have licensed an ergative case marker.  

\exg.sondu  consi-saŋ  inca-ma=ga    khet-uks-u-ci\\
Sondu   be\_happy{\sc -sim} sell{\sc -inf=gen} carry\_off{\sc -prf-3.P[pst]-3nsg.P}\\
\rede{Happily, Sondu carried them (the fish) off to sell them.} \source{01\_leg\_07.229}


The above mentioned coreference constraint is strictly semantic, applying irrespective of the question of argument realization. In \Next, the subject is a non-canonically marked experiencer, realized as the possessor of the laziness (cf. §\ref{exp}). As long as the argument is highest-ranking in terms of semantic roles, i.e., the  most agent-like argument, it qualifies for the coreference constraint of the simultaneous converb.\footnote{The reduplication of the converb, as in this particular example, signifies either ongoing or iterative events.}

\exg.	o-pomma ke-saŋ ke-saŋ kam cog-wa\\
	{\sc 3sg.poss-}laziness come\_up-{\sc sim} come\_up-{\sc sim} work do{\sc -npst[3sg.A;3.P]}\\
		\rede{He does the work lazily.}
		
The constraint on coreference can be weakened under certain conditions: in constructions with unspecific or generic reference, it is not always observed (see \Next[a]). The sentence in \Next[b] is interesting because the subjects of two converbal clauses have to be combined to yield the reference of the main clause subject. 

\ex.\ag.	sala len-saŋ len-saŋ len khe-meʔ=na\\
		matter exchange-{\sc sim} exchange-{\sc sim} day   go{\sc -npst[3sg]=nmlz}\\
		\rede{Chatting and chatting the day goes by.} 
		\bg.  lok kho-saŋ                 yunca-saŋ              ghak pog-i-ŋ\\
		anger scratch{\sc -sim} laugh{\sc -sim} all stand\_up{\sc -1pl-excl}\\
		\rede{Partly angry, partly laughing, we all got up.} \source{40\_leg\_08.042}
		
Despite the close bonds between converbal clause and main clause, the converbal clause may have considerable length and internal complexity, as shown in example \Next[a], which features embedded direct thought in its converbal clause. One also finds sequences of several converbal clauses linked to one main verb, as in \Next[b] from a procedural text (cf. also \Last[b]).

	\ex.\ag.ka=go   yapmi isiʔ ŋan=na   rahecha mis-saŋ  u-ma    las-apt-uks-u\\
	{\sc 1sg=top} person ugly   {\sc cop.1sg=nmlz.sg}  {\sc mir} think{\sc -sim} {\sc 3sg.poss-}mother   return{\sc -V2.bring-prf-3.P[pst]}\\
	\rede{Thinking: “I am a bad person!”, he brought his mother home.} \source{01\_leg\_07.081}
	\bg. maŋgaŋba=ŋa   hon=na      yoŋ-saŋ     munthum   thak-saŋ         haks-wa=na\\
		ritual\_specialist{\sc =erg} that\_very{\sc =nmlz.sg} shake{\sc -sim} ritual\_knowledge recite{\sc -sim} send{\sc -npst[3.P]=nmlz.sg}\\
		\rede{Shaking that (gourd) and reciting the Munthum, the Manggangba does the worship.} \source{01\_leg\_07.135}
	
It should also be mentioned that this converbal structure is  used for a periphrastic continuative aspect construction that is probably calqued upon a similar structure in Nepali, too (see \Next, and §\ref{tense}).

\exg.uŋ tamba  pu-saŋ     pu-saŋ    khy-a-ma=hoŋ,  \\
		{\sc 3sg} slowly grow{\sc -sim} grow{\sc -sim} go{\sc [3sg]-pst-prf=seq}\\
		\rede{Having grown slowly, ...} \source{01\_leg\_07.005}	


The simultaneous converb is interesting with regard to scope and focus properties, because it shows that negation and illocutionary force operators may show distinct behavior. 

Let us first consider negation. Always marked on the main verb, the negation scopes over the whole event, ([A while B]{\sc neg}). The focus of the negation, however, is attracted by the converbal clause, also known as \rede{negative transport} \citep{Horn1989A-natural}. As pointed out by \citet{Bickel1993Belhare} on the corresponding converbal construction in Belhare, the converbal clause conveys rhematic information, elaborating on the main predication, and as such qualifies for being focussed on. A sentence like \Next[a] cannot be interpreted with only the main predicate being  negated, as the whole sentence is under the  scope  of the negation. To convey an interpretation with negation focussing on the main clause, another construction, for instance the causal construction in \Next[b], has to be used (cf. also \citet{Bierkandtetal_Scope}). A reading with the main clause negated is not possible under any circumstances.

\ex.\ag.	sala len-saŋ kam n-jog-u-m-nin=ha\\
			matter exchange-{\sc sim} work {\sc neg-}do{\sc -3.P[pst]-1pl.A-pl.neg=nmlz.nsg}\\
			*\rede{Chatting, we didn’t work.}\\
			\rede{We didn’t work chatting (but quietly).}
	\bg.	sala lem-ma=ŋa kam n-jog-u-m-nin=ha\\
			matter exchange{\sc -inf=erg} work {\sc neg-}do{\sc -3.P[pst]-1pl.A-pl.neg=nmlz.nsg}\\
			\rede{As/Because we chatted, we didn’t work.}
		
Negating both subevents at the same time is impossible as well. One sub-clause has to be foregrounded, similar to the effect of dissociating \emph{figure} and \emph{ground} that is known from Gestalt psychology (see e.g. \citet{Jackendoff1983Semantics}, called “Rubin effect” in \citet[48]{Bickel1991Typologische}). In other types of clause linkage, the choice may fall on either of the clauses, but the simultaneous converb allows only one reading with regard to negation. It is thus different from the supine converb, where negation could as well have a coordinative reading, i.e. with both subevents negated.

\exg.	chem lu-saŋ n-lakt-i-ŋa-n=ha\\
		song sing-{\sc sim} {\sc neg-}dance{\sc -1pl-excl-neg=nmlz.nsg}\\
		*\rede{We didn’t sing and didn’t dance.}\\
		\rede{We didn’t dance singing.}
		
		
Looking at the behavior of illocutionary operators, the picture is different, though. Illocutionary operators have scope over the whole event, too ([A while B]{\sc illoc}), but this may result either in focus on the converbal clause (with the main clause presupposed, see the question in \Next[a]), or in focus on both clauses (with nothing presupposed, see the question in \Next[b] and the imperatives in \Next[c] and \Next[d]). Thus, illocutionary operators behave differently from polarity operators in the simultaneous converb construction. 

\ex.\ag.	kos-saŋ tai-ka=na, a-na=u?\\
		walk\_around{\sc -sim} come{\sc [npst]-2sg=nmlz.sg} {\sc 1sg.poss-}eZ={\sc voc}\\
		\rede{Do you come walking, sister? (Did you come to us on a walk?)}  \source{36\_cvs\_06.564}
\bg.	chem lus-saŋ lakt-i-g=ha=i?\\
		song   sing-{\sc sim} dance{\sc -2pl[pst]-2-nmlz.nsg=q}\\
		\rede{Did you sing and dance?}
\bg.	chem lu-saŋ lakt-a-ni!\\
		song   sing-{\sc sim} dance{\sc -imp-pl}\\
		\rede{Sing and dance!}
\bg. hoŋkha so-saŋ   so-saŋ   paŋ=be   las-a-khy-a     yu    a-cya\\
 those\_very look{\sc -sim} look{\sc -sim} house{\sc =loc} return{\sc -imp-V2.go-imp}  {\sc excla}   {\sc 1sg.poss-}child\\
 \rede{Look at those (sticks marking the way) and go home, my son.} \source{01\_leg\_07.078}

	
\section{The negative converb \emph{meN...le}}\label{menle}

The negative converb is marked by the prefix \emph{meN-} and the suffix \emph{-le} being attached to the uninflected verb stem. The reason not to analyze these markers as a circumfix is that the prefix \emph{meN-} occurs as negation marker in other syntactic contexts as well, for instance with infinitives and nominalizations, and occasionally in comitative clause linkage. The negative converb is used to express that the event in the main clause will take place without another event, as shown in example \Next. Apart from the negation, its semantics are rather unspecified, e.g. with regard to the temporal relation obtaining between the clauses. In \Next, the events are in a sequential relationship; in \NNext they happen at the same time. Roughly 14\% of negative converbal clauses were found center-embedded.
	
	\ex.\ag.\label{menjale}ka cama men-ja-le ŋ-im-meʔ-ŋa-n=na\\
	{\sc 1sg} food   {\sc neg-}eat{\sc -cvb} {\sc neg-}sleep{\sc -npst-1sg-neg=nmlz.sg}\\
	\rede{I will not go to sleep without eating.}
	\bg.yo=na paŋ=be men-da-le hiks-a-ab-a-ŋ=na\\
across{\sc =nmlz.sg} house{\sc =loc} {\sc neg}-arrive-{\sc cvb} return{\sc -pst-V2.come-pst-1sg=nmlz.sg}\\
\rede{I came back without reaching the house across.}

Although the verb in the converbal clause does not carry inflectional markers, there is no constraint on the coreference of any arguments. The identification of the referents is resolved by the context alone. In my Yakkha corpus, the S or A of the converbal clause is not controlled by the main clause S or A argument in  42.9\% of the cases. An example is given in \Next.

\exg.u-ppa=ŋa tha men-dok-le nasa-lapmana=nuŋ phurluŋ khet-uks-u-ci=hoŋ hoŋma=be    khy-a-ma.\\
	{\sc 3sg.poss-}father={\sc erg} knowledge {\sc neg}-get-{\sc cvb} fish-rod{\sc =com}  small\_basket carry\_off{\sc -prf-3.P[pst]-3nsg.P=seq} river={\sc loc} go{\sc [3sg]-pst-prf}\\
		\rede{Without his father noticing, he (the son) carried off the fishing net and the basket and went to the river.} \source{01\_leg\_07.210}

	
A rather unexpected finding is that the negative converbal clause can be turned into an adnominal modifier by means of the nominalizers \emph{=na} and \emph{=ha}. This possibility is not attested for the other converbal clauses.

\ex.\ag. men-sen-siʔ-le=na    mendhwak\\
{\sc neg-}clean{\sc -V2.prevent-cvb=nmlz.sg} goat\\
\rede{an uncastrated goat} \source{31\_mat\_01.071}
 \bg. u-laŋ men-da-le=na piccha\\ 
	{\sc 3sg.poss}-leg {\sc neg-}come-{\sc cvb=nmlz.sg} child\\ 
	\rede{the child that cannot walk yet}
	
As for operator scope, the scope of negation includes the link between the clauses, and is thus over the whole sentence  ([neg.A and B]{\sc neg}), as was illustrated in example \ref{menjale} above. 


The illocutionary operators have scope over the whole sentence as well ([neg.A and B]{\sc illoc}) and, as we have already seen for the other converbs, the converbal clause often attracts focus. In example \Next[a], a question uttered when someone fell down, the event stated in the main clause is presupposed; the focus of the question lies on \emph{lambu mensoʔle}. Example \Next[b] illustrates that the converbal clause may contain presupposed information as well; it is taken from a discussion about learning methods in which the speaker stresses the importance of listening for learning.

	\ex.\ag. lambu men-soʔ-le lam-a-ga=na?\\
		way   {\sc neg-}look{\sc -cvb} walk{\sc -pst-2=nmlz.sg}\\
		\rede{Did you walk without watching the road?}
	\bg.	meŋ-khem-le i=ya nis-wa-m=ha?\\
			{\sc neg}-listen-{\sc cvb} what{\sc =nmlz.nsg} know{\sc -npst-1pl.A[3.P]=nmlz.nsg}\\
			\rede{What will we know without listening?}
	

In \Next[a], the converbal clause states a (negative) condition for the deontically modalized main clause: the event in the main clause has to happen within a time span specified by the converbal clause, which can be paraphrased by \rede{as long as not (X)}. The deontic modality has scope over the whole sentence, and the negative converbal clause is even focussed on. The condition it contains is integral to the requirement stated. Similarly, in \Next[b] a, the  converbal clause stands in a conditional relation to the main clause. Optionally, the Nepali postposition \emph{samma} \rede{until} can be added to the converbal clause to emphasize this. 

\ex.\ag. om   me-leŋ-le  las-a=hoŋ  pheri to  thithi    em-diʔ-ma=bu\\
bright {\sc neg}-become-{\sc cvb} return{\sc [3sg]-sbjv=seq} again up upright stand{\sc -V2.give-inf[deont]=rep}\\
\rede{Before the daylight, it (the rock) had to return and stand upright again (they say).} \source{37\_nrr\_07.051}
\bg. bagdata  men-nak-le    samma, [...]\\
marriage\_finalization {\sc neg}-ask-{\sc cvb}  until [...]\\
\rede{As long as the Bagdata (ritual) is not asked for, (the marriage is not finalized).} \source{26\_tra\_02.030}
%\bg. nna  mem-boŋ-kheʔ-le              sʌmma wa-ya bhoŋ, [...]\\
%that {\sc neg-}fall\_over{\sc -V2.go-cvb} until exist{\sc -[3sg]sbjv} {\sc cond}\\
%\rede{As long as it exists without falling, ...} \source{18\_nrr\_03.019}


Constituents within the clause may carry focus markers. Consider example \Next, with the interrogative pronoun being focussed on by the additive focus particle \emph{=ca}, that results in the exhaustive negation “any” in combination with the negative converb. 

\exg.kanciŋ i=ca meŋ-ka-le sok-khusa=se ca-ya-ŋ-ci-ŋ\\
{\sc 1du} what{\sc =add} {\sc neg}-say-{\sc cvb} look{\sc -recip=restr} eat.{\sc aux-pst-excl-du-excl}\\
\rede{Without saying anything, we just looked at each other.} \source{40\_leg\_08.070}


\section{Comitative clause linkage in \emph{=nuŋ}}\label{com-cl}

Comitative clause linkage is the semantically least specified  clause linkage type. It covers a wide functional range, specifying the  manner, time span  or some other circumstance under which the main event proceeds. The marker \emph{=nuŋ} is homophonous with the comitative case marker, thus conforming to a common Tibeto-Burman tendency of utilizing case markers as clause linkage markers. This clause linkage type is rather rare in the present corpus.

The comitative clause linkage is not only semantically rather underspecified, but it is not quite restricted in formal terms either. The clitic \emph{=nuŋ} may attach to uninflected stems (see \Next[a]), to infinitives (see \Next[b] and (c)), or to inflected verbs (see \Next[d]), yielding more or less similar manner or circumstantial readings, paraphrasable by \rede{in a way that} (as distinct from causal or consecutive \rede{so that}). 

There are no constraints on coreference. In \Next[a], the referents of the  S and A arguments, respectively, are not identical, while  they are in \Next[b]. The example in \Next[a] could alternatively be expressed by the negative converb construction discussed  in §\ref{menle}. 
		
\ex. \ag.	kaniŋ asen      men-ni=nuŋ         men-ni=nuŋ=ca                isisi leks-a-ma=ha\\
{\sc  1pl[erg]} yesterday {\sc neg-}know{\sc =com.cl} {\sc neg-}know{\sc =com.cl=add} ugly happen{\sc [3sg]-pst-prf=nmlz.nsg}\\
\rede{Yesterday, without us noticing, too, something bad has happened.} \source{41\_leg\_09.064}
\bg. kam=ca         cok-ma            haʔlo mimik,   ya-ma=nuŋǃ\\
work{\sc =add} do{\sc -inf} {\sc excla} a\_little be\_able{\sc -inf=com.cl}\\
\rede{One also has to work, man — at least a little (in a way that one manages to do it/as much as one can)ǃ}   \source{28\_cvs\_04.326}
\bg.a-pum si-ma=nuŋ uŋ-wa\\
{\sc 1sg.poss-}grandfather kill{\sc -inf=com.cl} drink{\sc -npst[3.P]}\\
\rede{My grandpa drinks in a way that will make him drunk (i.e. too much).}
	   \bg.	khaʔniŋgo liŋkha   ekdam     cog-a-nuŋ        cog-a-nuŋ       bis  wora khibak=ca  ipt-i-ci\\
	but a\_clan very\_much do{\sc [3sg]-sbjv=com.cl} do{\sc [3sg]-sbjv=com.cl} twenty {\sc clf}  rope{\sc =add} twist{\sc [3sg.A]-compl-3nsg.P}\\
		\rede{But the Linkha man, diligently (working and working) made twenty ropes, too.} \source{11\_nrr\_01.008}	


Comitative clauses, if inflected at all, are always in the subjunctive; indicative morphology (tense/aspect marking) is not expressed on them. Yakkha has two sets of subjunctives (cf. §\ref{mood}), both used in various irrealis contexts and in subordinate clauses. The first set (the Nonpast Subjunctive) is marked by the absence of any marking except person; it is also found in independent adhortative and optative clauses. The second set is in most cases identical to the past indicative paradigm, and is hence called Past Subjunctive. It is found in adverbial clauses and in counterfactuals. The difference between the two sets becomes evident in comitative clause linkage, too: the temporal reference of the main clause determines which set has to be used. In \Next[a] with nonpast reference, the Nonpast Subjunctive applies, while in \Next[b] with past reference the Past Subjunctive applies (the stem-final /s/ in \Next[b] surfaces only before vowels).


\ex.\ag.ka kucuma kha=nuŋ pi-wa-ŋ=ha\\
{\sc 1sg[erg]} dog   be\_satisfied{\sc [sbjv;3sg]=com.cl} give{\sc -npst[3.P]-1sg.A=nmlz.nsg}\\
\rede{I will feed the dog sufficiently (in a way that it will be satisfied).}
\bg. ka kucuma khas-a=nuŋ pi-ŋ=ha\\
{\sc 1sg[erg]} dog   be\_satisfied{\sc [3sg]-sbjv=com.cl} give{\sc [pst;3.P]-1sg.A=nmlz.nsg}\\
\rede{I fed the dog sufficiently (in a way that it was satisfied).}

The clause is not only reduced with regard to the tense/mood distinction; sentence-level markers and the main clause nominalization are not possible either. 

Adverbial clauses in \emph{=nuŋ} may also  translate into \rede{as long as} (Nepali: V-\emph{in-jhel}). This usage results in a less tight kind of clause linkage, as reflected by its scope properties (see below). Comitative clauses with an \rede{as long as} reading are always inflected for the subjunctive. The nonpast indicative is not possible in \Next, even though the proposition in the adverbial clause (i.e. \rede{you are in Tumok}) has realis status.% (*wameciganuŋ)

\exg.\label{ex-nung}tumok=pe wa-ci-ga=nuŋ cuwa uŋ-a-c-u\\
Tumok{\sc =loc} live{\sc [sbjv]-du-2=com.cl} beer    drink{\sc -imp-du-3.P}\\
\rede{As long as you live in Tumok, drink millet beer.}


The comitative is also used to derive lexical adverbs. Attached to uninflected stems of (ingressive-)stative verbs, the marker creates adverbs and predicative adjectives, such as \emph{cinuŋ} \rede{cold, chilly}, \emph{nunuŋ} \rede{well}, \emph{limnuŋ} \rede{sweet(ly)} or \emph{neknuŋ} \rede{softly} in \Next.  Some fossilized adverbs are found as well: they look like inflected verbs to which the comitative is added, for instance \emph{ŋkhumdinuŋ} \rede{not tasty}, but corresponding verbs are not available synchronically (see also Chapter \ref{adj-adv}).
 
\exg.khyu  nek=nuŋ leŋ-me\\
curry\_sauce    be\_soft{\sc =com} become{\sc [3sg]-npst}\\
\rede{The curry will become soft.} \source{28\_cvs\_04.054}


The different semantic possibilities of the comitative clause linkage result in different scope and focus options, too. In the manner/circumstance reading, the negation scope covers the whole complex clause, and the adverbial clause attracts focus, in the same way as the  converbal clauses discussed above (see \Next[a]). 

If, however, the clause has the reading \rede{as long as}, the negation has scope over the main clause only, and consequently, it cannot reach into the adverbial clause (compare the intended interpretation with the only possible interpretation in \Next[b]). This shows that one marker can participate in two very different kinds of clause linkage. In this particular case, the inflectional properties of the verb (infinitive vs. inflected for the subjunctive) match nicely with the (overall vs. main clause only) operator scope and focus properties.

	\ex. \ag.si-ma=nuŋ ŋ-uŋ-wa-ŋa-n=ha\\
	kill{\sc -inf=com.cl} {\sc neg-}drink{-npst[3.P]-1sg.A=nmlz.nsg}\\
	\rede{I will not drink in a way that I get drunk.}
	\bg.hiŋ-ŋa=nuŋ curuk ŋ-uŋ-wa-ŋa-n=na\\
survive{\sc -1sg[sbjv]=com.cl} cigarette   {\sc neg-}smoke{\sc -npst-1sg.A[3.P]-neg=nmlz.sg}\\
*\rede{I will not smoke cigarettes for my whole life. (i.e. \rede{I will stop in some years.})} only:\\
\rede{As long as I live, I will not smoke cigarettes.}\footnote{The main reading of the verb \emph{uŋma} is  \rede{drink}, but it can also refer to consuming other substances, as for instance smoking cigarettes or water-pipes (\emph{hukka}).}

The illocutionary operators are assumed to show the same divide with regard to the two different readings of adverbial clauses in \emph{=nuŋ}. Compare example \ref{ex-nung} above with the deontic clause \Next. Unfortunately, the current data set does not contain examples of imperatives or questions containing adverbial clauses in \emph{=nuŋ} to back up this assumption, because this clause linkage type is comparatively rare in the corpus.

	\exg.si-ma=nuŋ mẽ-uŋ-ma\\
	kill{\sc -inf=com.cl} {\sc neg-}drink{\sc -inf[deont]}\\
	\rede{One should not drink in a way that one gets drunk.}


\section{Conditional clauses in \emph{bhoŋ}}\label{adv-cl-cond}


Conditional clauses spell out circumstances that have to apply for the proposition in the main clause to hold true.  The conjunction \emph{bhoŋ} (carrying its own stress) may link finite clauses or infinitival conditional clauses  to a main clause. It is also employed in complemental clauses (see §\ref{fin-comp}) and in purpose clauses (see §\ref{adv-cl-fin-purp} below). The dependent clauses it marks are larger than converbial clauses, and also larger than the \emph{=nuŋ}-clauses discussed above: conditional clauses contain inflected verbs in subjunctive or indicative mood, and in contrast to the comitative clause linkage the verbs may even carry the nominalizing  clitics \emph{=na} and \emph{=ha}. 

Conditional clauses can be divided into those containing realis conditions, those containing irrealis conditions and those containing general or habitual conditions. This is also reflected in their  distinct formal properties: realis conditions can show indicative morphology (tense/aspect markers, see \Next), with the nonpast indicative if the condition holds at the time of speaking. Irrealis conditions (both hypothetical and counterfactual) are always marked for the past subjunctive (see \NNext). Generic and habitual conditional clauses, i.e. those without specified referents, may be in the infinitive (see \NNext[c]). Most of the examples in the  corpus are in the Past Subjunctive, i.e. irrealis conditionals are more frequent in discourse than realis conditionals (cf. \citet[463]{Genetti2007_Newari} for the same observation in Dolakha Newar). Conditional clauses may also have a temporal reading in Yakkha. 


	\ex.\ag.Kamala=ŋa mi tupt-wa=na bhoŋ tupt-u-ni \\
	Kamala{\sc =erg} fire   light{\sc -npst[3.P]=nmlz.sg} {\sc cond} burn{\sc -3.P-opt}\\
\rede{If Kamala lights the fire, may she do it (I do not care).}
	\bg.	batti n-da-me-n=na. wa-ni haʔlo! n-da-me-n=na bhoŋ, n-da-nin-ni haʔloǃ\\
	electricity   {\sc neg-}come{\sc [3sg]-npst-neg=nmlz.sg} exist{\sc [sbjv]-opt} {\sc excla} {\sc neg-}come{\sc [3sg]-npst-neg=nmlz.sg} {\sc cond} {\sc neg-}come{\sc [3sg]-neg-opt} {\sc excla} \\
	\rede{The electricity  does not come. It's alright! If it does not come, may it not come, thenǃ}
\bg. a-cya            lambu=be    pham-di-meʔ=na bhoŋ,\\
{\sc 1sg.poss-}child way{\sc =loc} entangle{\sc -V2.give-npst=nmlz.sg} {\sc cond}\\
\rede{When my child gets confused on the road, ...}  \source{01\_leg\_07.072}\\
(The speaker is expecting it to happen; it comes true later in the story.)

	
	\ex.\ag.kamala=ŋa mi tupt-u bhoŋ hand-wa=na\\
	Kamala{\sc =erg} fire   light{\sc -3.P[sbjv]} {\sc cond} burn{\sc -npst[3.P]}\\
	\rede{If Kamala lights the fire, it will burn.}
		\bg.\label{cond-sbjv}ka  mas-a-bhy-a-ŋ bhoŋ, \\
{\sc 1sg} get\_lost{\sc -sbjv-V2.give-sbjv-1sg} {\sc cond}\\
\rede{In case I get lost, ...} \source{18\_nrr\_03.016}\\
(The speaker is not expecting it to happen; it does not come true.)
\bg. chep-ma bhoŋ m-muʔ-ni-me-n=ha. men-chep-ma bhoŋ muʔ-ni-me=ha\\
write{\sc -inf} {\sc cond} {\sc neg-}forget{\sc -compl[3sg]-npst=nmlz.nsg} {\sc neg-}write{\sc -inf} {\sc cond} forget{\sc -compl[3sg]-npst=nmlz.nsg}\\
\rede{If one writes it down, one will not forget it. If one does not write it down, one will forget it.}\footnote{This clause could also read as \rede{if we write it down, ...}, as detransitivized clauses may also refer to first person nonsingular agents, see §\ref{detrans}.}


Since conditional clauses contain inflected verbs, they may have their own value for polarity \Next[a]. Negative polarity  on the main verb can either have  scope over the main clause only (see \Next[a]) or over the whole sentence (see \Next[b], here with the conditional clause attracting the focus of the negation).

\ex.\ag.	paŋ=be ta-meʔ-ma n-yas-u-ga-n bhoŋ, aniŋ=ga=ca n-leŋ-me-n, ŋ=ga=ca n-leŋ-me-n\\
			house={\sc loc} arrive{\sc -caus-inf} {\sc neg-}be\_able{\sc -3.P-2.A-neg[sbjv]} {\sc cond}  {\sc 1pl.excl=gen=add} 
			{\sc neg-}be{\sc -npst[3sg]-neg} {\sc 2sg.poss=gen=add}  {\sc neg-}be{\sc -npst[3sg]-neg}\\
			\rede{If you cannot bring it home, it will neither be ours nor yours.} \source{37\_nrr\_07.015}
	\bg.	a-hoʔma=ci ŋ-gy-a bhoŋ sidhak n-ja-wa-ŋa-n=ha\\ %, ulamphaŋ ... chen cawaŋha\\
			{\sc 1sg.poss-}cough={\sc nsg}  {\sc 3pl-}come\_up{\sc -sbjv} {\sc cond} medicine {\sc neg-}eat{\sc -npst-1sg-neg=nmlz.nsg}\\ %cold ... {\sc top} eat.{\sc npst.1sg>3sg=nmlz}\\
			\rede{I don’t take medicine when I have a cough (but when I have a cold).} 

			
Illocutionary force operators generally have scope over the whole sentence, with the conditional clause specifying the question, assertion or command contained in the main clause. 

\ex.\ag. wandik=ŋa njiŋ-phaŋ  phalumba            ta-ya             bhoŋ i    lum-me-c-u-ga=na? \\
tomorrow{\sc =ins} {\sc 2du.poss-}FyB fourth\_born\_male come{\sc [3sg]-sbjv} {\sc cond} what tell{\sc -npst-du-3.P-2=nmlz.sg}\\
\rede{If your uncle Phalumba comes tomorrow, what will you tell him?} \source{40\_leg\_08.028}
\bg. ucun n-leks-a-n bhoŋ pheri chept-u-so\\
nice {\sc neg-}become{\sc -sbjv-neg} {\sc cond} again write{\sc -3.P-V2.see[imp]}\\
\rede{If it does not turn out nice, try and write again.}


Since conditionals provide a background against which the main clause unfolds, it is not surprising that the particle \emph{=ko} that marks topical constituents is often found on conditional clauses \Next.

\exg. n-da-ci      bhoŋ=go     im-m=ha=ci   lai,  ca-m=ha=ciǃ\\
{\sc 3pl.A-}bring{\sc -3nsg.P} {\sc cond=top} buy{\sc -inf[deont]=nmlz.nsg=nsg}  {\sc excla} eat{\sc -inf[deont]=nmlz.nsg=nsg}\\
\rede{If they bring them (fish), we have to buy some, we have to eat them!} \source{13\_cvs\_02.077}

The inherent topicality of conditionals notwithstanding, many conditional clauses may host focus markers as well, both restrictive \emph{=se} (expressing that this is the only condition under which the main clause obtains) or additive \emph{=ca} (expressing that the condition is added to all conceivable conditions). Examples can be found in \Next. Additive focus on conditional clauses may yield a concessive reading, but the standard way of expressing  a concessive is by a combination of the sequential marker \emph{=hoŋ} and additive focus particle \emph{=ca} (see §\ref{adv-cl-conc} below).

\ex.\ag.\label{ex-ngkhotan}ŋ-khot-a-n  bhoŋ=se          kaniŋ   mimik    in-u-ca-wa-m-ŋ=ha\\
{\sc neg-}be\_enough{\sc [3sg]-sbjv-neg} {\sc cond=restr} {\sc 1pl} a\_little buy{\sc -3.P-V2.eat-npst-1pl-excl=nmlz.nsg}\\
\rede{Only if it is not enough, we buy some.} \source{28\_cvs\_04.038}
\bg. ka  m-ma-ya-ŋa-n bhoŋ=ca  sondu=ŋa    nda sukha=ŋa   hiŋ-me-ka=na\\
{\sc 1sg} {\sc neg-}exist{\sc -sbjv-1sg-neg} {\sc cond=add} Sondu{\sc =erg} {\sc 2sg} happiness{\sc =ins} survive{\sc -npst-2=nmlz.sg}\\
\rede{Even if I am no more, Sondu, you will survive easily.} \source{01\_leg\_07.193 }
\bg.aspatal=be     kheʔ-ma    bhoŋ=ca,       heʔne kheʔ-ma=na?\\
hospital{\sc =loc} carry\_off{\sc -inf[deont]} {\sc cond=add} where carry\_off{\sc -inf[deont]=nmlz.sg} \\
\rede{Even if we have to take him to a hospital, where to take him?} (i.e. there is no hospital) \source{36\_cvs\_06.175}
\bg. wandik nniŋ-cya=ci, jʌnma n-jog-a bhoŋ=ca   aphno-aphno paisa=ŋa hiŋ-m=ha=ci\\
tomorrw {\sc 2pl.poss-}child{\sc =nsg} birth  {\sc 3pl-}do{\sc -sbjv} {\sc cond=add} own-own money{\sc =ins} raise{\sc -inf=nmlz.nsg=nsg}\\
\rede{Later, when your children are born, too, you have to raise them with your own money.} \source{28\_cvs\_04.141}

Example \Last[c] is an example of a \rede{speech-act conditional} \citep[267]{Thompsonetal2007_Adverbial}. Speech-act conditionals do not primarily relate to the content of the main clause, but to the fact that the act of communication as such is taking place, as e.g. in \rede{In case you  did not know, she got married}. Thus, by definition, in  speech-act conditionals the illocutionary force has scope only over the main clause. Another example is shown in \Next. 

\exg.	jeppa cok-ma  bhoŋ  i=ha=ca                        im-ma  por-a          n-joŋ-me-ŋa-n\\
		true  do{\sc -inf} {\sc cond}  what={\sc nmlz.nsg=add} buy{\sc -inf} have\_to{\sc -nativ} {\sc neg-}do{\sc -npst-1sg-neg}\\
		‘To be honest, I do not have to buy anything.’\\
		(lit.: ‘If one does true, I do not have to buy anything.’) (28\_cvs\_04.187)

It is generally possible for constituents inside adverbial clauses to be focussed on. It is my impression that emphatic markers and constituent focus (e.g. by \emph{=se}, \emph{=maŋ} and \emph{=ca}, see Chapter \ref{particles}), such as in \Next, are more often found in the group of inflected adverbial clauses, such as conditional clauses and sequential clauses, than in converbal clauses.

\ex.\ag.\label{coughcond}a-hoʔma=ci=se ŋ-gy-a bhoŋ sidhak n-j-wa-ŋa-n=ha.\\
{\sc 1sg.poss-}cough{\sc =nsg=restr}  {\sc 3pl-}come\_up {\sc cond} medicine {\sc neg-}eat{\sc -npst[3.P]-1sg.A-neg=nmlz.nsg}\\
\rede{When I am just coughing, I do not take medicine.}  
\bg. eko=se          ŋ-ab-u                  bhoŋ, yapmi  sy-a-ma=na                       miʔ-ma=na.\\
one{\sc =restr} {\sc 3pl.A-}shoot{\sc -3.P[sbjv]} {\sc cond} person die{\sc -pst-prf=nmlz.sg} think{\sc -inf[deont]=nmlz.sg}\\
\rede{If they fire just once, one has to consider that someone has died.} \source{29\_cvs\_05.044}
\bg.nna,  nani,  nna  luŋkhwak=maŋ   khet-wa-ga=na                   bhoŋ, seʔni=ŋa       naʔmasek  lam-ma.\\
that girl that stone{\sc =emph} carry\_off{\sc -npst-2=nmlz.sg} {\sc cond} night{\sc =ins} night\_time walk{\sc =inf[deont]}\\
\rede{This stone, child, if you take away this very stone, you have to walk at night.} \source{37\_nrr\_07.012}
\bg. ŋkhiŋ=ca       n-leks-a-n         bhoŋ, \\
that\_much{\sc =add} {\sc neg-}become{\sc [3]-sbjv-neg} {\sc cond}\\
\rede{If that much is not possible either, ...}	\source{37\_nrr\_07.094}
	
	

\section{Purpose clauses in \emph{bhoŋ}}\label{adv-cl-fin-purp}

In addition to conditional finite relations, clauses marked by \emph{bhoŋ} may also express purposes, intentions, cognitive reasons and goals, or states of mind in general (see \citealt{Bickel1993Belhare} for a similar observation in Belhare).  Purpose clauses frequently have optative finite forms, as if they were a direct quote from the subject of the main clause (see \Next[a] and  \Next[b]). Example \Next[c], however, shows that the purpose clause can also take the perspective of the speaker. Purpose clauses may also be marked for indicative mood (see \NNext[a]), or have deontic finite forms (see \NNext[b]). 

The formal similarity between purposive and speech or thought representation structures (see §\ref{utterance-pred}) can be explained by the etymological origin of the marker \emph{bhoŋ}. It is assumed that, as in Belhare \citep{Bickel1993Belhare},  it developed from a combination of the reportative  marker \emph{=pu} and the sequential marker \emph{hoŋ}.\footnote{\citet{Bickel1993Belhare} mentions the contracted form [muŋ] in Belhare, combined of the quotative marker \emph{mu} (an allomorph of \emph{-bu \ti -phu}, which is cognate with Yakkha \emph{=pu}), and the ablative/sequential marker \emph{huŋ} (cognate to Yakkha \emph{=hoŋ}).}    Since the reportative marker \emph{=pu} is a clitic and gets voiced after vowels, this also explains why \emph{bhoŋ} has a voiced initial despite being an independent word with regard to stress.

\ex.\ag.nam phen-ni     bhoŋ, nam=bhaŋ   leŋ-ma=na; wasik ta-ni      bhoŋ wasik=phaŋ   luŋkhwak leŋ-ma=na.\\
sun shine{\sc [3sg;sbjv]-opt} {\sc purp}   sun{\sc =abl}   turn{\sc -inf[deont]=nmlz.sg}  rain come{\sc [3sg;sbjv]-opt} {\sc purp} rain{\sc =abl} stone turn{\sc -inf[deont]=nmlz.sg}\\
\rede{In order for the sun to shine, it (the stone) has to be turned away from the sun. In order for the rain to come, the stone has to be turned away from the rain.} \source{37\_nrr\_07.116-7}
\bg.yaks-u-ni bhoŋ lept-u-ris-u-ŋ=na\\
strike{\sc -3.P-opt} {\sc cond}  throw{\sc -3.P-V2.place-3.P[pst]-1sg.A=nmlz.sg}\\
\rede{I threw it at him, so that it would hit him.} 
\bg.ap-ŋa-ni bhoŋ ka-ya-ŋ=na\\
come{\sc -1sg-opt} {\sc purp} call{\sc -pst-1sg=nmlz.sg}\\
\rede{He called me, so that I would come.} 

Purpose clauses have the same internal structure as complement clauses, but they function differently in that they adverbially modify the main clause, and hence are optional, whereas complement clauses function as obligatory arguments, without which the main clause would be incomplete.
 In \Next[a], the purpose clause contains quoted speech, which is evident from the use of variables such as the speech-act participant pronoun \emph{nniŋda} and the deictic adverb \emph{nhe}. The conjunction \emph{bhoŋ} in \Next[a] is ambiguous between a purpose marker and a quotative marker. Parellel  structures with a purely quotative function of \emph{bhoŋ} can also be found (see \Next[b], translated with \emph{are} in Nepali). 

\ex.\ag. nniŋda nhe  wa-ma      n-dokt-wa-m-ga-n=ha  bhoŋ ikt-haks-a-ma-c-u-ci\\
	{\sc 2pl[erg]} here live{\sc -inf} {\sc neg-}get\_to\_do{\sc -npst-2pl.A[3.P]-2-neg=nmlz.nsg} {\sc purp} chase-{\sc V2.send-pst-prf-du-3.P-3nsg.P}	\\
	\rede{You will not get the chance to live hereǃ, (they [dual] said) and chased them away.} OR \\
	\rede{They chased them away, so that they would not get the opportunity to live there.} \source{22\_nrr\_05.012-3}
	\bg. haku miyaŋ ŋkha yabenpekhuwa=ci=le   soʔ-meʔ-ma=ci=em    bhoŋ yabenpekhuwa=ci=ja    n-soʔ-met-uks-u-ci\\
	now a\_little those healer{\sc =nsg=ctr} look{\sc -caus-inf[deont]=nsg=alt} {\sc quot} healer{\sc =nsg=add} {\sc 3pl.A-}look{\sc -caus-prf-3.P-3nsg.P}\\
	\rede{Now we better show the matter to those healers; (they said) and they also showed it to the healers.} \source{22\_nrr\_05.072}

	
Finally, clauses marked by \emph{bhoŋ} are also found with main verbs such as \emph{soʔma} \rede{look}, \emph{kuma} \rede{wait} and \emph{yokma} \rede{search}, illustrated in \Next. The activity in the main clause is done not in order to achieve whatever is expressed in the purpose clause, but with the goal to acquire knowledge or to achieve a state of mind about the proposition in the purpose clause.  These clauses always contain indirect speech or questions (see also \LLast[c] for another example of indirect speech). The perspective is anchored in the speaker and the speech situation, not in the subject of the clause. In \Next[a], if the perspective of the subject had been taken, the verb \emph{tayamacuha} should have been \emph{tayamacugha}  \rede{how much have you brought}.

	\ex. \ag.paŋ=be     a-ma=ŋa       ikhiŋ   ta-ya-ma-c-u=ha     bhoŋ khesup so=niŋa\\
	house{\sc =loc} {\sc 1sg.poss-}mother{\sc =erg} how\_much bring{\sc -pst-prf-du-3.P=nmlz.nc} {\sc purp} bag look{\sc -[3.P;pst]=ctmp}\\
	\rede{At home, when mother looked into the bag to see how much we had brought, ...} \source{40\_leg\_08.025}
	\bg.m-ba=ŋa                nasa ta-wa-ci=ha                      bhoŋ ku-ma=ŋa                  ku-ma=ŋa!\\
{\sc 2sg.poss-}father{\sc =erg} fish bring{\sc -npst-3nsg.P=nmlz.nsg} {\sc purp}  wait{\sc -inf[deont]=erg.cl} wait{\sc -inf[deont]=erg.cl}\\
\rede{Because I have to wait and wait for your father to bring the fish!} \source{01\_leg\_07.205}
	\bg. kucuma heʔne waiʔ=na bhoŋ yok-ma-si-me-ŋ=na\\
	dog where exist{\sc [npst;3sg]=nmlz.sg} {\sc purp} search{\sc -inf-aux.prog-npst-1sg=nmlz.sg}\\
	\rede{I am looking for the dog (where the dog is).}
\bg. jal hetne het-u=na bhoŋ hoŋma=ga u-lap-ulap lukt-a-ma\\
net where get\_stuck{\sc -3.P=nmlz.sg} {\sc rep/purp} river{\sc =gen} {\sc 3sg.poss}-wing{\sc 3sg.poss}-wing run{\sc [3sg]-pst-prf}\\
\rede{He ran along the river bank (in order to see) where the net got stuck.}  \source{01\_leg\_07.216}



\section{Sequential clause linkage and narrative clause-chaining in \emph{=hoŋ}}\label{adv-cl-seq}

The sequential marker \emph{=hoŋ}  indicates that the dependent clause event and the main clause event take place in a temporal sequence.  This clause linkage marker is attached to the  inflected verb (without \emph{=na} or \emph{=ha}) or to an infinitival form (see e.g. example \ref{camahong}). Sequential clauses can be in the indicative or in the subjunctive. There is no constraint on the corefence of arguments, but the S/A arguments are shared in 60.8\% of the occurrences in the corpus (see \Next). Center-embedding is attested only marginally (2.0\% of 307 sequential clauses), shown in \NNext[a] and (b). Center-embedding can be attested semantically or via the case marking of the arguments. In \NNext[a], the dual pronoun belongs to the last two verbs, while the sequential clause with \emph{tupma} \rede{meet} has plural reference. In \NNext[b], the subject is in the unmarked nominative that can only come from the intransitive main verb, and hence the transitive sequential clause must be center-embedded. Since overt arguments are rare, one cannot always assess the variable of center-embeddedness in the data. As \Next[b] and \NNext[b] have already shown, the sequential clause may also entail a consecutive reading; \NNext[c] is another example illustrating this.
	
	\ex.\ag.nam lom-me=hoŋ phoʈo khic-a cog-u-m.\\
	sun come\_out{\sc -npst[3sg]=seq} photo press{\sc -nativ} do{\sc -3.P-1pl.A[sbjv]}\\
	\rede{Let us take photos when the sun comes	out.}
	\bg.         nam=ŋa   heco=hoŋ,                        liŋkha   sarap pi=na.\\
	sun{\sc =erg} win{\sc [pst;3.P]=seq} Linkha spell give{\sc [pst;3.P]=nmlz.sg}\\
	\rede{As the sun had won, it put a spell on the Linkha man.} \source{11\_nrr\_01.020}
	\bg.nna  tiʔwa=go       majhya                 paghyam=ŋa   napt-het-a-ŋ=hoŋ                         kobeŋ-kobeŋ       lam-a-khy-a, ...\\
	that pheasant{\sc =top} a\_title  old\_man{\sc =erg} snatch{\sc -V2.carry.off-pst-1sg.P=seq} continously-{\sc redup} walk{\sc [3sg]-pst-V2.go-pst}\\
	\rede{That old  Majhya snatched the pheasant from me and left quickly, ...} \source{40\_leg\_08.037}
\bg.   mela=be [...] suku=ŋa     u-ppa             u-ma=ci=ca                        mund-y-uks-u-ci=hoŋ                           phaps-a-khy-a-ma. \\
fair{\sc =loc} [...] Suku{\sc =erg} {\sc 3sg.poss-}father {\sc 3sg.poss-}mother{\sc =nsg=add} forget{\sc -compl-prf-3.P-3nsg.P=seq} entangle{\sc [3sg]-pst-V2.pst-pst-prf}\\
\rede{In the fun fair [...], Suku forgot her parents and got lost.} \source{01\_leg\_07.152}


\ex.\ag.	kanciŋ to tub-i=hoŋ uks-a-ŋ-ci-ŋ=hoŋ  yo tas-a-ŋ-c-u-ŋ=ba.\\
			{\sc 1du} up meet{\sc -1pl[pst]=seq}  come\_down-{\sc pst-excl-du-excl=seq}  over\_there arrive{\sc -pst-excl-du-3.P-excl=emph}\\
			‘After we (plural) had met, we (dual) went down and then we (dual) arrived over there.’ \source{36\_cvs\_06.395}
	\bg.	paghyam=ca piccha nis-uks-u=hoŋ cond-a-sy-a-ma.\\
			old\_man={\sc add} child  see{\sc -prf-3.P=seq}  be\_happy{\sc [3sg]-pst-mddl-pst-prf}\\
			‘The old man was happy, too, when he saw the child.’ \source{01\_leg\_07.296}
	\bg.		chippakekek nis-u-ŋ=hoŋ yamyam ca-ŋ=ha\\
	disgusting see{\sc -3.P[pst]-1sg.A=seq} [eating]hesitantly eat{\sc [pst]-1sg.A=nmlz.nc}\\
	\rede{It smelled awfully, and so I ate hesitantly.}
			
	
	
Deriving from its sequential semantics, a secondary function can be observed as well: clauses marked by \emph{=hoŋ} can also express chains of coordinated events in narratives, a function comparable to the \rede{narrative converb} \citep{Nedjalkov1995Some}.  Some examples are shown in \Next. The clause-initial conjunction \emph{nhaŋ(a)} has a very similar function, and is historically derived from a demonstrative and the sequential marker. As \Next[b] shows, \emph{=hoŋ} is not restricted to verbal hosts. It may attach to adverbs, nouns and demonstratives in non-verbal clauses (i.e. to predicates of copular clauses).\footnote{The marker can, not function as a nominal coordinator. Nouns are coordinated by the comitative marker \emph{=nuŋ}, though.}

	\ex.\ag. nhaŋ    ŋ-und-wa-ci=hoŋ                   pheri, haiko=na=be           ŋ-khe-me=ha=ci.\\
		and\_then {\sc 3pl.A-}pull{\sc -npst-3nsg.P=seq} again other{\sc =nmlz.sg=loc} {\sc 3pl-}go{\sc -npst=nmlz.nsg=nsg}\\
		\rede{They go a bit further and then they pull them (the fish) out.} \source{13\_cvs\_02.13}
\bg. nhaŋ hattabatta   lukt-ab-a=hoŋ muccok=hoŋ paŋ=be ta-ya, nhaŋa    paŋ=be=hoŋ pheri lukt-ab-a=hoŋ sidhak end-a-bhy-a-ŋ=ba.\\
	and\_then hastily run{\sc -V2.come-pst[3sg]=seq} lifting\_lightly{\sc =seq} house{\sc =loc} come{\sc -pst[3sg]}, and\_then house{\sc =loc=seq}  again run{\sc -V2.come-pst[3sg]=seq}  medicine apply{\sc -pst-V2.give-pst-1sg.P=emph}\\
\rede{Then, he came running quickly, he lifted me up, and came to the house, and then, as (we were) in the house, he came running again and applied medicine (on my wounds).} \source{13\_cvs\_02.053-54}
\bg.	ka=ca    om  mit-a-ŋ=hoŋ       n-sy-a=ha              wa=ci             solok       ta-ŋ-ci-ŋ\\
		{\sc 1sg=add} yes think{\sc -pst-1sg=seq} {\sc 3pl-}die{\sc -pst=nmlz.nsg} chicken={\sc nsg} immediately bring{\sc [pst]-1sg.A-3nsg.P-1sg.A}\\
		‘I agreed and quickly brought the dead chicken.’ \source{40\_leg\_08.072}
		

The scopeof negation  is less restricted than in the converbal clauses. It may reach over  the whole sentence, or be restricted to the main clause. The first possibility is shown in \Next[a], with the focus of the negation attracted by the sequential clause. Pure main clause scope can be found in sentences like \Next[b]. 

\ex.\ag.\label{camahong}cama ca-ma=hoŋ sidhak men-ja-ma, ondaŋ=se ca-ma\\
			food eat{\sc -inf=seq} medicine {\sc neg-}eat{\sc inf[deont]} before={\sc restr} eat{\sc -inf[deont]}\\
			‘You don’t have to take this medicine after eating, but before.’
	\bg.	hiʔwa u-yin ind-wa=ci=hoŋ na u-yin=go m-beŋ-me-n\\
			air {\sc 3sg.poss-}egg lay-{\sc npst=3nsgP=seq} this {\sc 3sg.poss-}egg={\sc top}  {\sc neg-}break{\sc -npst[3sg]-neg}\\
			‘After he has laid his eggs in the air, this egg does not break.’ \source{21\_nrr\_04.040}

			
For illocutionary operators all configurations are possible, and some sentences are even ambiguous, as \Next[a]. In addition,  coordinative scope can be found, with the illocutionary force applying to each subclause separately (see the second reading of \Next[a], and \Next[b]).\footnote{Coordinative scope has proven to be difficult to distinguish from overall scope in most cases, and was thus disregarded for the classification of clause linkage types here.} Note that for the coordinative reading in imperatives the verb of the clause marked with \emph{=hoŋ} has to be in subjunctive mood. The imperative \emph{lab-u=hoŋ} is not possible in \Next[b]).

\ex. \ag. kamniwak sori yuŋ-i=hoŋ uŋ-u-m\\
	friend together sit{\sc -1pl=seq} drink-3.P-1pl.A[sbjv]\\
		 \rede{(We) friends having sat down together, let us drink.} OR\\
		 \rede{Let us friends sit down together and drink!}
	\bg.\label{ex_hong_imp_coord}nda cattu=nuŋ lab-u-g=hoŋ tokhaʔla ky-a!\\
			{\sc 2sg} firmly grab{\sc -3.P-2sg.A[sbjv]=seq} upwards come\_up{\sc -imp}\\
			‘Grab it firmly and come up!’ \source{01\_leg\_07.329}
			

An example of main clause scope is given in \Next[a]. Here, the first part of the sentence is uttered as a statement in surprise and it thus does not fall under the scope of the question in the main clause. In \Next[b], though, the question word is part of the sequential clause, and the main clause is presupposed, as the question is not about whether the addressee survived, but how he was able to hold such a thin rope.

\ex.\ag.\label{ex_sondu}sondu  khaʔla=na      cuŋ=be  tek   me-waʔ-le  jal kapt-uks-u-g=hoŋ hetnaŋ    ta-e-ka=na?\\
		Sondu like\_this={\sc nmlz} cold={\sc loc} clothes {\sc neg}-wear-{\sc cvb} net carry{\sc -prf-3.P-2.A=seq}  where\_from come{\sc -npst-2=nmlz.sg}\\
		‘Sondu, without clothes in this cold and carrying this net – where do you come from?’ \source{01\_leg\_07.232}	
\bg. khaʔla=na          mi=na  khibak=ŋa   imin tapt-a-g=hoŋ          hiŋ-a-ga=na?\\
like\_this{\sc =nmlzs.g} small{\sc =nmlz.sg} rope{\sc =erg} how hold{\sc -pst-2=seq} survive{\sc -pst-2=nmlz.sg} \\
\rede{How did you survive, holding such a thin rope?} (literally: \rede{You survived, holding such a thin rope HOW?}) \source{01\_leg\_07.343}

		
The sequential clause may also get focussed on as a whole, for instance by the restrictive focus particle, shown in example \Next from a conversation. 
		
		\ex. \ag.heʔniŋ lam-me-ci-g=ha?\\
		when walk{\sc -npst-du-2=nmlz.nsg}\\
		\rede{When will you (dual) set off?}
		\bg.cama ca-i-wa-ŋ=hoŋ=se lam-me-ŋ-ci-ŋ=ha \\
		rice eat{\sc -1pl-npst-excl=seq=restr} walk{\sc -npst-excl-du-excl=nmlz.nsg}\\
		\rede{We (dual) will set off only after we (all) had our meal.}

	
The coordinative reading of  \emph{=hoŋ} is also employed in a narrative strategy  to build up continuity, known as \rede{tail-head linkage} \citep[39]{Ebert2003Equivalents}. The previous clause, or just the verb, is repeated in a sequential clause, before adding new information in the main clause, as illustrated in \Next.

\exg.ŋkhoŋ=ca  a-chik ekt-a=na belak=ŋa  esap-esap thaŋ-ma    yas-a-ŋ. to  thaŋ-a-by-a-ŋ=hoŋ khaʔla   so-ŋ=niŋa=go eko maŋpha=na hoŋ nis-u-ŋ.\\
afterwards{\sc =add} {\sc 1sg.poss-}anger make\_break{\sc -pst=nmlz.sg} time{\sc =ins} swiftly-swiftly climb{\sc -inf} be\_able{\sc -pst-1sg} up climb{\sc -pst-V2.give-pst-1sg=seq} like\_this look{\sc [pst]-1sg.A=ctmp=top} one huge{\sc =nmlz.sg} hole see{\sc [pst]-3.P-1sg.A}\\
\rede{Nevertheless, as I was so angry, that I managed to climb up. When I had climbed up into the tree and looked, I saw
a large hole.} \source{42\_leg\_10.022-3}

\section{Concessive clauses in \emph{=hoŋca}}\label{adv-cl-conc}

It is a crosslinguistically common pattern for concessive clauses to be constructed by means of an additive focus marker or a scalar operator \citep[980]{Koenig1993_Focus}. Yakkha employs this strategy, too, combining the sequential clause linkage marker  \emph{=hoŋ} (see above) with the additive focus marker \emph{=ca} (see Chapter \ref{ptcl-additive}), as shown in \Next.\footnote{Occasionally, the  cotemporal clause linkage marker \emph{niŋ} is found in concessive clauses, too.} Concessive adverbial clauses indicate that the condition expressed in the adverbial clause is in contrast to the expected conditions, or that the condition is not relevant for the assertion to be true. As in \Next[a] the concessive is employed together with another converb (\emph{-saŋ}),  \emph{=hoŋ} is proved to not have sequential semantics any more in concessive clauses.

\ex. \ag. ropa  lamdhaŋ=be    cayoŋwa  lin-ca-saŋ=hoŋ=ca                       cama=ŋa   khot-u-co-nes-uks-u-ci\\
paddy\_field field{\sc =loc} food plant{\sc -V2.eat-sim=seq=add} food{\sc =erg} be\_enough{\sc -3.P-V2.eat-V2.lay-prf-3.P-3nsg.P}\\
\rede{Even though they ate what they planted in the field, the food was enough for them.} \source{01\_leg\_07.063}
\bg. marej end-a=hoŋ=ca khumdu sa\\
 pepper insert{\sc [3sg]-pst=seq=add} tasty {\sc cop.pst[3]}\\
 \rede{Even though pepper was added, it was tasty.}
 \bg. yok-ma=bu, cek-ma=i men-ni-ma=hoŋ=ca ceŋ-soʔ-ma=bu\\
 search{\sc -inf[deont]=rep} speak{\sc -inf=emph} {\sc neg-}know{\sc -inf=seq=add} speak{\sc -V2.look-inf[deont]=rep}\\
 \rede{One has to search for it (the language). Even though one does not know it, one has to try and speak, she said.}\source{07\_sng\_01.11}
 
 In §\ref{adv-cl-cond}, speech-act conditionals have been introduced \citep{Thompsonetal2007_Adverbial}. The same phenomenon is also found with concessive clauses. In \Next, the speaker excuses himself for talking about sensitive topics despite the fact that women are present. 
 
 \exg.nhe, mamu wanne=hoŋ=ca,                baru   khaʔla   cok-ma=na,                  nna, lakhe     coŋ-siʔ-ma=na.\\
 here girl exist{\sc [3;npst]=seq=add} instead like\_this co{\sc -inf[deont]=nmlz.sg} that castrated do{\sc -V2.prevent-inf[deont]=nmlz.sg}\\
 \rede{Here - even though women are present -  instead, one should do it like this, one should castrate them.} \source{28\_cvs\_04.228}
 
 Related to concessive clauses are \emph{exhaustive} clauses. They contain a question word functioning as a pro-form that may stand for the greatest conceivable temporal extension in \Next[a] and for any conceivable location in \Next[b]. Such clauses can either be marked just by the additive focus marker, or by the concessive \emph{=hoŋca}.
 
 \exg. ikhiŋ   kus-u=ca                        mima  n-lond-a-ma-n\\
 how\_much wait{\sc -3.P=add} mouse {\sc neg-}come\_out{\sc -pst-prf-neg}\\
 \rede{However much it waited, the mouse did not come out.} \source{04\_leg\_03.008}
\bg. heʔne khe-i-ga=hoŋ=ca      [...]         ka=ca         hakt-a-ŋ          au. \\
where go{\sc -2pl-2=seq=add} [...] {\sc 1sg=add} send{\sc -imp-1sg.P} {\sc insist}\\
\rede{Wherever you go, (...) send (a message) to me too.} \source{01\_leg\_07.276}


\section{Cotemporal linkage in \emph{=niŋ(a)}}\label{sim-finite}

The marker  \emph{=niŋ \ti =niŋa} combines clauses that refer to  events that happen at the same time, but that do not necessarily have coreferential S or A arguments. The two forms occur in free variation; no phonological or functional motivation for the alternation could be found. The event in the main clause generally unfolds against the background provided by the adverbial clause, as e.g. in \Next. Given its semantics, it is not surprising that the marker may  host the  topic particle \emph{=go} (see \Next[b]). The adverbial clause can be inflected for either one of the subjunctives (see §\ref{mood}), but also for different  tenses in the indicative mood. Generic statements are in the infinitive, as in  \Next[a].

As in the sequential construction, the S and A arguments do not have to be coreferential (see \Next), but in 37\% of the clauses they are. Center-embedding is as rare as in the sequential clause linkage pattern, but it is technically possible.

 \ex.\ag. uthamlaŋ uimalaŋ lam-ma=niŋa laŋ=ci n-sa-ma=ha=ci\\
		steeply\_uphill steeply\_downhill walk{\sc -inf=ctmp} leg{\sc =nsg} {\sc 3pl-}become{\sc .pst-prf=nmlz.nsg=nsg}\\ 
	 \rede{While  walking steeply uphill and downhill, the legs got stronger.}
	\bg.thawa=bhaŋ    to  ŋ-khy-ama=niŋ=go                    mamu nnhe=maŋ    wet=na=bu\\
ladder{\sc =abl} up {\sc 3pl-}go{\sc -prf=ctmp=top} girl there{\sc =emph} exist{\sc [3sg]=rep}\\
\rede{As they have gone up on the ladder, the girl is right there!} \source{22\_nrr\_05.111}


The scope properties are similar to those of the sequential clauses. With regard  to negation, overall and main clause scope are possible. Overall scope is shown in \Next[a], and main clause scope in \Next[b], which acquires a causal reading in addition to the basic cotemporal one. 

\ex.\ag.ta-ya-ŋ-ci-ŋ=niŋ cama n-ni-n=ha\\
		come{\sc -pst-excl-du-excl=ctmp} food {\sc neg-}cook{\sc [pst;3.P]-neg=nmlz.nsg}\\
		‘She wasn’t preparing food when we came.’
	\b.\label{ex-tikule-do}\gll	pyak=ŋa m-phat-uks-u=niŋa, Tikule=ŋa   i=ya=ca cokla cok-ma    por-a n-jog-ama-n\\
		many={\sc erg} {\sc 3pl.A-}help{\sc -prf-3.P=ctmp} Tikule={\sc erg} what{\sc =nsg=add} work  do{\sc -inf} must{\sc -nativ} {\sc neg-}do{\sc -prf[3sg]-neg}\\
		‘As so many helped him, Tikule did not have to do anything.’ \source{01\_leg\_07.019}



\section{Counterfactual clauses in \emph{=niŋ(go)bi} or  \emph{=hoŋ(go)bi} }\label{adv-cl-count}

Counterfactual clauses are marked mostly by \emph{=niŋ} for cotemporal clauses and occasionally by \emph{=hoŋ} for sequential clauses. The adverbial clause can be infinitival or inflected for the subjunctive, and is often marked by the topic particle \emph{=ko} (voiced [go]  in intervocalic and postnasal position, see §\ref{voicing}). Furthermore, both the adverbial clause and the main clause host the irrealis marker \emph{=pi} ([bi] due to the voicing rule). Clauses as in \Next can only have a counterfactual reading (i.e. it is established knowledge at the time of speaking that the condition does not obtain); they cannot be understood hypothetically. The irrealis marker, however, also occurs in hypothetical statements.



\ex.\ag. ka nis-u-ŋ=niŋ=bi ikhiŋ lu-ŋ=bi.\\
	{\sc 1sg[erg]} 	know{\sc -3.P[sbjv]-1sg.A=ctmp=irr} how\_much tell{\sc -1sg.A[3.P;sbjv]=irr}\\
	\rede{If I knew it, how much would I tellǃ}
	 \bg. diana=ca  piʔ-ma=hoŋ=go=bi   cond-a-sy-a=bi=baǃ\\
	Diana{\sc =add} give{\sc -inf=seq=top=irr} be\_happy{\sc -sbjv-mddl-sbjv[3sg]=irr=emph}\\
	\rede{After giving them to Diana too, she would have been happy!} \source{13\_cvs\_02.075}
	\bg. encho=maŋ     jal lep-ma   cind-a-ŋ-ga=niŋ=bi     hen   tuʔkhi n-ja-ya-ŋa-n    loppi.\\
	long\_ago{\sc =emph} net throw{\sc -inf} teach{\sc -pst-1sg.P-2.A=ctmp=irr} today trouble {\sc neg-}eat{\sc -sbjv-1sg-neg} probably\\
	\rede{If we had gone, I would have sent it to you.} \source{01\_leg\_07.252}
	\bg.nniŋ=ga ten a-sap n-thakt-u-n=niŋ=go=bi, ka n-da-ya-ŋa-n=bi.\\
	{\sc 2pl=gen} village {\sc 1sg.poss-[stem]} {\sc neg-}like{\sc -3.P-neg=ctmp=top=irr} {\sc 1sg} {\sc neg-}come{\sc -pst-1sg-neg=irr}\\
	\rede{If I did not like your village, I would not have come here.}

 

\section{Interruptive clauses in \emph{=lo}}\label{adv-cl-int}

The clause linkage marker \emph{=lo} does not figure prominently in the current corpus. It signals that a certain event takes place within the time span of another event, often interrupting it (see \Next[a]), or having an effect  contrary to the one expected (see \Next[b]). The verb in the adverbial clause often includes the inceptive V2 \emph{-heks} (see Chapter \ref{verb-verb}), which signifies that the action has just begun and has not yet been not completed (see \Next[a] and \Next[b]). With telic verbs, this implies that the event has not reached its end point yet \Next[c]. The marker is probably cognate with a comitative marker found e.g. in Belhare and Bantawa, which is also employed in clause linkage in Belhare \citep{Bickel1993Belhare, Doornenbal2009A-grammar}. As the examples are limited for this kind of clause linkage, the description of this clause linkage strategy cannot go into further detail.

\ex.\ag.a-ppa,             eh,  a-ppaǃ             cekt-heks-a=lo                   swak  wa-ya-by-ama\\
{\sc 1sg.poss-}father he {\sc 1sg.poss-}father talk{\sc -V2.cut-pst=itp} silent be{\sc -pst-V2.give-prf[3sg]}\\
\rede{He started calling: Father, hey, father!, when he suddenly fell silent.} \source{01\_leg\_07.179}
\bg.jal so-heks-u=lo                  mas-a-by-ama\\
net watch{\sc -V2.cut-3.P=itp} lose{\sc -pst-V2.give-prf[sbjv]}\\
\rede{As he was about to watch the net, it got lost.} \source{01\_leg\_07.217}
\bg.tokhaʔla khem-me,             to  khem-meʔ=lo           pheri heko=na=ga            u-cya            leŋ-me.\\
upwards go{\sc -npst[3sg]} up go{\sc -npst[3sg]=itp} again other{\sc =nmlz.sg=gen} {\sc 3sg.poss-}child become{\sc -npst[3sg]}\\
\rede{It flies up, and within the time it flies up, again this other one has a child (egg).} \source{21\_nrr\_04.046}
	
	