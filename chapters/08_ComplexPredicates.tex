\chapter{Complex predication}\label{verb-verb}

This chapter deals with complex predication, i.e. with predicates that consist of multiple verbal  stems. Yakkha follows a common South Asian pattern of complex predication where the verbs do not combine freely, but where  a class of function verbs (cf. \citealt{Schultze-Berndt2006_Taking}) has undergone grammaticalizations and lexicalizations (since not all verb-verb combinations are transparent).\footnote{The concatenation of verbs to specify the verbal semantics is a frequent pattern in South Asia and beyond; see for instance \citet{Butt1995The-structure, Hook1991_Emergence, Masica2001The-definition, Nespital1997Hindi, Pokharel1999Compound} on Indo-Aryan languages, and \citet{Matisoff1969The-syntax, DeLancey1991The-origin,  Bickel1996Aspect, Ebert1997A-grammar, Doornenbal2009A-grammar, Kansakar2005Classical} on other Tibeto-Burman languages, and \citet{Peterson2010_Kharia} on a Munda language.}

 They are employed in various semantic domains; they specify the temporal structure or the spatial directedness of an event, they change the argument structure of a predicate, and  they may also pertain to other kinds of information, such as modality, intentionality or the  referential properties of the arguments. Notably, there are semantic restrictions; the function verbs select lexically defined subsets of verbal hosts, a matter which leaves potential for a deeper investigation. The two simple examples below show the verb \emph{piʔma} \rede{give} functioning as a benefactive marker (see \Next[a]), and the verb \emph{kheʔma} \rede{go} functioning as a marker of directedness away from a reference point (see \Next[b]).
 
 	\ex.\ag. ka katha lend-a-by-a-ŋ\\
	{\sc 1sg} story  exchange{\sc -imp-V2.give-imp-1sg.P}\\
	\rede{Tell me a story.}
	\bg.kisa lukt-a-khy-a=na\\
	deer run{\sc -pst-V2.go-pst[3sg]=nmlz.sg}\\
	\rede{The deer ran away.}
 
 

Both lexicalized and grammaticalized instances can be found among the complex predicates in Yakkha, and the line between lexicalizations and grammaticalizations is not always easy to draw. Most of the function verbs (V2s) display multiple functions, which are in close interaction with the lexical semantics of the verbal base they combine with. The expressive potential of function verbs is vast, and the productivity and transparency of complex predicates shows great variability, a fact that supports the view that the boundary between grammaticalization and lexicalization cannot always be drawn sharply. 

This chapter is organized as follows:  \sectref{verb-verb-formal} introduces the formal properties of the Yakkha CPs and \sectref{verb-verb-functional} discusses the functional range of each V2.

\section{Formal properties}\label{verb-verb-formal}
 
Complex predicates (CPs) are basically defined as expressing one event in a monoclausal structure that contains a sequence of verbs \citep{Givon1991Some-substantive}. This makes complex predication similar to the definition of serial verb constructions \citep{Aikhenvald2006_Serial, Durie1997_Grammatical}, and yet there are significant differences, as we shall see below. 
 
In Yakkha, usually two (and maximally four) verbal roots may be combined to yield a more specific verbal meaning.\footnote{I have no evidence for predicates consisting of more than four stems in my Yakkha corpus, but I do not have negative evidence either.} The basic structure of a CP in Yakkha is as follows: the first verbal stem carries the semantic weight, and the second stem (the function verb or \rede{V2}) takes over the “fine-tuning” of the verbal semantics,  as in \Next,\footnote{As several stems have more than one function, depending on their lexical host, I have decided to gloss them with their lexical meaning.} where the function verb \emph{-nes}, with the lexical meaning \rede{lay}, contributes aspectual (continuative) information. The class of V2s in Yakkha is closed (synchronically) and relatively small; it comprises just twenty-five verbs.\footnote{There are a few V2 that only occur once in my data, and that are not treated further here, as generalizations about their function in complex predication are not possible yet: \emph{yukt} \rede{put down (for)}, \emph{cok} \rede{make} and \emph{rokt} (*\emph{tokt}) \rede{get}.} Most of the V2s have a corresponding lexical verbal stem, but there are also three morphemes that behave like function verbs without having a transparent verbal etymology (treated here as well, because of their similarity to “proper” function verbs. Complex predicates (including transparent and non-transparent CPs) roughly make up one third of the verbal lexicon. In the recorded  data of natural discourse, CPs make up only 17\% (across genres), but the current size of the corpus does not allow any strong statistical claims.



\exg. ka yog-u-nes-wa-ŋ=ha\\ 
{\sc 1sg[erg]} search{\sc -3.P-V2.lay-npst[3.P]-1sg.A=nmlz.nc}\\ 
	\rede{I (will) keep searching for it.} \source{18\_nrr\_03.008}


 As mentioned in the introduction, Yakkha has both grammaticalized and lexicalized CPs; one and the same V2 may have simultaneously developed a regular and productive function and an unpredictable, idiomatic meaning, which is not too surprising, as both developments have their origin in the metaphorical extension of verbal meanings. The distinction between lexicalized and grammaticalized forms is gradual, which has long been acknowledged in the typological literature  \citep{Lehmann2002_Thoughts, Diewald2010_Some, Lichtenberk1991_Gradualness, Himmelmann2004_Lexicalization} and in methodological approaches to grammar writing and lexicography \citep{Schultze-Berndt2006_Taking, Mosel2006_Grammaticography, Enfield2006_Heterosemy}.\footnote{Cf. also the distinction between \rede{collocation} and \rede{construction} in \citet{Svensen2009_Handbook}.} Structurally, there is no way to distinguish lexicalized and grammaticalized CPs; they show completely identical behavior. Thus, although it may hold on the level of individual tokens, the distinction between symmetrical and asymmetrical complex predicates that is made in \citet{Aikhenvald2006_Serial},  is not useful in determining the different types of CPs in Yakkha. There is, however, a tendency towards grammaticalization in the function verbs. All of the V2s have a grammaticalized function, and just some of them appear in idiosyncratic verb combinations as well. In order to capture the correspondences between the lexical semantics of the V2s on the one hand and their lexicalized and grammaticalized occurrences on the other hand, an excursus into the lexicon is inevitable in this chapter. 
 
The CPs in Yakkha roughly match criteria (a)—(e) of the definition of serial verb constructions in \citet[1]{Aikhenvald2006_Serial}: 


\begin{itemize}
\item (a) The verbs act together to refer to one single event.
\item (b) No overt marker of coordination, subordination, or syntactic dependency may occur.
\item (c) CPs are monoclausal (clause-final markers occur only after the last verb).
\item (d) CPs share tense, aspect and polarity values (i.e. these values can only be specified once).\footnote{However, tense and aspect interact with the meanings of the V2 independently of the lexical verbs, and some V2 block certain tense/aspect markers, e.g. the immediate prospective V2 \emph{-heks} \rede{be about to} is not possible with imperfective aspect.}
\item (e) CPs share core (and other) arguments.
\item (f) Each component of the construction must be able to occur on its own.
\end{itemize}



Criterion (a) and the question what constitutes one event is not trivial; most events one can think of are inherently complex and consist of several subevents. The criterion developed in \citet{Bohnemeyeretal2007_Principles} proved to be  useful in answering this question. It refers to the tightness of packaging of subevents that constitute one complex event. Bohnemeyer et al.\todo{Please use the citation commands \textbackslash citet and \textbackslash citep} call this criterion the \emph{Macro-Event-Property (	extsc{mep})}: 
\begin{quote}
A construction has the 	extsc{mep} if it packages event representations such that temporal operators necessarily have scope over all subevents. \citep[504--5]{Bohnemeyeretal2007_Principles}
\end{quote}

This criterion applies to all CPs in Yakkha, regardless of their individual functions. 

Criterion (b) distinguishes Yakkha CPs from infinitive constructions with auxiliaries, from complement-taking verbs and from periphrastic tense forms involving converbal markers. 

Monoclausality, criterion (c), usually correlates with eventhood.\footnote{But cf. \citet{Foley2010_Events} for a different view and counterexamples.} As the CPs even constitute one word (by the criteria of stress, morphophonological rules and clitic placement), the question of monoclausality is trivial in Yakkha. Further formal criteria are clause-final markers such as  nominalizers, converbs; they are never found inside a CP.  

Criterion (d) is restricted to modal and polarity markers in Yakkha, while the V2s interact with tense and aspect markers in their own ways. 

Most definitions of serial verbs have the requirement that at least one argument should be shared. In nearly all of the Yakkha CPs in my data, all arguments are shared; the CPs are formed by nuclear juncture in the sense of \citet[190]{Foleyetal1984Functional}. 

Yakkha CPs differ from  serial verbs as defined above, and also from function verbs as they are defined by \citet[362]{Schultze-Berndt2006_Taking} in criterion (f): not all function verbs can be found synchronically as independent lexical verbs. One morpheme (the middle marker \emph{-siʔ}) looks and behaves like a verbal stem, but can be traced back to a Proto-Tibeto-Burman suffix.\footnote{The cognate Belhare reflexive marker \emph{-chind} is a result of exactly  the same development \citep[560]{Bickel2003Belhare}.} The fact that suffixes even got reanalyzed as V2s show how salient complex predication is in the organization of the Yakkha verbal system. Another hybrid marker is \emph{-i \ti -ni}, tentatively called \emph{transitive completive} here. It occurs in paradigmatic opposition to another V2 \emph{-piʔ \ti -diʔ} \rede{give} which is found on intransitive verbs, yielding causative-inchoative correspondences like \emph{maʔnima} \rede{lose} - \emph{mandiʔma} \rede{get lost}. This marker (\emph{-i \ti -ni}) has no corresponding lexical verb either, and it does not license the typical double inflection that is found on CPs (see the discussion of the morphological structure of the CPs that follows this paragraph). But its occurrence in infinitives and its opposition to another V2 make it look like a V2 itself. Note that these two markers \emph{-siʔ} and \emph{-i \ti -ni}, although discussed here along with V2s, are not labelled \rede{V2} in the glosses.

In what follows I will outline the morphological structure of the CPs. The Yakkha pattern (and generally the Kiranti pattern) of complex predication differs from what we know from its Indo-Aryan sister construction (mostly termed \emph{(explicator) compound verbs} in the Indo-Aryan descriptive tradition). In Indo-Aryan CPs, the inflection  typically applies only to the V2 \citep{Montaut2004Hindi, Butt1997_Complex, Hook1991_Emergence}. In Yakkha, both verbs take inflectional material, though their inflection is subject to certain rules. They are laid out below, similar to Doornenbal's analysis of Bantawa CPs \citep[251]{Doornenbal2009A-grammar}.

\begin{itemize}
\item Prefixes attach to the first verb (V.lex).
\item The full suffix string attaches to the final, typically the second, verb (V2).
\item The V.lex takes maximally one inflectional suffix, and only if it has purely vocalic quality (i.e. \emph{-a} \rede{{\sc pst/imp/sbjv}}, \emph{-i} \rede{1/2{\sc pl}} or \emph{-u} \rede{3.P}).
\item There is no morphology on the first verb that is not underlyingly present in the complete suffix string, i.e. no morphologically empty \rede{dummy elements} are inserted.
\item Only inflectional suffixes, but not phrasal clitics,\footnote{Placing clitics between the verbal stems is indeed possible, for instance in Chintang \citep{Bickeletal2007Free}.} clause-level particles or clause linkage markers attach to the V.lex. 
\item Marked vowel or consonant sequences may block the inflection of the V.lex (for details see further below).
\end{itemize}

This pattern is henceforth called \emph{recursive inflection} (following the terminology and analyis in  \citealt{Bickeletal2007Free} on Chintang). As these rules show, the recursive inflection is both phonologically and morphologically informed. A prosodic constraint  requires a disyllabic host for the V2 , but the fulfillment of this requirement is conditioned by the availability of inflectional material, i.e. no dummy material is inserted. Example \Last above and example  \Next illustrate the recursive inflection: the first verb hosts the prefix and maximally one inflectional suffix, while the full suffix string and further material attach to the second verb. The  V2 \emph{-kheʔ} \rede{go} indicates the directedness of the movement away from a point of reference, and the V2 \emph{-nes} \rede{lay} indicates continuative aspect in \Last. 

\exg.asen lukt-i-khe-i-ŋ=ha\\
	yesterday run{\sc -1pl-V2.go-1pl-excl=nmlz.nsg}\\
	\rede{Yesterday we ran away.}  
 
Suffixes containing consonants cannot stand between lexical verb and V2. The infinitive marker \emph{-ma}, for instance, only attaches to the second verb, hence the verb in \Last[b] has the citation form \emph{khuŋkheʔma}. In this respect, Yakkha is different from closely related languages  such as Bantawa and Puma, where the infinitive marker attaches to both verbs in a CP \citep{Doornenbal2009A-grammar, Bickeletal2006The-Chintang}.\footnote{\citet[255]{Doornenbal2009A-grammar}, for instance, provides  the infinitive of \rede{forget}: \emph{manmakhanma}.} Yakkha CPs seem to be more tightly fused than the corresponding constructions in neighbouring languages, also with respect to other features such as stress and clitic placement (cf. \sectref{voicing}). 

Certain phonological conditions may block the inflection of the first verb, too, namely V2 stems that start in  /h/ or in a vowel (or stems that consist merely of a vowel). This is exemplified for /h/-initial V2 in \Next, and for vowel-initial V2 in \NNext.

\ex.\ag. so-haks-u-ci se=ppa\\
look{\sc -V2.send-3.P[pst]-3nsg.P} {\sc restr=emph}\\
\rede{He just glanced at them (his eyes following them as they went).} \source{34\_pea\_04.044}
\bg.yuŋ-heks-a!\\
sit{\sc -V2.cut-imp}\\
\rede{Sit here (while I go somewhere else).}


\ex.\ag.u-laŋ=ci leʔ-end-u-ci=ha\\
{\sc 3sg.poss-}leg{\sc =nsg} drop{\sc -V2.insert-3.P[pst]-3nsg.P=nmlz.nsg}\\
\rede{It (the plane) lowered its landing gear.}
\bg.jhyal peg-end-u=na\\
window shatter{\sc -V2.insert-3.P[pst]=nmlz.sg}\\
\rede{He (accidentally, unfortunately) shattered the window.}

Furthermore, when  certain stem combinations result in phonologically marked sequences like CV-V(C) and CVʔ-V(C), [n] may be inserted (see  \Next and  \sectref{nas-strat} for the exact conditions, e.g. for the reason why this does not happen in cases like \Last[a]).

\ex.\a.\glll leʔ-nen-saŋ\\
/leʔ-end-saŋ/\\
drop{\sc -V2.insert-cvb}\\
\rede{dropping}
\b.\glll pin-nhaŋ-saŋ\\
/piʔ-haks-saŋ/\\
give{\sc -V2.send-cvb}\\
\rede{marrying off (one's daughter)}


The first verb and the function verb do not necessarily have the same valency, as several of the previous examples and \Next show. In \Next[a], the sequence is transitive-intransitive, yielding an intransitive predicate; the sequence in \Next[b]  is notably labile-ditransitive, yielding an intransitive predicate. Thus, either  verb can be relevant for the argument structure of the whole predicate. However, the components of a CP are nearly always synchronized with respect to their valency; in general, the inflectional morphology attaching to the CP must be either from the transitive or from the intransitive paradigm. 

The two predicates shown in \Next both have non-transparent, lexicalized semantics, but there is a difference in the relations between first verb and V2. In \Next[a], the verb \emph{khus} with the independent meaning \rede{steal} acquires a new meaning in combination with a V2 with motion verb semantics. Apart from \rede{go}, other V2s are possible as well, such as \emph{-ra} \rede{come}, that can indicate that someone came fleeing. The verb \emph{maks} in \Next[b] does not occur independently, only in combination with various V2s. The  V2 here mainly specifies the intransitive valency.\footnote{It might seem anti-intuitive, but the function of the V2 \rede{give} is indeed detransitivization, similar to so-called \rede{give} passives (more below).} 

\ex.\ag.khus-a-khy-a-na\\
steal{\sc -pst-V2.go[3sg]-pst=nmlz.sg}\\
\rede{He escaped.}
\bg. maks-a-by-a=na\\
be\_surprised{\sc -pst-V2.give[3sg]-pst=nmlz.sg}\\
\rede{He was surprised.}


V2s can also be further grammaticalized to become suffixes and lose their verbal qualities. For instance, the etymological source of the two nonpast allomorphs \emph{-meʔ} and \emph{-wa} are most certainly the two verbal stems \emph{meʔ \ti me} \rede{do, apply}\footnote{At least diachronically, as evidence from neighbouring languages shows. In Yakkha, this verb only means \rede{put (waistband) around the waist}.} and \emph{wa} \rede{exist} (their choice depending on the participant scenario, see \sectref{npst}). These two morphemes occupy different slots in the verbal inflection,  they do not occur in the infinitives, and they do not license the recursive inflection, which shows that they are not treated as V2s any more. In a similar way, the perfect tense markers \emph{-ma} and \emph{-uks} seem to have developed from verbal stems.\footnote{\citet[66]{Butt2010_Light} makes a point that function verbs (\rede{vector verbs } in \citet{Butt2010_Light} and most works from the  Indo-Aryan descriptive tradition) are a class distinct from, e.g., auxiliaries. This is confirmed by the Yakkha data as well as data from neighbouring languages such as Chintang \citep{Bickeletal2007Free}, as function verbs and auxiliaries can co-occur in one clause. However, her claim that vector verbs are not subject to historical change as much as auxiliaries are cannot be confirmed in light of the verbal origin of some verbal suffixes in Yakkha and other Kiranti languages. The grammaticalization path proposed in \citet[108]{Hopperetal1993Grammaticalization}, namely: \emph{lexical verb (>vector verb) > auxiliary > clitic > affix} is unlikely, as the CPs are already one word phonologically, and the historical change towards an affix apparently took place without the intermediate stage of an auxiliary.} 


\section{The functions of the V2s}\label{verb-verb-functional}

\tabref{V2-table} provides an overview of the various V2 stems, their productive functions and their lexical origin, as far as it could be determined. As one can see, it is rather the norm that the V2s have more than one meaning or function; they are multi-faceted, which reflects the degree of their grammaticalization. Instances of lexicalizations will be discussed in the corresponding sections on each V2. The grammatical functions and the occurrence of a V2 in a lexicalized CP usually show some semantic parallels. 

As the table shows, many V2s are motion verbs; they are used to specify events with respect to their  relation to the surrounding landscape, along the two para\-meters of (i) the cline of the hill and (ii) the directedness towards or away from a point of reference, which is often, but not necessarily, the speech situation. 

For the V2s whose functions are mostly related to argument structure (\emph{-piʔ}, \emph{-i}, \emph{-ca}, \emph{-siʔ}) the reader is also referred to \sectref{trans-op}. Some V2s, especially those specifying the spatial  orientation,  can  only attach to a host that matches in transitivity, while others are not constrained regarding transitivity. They inflect intransitively when their lexical host  verb is intransitive, and transitively when the lexical verb is transitive. The valency values in the table have to be understood as maximally possible values.

The functions of the V2s pertaining to the temporal structure of a predicate have to be understood as tentative labels; since in-depth analysis of the verbal semantics, tense and aspect in Yakkha goes beyond the scope of this work and deserves a study in its own right.





\begin{table}[htp]
\resizebox{\textwidth}{!}{
\small
\begin{tabular}{lp{4.5cm}p{2.5cm}l} 
\lsptoprule
{\sc V2} & {\sc function}& {\sc lexical meaning}&{\sc valency}\\
\lsptoprule
\emph{-piʔ}&(a) benefactive/malefactive& \rede{give} & 3\\
 &(b) affected participants& &\\
(\emph{-piʔ} \ti  \emph{-diʔ})&(c) telic, completive (intrans.)& &1\\
\emph{-i \ti -ni}&completive (trans.)&(only V2)&2\\
\emph{-ca}&(a) reflexive, &\rede{eat}&2\\
 &(b) autobenefactive,& &(but often \\
 &(c) middle (intentional)& &detransitivized)\\
\emph{-siʔ}&middle (unintentional)&(only V2)&1\\
\emph{-kheʔ}&(a) motion away&\rede{go}&1\\  
&(b) telic (S/P arguments)&&\\ 
\emph{-ra} (*\emph{ta})&motion towards &\rede{come (from further away)}&1\\ 
\emph{-raʔ} (*\emph{taʔ})&caused motion towards &\rede{bring (from further away)}&3\\%27.11 example
\emph{-uks} & motion down towards&\rede{come down}& 1\\
\emph{-ukt} &caused motion down towards&\rede{bring down}& 3\\
\emph{-ap}&   motion towards& \rede{come (from close nearby)}&1\\
\emph{-apt}&  caused motion towards& \rede{bring (from close nearby)}&3\\
\emph{-ris} (*\emph{tis})&caused motion to a distant goal & \rede{invest, place} &3\\
\emph{-bhes}&caused horizontal motion &\rede{send, bring here}&3\\
\emph{-end \ti -neN}& (a) caused motion downwards,&\rede{insert, apply}&3\\
&(b) accidental actions, regret&&\\
\emph{-ket}&caused motion up and towards&\rede{bring up}&3\\
\emph{-haks} \ti \emph{-nhaŋ}& (a) caused motion up and away &\rede{send}&3\\
&(b) irreversible caused change-of-state&  &\\
\emph{-khet \ti -het}&(a) caused motion away &\rede{carry off}&3\\
&(b) telic, excessiveness (transitive)&&\\
\emph{-a \ti -na}&do X and leave object behind&(only V2)&3\\
\emph{-nes}&continuative&\rede{lay}&3\\
\emph{-nuŋ}&continuative&(probably \emph{yuŋ} \rede{sit})&1\\
\emph{-bhoks}&punctual, sudden&\rede{split}&2\\
	& events& &\\
\emph{-heks}&(a) immediate prospective&\rede{cut}&2\\
 &(b) do separately& & \\
\emph{-ghond}&spatially distributed events &\rede{roam}&2\\
\emph{-siʔ}&prevent, avoid&(probably \emph{sis} \rede{kill})&2\\
\emph{-soʔ}&find out, experience&\rede{look}&2\\
%\emph{-rokt} (*\emph{tokt})&get, achieve&\rede{get}&2\\ only 1 ex.
%cok - do - only 1 ex.
\lspbottomrule
\end{tabular}
}
\caption{Yakkha V2s, their functions and their lexical origins}\label{V2-table}
\end{table}



\subsection{The V2 \emph{-piʔ} (benefactive, affected participants)}\label{V2-give}%\rede{give} 

Verbs of  giving are often found grammaticalized as benefactive markers crosslinguistically. In Yakkha, the verb \emph{piʔma} \rede{give} has acquired the following grammatical functions: benefactive/malefactive, indicating affected participants (without necessarily expressing a causer), and a completive notion, translatable as \rede{already}, \rede{inevitably} or \rede{definitely}. Furthermore, it is found as a marker of intransitive valency in intransitive-causative pairs of lexicalized CPs.
 
In the benefactive function, the argument structure of the predicate changes, and a beneficiary participant is added as G argument to the verbal argument structure (note the agreement with the first person patient in \Next[b]). The morphosyntactic properties of the benefactive derivation are discussed in \sectref{benefactive}. 

\ex. \ag. end-u-bi-ŋ=ha\\
		insert{\sc -3.P[pst]-V2.give-1sg.A=nmlz.nsg}\\
	\rede{I poured her (some sauce).}
	\bg. ka katha lend-a-by-a-ŋ\\
	{\sc 1sg} story  exchange{\sc -imp-V2.give-imp-1sg.P}\\
	\rede{Tell me a story.}

The effect on the \rede{beneficiary} is not necessarily a desirable one; the V2 can also be employed to 
convey malefactive or at least undesirable events, as shown by \Next. Beneficiaries and also negatively affected participants gain syntactic properties that are typical of arguments; they  trigger agreement in the verb and they qualify as antecedents of reciprocal derivations (see \sectref{benefactive}). 

\ex.\ag.yakkha ceʔya cek-ma=niŋa limbu ceʔya ceŋ-bi-me=haǃ\\
Yakkha language  speak\sc{-inf=ctmp} Limbu language speak{\sc -V2.give-npst[1.P]=nmlz.nsg}\\
\rede{While talking in Yakkha, they answer in Limbuǃ}
\bg.khus-het-i-bi\\
steal{\sc -V2.carry.off-compl-V2.give[3.P;pst]}\\
\rede{He stole it (the basket) from him and carried it away.}\source{34\_pea\_04.024}
\bg. a-nuncha a-namcyaŋ=be thokt-a-by-a-ŋ=na!\\
{\sc 1sg.poss-}younger\_sibling {\sc 1sg.poss-}cheek\sc{=loc}  spit{\sc -pst-V2.give-pst-1sg.P=nmlz.sg}\\
\rede{My little brother spat on my cheeks!}


The   V2 \emph{-piʔ} \rede{give} can also indicate that some participant is affected by the event in undesirable ways, a common function in East Asian languages, also known as \emph{adversative passive} or \emph{\rede{give} passive} \citep{Keenanetal2007Passives, Yapetal1998_give}. This usage differs from the benefactive in semantic and in formal ways.  In the benefactive derivation the lexical verb can be marked by a suffix \emph{-t} under certain conditions (see \sectref{benefactive}). This is not possible in non-benefactive functions of  \emph{-piʔ}.  Furthermore, the resulting CP is always intransitive. A volitional agent  and an intentional action are not necessarily implied. The affected participant is often non-overt, and its reference is retrieved from the context or from possessive marking (see \Next[c]-\Next[d]). The affected-participant usage of \emph{-piʔ} can be distinguished from the benefactives also by a special infinitival form of this V2, the suppletive stem \emph{-diʔ}, which is only found in the intransitive usage of \rede{give}. 

\ex.\ag.wasik n-da-ya-n, nnakha ghak her-a-by-a=hoŋ\\
rain {\sc neg-}come{\sc [3sg]-pst-neg} that all dry{\sc -pst-V2.give[3sg]-pst=seq}\\
\rede{It did not rain, (and) after all that (i.e. their crops) dried up, ...}\source{14\_kth\_02.005}
\bg. ka tug-a-by-a-ŋ=na\\
{\sc 1sg} get\_ill{\sc -pst-V2.give-pst-1sg=nmlz.sg}\\
\rede{I got ill.}
\bg. a-khon thot-a-by-a=na\\
{\sc 1sg.poss-}neck get\_stiff{\sc -pst-V2.give[3sg]-pst=nmlz.sg}\\
\rede{My neck got stiff.}
\bg. a-yaŋ mas-a-by-a=ha [masabhya]\\
{\sc 1sg.poss-}money lose{\sc -pst-V2.give[3sg]-pst=nmlz.nsg}\\
\rede{My money got lost.}


The following semantic minimal pair also illustrates the semantic nuance added by  \emph{-piʔ} in contrast to \emph{-kheʔ}. Example \Next[a] is a  statement about  food that has been rotten for some time past, while in \Next[b] this fact is a new discovery that forces people to change their plans for the meal.

\ex.\ag.kind-a-khy-a=na\\
decay{\sc -pst-V2.go[3sg]-pst=nmlz.sg}\\
\rede{It is rotten (since long ago.)}
\bg. kind-a-by-a=na\\
decay{\sc -pst-V2.go[3sg]-pst=nmlz.sg} \\
\rede{It is rotten (but we had the plan to eat it now).}


The V2 \emph{-piʔ  \ti -diʔ} is also found in lexicalized CPs, contributing transitivity information. Certain lexical stems never occur independently; they have to be in a complex predicate construction. Their valency is not specified; different  V2s may combine with them to specify their transitivity. There are two  corresponding sets of predicates: one intransitive, built by adding the V2 \emph{-piʔ  \ti -diʔ} \rede{give}, and one transitive, built by adding the marker \emph{-i \ti -ni}  (see \sectref{V2-compl} below) to the lexical verb. This alternation is illustrated by \Next and \NNext. The alternations do not always have  the same direction in terms of argument structure. In \Next, the intransitive subject corresponds to the P argument in the corresponding transitive predicate, while in \NNext, it corresponds to the A argrument. \tabref{pi-ni} provides further examples of this symmetrical alternation.


\ex.\label{ex-pini1}\ag.maŋmaŋ-miŋmiŋ m-maks-a-by-a-ma\\
	surprised{\sc -redupl} {\sc 3pl-}wonder{\sc -pst-V2.give-pst-prf }\\
	\rede{They were utterly surprised.} \source{22\_nrr\_05.026}
 	\bg.ka nda mak-ni-meʔ-nen=na\\
	{\sc 1sg[erg]} {\sc 2sg} wonder{\sc -compl-npst-1>2=nmlz.sg}		\\
	\rede{I will surprise you.} 
	
\ex.\label{ex-pini2}\ag.ka mund-a-by-a-ŋ=na\\
{\sc 1sg} forget{\sc -pst-V2.give-pst-1sg=nmlz.sg}\\
\rede{I was forgetful.}
\bg.muʔ-ni-nen=na\\
forget{\sc -compl-1>2=nmlz.sg}\\
\rede{I forgot you.}


\begin{table}[htp]
\begin{centering}
\begin{tabular}{llll} 
\lsptoprule
\multicolumn{2}{c}{{\bf intransitive}}&\multicolumn{2}{c}{{\bf transitive}}\\
\midrule
\emph{mundiʔma}&\rede{be forgetful}&\emph{muʔnima}&\rede{forget}\\
\emph{maŋdiʔma}&\rede{be surprised}&\emph{maknima}&\rede{surprise}\\
\emph{mandiʔma}&\rede{get lost}&\emph{maʔnima}&\rede{lose}\\
\emph{phomdiʔma}&\rede{spill, get spilled}&\emph{phopnima}&\rede{spill}\\
\emph{himdiʔma}&\rede{(be) spread}&\emph{hipnima}&\rede{spread}\\
%\emph{}&\rede{}&\emph{}&\rede{}\\
\lspbottomrule
\end{tabular}\\
\caption{Transitivity alternations indicated by \emph{-piʔ \ti -diʔ} and \emph{-(n)i}}\label{pi-ni}
\end{centering}
\end{table}

A different kind of lexicalization is shown in \Next. Here, the lexical verb has an independent meaning (see \Next[a]), but it changes in unpredictable ways in the CP. However, the notion of an affected participant remains valid in this example, too.  

\ex.\ag.khap yoŋ-ma tarokt-uks-u\\
roof  shake{\sc -inf} start{\sc -prf-3.P[pst]}\\
\rede{The roof started shaking.}\source{27\_nrr\_06.031}
\bg. pik yoŋ-a-by-a=na\\
cow shake{\sc -pst-V2.give[3sg]-pst=nmlz.sg}\\
\rede{The cow got scared.}  


Furthermore, the  V2 \emph{-piʔ} may emphasize the orientation towards an end point or to the completion of an event, best translatable with the adverb \emph{already} in English (see \Next).

\ex.\ag. mi=go sy-a-by-a-ma [sebyama]\\
 fire{\sc =top} die{\sc -pst-V2.give[3sg]-pst-prf}\\
 \rede{But the fire has gone out already.}  (said to indicate that there is no need to extinguish it)
\bg.makai end-i-bi-g=ha?\\
 corn insert{\sc -2pl-V2.give-2=nmlz.nsg}\\
 \rede{Did you already plant the corn?}\source{06\_cvs\_01.080}
 \bg. ca-ya-by-a-ŋ=na\\
 eat{\sc -pst-V2.give-pst-1sg=nmlz.sg}\\
 \rede{I already finished eating (the whole procedure is done, including washing hands).}


Finally, \emph{-piʔ \ti -diʔ} can also express that something happens immediately,  inevitably, without delay or with certainty. Such a function, again, is only found with intransitive verbs (see \Next). 

	\ex.\ag. duru nam-ma=hoŋ a-chippa ŋ-gen-di-me\\
	cow\_milk smell{\sc -inf=seq} {\sc 1sg.poss-}disgust {\sc 3pl-}come\_up{\sc -V2.give-npst}\\
	\rede{Smelling milk, I will certainly get disgusted.} 
 	\bg. am-di-me-ŋ=na\\
		come{\sc -V2.give-npst-1sg=nmlz.sg}\\
	\rede{I will come without delay.} 
	
To conclude, this V2 shows an immense variety of functions (the Nepali translations need as much as three different verbs to cover the range of this marker: \emph{dinu} \rede{give}, \emph{hālnu} \rede{insert} and \emph{saknu} \rede{finish}), and a deeper understanding of the interactions of this V2 with the respective lexical hosts would require more research.



\subsection{The quasi-V2 \emph{-i \ti -ni} (completive)}\label{V2-compl}

The marker  \emph{-i \ti -ni}  partly behaves like a V2,\footnote{Like V2 stems, it appears in the infinitival form of a CP, in contrast to inflectional affixes that never occur in infinitives, even when they have V2 origin (cf. \sectref{verb-verb-formal}). It does not license the recursive inflection pattern, though.} although is does  not  correspond to a lexical verb. It  marks  completed transitive actions (see \Next and \sectref{completive}). As it codes transitivity information, it stands in complementary distribution with the intransitive completive use of the V2 \emph{-piʔ} \rede{give}, as examples \ref{ex-pini1} and \ref{ex-pini2} and \tabref{pi-ni} in  \sectref{V2-give} above have illustrated. The  alternation between \emph{-i} and \emph{-ni} is phonologically conditioned: the allomorph \emph{-ni} surfaces before consonants (see \sectref{morphophon}). If \emph{-i} is followed by a vowel, it may also become a glide, as in \Next[c].

\ex.\ag.uŋci-camyoŋba  ŋ-geks-a-n,   nam=ŋa   ghak her-i\\
{\sc 3nsg.poss-}food {\sc neg-}ripen{\sc [3sg]-pst-neg} sun{\sc =erg} all dry{\sc -compl[pst;3.P]}\\
\rede{Their food did not ripen, the sun dried up everything.}\source{14\_nrr\_02.004}
\bg.nhaŋ     pik=ci=ca  chuʔ-ni-ma     pʌrne=bu,  luŋkhwak=ca  hoʔ-ni-ma     pʌrne\\
and\_then cow{\sc =nsg=add} tie{\sc -compl-inf[deont]} have\_to{\sc =rep}  stone{\sc =add}  pierce{\sc -compl-inf[deont]} have\_to\\
\rede{And he both had to tie the cows, they say, and he had to hole out a (grinding) stone.}\footnote{Context: the protagonist has to finish tasks within one night in order to win a bet.} \source{11\_nrr\_01.005}
\bg.n-chimd-y-uks-u-n-ci-ŋa-n=ha\\
{\sc neg-}ask{\sc -compl-prf-3.P[pst]-neg-3nsg.P-1sg.A-neg=nmlz.nsg}\\
\rede{I have not finished asking them.}

There are also some  lexicalized instances of this marker. In \emph{toknima} \rede{touch}, for instance, the interpretation is holistic and cannot be achieved by analytic decomposition of the predicate into its components. The verbal stem \emph{tok(t)} means \rede{get} when it occurs independently. Another example is \emph{themnima} \rede{compare}, with a stem \emph{themd} that is not attested independently.


\subsection{The V2 \emph{-ca} (reflexive, middle, autobenefactive)}\label{V2-eat}%\rede{eat} 

 The polysemous V2 \emph{-ca} \rede{eat} covers both grammatical and lexical functions. It has grammaticalized into a reflexive marker, characterized by detransitivizing effects on the syntax. Related to these functions, but semantically distinct, is the employment in autobenefactive derivations and in lexical compounding. The lexicalized CPs are verbs of  grooming and social interaction: what  they all have in common is the typically intended and beneficial affectedness of the subject. In this function, \emph{-ca} does not necessarily have a detransitivizing effect.
 
 The reflexive is constructed by attaching  \emph{-ca} \rede{eat} to the lexical verb (see \Next). The resulting CP is always intransitively inflected. The A and P arguments have identical reference and thus they are expressed by a single noun phrase, which is in the nominative case and triggers agreement on the verb. See \sectref{refl} for a detailed discussion of the reflexive function of \emph{-ca}). 
 
 \exg. babu=ci n-jond-a-ca-ya-ci\\
boy{\sc =nsg} {\sc 3pl}-praise-\sc{3.P-V2.eat-pst-nsg}\\
\rede{The boys praised themselves.} 

In the following, the autobenefactive effect of \emph{-ca} will be described. Example \Next[a] shows the stem \emph{phaʔ} \rede{knit, weave, plait}, which is typically transitive, with  the result of the activity as object. However, the addition of  \emph{-ca} changes the interpretation to \rede{knit something for oneself} shown in \Next[b]. The verbal peson marking also changes to intransitive, but a P argument can still be expressed; semantically this is still a transitive verb.  

\ex. \ag.tamphwak phaʔ-uks-u-g=ha\\
		hair weave{\sc -prf-3.P[pst]-2.A=nmlz.nc} \\
	\rede{Did you plait (your) hair?} %check exactly whether inalienable poss.!
\bg. ka phurluŋ phan-ca-me-ŋ=na\\
		{\sc 1sg} little\_box weave{\sc -V2.eat-npst-1sg=nmlz.sg} \\
	\rede{I weave a \emph{phurlung} (little box out of bamboo stripes) for myself.}


The verb \emph{soʔ} \rede{look} in \Next is also transitive. It changes to intransitive inflection when  \emph{-ca} is attached, and the former P argument is now marked with a locative \Next[b]. This is not a reflexive construction, the semantics do not entail that the A argument looks at photos of herself. Rather, the V2 alters the semantics to the effect that a specific P argument is not necessary. If it is overtly expressed, it hosts a locative case marker. Omitting overt arguments is fine in both clauses, as all arguments can be dropped easily in Yakkha, but in (a) a P argument is still implied, which is not the case for (b). The typical situation here is that someone is looking at nothing in particular, but enjoying a nice view, or someone who dreams with his eyes wide open.

As already mentioned, the valency of the lexical verb is not necessarily changed in the autobenefactive. In contrast to the reflexive and the reciprocal,  intransitive verbs can serve as input to this derivation, as the verb \emph{kheʔma} \rede{go} in \NNext. The V2 here indicates an action that is intended for one's own enjoyment, i.e. going to the police post without a particular reason, but just to have a chat with the policemen.

\ex. \ag.  so-ks-u-ga=na=i?\\
look{\sc -prf-3.P[pst]-2.A=nmlz.sg=q} \\
\rede{Have you looked at it?}
	\bg. ka (phoʈo=be) son-ca-me-ŋ=na\\
	{\sc 1sg} (phoʈo{\sc =loc}) look{\sc -V2.eat-npst-1sg=nmlz.sg}	\\
\rede{I look dreamily (at the photos).}


\exg. a-ppa pulis=be khen-ca-meʔ=na\\
	{\sc 1sg.poss}-father police{\sc =loc} go{\sc -V2.eat[3sg]-npst=nmlz.sg}	\\
\rede{Father goes to the police (to have a chat).}


Another verb illustrating the use of the V2 \emph{-ca} is \emph{koncama} \rede{take a walk}, derived from the transitive verb \emph{kot} \rede{walk (around, from place to place)} in \Next[a], from a poem about a butterfly.\footnote{The stem is realized as [kos] due to assimilation to the following sibilant.}  The underived verb \emph{kot} is transitively inflected and takes the respective stations as objects, but never the goal of a movement. As  \Next[b] shows, the V2 adds the notion of \rede{consuming} and enjoying a walk.

	\ex. \ag. phuŋ  phuŋ  kos-saŋ\\
	flower flower walk{\sc -sim} \\
	\rede{walking from flower to flower} \source{04\_leg\_03.038} \\
\bg.kon-ca-se khe-i?\\
	walk-{\sc V2.eat-sup} go-{\sc 1pl[sbjv]}\\
	\rede{Shall we go for a walk?}


To wrap up, the self-benefactive use of the V2 \emph{ca} may, but does not have to result in detransitivization. The P argument can still be expressed, but it is typically less central to the event. 

The V2 \emph{-ca} is also found in verb-verb sequences with holistic, unpredictable meanings, where the V2 interacts individually with the respective verbal meanings. Occasionally \emph{-ca} refers to the literal eating in verb-verb sequences, as in \emph{sincama} \rede{hunt, i.e. kill and eat}, \emph{ŋoncama} \rede{fry and eat}, \emph{komcama} \rede{pick up and eat} and \emph{hamcama} \rede{devour, bite and eat}. This transparent usage is possibly the etymological source from which the various grammaticalized functions and metaphorical meanings have  emerged. A few  examples suggest that \emph{-ca} may also convey adversative contexts, pretty much the opposite of the autobenefactive notion. 

\exg. moŋ-ca-khuba babu\\
beat-{\sc V2.eat-S/A.nmlz} boy \\
\rede{the boy who gets beaten up (regularly)}
\bg. cuŋ=ŋa n-laŋ=ci khokt-u-co-ci=ha? \\
cold{\sc=erg} {\sc 2sg.poss}-leg{\sc =nsg} chop{\sc -3.P[pst]-V2.eat[3.P;pst]-nsg=nmlz.nsg} \\
\rede{Did your legs get stiff from the cold?} (Lit.: \rede{Did the cold chop off and eat your legs?})




The lexicalized predicates with \emph{-ca} are presented in \tabref{ca}. Their semantics are non-compositional and non-transparent. They cover bodily functions and sensations, social interactions and actions performed for one's own  benefit or enjoyment. Formally, they are not different from the  reflexive and autobenefactive examples shown above, but transitive predicates are more frequent within the lexicalized predicates. Examples of the metaphorical predicates are provided in \Next.  

\begin{table}[htp]
\begin{center}
{\small
\begin{tabular}{llll}
\lsptoprule
{\sc verb} & {\sc gloss} & {\sc lex. stem}& {\sc gloss}\\
\midrule
\emph{chemcama}&\rede{tease} &\emph{chemd}&\rede{tease}\\
\emph{hencama}&\rede{defeat} &\emph{ hes}&- \\
\emph{lemcama}&\rede{cheat, deceive} &\emph{lem}&\rede{flatter, persuade}\\
\emph{luncama}&\rede{backbite}&\emph{luʔ}&\rede{tell}\\
\emph{incama }&\rede{sell}&\emph{in}&\rede{buy}\\
&\rede{buy and eat}&\emph{in}&\rede{buy}\\
\emph{oncama }&\rede{overtake, outstrip}&\emph{ond}&\rede{block}\\
\emph{huŋcama}&\rede{bask}&\emph{huŋ}&- \\
\emph{incama}&\rede{play}&\emph{is}&\rede{rotate, revolve}\\
\emph{suncama}&\rede{itch}&\emph{sus}&\rede{get sour}\\
\emph{yuncama}&\rede{laugh, smile}&\emph{yut}&\rede{sharpen}\\
\lspbottomrule
\end{tabular}
}
\end{center}
\caption{Some lexicalizations with the V2 \emph{-ca} \rede{eat}}\label{ca}
\end{table}




\ex.\ag. haiko=ha=ci lem-u-ca-ŋ-ci-ŋ=ha\\
other{\sc =nmlz.nsg=nsg} flatter{\sc -3.P[pst]-V2.eat-1sg.A-nsg.P-1sg.A=nmlz.nsg}\\
\rede{I cheated the others.}
\bg. ka mi huŋ-u-ca-ŋ=na\\
{\sc 1sg[erg]} fire bask{\sc -3.P[pst]-V2.eat-1sg=nmlz.sg}\\
\rede{I basked in the heat of the fire.}
\bg. ka uŋ luk-ma=be ond-u-ca-ŋ=na\\
{\sc 1sg[erg]} {\sc 3sg} run{\sc -inf=loc} block{\sc -3.P[pst]-V2.eat-1sg.A=nmlz.sg}\\
\rede{I outstripped him in running.}

The most transparent use of \emph{ca} is shown in \Next: the V2 retains its lexical meaning. The same content could as well be expressed in two independent clauses. The verbs participating in the CP share all arguments, which motivates the choice of a CP instead of two clauses. 

\exg. ka makkai ŋo-c-wa-ŋ=ha\\
{\sc 1sg[erg]} corn fry{\sc -V2.eat-npst[3.P]-1sg.A=nmlz.nsg}\\
\rede{I make popcorn and eat it.}


Having looked at the whole range of functions of the V2 \emph{-ca} \rede{eat}, it is obvious that perceiving this marker merely as a syntactic valency-decreasing device is not justified. Rather, the common core of all the uses of this V2  is the volitionally and beneficially affected agent (with the exception of a few adversative usages). 

The affectedness of the agent is also a property of the literal meaning of the verb \emph{eat}. As pointed out in  \citet[37]{Naess2009_How}, it is central to the semantics of \emph{eat} and \emph{drink} verbs. Næss argues that this semantic property of the event also makes it less prototypically transitive, as the agent shares the property of affectedness with patient arguments. Thus, the A and the P are not maximally distinct in \emph{eat} and \emph{drink} verbs. Verbs of eating often exhibit properties typical of intransitive verbs crosslinguistically, which possibly also gave rise to the grammaticalization of \emph{-ca} into detransitivizing markers of reflexivity, autobenefactive, adversative notions  and reciprocality (see \sectref{refl3}) in Yakkha. 

The extreme polysemy of the V2 \emph{-ca} is not at all surprising, as the activity of eating is universal to human existence, and is thus expected to be a rich source for metaphors \citep{Newman2009_A-cross-linguistic}. The use of \rede{eat} in the expression of experiental events is a prominent pattern in Central and South Asia generally, that is thus found in other Tibeto-Burman languages, and in Indo-Aryan languages (\citealt[154]{Hooketal2009_The-semantic}, \citealt{Pramodini2010_Eat}). While the grammaticalization to passive markers is also found elsewhere \citep[122]{Heineetal2002_World}, the use of \rede{eat} as marker of reflexivity is, to my best knowledge, not yet reported for other languages, although such a development is, for the reasons laid out above, very plausible.


\subsection{The quasi-V2 \emph{-siʔ} (middle, unintentional)}\label{V2-mddl}

The marker \emph{-siʔ} is not a verb historically, but it behaves according to the V2 pattern, triggering the recursive inflection and being part of the infinitives, which is why it is mentioned here as well (see below for its historical development). The morpheme is found with detransitivizing function as a middle  marker (see \Next and \sectref{middle}),  but also in less transparent intransitive verb-verb combinations, that will be shown in this section. The grammatical and lexical functions of  \emph{-siʔ} all share the semantic feature of indicating unintentional  or involuntary  actions. All verbs derived by the middle involve animate or human subjects.

\ex. \ag.kamala=ŋa sakhi phaps-u=na  \\
		Kamala{\sc =erg} thread entangle-{\sc 3.P[pst]=nmlz.sg}\\
	\rede{Kamala entangled the thread.}
 	\bg.mendhwak=ci phaps-a-sy-a-ci\\
	goat{\sc =nsg} entangle-{\sc pst-mddl-pst-[3]du}\\
	\rede{The two goats lost their way.}
	
The component of unintentional actions is best illustrated in \Next. The simple verb \emph{tupma} \rede{meet} is  intransitive and inherently reciprocal (see \Next[a]). The middle specifies that the event happened spontaneously and unintentionally (see \Next[b]). The examples in \NNext show further middle verbs. 
	
\ex. \ag.wandik tub-i?\\
		tomorrow meet{\sc -1pl[sbjv]}\\
	\rede{Shall we meet tomorrow?}
 	\bg.tub-a-sy-a-ŋ-ci-ŋ=ha\\
		meet-{\sc pst-mddl-pst-excl-[1]du-excl=nmlz.nsg}\\
	\rede{We (dual) ran into each other.}
	
\ex. \ag.cuwa=ŋa hipt-a-sy-a-ga=na=i?\\
		beer{\sc =ins} choke-{\sc pst-mddl-pst-2=nmlz.sg=q}	\\
	\rede{Did you choke on the beer?}   
 	\bg.ka kolem=na=be sos-a-sy-a-ŋ=na\\
	{\sc 1sg} slippery={\sc nmlz.sg=loc} slide{\sc -pst-mddl-pst-1sg=nmlz.sg}  	\\
	\rede{I slipped on the slippery ground.}
	\bg. chimd-a=nuŋ chimd-a=nuŋ tops-a-sy-a-ŋ=na\\
	 ask-{\sc pst[3sg]=com} ask-{\sc pst[3sg]=com} confuse{\sc -pst-mddl-pst-1sg=nmlz.sg}\\
	\rede{As she asked and asked, I got confused.}  

Example \Next shows lexicalized predicates containing \emph{-siʔ}, all from the experiential domain. A look at the independent meanings of the lexical roots here shows that the semantics of these predicates are non-compositional, and non-transparent.

\ex. \ag.cond-a-sy-a=bi=ba\\
		praise{\sc [3sg]-pst-mddl-pst=irr=emph}\\
	\rede{She would have been happy.}  \source{13\_cvs\_02.056}
 	\bg.ŋond-a-sy-a-ga=na=i?\\
	remain-{\sc pst-mddl-pst-2=nmlz.sg=q}	\\
	\rede{Do you feel shy?}


The middle marker is also used in imperatives, with the function of turning commands into implorations. By using the middle, the speaker acknowledges the affectedness of the addressee, or he implies that it is against the addressee's will (e.g. in \Next[b]). In this function, the volition of the subject is allowed and even  required, as \Next[b] and \Next[c] show. 

	 \ex.\ag. pog-a!\\
	stand\_up-{\sc imp}\\
	\rede{Stand up!}
	 \bg.pog-a-sy-a\\
	stand\_up-{\sc imp-mddl-imp}\\
	\rede{Please, stand up.}  (to a person who was not willing to stand up)
\bg.yok-t-a-by-a-sy-a, so-t-a-by-a-sy-a\\
		search-{\sc ben-imp-V2.give-imp-mddl-imp}, look{\sc -ben-imp-V2.give-imp-mddl-imp}\\
	\rede{Please search her (the missing girl) for us, look for us.} \source{22\_nrr\_05.084--5}
 

As for the etymology of the marker, it behaves like a V2 in the Kiranti languages, but it is a reconstructed suffix already in Proto-Tibeto-Burman. A sibilant suffix with middle semantics is well attested in many Tibeto-Burman languages, e.g. in Dulong-Rawang and Padam \citep[1944]{LaPolla1996_Middle} and in languages of the West Himalayish branch \citep[471]{Matisoff2003Handbook}. The verbal behavior of the marker is a Kiranti innovation, resulting from reanalysis of a suffix to a verbal stem, under the pressure to have a structure that is analogous to the other verbal operations that are marked by V2s in Kiranti languages (see \citet[560]{Bickel2003Belhare} for the same development in Belhare).\footnote{In some Kiranti languages the marker has reflexive and reciprocal semantics, e.g. in Limbu \citep[86]{Driem1987A-grammar}, in Kulung \citep[61]{Tolsma1999A-grammar} and in Chintang \citep[300]{Bickeletal2010Ditransitives}. In Thulung and Wambule it is found as a detransitivizer or \rede{stativizer} (\citet[209]{Lahaussois2003_Thulung}, \citet[351]{Opgenort2004A-Grammar}).}  The middle marker in Yakkha does not have a stativizing effect on the temporal structure of the verb (reconstructed as a proto-function of this morpheme, \citet[471]{Matisoff2003Handbook}). Most of the verbs have ingressive-stative Aktionsart, which is why they are usually marked for past while referring to present (stative) events.

\subsection{The V2 \emph{-kheʔ} (motion away, telicity)}\label{V2-go}% \rede{go} 

The V2 \emph{kheʔ} \rede{go} is  found with intransitive telic verbs, emphasizing their orientation towards an end point, or the irreversibility of an event. Example \Next[a], for instance, was uttered to emphasize that the subject is already fast asleep, implying that it is useless to try and wake up that person. The other examples in \Next illustrate the application of this V2 to indicate irreversible intransitive events.

\ex. \ag.  ips-a-khy-a=na\\
	  fall\_asleep{\sc -pst-V2.go[3sg]-pst=nmlz.sg}\\
\rede{She has fallen asleep (better not wake her up).}
\bg.a-ma sy-a-khy-a,                    i    cok-ma=na?\\
{\sc 1sg.poss-}mother die{\sc -pst-V2.go[3sg]-pst=nmlz.sg} what do{\sc -inf=nmlz.sg}\\
\rede{My mother has died, so what to do?}\source{06\_cvs\_01.020}
\bg.wa   bhale ka-ya-khy-a\\
cock cock call{\sc -pst-V2.go[3sg]-pst}\\
\rede{The cock  crowed already.}\footnote{Context: the time is over and the protagonist loses his bet.} \source{11\_nrr\_01.011}
\bg. ulippa paŋ hor-a-khy-a=na\\
	old house crumble{\sc -pst-V2.go[3sg]-pst=nmlz.sg}\\
	\rede{The old house crumbled down.}


The development of a motion verb into a marker of telicity is common; the motion semantics get extended to a movement in time and changes-of-state in general. Examples that involve literal motion events are, however, also fairly frequent with \emph{-kheʔ}, as the verbs in \Next illustrate. In these examples, the lexical verb denotes the kind of motion, and the V2 specifies the direction away from a point of reference.

	\ex.\ag. chemha phom-khem-me=ha\\
	liquor spill{\sc [3sg]-V2.go-npst=nmlz.nc}\\
	\rede{The liquor will spill.}
	\bg.  uimalaŋ=be sos-a-khy-a=na\\
	 steep\_slope\_downwards{\sc =loc} lie\_slanted{\sc [3sg]-pst-V2.go-pst=nmlz.sg}\\
	\rede{He slipped on the steep slope (landing on his back and sliding off).}
		\bg. kakkulik kaks-a-khy-a=na\\
	rolling fall{\sc [3sg]-pst-V2.go-pst=nmlz.sg}\\
	\rede{She tumbled down (in somersaults).}\source{06\_cvs\_01.020}
	\bg. nhaŋ    pes-a-khy-a-ma\\
	and\_then fly{\sc [3sg]-pst-V2.go-pst-prf}\\
	\rede{And then he has flown away.}  \source{21\_nrr\_04.030}
\bg.limbu=ci            nhaŋ    n-las-a-khy-a-ma\\
Limbu\_group{\sc =nsg} after\_that {\sc 3pl-}return{\sc -pst-V2.go-pst-prf}\\
\rede{The Limbus went back afterwards.} \source{22\_nrr\_05.035}
	

Lexicalized complex predicates are possible as well. In \Next, the combination of \emph{khus} \rede{steal} and V2 \emph{-kheʔ} \rede{go} has acquired the meaning \emph{escape}, without an action of stealing being implied. As in the examples in \Last, the V2 specifies also the direction away from a point of reference. The same is possible with the V2 \emph{-ra} \rede{come} to indicate that someone comes fleeing from a location further away.

\exg.ŋ-khus-a-khy-a-n=na\\
{\sc neg-}escape{\sc -pst-V2.go[3sg]-pst-neg=nmlz.sg}\\
\rede{He did not escape.}

In some instances, the V2 \emph{-kheʔ} retains its original lexical meaning, and simply means \rede{go}, as in the sequence of events shown in \Next. The transitive verbs \rede{fry} and \rede{eat} have been detransitivized to synchronize the argument structure with the final verb \rede{go}. So far, this is the only instance where the participants do not bear equal relations to the verb, as the location where the frying and the eating  take place is in a source relation to the action of going. 

\exg.camraŋ=be    cama i=ya           n-ni-ca-ya-khy-a-ma=hoŋ\\
Camrang{\sc =loc} food what{\sc =nmlz.nc} {\sc 3pl-}fry{\sc -V2.eat-pst-V2.go-pst-prf=seq}\\
\rede{After they have fried and eaten some food in Camrang and gone away, ...}\source{22\_nrr\_05.034}
 
The whole range of   \emph{-kheʔ}  in CPs, thus, represents a con\-tinuum from the lexical meaning of \rede{go},  via unspecific motion away from a point of reference, to a grammaticalized and regular telic function, by metaphorically extending a movement in space to a movement in time.
	
\subsection{The V2 \emph{-ra} (motion towards)}\label{V2-comeneut} %\rede{come}}

The V2 \emph{-ra} (*\emph{ta}) \rede{come (from further away)} specifies an event in terms of  a motion towards a point of reference, while being  unspecified for the uphill/downhill distinction. The lexical source verb is \emph{tama}, but  initial /t/ becomes [r] intervocalically in all V2 stems.

\ex.\ag.arap=phaŋ khus-a-ra-ya=na\\
Arab\_countries{\sc =abl} escape{\sc [3sg]-pst-V2.come-pst=nmlz.sg}\\
\rede{He came escaping from (working in the) Arab countries.}
\bg. cuncula [...] ten=be las-a-ra-ya-ma\\
Cuncula [...] village{\sc =loc} return{\sc [3sg]-pst-V2.come-pst-prf}\\
\rede{Cuncula has returned home.} \source{01\_leg\_07.307}
\bg.dharan men-da-le hiks-a-ra-ya=na\\
Dharan {\sc neg-}reach\sc{-cvb} turn\_around{\sc [3sg]-pst-V2.come-pst=nmlz.sg}\\
\rede{Without reaching Dharan, he turned around and came back.}

An actual movement in space is not implied here either, equal to \emph{-kheʔ} discussed above. The function of \emph{-ra} can also be metaphorically extended, just as in the English translation (see \Next).

\exg. hiŋ-a-ra-ya=na\\
survive{\sc [3sg]-pst-V2.come-pst=nmlz.sg}\\
\rede{He came back to life.}


\subsection{The V2 \emph{-raʔ}  (caused motion towards)}\label{V2-bringneut}%\rede{bring}

The lexical source verb of  \emph{-raʔ} (*\emph{taʔ})   refers to bringing something from further away, and this meaning component is preserved in the CPs, too. It specifies transitive events for directedness towards a point of reference. As such, the V2 can either modify a motion verb (see \Next[a]) or express a sequence of events (see \Next[b]).

\ex.\ag.nhaŋ    ak=ka kamnibak=ci hip-paŋ tikt-u-ra-wa-ŋ-ci-ŋ\\
and\_then {\sc 1sg.poss=gen} friends{\sc =nsg} two{\sc -clf.hum} lead{\sc -3.P-V2.bring-npst-1sg.A-3nsg.P-1sg.A}\\
\rede{And I will bring along two of my friends.}\source{14\_nrr\_02.24}
\bg.eko phuŋ chikt-u-ra=na\\
one flower pluck{\sc [3sg.A;pst]-3.P-V2.bring=nmlz.sg}\\
\rede{She plucked a flower and brought it.}


It is also possible to turn a transitive verb into a motion verb by adding \emph{-raʔ} (see \Next). Since the stem \emph{momd} \rede{cover} is transitive, the V2 has to be transitive, too. 

\exg. eŋ=ga ten khibrumba=ŋa momd-u-ra=na\\
\sc{1pl.incl=gen} village fog{\sc =erg} cover{\sc [3sg.A;pst]-3.P-V2.bring=nmlz.sg}\\
\rede{The fog came (lit.: brought) covering our village.}


\subsection{The V2 \emph{-uks} (motion down towards)}\label{V2-comedown}%\rede{come down} 

If intransitive motion  is directed downwards and towards a point of reference, the V2 \emph{-uks} \rede{come down} is used to specify this path.

\exg. taŋkhyaŋ ka-ya=na=hau. ikhiŋ=na! hor-uks-heks-a=na\\
sky speak{\sc [3sg]-pst=nmlz.sg=excla} such{\sc =nmlz.sg} burst{\sc -V2.come.down-V2.cut-pst=nmlz.sg}\\
\rede{It thundered, indeed. Such a loud one! It (the sky) is about to break down on us.} \source{13\_cvs\_02.088--89}

\subsection{The V2 \emph{-ukt}  (caused motion down towards)}\label{V2-bringdown}%\rede{bring down}

The V2 \emph{-ukt} \rede{bring down} denotes caused motion down and towards the deictic center, both with monotransitive and ditransitive verbs. The resulting CP has the argument realization of the indirective frame, showing agreement with the T argument (cf. Chapter \ref{verb-val}). This verb is compatible with the adverb \emph{mo} \rede{downhill} (versus \emph{yo} and \emph{to}), although this would be a somewhat redundant expression. Note that in combination with CV or CVʔ stems that also have /u/ as stem vowel, the stems fuse into one syllable, as shown in \Next[b].

\ex.\ag.ŋ-gamnibak (mo) tikt-ukt-u\\
{\sc 2sg.poss-}friend (down) guide{\sc -V2.bring.down-3.P[imp]}\\
\rede{Bring your friend down here.}
\bg.ʈhuŋkha=bhaŋ siŋ khu-kt-u-m-ŋ=ha\\
steep\_slope{\sc =abl} wood carry{\sc -V2.bring.down-3.P[pst]-1pl.A-excl=nmlz.nc}\\
\rede{We brought down fire wood from the steep slopes.}


\subsection{The V2 \emph{-ap} (motion towards, from close nearby)} \label{V2-come}

The V2 \emph{-ap} \rede{come} denotes intransitive motion towards a point of reference, crucially from close nearby and from the same  level with respect to the inclination of the hill, e.g. from a neighbouring house which is on the same elevation level. Such predicates are compatible with the adverb \emph{yondaŋ}, which refers to sources on the same level.

\exg.yo=na paŋ=bhaŋ las-a-ab-a=na\\
across{\sc =nmlz.sg} house{\sc =abl} return-{\sc [3sg]-V2.come-pst=nmlz.sg}\\
\rede{She came back from the house across.}


\subsection{The V2 \emph{-apt} (caused motion towards, from close nearby)} \label{V2-bring}%\rede{bring (from close distances)} 

The V2 \emph{-apt} \rede{bring}  expresses caused motion towards  a point of reference, from nearby and from the same elevation level with respect to the hill, in analogy to the intransitive \emph{-ap} above. The resulting CP has the argument realization of the indirective frame (cf. \sectref{three-arg-frame}). As example \Next[b] illustrates, this V2 is also used for small-scale movements.

\ex.\ag.ŋ-gamnibak tikt-apt-u\\
{\sc 2sg.poss-}friend  guide{\sc -V2.bring-3.P[imp]}\\
\rede{Bring your friend here.}
\bg.jhola peʔleʔle und-apt-u-ga=i\\
bag {\sc ideoph} pull{\sc -V2.bring-3.P[imp]-2=emph}\\
\rede{Pull out the bag (from behind the heap of clothes, towards oneself).}


The transitivity of the two verbal stems has to match. In \Next, no literal \rede{bringing} of the substance is involved, at least not if bringing is understood as carrying something in a container outside one's own body.

\exg.nhaŋ chemha=ca uŋ-apt-a-ŋ-c-u-ŋ=ba\\
and\_then liquor\sc{=add}  drink{\sc -V2.bring-pst-excl-du-3.P-excl=emph}\\
\rede{We drank liquor and came (lit.: brought it) here.}\source{36\_cvs\_06.398}


\subsection{The V2 \emph{-ris} (caused motion to a distant goal)}\label{V2-place}%\rede{place, invest} 

The V2 \emph{-ris} (*\emph{tis}), with the lexical meaning \rede{place, invest} (e.g. place a pot on the fire, invest money in some project),  indicates caused motion towards a distant goal, implying that the object will remain there. The resulting predicate again exhibits  the argument realization of the indirective frame (cf. Chapter \ref{verb-val}). This V2 is not specified for the vertical dimension, and it is compatible with adverbial specifications for either \emph{mo} \rede{down}, \emph{to} \rede{up} or \emph{yo} \rede{across}, but naturally not with adverbials expressing proximity to the deictic center.\footnote{In both examples, two V2 are involved. The V2 \emph{-a \ti -na} \rede{leave} fuses with the inflectional material (/-u-a-u/) to result in [o].}

\ex.\ag.ŋ-gamnibak  u-paŋ=be tikt-u-ris-o\\
{\sc 2sg.poss-}friend {\sc 3sg.poss-}house{\sc =loc}  lead{\sc -3.P-V2.place-V2.leave[3.P;imp]}\\
\rede{Deliver your friend at his home.}
\bg. uŋci-ten=be ikt-u-ris-o\\
{\sc 3nsg.poss-}village{\sc =loc} chase{\sc -3.P-V2.place-V2.leave[3.P;imp]}\\
\rede{Chase them to their village.}
 

\subsection{The V2 \emph{-bhes} (caused horizontal motion towards)} \label{V2-bhes}%\rede{bring horizontally}

The V2 \emph{-bhes}  has the lexical meaning \rede{send [towards]}, \rede{bring [towards]}. It signifies that  caused motion takes place on the same elevation  level and towards the point of reference, for either  small-scale or large-scale movements.  Example \Next[a] shows  transfer from a very short distance, the application of the blessing on the forehead (by sticking cooked rice on the forehead). Example \Next[b] shows the employment of this V2 for a large-scale movement. This V2 is only compatible with adverbs derived from the root \emph{yo} \rede{across}. 


\ex.\ag.ʈika ʈal-a n-jog-u-bhes-u=hoŋ\\
blessing\_on\_forehead stick{\sc -nativ} {\sc 3pl.A-}make{\sc -3.P-V2.bring-3.P[pst]=seq}\\
\rede{After they applied the blessings, ...} (as remembered by a bride who got blessed) \source{25\_tra\_01.049}
\bg. nhaŋ,    nna,  lalubaŋ=nuŋ phalubaŋ ŋ-ikt-a-bhes-uks-u-ci\\
and\_then that Lalubang{\sc =com}  Phalubang {\sc 3pl.A-}chase{\sc -pst-V2.bring-V2.prf-3.P-3nsg.P}\\
\rede{And then, they chased Lalubang and Phalubang here.} \source{22\_nrr\_05.023} 


\subsection{The V2 \emph{-end}  (caused motion downwards)} \label{V2-insert}%\rede{apply, insert}

The V2 \emph{-end} (\ti \emph{-nen} before consonants)\footnote{Cf. the citation forms \emph{leʔnemma}, \emph{lepnemma} and \emph{huʔnemma} for the examples in \ref{leendu}.} has the lexical meaning \rede{apply, insert}. As a V2, it indicates caused motion downwards, as shown by the examples in \Next. Here, the motion is not specified for the direction towards or away from a point of reference.
 
\ex.\label{leendu}\ag. u-laŋ=ci leʔ-end-u-ci=ha\\
 {\sc 3sg.poss-}leg{\sc =nsg} drop{\sc -V2.insert-3.P[pst]-3nsg.P=nmlz.nsg}\\
\rede{It (the plane) lowered its  landing gear.}
\bg. nhaŋ    phoʔ   n-lept-end-wa\\
	and\_then {\sc ideoph}	{\sc 3pl.A-}throw{\sc -V2.insert-npst[3.P]}\\
	\rede{They throw it down swiftly (the fishing net).} \source{13\_cvs\_02.009}
	\bg. hut-end-u-ŋ=na\\
	push{\sc -V2.insert-3.P[pst]-1sg.A=nmlz.sg}\\
	\rede{I pushed him down.}


The V2 \emph{-end} can also express caused motion downwards as a result of another action (see \Next). Furthermore, using this V2 may also convey regret. Saying \emph{pegenduŋna} (\rede{I shattered it}) sounds more regrettingly than the underived  \emph{peguŋna}.

\ex.\ag. tabek=ŋa siŋ=ga u-whak cen-end-u=na\\
khukuri\_knife{\sc =ins} tree{\sc =gen} \sc{3sg.poss-}branch chop{\sc -V2.insert-3.P[pst]=nmlz.sg}\\
\rede{He cut down the branch with a Khukuri knife.}  (Cut-and-break clips, \citealt{Bohnemeyeretal2010_cut})
\bg. u-sa seps-end-u\\
{\sc 3sg.poss-}fruit pluck{\sc -V2.insert-3.P[imp]}\\
\rede{Get the fruits down (plucking).}


\subsection{The V2 \emph{-ket} (caused motion up and towards)}\label{V2-bringup}%\rede{bring up} 

The V2 \emph{-ket}  \rede{bring up}, signifies caused motion up and towards a reference point (see \Next). There is a corresponding intransitive stem \emph{-keʔ} \rede{come up}, but it has not been found yet as V2.  

\ex.\ag.na   eko=ŋa=go  thend-u-get-uks-u=ba\\
this one{\sc =erg=top} lift{\sc -3.P-V2.bring.up-prf-3.P[pst]=emph}\\
\rede{Someone has lifted it up (carried it up in one's hands).} \source{37\_nrr\_07.082}
\bg. thithi end-u-get-uks-u=na\\
upright insert{\sc -3.P-V2.bring.up-3.P[pst]-prf=nmlz.sg}\\
\rede{He inserted it upright at an elevated place.}\source{37\_nrr\_07.083}


The movement upwards may also happen on a very small scale. In \Next, for instance, the speaker does not refer to someone plucking further downhill, but just one meter below herself.\footnote{In Kiranti languages, the topography-based specification of events reaches a level of much greater distinction than speakers of European languages  are generally used to (cf. Chapter \ref{ch-geomorphic}, and  \citealt{Bickel1999Cultural, Bickel2001Deictic, Ebert1999The-up---down, Gaenszle1999Travelling}).}

\exg. maŋkhu seps-u-get-u=ha\\
garlic pluck{\sc -3.P-V2.bring.up-3.P[pst]=nmlz.nsg}\\
\rede{She plucked and brought up the garlic.}

Furthermore, at least one example suggests that \rede{bring up} can also be understood metaphorically, as referring to a movement in time, where the past is equated with lower altitude.\footnote{The postposition \emph{nhaŋto} \rede{since} in this example literally means \rede{and then up}, providing  support to this hypothesis. Further support for the hypothesis that the past is conceptualized as \rede{below} comes from an idiomatic Noun-Verb Predicate, \emph{setni keʔma}, literally \rede{bring up the night}, which refers to staying awake until the morning.}

\exg. nna  namda ka       piccha nhaŋto nis-u-get-u-ŋ=na.\\
that festival \sc{1sg[erg]} child since see\sc{-3.P-V2.bring.up-3.P[pst]-1sg.A=nmlz.sg}\\
\rede{I have been attending this festival since childhood.}  \source{41\_leg\_09.006}


\subsection{The V2 \emph{-haks} (caused motion away, irreversibility)}\label{V2-send}%\rede{send} 

The V2 \emph{-haks \ti -nhaŋ} \rede{send (things)} expresses caused motion away from a point of reference (see \Next),  and away from the agent (in contrast to \emph{-khet} \rede{carry off} described below).  Although its lexical meaning is \rede{send (things)}, as a V2 it is also used with animate T arguments, including human referents.


\ex.\ag.ka mima o-hoŋ=be hut-haks-u-ŋ=na\\
{\sc 1g[erg]} mouse  {\sc 3sg.poss-}hole{\sc =loc} push{\sc -V2.send-3.P[pst]-1sg.A=nmlz.sg}\\
\rede{I pushed the mouse (back) into her hole.}
\bg.ʈebul=be cuwa tug-haks-u=ha\\
table{\sc =loc} beer wipe{\sc -V2.send-3.P[pst]=nmlz.nsg}\\
\rede{She wiped the beer from the table.}\footnote{Despite the ablative semantics of the verbs, the locative is the standard case choice with this verb.}
\bg. luŋkhwak luŋkhwak seg-haks-u\\
stone stone chose{\sc -V2.send-3.P[imp]}\\
\rede{Sort out stone by stone (from the grains).}
\bg.wasik ta-ya=hoŋ hoŋma uks-a, ŋkhoŋ yokhaʔla chekt-haks-a=na\\
rain  come\sc{[3sg]-pst=seq} river come\_down\sc{[3sg]-pst} and\_then across block{\sc -V2.send-pst[3sg]=nmlz.sg}\\
\rede{After the rain, a river came down, and we redirected it.}\footnote{The verb \emph{chekthaksana} is detransitivized and passive-like structurally, but this structure can also express first person nonsingular agents (see \sectref{detr-pass}).}


Like the V2s \emph{-kheʔ} \rede{go} and \emph{-khet} \rede{carry off}, \emph{-haks}  also conveys telicity and actions with irreversible consequences, as illustrated by the examples in \Next. This V2 was  particularly prominent in the data elicited with the cut-and-break clips   \citep{Bohnemeyeretal2010_cut}.

\ex.\ag.solop miyaŋ eg-haks-u-su\\
immediately a\_little break{\sc -V2.send-3.P[pst]-pst.prf}\\
\rede{Immediately he broke off a little.} \source{04\_leg\_03.079}
\bg. a-yaŋ cum-haks-u-ŋ=ha\\
{\sc 1sg.poss-}money hide{\sc -V2.send-3.P[pst]-1sg.A=nmlz.c}\\
\rede{I mislaid my money.}
\bg. hu-haks-u=na\\
accuse{\sc -V2.send-3.P[pst]=nmlz.sg}\\
\rede{He accused her.}\footnote{This is probably a metaphorical use of \emph{huʔ} \rede{burn}.}
\bg. a-phu=ŋa cekt-haks-u=ha\\
{\sc 1sg.poss-}eB\sc{=erg} talk{\sc -V2.send-3.P[pst]=nmlz.nc}\\
\rede{My elder brother did/finalized the talking.}\footnote{Context: wedding negotiations.} \source{36\_cvs\_06.363}
\bg.a-niŋwa=be=ha ceʔya lu-haks-u=ha\\
{\sc 1sg.poss-}mind{\sc =loc=nmlz.nc} matter tell{\sc -V2.send-3.P[pst]=nmlz.nc}\\
\rede{She blabbered out my secret thoughts.}


The V2 \emph{-haks} may also attach to the lexical verb \emph{haks}, an option that has not been found with other V2s so far. It implies that something was sent via an intermediate station, e.g. another house where the adressee has to go and get his things, or via a post office. 

\exg. salen haks-haks-u=na\\
message send{\sc -V2.send-3.P=nmlz.sg}\\
\rede{He sent the message (via some institution).}


Furthermore, this V2 is frequently used when the P or T argument of a verb is human (discussed in detail in \sectref{t-sap}). This is surprising because the lexical verb \emph{haks} \rede{send} implies inanimate T arguments. There is a strong tendency for referentially high objects (mostly P and T; as G arguments are expected to be referentially high anyway) to occur in a complex predicate construction, and one V2 choice for this is obviously \emph{-haks}.

\ex.\ag. i=ca n-lu-n-ci-n=ha, so-haks-u-ci se=ppa\\
what{\sc =add} {\sc neg-}tell{\sc -neg-3nsg.P-neg=nmlz.nsg}  look{\sc -V2.send-3.P[pst]-3nsg} {\sc restr=emph}\\
\rede{He did not say anything to them, he just glanced at them.} \source{34\_pea\_04.044}
\bg.kaniŋ na=haŋ    iŋ-nhaŋ-ma=na\\
{\sc 1pl[erg]} this{\sc =abl} chase{\sc -V2.send-inf[deont]=nmlz.sg}\\
\rede{We have to chase him away from here (this place).} \source{21\_nrr\_04.010}
\bg. kaniŋ lon-nhaŋ-ma sin\\
{\sc 1pl} take\_out{\sc -V2.send-inf[deont]} {\sc cop.1pl.incl}\\
\rede{He has to expel us.}		

\subsection{The V2 \emph{-khet} (caused motion along with A)}\label{V2-carryoff}%\rede{carry off}

The V2 \emph{-khet \ti -kheʔ \ti -(h)et} indicates that the  object is carried off or is in some way separated from its original location, remaining with the A argument. The lexical source verb is \emph{khet} \rede{carry off}. In the infinitives, the form is always \emph{-kheʔ}, while in the inflected forms, the V2 surfaces as \emph{-het} or even \emph{-et}.  These predicates either express a manner of caused motion away from a reference poin, as in \Next[a] and \Next[b], or a sequence of doing something and literally carrying off the object, as in  \Next[c] and \Next[d].

%*** check: is there het and et? -> no, all the same.

\exg.ghak yaŋ-het-i=nuŋ=ga  wasik\\
all flush{\sc -V2.carry.off-compl=com=gen} rain\\
\rede{a rain that flushed away everything} \source{38\_nrr\_07.076}
\bg. jaŋgal=be sa-het-u=hoŋ\\
forest{\sc =loc} lead\_by\_rope{\sc -V2.carry.off-3.P[pst]=seq}\\
\rede{He led it (the goat) into the jungle, ...} \source{20\_pea\_02.026}
\bg. yubak coŋ-kheʔ-ma\\
goods shift\sc{-V2.carry.off-inf[deont]}\\
\rede{The goods have to be unloaded and carried off.}
\bg. khus-het-uks-u=ha hola\\
steal{\sc -V2.carry.off-prf-3.P[pst]=nmlz.nsg} probably\\
\rede{He probably stole it and carried it off.} \source{20\_pea\_02.014}

Like \emph{-raʔ} \rede{bring} described above, \emph{-khet} \rede{carry off} can also be used  metaphorically, carrying various interpretations. In  \Next[a], it is employed to satisfy the requirement of matching valency within a CP. The verb \emph{-kheʔ} \rede{go} would be impossible here  because it is intransitive, while \emph{uŋ} \rede{drink} is transitive. The same holds for transimpersonal verbs (see \Next[b] and (c)).\footnote{Transimpersonal verbs always show  transitive person marking, but an overt A argument cannot be expressed.} With such verbs the V2 has to be transitive. In these examples, \emph{-khet} has a  telicizing effect, similar to the effect of \emph{-kheʔ} \rede{go} in intransitives. The lexical stems \emph{lokt} and \emph{hand}  have ingressive semantics, i.e. when they are inflected for past, they refer to ongoing events. After \emph{-khet} has been added, they are oriented towards the end point, as these examples show.

\ex.\ag.khem uŋ-het-u-ŋ=na\\
before drink\sc{-V2.carry.off-3.P[pst]-1sg.A=nmlz.sg}\\
\rede{I had drunken it before and left.}
\bg. maŋcwa lokt-het-u=ha\\
water boil\sc{-V2.carry.off-3.P[pst]=nmlz.sg}\\
\rede{The water boiled down.}
\bg. micuʔwa hand-het-u=na\\
bamboo\_torch burn\sc{-V2.carry.off-3.P[pst]=nmlz.sg}\\
\rede{The bamboo torch burned down.}

Another interpretation of \emph{-khet} was found e.g. with cognition verbs such as \emph{miʔma} \rede{think, hope, want, remember}, but also with other verbs. In \Next the V2 functions as a marker of degree, intensity or excessiveness.

\ex.\ag.  nna  hoŋ=be    iha=le   weʔ=na        bhoŋ soʔ-ma    mit-het-u-ŋ\\
that hole{\sc =loc} what{\sc =ctr} exist{\sc [3sg]=nmlz.sg} {\sc comp} look{\sc -inf} think{\sc -V2.carry.off-3.P[pst]-1sg.A}\\
\rede{I badly wanted to see what was inside that hole.} \source{42\_leg\_10.024}
\bg.maŋpha tas-het-u=ha\\
very\_much arrive{\sc -V2.carry.off-3.P[pst]=nmlz.nc}\\
\rede{It became very/too much.}


\subsection{The quasi-V2 \emph{-a \ti -na} (do X and leave object)}\label{V2-leave}%\rede{leave sth. and come back}

The marker \emph{-a \ti -na} expresses that the action was carried out at a location not identical to the point of reference and that the subject has returned, leaving the object there, like for instance in \emph{pheʔnama} \rede{drop someone at X} and \emph{eʔnama} \rede{enroll someone} (e.g. in a boarding school). There is no corresponding independent simple verb \emph{ama} or \emph{nama}, but there is the complex verb \emph{naʔnama} with the meaning \rede{to leave}, which looks as if the first verb and the V2 are identical. 

This marker, due to its limited phonological content, undergoes several morphophonological operations, like ablaut and the insertion of consonants, so that it is not always easy to distinguish \emph{-a} from other morphological material in the verbal inflection. When the first suffix following the stem contains a consonant, the V2 surfaces as \emph{-na}, e.g. in \emph{naʔnanenna} \rede{I left you}. If the stems are followed by the suffix \emph{-u}, the sequence /-u-a-u-/ will be realized as [-u(ʔ)o], [-o(ʔ)o] or simply [o] (see \Next). If there is the underlying sequence /-a-a/, it will either be realized as [aya] or as [aʔa] (see \NNext).  Furthermore, the ablaut (/a/ to [o]) triggers a change of  [-uŋha]  in the suffix string to [-oŋha]. 
 
 \ex.\a.\glll tisuona\\
 /tis-u-a-u=na/\\
 place{\sc -pst-V2.leave-1sg.P-2.A=nmlz.sg}\\
 \rede{You delivered him (and returned).}
 \b.\glll nyubak kamalabe haktoʔoksoŋha\\
 /n-yubak kamala=be/ /hakt-u-a-uks-u-ŋ=ha/\\
 {\sc 2sg.poss-}goods Kamala{\sc =loc} send{\sc -3.P-V2.leave-prf-3.P[pst]-1sg.A=nmlz.nsg}\\
 \rede{I have sent your goods to Kamala (so you can get them there).}
 \b.\glll umaŋachen (...) lambu lambu yaksaŋnuŋ seula               eksaŋ         yuksoʔokso\\
 /u-ma=ŋa=chen (...) lambu lambu yaksaŋ=nuŋ/ /seula               ek-saŋ         yuks-u-a-uks-u/\\
{\sc 3sg.poss-}mother{\sc =erg=top} (...) road road  grass{\sc =com} green\_stalk break{\sc -sim} put{\sc -3.P-V2.leave-prf-3.P[pst]}\\
\rede{His mother (...) broke off some gras and stalks along the road and left them (to help the son orient himself back home).}   \source{01\_leg\_07.072}
 
 
 \ex.\a.\glll tisayaŋgana\\
 /tis-a-a-ŋ-ga=na/\\
 place{\sc -pst-V2.leave-1sg.P-2.A=nmlz.sg}\\
 \rede{You delivered me (and returned).}
 \b.\glll pasupatinathpe phesaʔaŋna\\
 /pasupatinath=pe phes-a-a-ŋ=na/\\
Pashupatinath{\sc =loc} bring{\sc -pst-V2.leave-pst-1sg.P=nmlz.sg}\\
 \rede{He brought me to Pashupatinath (and returned without me).}



\subsection{The V2 \emph{-nes}  (continuative)}\label{V2-lay}%\rede{lay}

The V2 \emph{-nes} \rede{lay} marks continuative events, i.e., events that are ongoing for longer than expected, and which are not oriented towards an end point. It is found with both transitive and intransitive verbs (see \Next.) Examples (a) to (c) show the combination of \emph{-nes} with activity verbs,  and example \Next[d] shows that in ingressive-stative verbs, the contintuative applies to the resulting state.
 
 \ex.\ag.wasik n-da-me-n=niŋa nam phen-a=na phen-a=na, phen-a-nes-a=na\\
 rain  {\sc neg-}come{\sc -npst-neg=ctmp} sun shine{\sc [3sg]-pst=nmlz.sg} shine{\sc [3sg]-pst=nmlz.sg} shine{\sc -pst-V2.lay-[3sg]-pst=nmlz.sg}\\
 \rede{While there is no rain, the sun was shining and shining, it kept shining.}\source{38\_nrr\_07.075 }
 \bg. leʔnamcuk kei m-mokt-u-nes-uks-u=ha\\
 whole\_day drum {\sc 3pl.A-}beat{\sc -3.P-V2.lay-prf-3.P[pst]=nmlz.nsg}\\
 \rede{They have kept playing the drums the whole day long.}
\bg. whaŋma=ŋa lupt-u-nes-u=na\\
sweat\sc{=erg} disperse{\sc -3.P-V2.lay-3.P[pst]=nmlz.sg}\\
\rede{She kept sweating (e.g. after a long run).}
 \bg. ka=ca hiŋ-a-nes-a-ŋ=na\\
 {\sc 1sg=add} survive{\sc -pst-V2.lay-pst-1sg=nmlz.sg}\\
 \rede{I have survived, too.}


\subsection{The V2 \emph{-nuŋ}  (continuative)}\label{V2-nung}%\rede{sit}
 
The V2 \emph{-nuŋ}  adds a continuative reading, similar to the function of \emph{-nes} described above. I tentatively suggest the verb \emph{yuŋ} \rede{sit} as the etymological source of this V2. Firstly, the grammaticalization of \rede{sit} into a continuative marker would be a very common development historically, and secondly, I have shown that the insertion of a nasal occurs in vowel-initial and /h/-initial V2s, so that replacing /y/ with [n] seems plausible, too. So far, all examples found with this V2 were intransitive or detransitivized (see \Next).

Punctual events, like in \Next[a], get an iterative reading when \emph{-nuŋ} is added. States and activities can also be extended by means of \emph{-nuŋ} (see \Next[b], \Next[c]). In several instances, the two V2s \emph{-nuŋ} and \emph{-nes} seem to be interchangeable without any change in meaning. However, while \emph{-nes} is more frequently combined with past tense, \emph{-nuŋ} is typically found in  nonpast contexts. The exact difference between \emph{-nuŋ} and \emph{-nes} cannot be established with certainty yet.
% (*how to differentiate prf 'has fallen down' and pst.continuative 'kept falling down' - maybe 2 diff. V2s? -> has fallen down is intr. and hence shows -ma in prfǃ)

\ex.\ag.a-laŋ=ci ŋ-aŋ-khe-nuŋ-me=ha\\
{\sc 1sg.poss-}leg{\sc =nsg} {\sc 3pl-}descend{\sc -V2.go-V2.sit-npst=nmlz.nsg}\\
\rede{My legs keep falling down (from the seat).}
\bg. heʔniŋ=ca ŋonsi-nuŋ-meʔ=na\\
when{\sc =add} feel\_shy{\sc -V2.sit[3sg]-npst=nmlz.sg}\\
\rede{She is always shy.}
\bg. tek leŋ-nuŋ-meʔ=na\\
clothes exchange{\sc -V2.sit[3sg]-npst=nmlz.sg}\\
\rede{She keeps changing her clothes.}



\subsection{The V2 \emph{-bhoks}  (punctual, sudden events)}\label{V2-split}%\rede{split}

The function of the V2  \emph{-bhoks}  has developed from the lexical meaning \rede{split}. Adding this V2 to a lexical verb results in a punctual reading, or in the implication that an event happens suddenly and unexpectedly (see \Next). 

\exg.a-nabhak yokt-u-bhoks-u-ŋ=na\\
{\sc 1sg.poss-}ear prick{\sc -3.P-V2.split-3.P[pst]-1sg.A=nmlz.sg}\\
\rede{Suddenly I pierced through my ear (after trying some time and then applying too much pressure).}


With telic verbs, the event is distilled to an end point (see \Next), while with activities and ingressive-stative verbs, like in \NNext, the initial point of an event is emphasized by \emph{-bhoks}.

	\ex.\ag. luŋkhwak thend-u-bhoks-u-ŋ=na\\
		stone lift{\sc -3.P-V2.split-3.P[pst]-1sg.A=nmlz.sg}\\
	\rede{I lifted the stone (with great difficulties, at once).} 
	\bg. mi mi=na et-u-ŋ=na, khatniŋgo ma leks-a-bhoks-a=na\\
	fire small{\sc =nmlz.sg} perceive{\sc -3.P[pst]-1sg.A=nmlz.sg} but big become{\sc [3sg]-pst-v2.split-pst=nmlz.sg}	\\
	\rede{It seemed to me that the fire was small, but suddenly it flamed up.}  
	
	
\ex. \ag.cumabya=ha ceʔya haku khom-bhoŋ-ma\\
hidden{\sc =nmlz.sg} language now dig{\sc -V2.split-inf[deont]}\\
\rede{We have to start digging out the (our) hidden language now.}  %\source{song: uthuk mundimna}
\bg. okt-a-bhoks-a-ma-ŋ=ba, hab-a-bhoks-a-ma-ŋ=ba\\
	shriek{\sc -pst-V2.split-pst-prf-1sg=emph} cry{\sc -pst-V2.split-pst-prf-1sg=emph}\\
	\rede{Suddenly I shrieked, I broke out in tears.} \source{13\_cvs\_02.034}
	 
%	talk{\sc -inf=ctmp=restr} remember{\sc -3.P-V2.split-3.P[pst]-1sg.A=nmlz.sg}\\
%	\rede{Only while talking, it suddenly came to my mind.}  

\subsection{The V2 \emph{-heks}  (immediate prospective,  do separately)}\label{V2-cut}%\rede{cut, saw}

The V2  \emph{-heks}  is used when the event denoted by the main verb is about to begin, as shown in  \Next. Its literal meaning is \rede{cut, saw}.  Note, again, that because of the inceptive semantics of many verbs, it is usually the past form that is used. The V2 may attach to verbs of any temporal structure, and restrictions on the semantics of the arguments (e.g. animacy or volition) were not encountered. With activities and states, the V2 conveys that the activity or state is about to start. With telic verbs, the V2 conveys that the end point is approaching. Example \NNext[b] shows a combination of a completive notion and the \rede{immediate prospective} notion.

\ex. \ag.  o-theklup leks-heks-a=na\\
	{\sc 3sg.poss-}half become{\sc -V2.cut[3sg]-pst=nmlz.sg}\\
	\rede{Almost half (of the book) is finished.}
	\bg. sabun mend-heks-a=na\\
	soap finish{\sc -V2.cut[3sg]-pst=nmlz.sg}\\
	\rede{The soap is about to be finished.}
	\bg. ucun=na lambu(=be) tas-heks-u-m=na\\
	nice{\sc =nmlz.sg} way(=loc)  arrive{\sc -V2.cut-3.P[pst]-1sg.A=nmlz.sg}\\
	 \rede{We are about to get to the nice road.}

	
\ex.\ag.la ʈoŋnuŋ leks-heks-a=na\\
	moon full become{\sc -V2.cut[3sg]-pst=nmlz.sg}\\
	\rede{The moon is about to be full.}
	\bg. hops-i-heks-u-ŋ=ha\\
	sip{\sc -compl-V2.cut-3.P[pst]-1sg.A=nmlz.nc}\\
	\rede{I am about to finish (the soup).}
	
	 

This V2 has a second meaning, translatable as \rede{do separately}. The corresponding construction in Nepali is \emph{[{\sc V.stem}]-dai garnu}. This usage of \emph{-heks} is often found in commands, for instance when the speaker encourages the hearer to start or go on with some activity while the speaker leaves the speech situation (see \Next). 


\ex.\ag.yuŋ-heks-a\\
sit\sc{-V2.cut-imp}\\
\rede{Sit down (while I leave for a moment).}
\bg. co-heks-u\\
eat\sc{-V2.cut-3.P[imp]}\\
\rede{Keep eating (without me).}
\bg. thukpa hops-heks-wa-ŋ=ha\\
soup sip\sc{-V2.cut-npst[3.P]-1sg.A=nmlz.nsg}\\
\rede{I am sipping soup (noone else does).}


	
\subsection{The V2 \emph{-ghond} (spatially distributed events)}\label{V2-roam}%\rede{roam} 


The V2 \emph{-ghond} has the literal meaning of \rede{roam, wander around}. This marker refers to actions and events that happen distributed over various locations, in the same manner as has been analyzed for the cognate Belhare marker \emph{-kon \ti -gon} \citep[163]{Bickel1996Aspect}. This V2 may attach to intransitive and transitive stems, and can be inflected either way, too (see \Next).


\ex.\ag.heʔne  maŋdu maŋdu kha luplum=ci=be           wa-ya-ghond-a             i-ya-ghond-a=niŋ=ca, ...\\
somewhere far far those den{\sc =nsg=loc} exist{\sc -pst-V2.roam-pst} revolve{\sc -pst-V2.roam[3sg]-pst=ctmp=add}\\
\rede{While he also used to live and walk around somewhere far, far away, in those caves, ...} \source{18\_nrr\_03.013}
\bg.na maghyam heʔniŋ=ca sis-u-ghond-wa=na\\
	this old\_woman when{\sc =add} kill{\sc -3.P-V2.roam-npst=nmlz.sg}		\\
	\rede{This old woman always walks around drunken.} (lit. she walks around being killed) 
\bg.	ijaŋ yoniŋ-kheniŋ     n-jiŋ-ghom-me=ha?\\
why thither-hither {\sc 3pl-}learn{\sc -V2.roam-npst=nmlz.nsg}\\
\rede{Why do they walk around learning (languages)?}


Note that in \Last[b], the experiencer is treated like s standard P argument  by case and the verbal person marking (indexed by the \rede{3.P} suffix). This does not prevent the experiencer argument from taking part in complex predication, which usually synchronizes the argument structure of the single components of one CP. This again shows the importance of generalized semantic roles as parameter along which the syntax of Yakkha is organized. 

Example \Next from a conversation clearly shows that the first verb is the semantic head, and that \emph{-ghond} has lost its lexical meaning \rede{walk around}. It  merely adds the notion of spatial distribution. In the answer \Next[b], the speaker refers to the event in question without using the V2. 

\ex.\ag.ŋkha i=ya                het-u-ghond-wa-ga?\\
that what{\sc =nmlz.nsg} cut{\sc -3.P-V2.roam-npst-2}\\
\rede{What are you cutting (at various places)?} (said reproachfully) \source{28\_cvs\_04.321}
\bg.are    haʔlo, ijaŋ me-heʔ-ma?    abbuiǃ\\
hey {\sc excla} why {\sc neg-}cut{\sc -inf[deont]} {\sc excla}\\
\rede{Goodness, why not to cut? Holy crackers!}  \source{28\_cvs\_04.323}

	

\subsection{The V2 \emph{-siʔ} (avoid, prevent)}\label{V2-avoid}
	
The V2 \emph{-siʔ} is always inflected transitively. It is probably etymologically connected to \emph{sis}  \rede{kill}.\footnote{In the Nepali translations, the  predicates were paraphrased using \emph{mārnu} \rede{kill}.} In a CP,  \emph{-siʔ} means \rede{avoid, prevent}. The lexical verb denotes an action that prevents something else from happening, like \rede{catch} in \Next[a] and \Next[b], and \rede{scold} in \Next[c].  The event which shall be avoided is not necessarily expressed overtly; it is usually obvious from the utterance context.
	
\ex.\ag.picha kaŋ-kheʔ-ma n-dokt-u-n=na, u-ma=ŋa lab-i-si=na\\
child fall{\sc -V2.go-inf} {\sc neg-}get{\sc -3.P[pst]-neg=nmlz.sg},  {\sc 3sg.poss-}mother{\sc =erg} catch{\sc -compl-V2.prevent[3.P;pst]=nmlz.sg}\\
\rede{The child could not fall down because its mother held it.}
\bg.lukt-heks-a=na, lam-siʔ-ma=na\\
run{\sc -V2.cut-pst[3sg]=nmlz.sg} catch\sc{ -V2.prevent-inf[deont]=nmlz.sg}\\
\rede{She is about to run away, we have to hold her.}
\bg. mokt-heks-uksa=na, nhaŋ thind-i-si-ŋ=na\\
beat{\sc -V2.cut-pst.prf[3.P]=nmlz.sg} and\_then scold{\sc -compl-V2.prevent[3.P;pst]-1sg.A=nmlz.sg}\\
\rede{He was about to beat him, so I scolded and stopped him.}



\subsection{The V2 \emph{-soʔ}  (experiential)}\label{V2-look}%\rede{look}

The V2 \emph{-soʔ} means \rede{look}, and it is used as experiential marker, translatable with \rede{try X and find out oneself} (see \Next). Note that this is not a complementation strategy,  as one cannot express clauses like \rede{I found out that X did Y} or \rede{I tried to X} by means of  this V2. Yakkha utilizes complement taking predicates to convey such meanings. The V2 is also not a means to express a conative, since it neither  reduces the valency nor implies that the attempt fails (see \citet{Vincent2013_Conative} for an overview of the different usages of the term \rede{conative}). The crucial meaning component of \emph{-soʔ} is \rede{experiencing something by trying out oneself}. The grammaticalization of  perception verbs to such a marker is common in South Asian and South East Asian languages. 

\ex.\ag.liŋmi=ŋa chapt-u-so!\\
straw{\sc =ins} thatch{\sc -3.P-V2.look[3.P;imp]}\\
\rede{Try and thatch (the roof) with straw!} (said as advice against tin roofs)
\bg. kheps-u-so!\\
listen{\sc-3.P-V2.look[3.P;imp]}\\
\rede{Listen and find out!}	
\bg. chimd-u-ŋ-so-ŋ?\\
ask{\sc -3.P-1sg-V2.look-1sg.A[3.P;sbjv]}\\
\rede{May I ask and find out?}

This V2 behaves exceptional with regard to the material that can stand between the verbal stems. Usually the first verbal stem can be inflected only by one suffix, and only if the suffix consists of a vowel. However, as shown in \Last[c], the inflection on the first stem can include a nasal, if a nasal is available in the inflection.



