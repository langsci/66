\chapter{Pronouns, demonstratives, quantifiers, numerals, interrogatives}\label{ch-pron}
  
This chapter describes the elements that can be found in the noun phrase,  mo\-di\-fying or replacing a head noun. It is structured as follows: §\ref{pers-pron} deals with the personal pronouns, §\ref{poss-pron} discusses the possessive pronouns, and  §\ref{dem-pron} the demonstratives. Section \ref{sec-indef} shows how indefinite reference is expressed, §\ref{sec-quant} deals with numerals and other quantifying elements. Section \ref{interr} then focusses on interrogative forms, including non-nominal interrogatives. 
  
\section{Personal pronouns}\label{pers-pron}

Yakkha personal pronouns are used to refer to persons, typically to participants whose reference has already been established in discourse. They can take the structural position of a noun phrase (of any participant role) or they can function as  heads of noun phrases, although the possibilities to be modified are restricted; relative clauses and demonstratives are not possible, for instance. Possible modifiers are quantifiers and numerals, but they follow the pronominal head, in contrast to noun phrases with nominal heads, which are mostly head-final. Pronouns, like noun phrases in general, are not obligatory, and they are frequently dropped in Yakkha.

The pronouns distinguish person and number. Clusivity, which is found  in possessive pronouns, possessive prefixes and in the verbal inflection,  does not play a role in the personal pronouns (compare \Next[a] and \Next[b]). An overview of the personal pronouns is provided together with the possessive pronouns in Table \ref{poss} below. The first and second person pronouns distinguish singular, dual and plural number. The morpheme \emph{-ci} conveys a dual meaning in the first and second person pronouns, as opposed to \emph{-ni} for plural. In the third person, \emph{-ci} simply has a nonsingular meaning.\footnote{Note that in contrast to the pronominal paradigm, the verbal inflection distinguishes dual number also in the third person (cf.  §\ref{verb-infl}).} 



\ex. \ag. 	kaniŋ khe-i-ŋ=a\\
			{\sc 1pl} go{\sc [pst]-1pl-excl=nmlz.nsg}\\
			\rede{We (without you) went.} 
	\bg.	kaniŋ khe-i=ha\\
			{\sc 1pl} go{\sc [pst]-1pl=nmlz.nsg}\\
			\rede{We (all) went.} 


\section{Possessive pronouns and nominal possessive inflection}\label{poss-pron}

\subsection{Possessive pronouns}

The possessive pronouns modify a head noun, indicating the possessor of the thing that is referred to by the noun (see \Next[a]). Since the head noun can be omitted when its reference has been established already, the possessive pronoun can also be the sole element in a phrase (see \Next[b]). 

The possessive pronouns  resemble the personal pronouns slightly, but they are sufficiently different and irregular so that they establish a separate paradigm. Except for the third person nonsingular form, the roots all look slighty different from the corresponding personal pronouns.  They all host the genitive enclitic \emph{=ga}, though. The possessive pronouns distinguish number and person, including clusivity, a category that is  absent from the personal pronoun paradigm.  The inclusive forms have no parallel in the personal pronouns. Table \ref{poss} provides an overview of personal and possessive pronouns and possessive prefixes. The capital /N/ stands for an unspecified nasal that assimilates to the following consonant  in place of articulation.

\ex. \ag.ak=ka kucuma sy-a-ma=na\\
{\sc 1sg.poss=gen} dog die{\sc [3sg]-pst-prf=nmlz.sg}\\
\rede{My dog has died.}
\bg.ak=ka=ca sy-a-ma=na\\
{\sc 1sg.poss=gen=add}  die{\sc [3sg]-pst-prf=nmlz.sg}\\
\rede{Mine has died, too.}


\begin{table}
{\small
\begin{centering}
\begin{tabular}{llll}
\toprule
				&{\sc personal pronoun}		&{\sc possessive pronoun}&		{\sc possessive	prefix}\\
\midrule
{\sc 1sg}		&\emph{ka}			&	\emph{akka }	&	\emph{a-}\\
{\sc 1du.excl}&	\emph{kanciŋ}		&	\emph{anciŋga} &	\emph{anciŋ-}\\
{\sc 1pl.excl}&	\emph{kaniŋ}		&	\emph{aniŋga} &	\emph{aniŋ-}\\
{\sc 1du.incl}&	\emph{kanciŋ}		&	\emph{enciŋga} &	\emph{enciŋ-}\\
{\sc 1pl.incl}&	\emph{kaniŋ}		&	\emph{eŋga} &		\emph{eN-}\\
\midrule
{\sc 2sg}&	\emph{nda} 			&	\emph{ŋga} &		\emph{N-}	\\
{\sc 2du}&	\emph{njiŋda}			&	\emph{njiŋga} &	\emph{njiŋ-}\\
{\sc 2pl}&	\emph{nniŋda}			&	\emph{nniŋga} 	&	\emph{nniŋ-}\\
\midrule
{\sc 3sg}&	\emph{uŋ}				&	\emph{ukka} 	&	\emph{u-} \ti \emph{o-}\\	
{\sc 3nsg}&	\emph{uŋci}			&	\emph{uŋciga} &	\emph{uŋci-}\\
\bottomrule
\end{tabular}
\caption{Personal and possessive pronouns, possessive inflection}\label{poss}
\end{centering}
}
\end{table}


\subsection{Possessive prefixes}

Alternatively to using possessive pronouns, relationships of possession can also be expressed by attaching a possessive prefix to the head noun, that refers to the possessee. The prefixes index the number and person of the possessor. Their form is similar to the possessive pronouns, which suggests that they have developed out of them. The nasals in the {\sc 1pl.incl} prefix \emph{eN-} and in the {\sc 2sg} prefix \emph{N-} assimilate in place of articulation to the first consonant of their nominal host (see \Next). The third person singular prefix \emph{u-} has the allomorph \emph{o-} before stems containing /e/ or /o/. The possessees can also be nouns referring to sensations, as in \Next[a]. 

The difference between using a pronoun or a prefix lies in the information structure. If the possessive relationship is focussed on, the pronoun has to be used.

\ex.\ag.n-yupma\\
{\sc 2sg.poss-}sleepiness\\
\rede{your sleepiness}
\bg.m-ba\\
{\sc 2sg.poss-}father \\
\rede{your father} 
\bg.eŋ-gamnibak\\
{\sc 1pl.incl.poss-}friend \\
\rede{our friend} 

Possessive prefixes only attach to nouns, and thus, they are affixes, not clitics. In co-compounds (see \Next[a]), and if two nouns are conjoined in a noun phrase (see \Next[b]), both nouns host the possessive prefix.\footnote{Admittedly, all examples of co-compounds or coordinated nouns with possessive marking in the current data set are from the domain of kinship terms.}

\ex.\ag.u-ppa   u-ma=ci=ca \\
{\sc 3sg.poss-}father {\sc 3sg.poss-}mother{\sc =nsg=add}\\
\rede{her parents, too}  \source{01\_leg\_07.152}
\bg.a-ma=nuŋ                a-na=ŋa                      y-yog-a-n-niŋ=bi\\
{\sc 1sg.poss-}mother{\sc =com} {\sc 1sg.poss-}sister{\sc =erg} {\sc neg-}search{\sc -sbjv[1.P]-neg-neg.pl=irr}\\
\rede{If my mother and sister had not searched for me, ...}  \source{42\_leg\_10.052}


\subsection{Obligatory  possession}\label{inh-poss}

Certain nouns nearly always  appear with possessive prefixes, even when no clear possessor has been  mentioned in the preceding discourse. They can hardly be expressed without belonging to another entity or person. The semantic domains which are relevant for obligatory possession are consanguineal kinship, spatial relations (relational nouns),  body parts and other part-whole relations that are not body parts in the strict sense, such as \emph{otheklup} \rede{half} or \emph{ochon} \rede{splinter}. So far, 118 obligatorily possessed nouns could be found, which makes up roughly 9\% of the nominal lexicon.\footnote{In  \citet[242]{Bickeletal2005_Obligatory} on obligatorily possessed nouns, this phenomenon is defined as “words for which an inflectional category of possession is obligatorily present”. In the current Yakkha data at least some exceptions can be found, so that I conclude that obligatory possession is rather a gradual phenomenon  in Yakkha. More data would be necessary in order to explain apparent exceptions and thus to paint a clearer picture of obligatorily possessed nouns in Yakkha.} Some of the  obligatorily possessed nouns are listed in Table \ref{inalien}. Since obligatory possession is also found in the expression of spatial relations, several adverbs and relational nouns originate in obligatorily possessed nouns (cf. §\ref{adv}).


\begin{table}[htp]
{\small
\begin{centering}
\begin{longtable}{ll}
\toprule
  \multicolumn{2}{c}{{\sc consanguineal kinship}} \\
\midrule
\emph{acya}&\rede{child}\\
\emph{aphu}&\rede{elder brother}\\
\emph{ana}&\rede{elder sister}\\
\emph{aphaŋ}&\rede{father's younger brother}\\
\emph{akoŋma}&\rede{mother's younger sister}\\
\midrule
\multicolumn{2}{c}{{\sc spatial and temporal relations}} \\
\midrule
\emph{ucumphak}&\rede{day after tomorrow}\\
\emph{ulum}&\rede{middle, center}  (relational noun)\\
\emph{oʈemma}&\rede{plains}\\
\emph{uyum}&\rede{side} (relational noun)\\
\emph{okomphak}&\rede{third day after today}\\
\midrule
\multicolumn{2}{c}{{\sc body parts}} \\
\midrule
\emph{unabhak}&\rede{ear}\\
\emph{umik}&\rede{eye}\\
\emph{unamcyaŋ}&\rede{cheek}\\
\emph{unacik}&\rede{face}\\
\emph{utamphwak}&\rede{hair}\\
\emph{umuk}&\rede{hand}\\
\emph{uʈaŋ}&\rede{horn}\\
\emph{ulaŋ}&\rede{leg}\\
\emph{uya}&\rede{mouth, opening}\\
\emph{ophok}&\rede{stomach}\\
\emph{osenkhwak}&\rede{bone}\\
\emph{uʈiŋ}&\rede{thorn, fishbone}\\ 
\midrule
\multicolumn{2}{c}{{\sc part-whole relations}} \\
\midrule
\emph{opoŋgalik}&\rede{bud}\\
\emph{uchuk}&\rede{corner} \\
\emph{upusum}&\rede{crust}\\
\emph{uyin}&\rede{egg}\\
\emph{otheklup}&\rede{half}\\
\emph{okhop}&\rede{husk of rice}\\
\emph{uhup}&\rede{knot}\\
\emph{ukhuppa}&\rede{lid, cover}\\
\emph{ophetrak}&\rede{petal}\\
\emph{ochon}&\rede{thorn, splinter}\\
\emph{oyok}&\rede{place}\\
\emph{uwha}&\rede{wound}\\ 
\bottomrule
\end{longtable}
\caption{Some obligatorily possessed nouns}\label{inalien}
\end{centering}
}
\end{table}



With kinship terms, the first person singular possessive prefix is the default option, e.g., in the citation forms in elicitations, in general statements and in vocatives (as using names to address people is considered impolite). There are some lexicalized terms like \emph{a-mum} \rede{grandmother}, \emph{a-pum} \rede{grandfather}, \emph{a-na} \rede{elder sister}, characterized by a shift of stress to the first syllable. Recall that prefixes generally do not belong to the domain to which stress is assigned. In words like \emph{a.ˈpaŋ} \rede{my house}, the domain of stress excludes the prefix, but several monosyllabic kin terms clearly have the stress on the first syllable: \emph{ˈa.mum}, \emph{ˈa.pum}, \emph{ˈa.na}, \emph{ˈa.ni}. Even though the stress does not treat the prefixes like prefixes any more, the words are still transparent, as \rede{his grandmother} is \emph{u.ˈmum}, not *\emph{u.ˈa.mum}.

Terms for non-consanguineal family relations like \emph{namba} \rede{father-in-law} or \emph{taŋme} \rede{daughter-in-law} do not fall within the domain of obligatory possession (see example \Next[a]).\footnote{I thank Ram Kumar Linkha for pointing this out to me.} This does not mean that possessive prefixes are prohibited, they are just less frequent. The difference is nicely illustrated in \Next[b], from a wedding description that contains many kinship terms. 

\ex.\ag. tabhaŋ heʔne tas-wa-ga=na\\
 	male\_in-law where arrive{\sc -npst-2=nmlz.sg}\\
 	\rede{Where will (your) husband arrive?} 
\bg.nhaŋa   jammai jammai jammai lokondi,     [...] u-chim u-phaŋ=ci \\
and\_then all all all companion\_of\_bride [...] {\sc 3sg.poss-}FyBW {\sc 3sg.poss-}FyB{\sc =nsg}\\
\rede{and then, they all, the bride's companions, her paternal aunts and uncles ...} \source{25\_tra\_01.091}
	
While the default option for kin terms is the first person prefix, for the other obligatorily possessed nouns it is the third person singular, as for instance in \emph{u-ʈiŋ} \rede{thorn}. We find some lexicalized instances here as well, for instance \emph{usa} \rede{fruit}, stressed on the first syllable and lexicalized from the more general noun \emph{sa}, translating as \rede{flesh, meat} and \rede{fruit flesh}. Another instance is \emph{uwa} \rede{nectar, honey, (any) liquid}, also stressed on the first syllable, with the original meaning \rede{water} or, more generally, \rede{liquid}. 

The obligatory possessive marking is also known from other Kiranti languages. Camling also has obligatory possessive marking on inherently relational nouns \citep[41]{Ebert1997Camling}. Similarly, \citet[98-100]{Doornenbal2009A-grammar} lists  classes of nouns that necessarily occur with possessive marking. In her grammar of Thulung, \citet[72]{Lahaussois2002Thulung} mentions that an otherwise rare combination of possessive prefix and genitive marking is frequently found with inalienably possessed nouns such as nouns from the domains of kinship and body parts.




\section{Demonstratives} \label{dem-pron}

The functional core of demonstratives is deixis. Demonstratives (just like pronouns and temporal adverbs such as \rede{tomorrow}) are deictic; their reference depends on a center that is established in the  particular utterance context and that may thus  change with that particular utterance context \citep{Buehler1934_Sprachtheorie, Fillmore1997_Deixis}. The point of reference is typically, but not necessarily, the speaker. 


There are two sets of demonstratives in Yakkha, one set based on proximity and distance to the deictic center (spatial as well as anaphoric, see §\ref{dem-pron-1}) and one set based on the inclination of the landscape, called \emph{geomorphic} in \citet{Bickel1997Spatial}). The latter are treated separately in §\ref{dem-pron-2} on the topography-based orientation system. The roots of the former set are pronominal in  their nature, but they can become adverbial via derivations (see §\ref{dem-set1rel}). 


\subsection{Proximal, distal and anaphoric deixis}\label{dem-pron-1}

Table \ref{dem-tab} shows the forms expressing the three-fold distinction between proximal, distal and anaphoric demonstratives. The proximal forms are used to refer to objects or people that are close to the speaker and can be touched or pointed at, while the distal forms are used for objects or people further away and also for referents that are not present in the speech situation. Narratives mostly use the distal forms, except in direct quotations. The anaphoric demonstratives are used to take up  reference to some participant that had  already been activated at a previous time in discourse, best translated as  \rede{that very (person/thing/event)}. The members of this set of demonstratives are also found in correlative clauses (see §\ref{correlative}). Demonstratives can be used adnominally (i.e., modifying a head noun) and pronominally (i.e. replacing a noun phrase) in Yakkha. Furthermore, demonstratives may replace personal pronouns in the third person, as the use of personal pronouns is considered somewhat rude.


\begin{table}[htp]
\begin{centering}
\begin{tabular}{llll}
\toprule
&			{\sc proximal} 	& {\sc distal}	& {\sc anaphoric}  \\
\midrule
{\sc sg}						&	\emph{na}		&	\emph{nna}		& \emph{honna}\\
{\sc nsg/}	&		\emph{kha}	&	\emph{ŋkha(ci)}	\ti	& \emph{hoŋkha(ci)}\\
{\sc non-count}	&			&	\emph{nnakha(ci)}	& \\
\bottomrule
\end{tabular}\\
\caption{Proximal, distal and anaphoric demonstratives}\label{dem-tab}
\end{centering}
\end{table}


Let us first take a look at the proximal-distal distinction.  In example \Next, the demonstratives are used in attributive function. The number distinction is encoded by the base forms for proximal deixis \emph{na} (singular) and \emph{kha} (nonsingular and non-countable reference).\footnote{The distinction between singular on the one hand and nonsingular/non-countable on the other hand is fundamental and robust in Yakkha, found not only in the  demonstratives but also in nominalizations and in verbal agreement.}  Distal deixis is expressed by adding either a prefix \emph{nna} or just a homorganic nasal to these roots (not segmented in the glosses).\footnote{In Belhare \citep[548]{Bickel2003Belhare}, the lexeme corresponding to \emph{nna} is \emph{ina}. The same sound correspondence (between nasal prefix and prefix \emph{i-}) is found between the Tumok and  the Kharang dialects of Yakkha.} No semantic difference between \emph{nnakha} and \emph{ŋkha} could be determined, and the latter seems like a contracted form of the former. In terms of stress assignment, these demonstratives may cliticize phonologically when they are used attributively, but they are generally able to carry their own stress. They naturally  carry stress when they occur on their own, e.g. \emph{khaci} \rede{these}.

\ex. \ag. na babu \\
	this boy\\
	\rede{this boy}  
	\bg. nna babu \\
		that boy\\
		\rede{that boy}
	\bg. kha babu=ci\\
		these boy{\sc =nsg}\\
		\rede{these boys} 
	\bg. ŋkha babu=ci\\
		those boy{\sc =nsg}\\
		\rede{those boys} 
		\bg. kha kham\\
		this mud\\
		\rede{this mud/soil} 
	\bg. ŋkha kham\\
		that mud\\
		\rede{that mud/soil} 

As example \Last shows, all demonstratives can appear as nominal modifiers (see also \Next). The non-countable reference of \emph{kha} can be illustrated by the difference between \emph{toŋba} \rede{beer served in a small barrel and drunken with a pipe} and \emph{cuwa} \rede{beer}. While the first has countable reference, the latter is treated as a substance and hence has non-countable reference. The demonstrative \emph{kha} may thus refer to nonsingular instances of count nouns (see \Next[b]) or to mass nouns  (see \Next[c]). This distinction of number and countability is also reflected in the sentence-final nominalizers in these examples, which are etymologically related to the demonstratives (discussed at length in §\ref{nmlz-uni}).

\ex. \ag. 	na toŋba imin et-u-ga=na?\\
			this beer\_in\_barrel how like{\sc -3.P[pst]-2=nmlz.sg}\\
			\rede{How do you like this tongba?}
			\bg.	kha toŋba=ci khumdu=ha=ci\\
			these beer\_in\_barrel{\sc =nsg} tasty{\sc =nmlz.nsg=nsg}\\
			\rede{These tongbas are tasty.}
			\bg.	kha cuwa(*=ci) khumdu=ha\\
			these beer{\sc (*=nsg)} tasty{\sc =nmlz.nc}\\
			\rede{This  beer (beer of this house/area) is tasty.}
				
The demonstratives may also head noun phrases, hosting the phrasal morphology and triggering agreement (see \Next). They are more restricted than nominal heads of noun phrases, as they cannot take adnominal modifiers.  
			
	\ex.\ag.	kha=ci ucun=ha=c=em, ŋkha=ci ucun=ha=c=em?\\
			these{\sc =nsg} nice{\sc =nmlz.nsg=nsg=alt} those{\sc =nsg} nice{\sc =nmlz.nsg=nsg=alt}\\ 
			\rede{Are these better, or those?}
	\bg. 	na=go ucun=na\\
			this{\sc =top} nice{\sc =nmlz.sg}\\
			\rede{This one is nice.}		

			
The anaphoric demonstratives identify referents that have already been activated in discourse, and are taken up again, as in \Next, from a pear story. The speaker introduces her narrative with the fact that she has seen a film. Then, the listener makes a joke, distracting away from the film (not included in the example).  The speaker re-introduces the topic with \emph{honna}.

\ex.\ag.ha,    imin ka-ma=ha? ka       khem    eko philm so-ŋ, men=na=i?\\
yes, how say{\sc -inf[deont]=nmlz.nsg}  {\sc 1sg} before one film watch{\sc [3.P;pst]-1sg.A} {\sc neg.cop=nmlz.sg=q}\\
\rede{Yes, how to start? I saw a film before, right?}\source{34\_pea\_04.005}
\bg. honna=be=jhen,  eko jaŋgal=we    eko yapmi  khy-a-masa,    men=na=i?    paghyam.\\
that\_very{\sc =loc=top} one jungle{\sc =loc} one person go{\sc [3sg]-pst-pst.prf} {\sc neg.cop=nmlz.sg=q} old\_man \\
\rede{In that (film), a man had gone into a jungle, right? An old man.}  \source{34\_pea\_04.011}

In \Next, a written narrative, the protagonist wants to go fishing to surprise his sick father. What happens is that he loses the fishing net in the strong currents of the river. The following is said about the net after narrating how he lost it:

\exg. honna           eko=se          jal wa-ya-masa=na\\
that\_very one{\sc =restr} net exist{\sc [3sg]-pst-pst.prf=nmlz.sg}\\
\rede{There had been only that very net.}  \source{01\_leg\_07.214}

Human reference is possible with \emph{honna} as well, exemplified by \Next.
 
\exg.nnakhaʔla   cok-saŋ       honna    yapmi  bhirik=phaŋ   lond-uks-u\\
like\_that do{\sc -sim} that\_very  person cliff{\sc =abl} take\_out{\sc -prf-3.P[pst]}\\
\rede{In this way, he rescued that (afore-mentioned) man from the cliff.} \source{01\_leg\_07.330}

In \Next, also a written narrative, the referent taken up from the previous clause is a cradle. 

\ex.\ag.uŋ=ŋa   hoŋma=ŋa   eko mina  yoŋ   yaŋ-kheʔ-ma-si-meʔ=na   nis-uks-u\\
{\sc 3sg=erg} river{\sc =ins} one small cradle flush{\sc -V2.carry.off-inf-aux.prog-npst=nmlz.sg} see{\sc -prf-3.P[pst]}\\
\rede{She saw a little cradle being carried off by the river. } \source{01\_leg\_07.288}
\bg.nhaŋ    uŋ=ŋa   hattapatta honna        yoŋ   lab-uks-u\\
and\_then {\sc 3sg=erg} hastily that\_very  cradle grab{\sc -prf-3.P[pst]}\\
\rede{And hastily she grabbed that cradle.} \source{01\_leg\_07.289}


The singular form  \emph{na} could be etymologically related to a topic particle of the same form, as it is still found in Belhare or Puma, for instance \citep[559]{Bickel2003Belhare, Bickeletal2009Puma}. Furthermore, the demonstratives \emph{na} and \emph{kha} have  developed into the nominalizers \emph{=na} and \emph{=ha} which show exactly the same distribution with regard to number and the count/mass distinction as the demonstratives (cf. §\ref{nmlz-uni}). On a final note, clause-initial coordinators like \emph{nhaŋ}, \emph{nnhaŋ}, \emph{khoŋ} and \emph{ŋkhoŋ} (all paraphrasable with \rede{and then} or \rede{afterwards}) are demonstratives with ablative marking historically.


\subsection{Demonstrative adverbs and quantifiers}\label{dem-set1rel}

The proximal-distal-anaphoric distinction is also present in a set of demonstrative adverbs and quantifiers, as summarized in Table \ref{pr-di-an-adv}. In \Next we can see some examples of anaphoric demonstrative adverbs based on the root \emph{hon}. The sentence in \Next[a] is uttered at the end of a narrative, and the adverbs refer to the content and amount of the events just told.\footnote{Quantifying expressions (both for amount and size) are the topic of §\ref{sec-quant} below.}  In \Next[b], \emph{hoŋkhaʔniŋ} refers to the time at which the events took place (specified in a previous sentence), and in \Next[c], \emph{honnhe} refers to the place just mentioned in the conversation.

\ex.\ag.liŋkha=ci=ga         lagi, hoŋkhaʔla=oŋ,    hoŋkhiŋ=se\\
Linkha\_clan\_member{\sc =nsg=gen} for like\_that{\sc =seq} that\_much{\sc =restr}\\
\rede{For the Linkhas, like that, that much only.} \source{11\_nrr\_01.042}
\bg.hoŋkhaʔniŋ ten=beʔ=na               yalumma          a-mum=ŋa   so-saŋ    ka-ya:\\
that\_very\_time village{\sc =loc=nmlz.sg} talkative\_granny  {\sc 1sg.poss-}grandmother{\sc =erg} look{\sc -sim} say{\sc [3sg]-pst}\\
\rede{At that time, a talkative old lady, watching, it said: ...}  \source{41\_leg\_09.041}
\bg.honnhe=maŋ khe-me-ŋ=na\\
right\_there{\sc =emph} go{\sc -npst-1sg=nmlz.sg}\\
\rede{I will go right there.} (in a talk about Mamling village, a new person shows up and states that she will go right to that village)	
	
\begin{table}[htp]
{\small
\begin{centering}
\begin{tabular}{llll}
\toprule
&			{\sc proximal} 	& {\sc distal}	& {\sc anaphoric}  \\
\midrule
{\sc location}&		\emph{nhe}  &	\emph{nnhe} 	& \emph{honnhe} \\
&\rede{here}	&\rede{there}	&\rede{where mentioned before}\\
{\sc time}&		\emph{khaʔniŋ} 	&	\emph{ŋkhaʔniŋ}  \ti \emph{nnakhaʔniŋ}	& \emph{hoŋkhaʔniŋ}\\
& \rede{this time, now}&\rede{that time, then}	&\rede{right at that time}\\
{\sc manner} 		&	\emph{khaʔla} 	&	\emph{ŋkhaʔla} \ti \emph{nnakhaʔla}& \emph{hoŋkhaʔla} \\
&\rede{like this}	&\rede{like that}	&\rede{like mentioned before}\\
{\sc amount}/&		\emph{khiŋ}  	&	\emph{ŋkhiŋ} \ti \emph{nnakhiŋ}	& \emph{hoŋkhiŋ} \\
{\sc size/}&\rede{this much}/&\rede{that much}/&\rede{as much as mentioned before}/	\\
{\sc degree}&\rede{this big}&\rede{that big}&\rede{as big as mentioned before}\\
\bottomrule
\end{tabular}\\
\caption{Demonstrative adverbs and quantifiers}\label{pr-di-an-adv}
\end{centering}
}
\end{table}

	

\section{Indefinite reference}\label{sec-indef}

Yakkha does not have a morphologically distinct class of indefinite pronouns; all pronouns and demonstratives are definite. There are, however,  several strategies to convey indefinite reference, including the use of simple nouns. Occasionally, the numeral \emph{eko} \rede{one} is also used for this purpose. In example \Next[a], \emph{eko} refers to an object in a future and hence irrealis statement; in \Next[b], \emph{eko} refers to a specific (i.e., known to the speakers), but still indefinite person (i.e., not determined in a way that the hearer can identify the referent).

\ex. \ag. uŋ mit-a:       haku eko paŋ  cok-ma    ta-ya=na\\
		{\sc 3sg} think{\sc [3sg]-pst}: now one house{\sc } make-{\sc inf} come{\sc [3sg]-pst=nmlz.sg}	\\
	\rede{He thought: Now the time has come to build a house.} \source{27\_nrr\_06.006}
\bg.aniŋ=ga  eko mamu mas-a-by-a-ma=na\\
{\sc 1pl.excl.poss=gen} one girl get\_lost{\sc [3sg]-pst-V2.give-pst=nmlz.sg}\\
\rede{One of our girls got lost.}	 \source{22\_nrr\_05.076}


Interrogatives can also function as indefinite pronouns, particularly in contexts where the referent is unknown to the speaker, as in \Next. Interrogatives as indefinite pronouns may head noun phrases and can be modified (see \Next[a]); they may also modify nouns themselves (see \Next[b]). Using interrogatives for indefinite reference is a very common strategy cross-linguistically, which can be explained by the functional similarity of the two. Both express an information gap and vagueness at the utterance level \citep[170]{Haspelmath1997_Indefinite}. 

 \ex.\ag. uŋci yuncamakekek i ŋ-ga-ya-masa\\
{\sc 3nsg} funny what {\sc 3pl-}say{\sc -pst-pst.prf}\\
 \rede{They had said something funny.} \source{41\_leg\_09.029}
 \bg. nhaŋa   desan-masan  n-da-me      i=ha \\
 and\_then malicious\_ghost {\sc 3pl-}come{\sc -npst} what{\sc =nmlz.nsg}\\
 \rede{And then, some scary ghosts will come.} \source{28\_cvs\_04.266 }
 
 As \Next shows,  information that is known to the speaker, but that she does not want to disclose, is also covered by the interrogative-indefinite polysemy.
 
  \exg. khy-a-ŋ=na=le,                           pheri khaʔla=maŋ=ba,        sala   i=ha                i=ha                ta-me\\
 go{\sc -pst-1sg=nmlz.sg=ctr} again like\_this{\sc =emph=emph} talk what{\sc =nmlz.nc} what{\sc =nmlz.nc} come{\sc [3sg]-npst} \\
 \rede{I just went, again, just like this, one talks about a little bit of this, a little bit of that.} (the speaker explains why she had gone, i.e., to talk, without specifying what they talked about) \source{28\_cvs\_04.319}
 
Exhaustive reference, i.e., including all imaginable referents in a given context, is expressed by attaching the additive focus particle \emph{=ca} to an interrogative pronoun (see \Next). This works with affirmative and with negated statements, in the latter case with the effect of exhaustive negation (see \Next[c]).

\ex.\ag.i=ha camyoŋba=ca a-sap thakt-wa-ŋ=ha\\
what{\sc =nmlz.nc} food{\sc =add} {\sc 1sg.poss-[stem]} like{\sc-npst[3.P]-1sg.A=nmlz.nsg}\\
\rede{I like any (kind of) food.}
\bg. eŋ=ga              niŋ=be    uŋci i=ha            cok-ma=ca            tayar  n-leŋ-me\\
{\sc 1pl.incl.poss=gen} name{\sc =loc} {\sc 3nsg} what{\sc =nmlz.nc} do{\sc -inf=add} ready {\sc 3pl-}become{\sc -npst}\\
\rede{They will be  ready to do anything in our name.} \source{01\_leg\_07.084}
\bg.ŋkhaʔla bhoŋ     lop ka  i=ha=ca                       n-nakt-a-ŋa-n,\\
like\_that {\sc cond} now {\sc 1sg} what{\sc =nmlz.nc=add} {\sc neg-}ask\_for{\sc -imp-1sg.P-neg}\\
\rede{If it is like that, do not ask me for anything right now.}  \source{27\_nrr\_06.025}

Occasionally, the interrogative pronoun can also be doubled, often in combination with markers of focus or emphasis (see \Next).

\ex. \ag.chippakekek=na         i=na=i                i=na           loʔwa=na\\
disgusting{\sc =nmlz.sg} what{\sc =nmlz.sg=emph} what{\sc =nmlz.sg} like{\sc =nmlz.sg}\\
\rede{like some disgusting, undefinable (thing)} \source{40\_leg\_08.054 }
\bg. eh,    ikhiŋ   mam=ha i=ya i=ya=le                         naŋ-me-c-u=ha baŋniŋgo     haʔlo\\
 oh, how\_much big{\sc =nmlz.nc} what{\sc =nmlz.nc} what{\sc =nmlz.nc=ctr} ask{\sc -npst-du-3.P=nmlz.nc} {\sc top} {\sc excla}\\
 \rede{Oh, (we had thought that) they would ask for something big!}\\
 (instead, they asked for a minor favor) \source{22\_nrr\_05.129}
 
 
Another strategy to express indefinite reference is to use an interrogative pronoun and to reduplicate  the fully inflected verb (see \Next). Additionally, the interrogative phrase may host a topic marker \emph{=ko}, which is not possible in interrogative utterances, since the inherent focus of interrogative phrases rules out topic marking on them. Both strategies help to disambiguate indefinite statements and interrogative utterances.

\ex.\ag.a-yaŋ heʔne mas-a-by-a=ha mas-a-by-a=ha\\
{\sc 1sg.poss-}money where get\_lost{\sc [3sg]-pst-V2.give-pst=nmlz.nc} get\_lost{\sc [3sg]-pst-V2.give-pst=nmlz.nc}\\
\rede{My money got lost somewhere.}
\bg.surke=ŋa isa=ge=ko khus-u-co-ya khus-u-co-ya\\
Surke{\sc =erg} who{\sc =loc=top} steal{\sc -3.P-V2.eat-pst} steal{\sc -3.P-V2.eat-pst}\\
\rede{Surke (a dog) stole (food) from someone's house.}
\bg. na inimma=be a-ppa  a-ma=ci                 heʔne m-phaps-a-khy-a                      m-phaps-a-khy-a\\
this market{\sc =loc} {\sc 1sg.poss-}father  {\sc 1sg.poss-}mother{\sc =nsg} where {\sc 3pl-}entangle{\sc -pst-V2.go-pst} {\sc 3pl-}entangle{\sc -pst-V2.go-pst}\\
\rede{My parents got lost somewhere in this market.}\footnote{The word \emph{inimma} is a neologism not widely in use.} \source{01\_leg\_07.163}

In practice, indefinite reference is often just realized by the omission of overt arguments, since overt personal pronouns are not required for accessible referents, not even for mentioning them for the first time. In \Next, the referent talked about is only introduced by the verbal agreement: people talk about someone they saw walking away, without recognizing who it was.

\exg.churuk uŋ-saŋ khy-a-ma=na. isa=ʔlo?\\
cigarette drink{\sc -sim} go{\sc [3sg]-pst-prf=nmlz.sg} who{\sc =excla}\\
\rede{He has gone, smoking a cigarette. But who was it??}


\section{Quantifiers, numerals and numeral classifiers}\label{sec-quant}

\subsection{Quantification, size and degree}

Yakkha has several quantifiers to indicate the amount, size,  degree or intensity of the concepts expressed by nouns, adjectives or verbs. They are listed in Table \ref{quant}, with the  word classes with which they combine. The form  \emph{maŋpha} \rede{much/very} is special insofar as it may also express  the degree of another quantifier, such as in \emph{maŋpha pyak} \rede{really much}. The table also includes deictic quantifiers and degree words.

\begin{table}[htp]
\begin{centering}
\begin{tabular}{lll}
\toprule
{\sc yakkha} & {\sc gloss} & {\sc domain} \\
\midrule
\emph{mi}& \rede{a little} & A \\
\emph{miyaŋ}& \rede{a little} & N, V, A \\
\emph{mimik}& \rede{a little}& N, V\\
\emph{ghak}& \rede{all/whole}&N\\
\emph{tuknuŋ}& \rede{completely}&V, A\\
\emph{pyak}& \rede{much/ many/ very}&N, V, A\\
\emph{maŋpha}& \rede{much/very}&A, {\sc quant}\\
\emph{ibibi}& \rede{very much/many}&N\\
\midrule
\emph{khiŋ}& \rede{this much/this big} (deictic)&N, V, A\\
\emph{ŋkhiŋ}& \rede{that much/that big} (deictic)&N, V, A\\
\emph{hoŋkhiŋ}& \rede{as much/big as stated before} (deictic)& N\\
\bottomrule
\end{tabular}\\
\caption{Quantifiers}\label{quant}
\end{centering}
\end{table}

The difference between \emph{mimik} and \emph{miyaŋ} (both: \rede{a little}) is subtle. Both can be found with nouns (see \Next) or verbs (see \NNext), but \emph{miyaŋ} is the typical choice with nouns, while  \emph{mimik} is found more often with verbs. Both words may also appear as proforms heading noun phrases, as  \Next[a] and \Next[c] show. 

\ex.\ag.nda=ca miyaŋ=se uŋ-uǃ\\
{\sc 2sg=add} a\_little{\sc =restr} drink{\sc -3.P[imp]}\\
\rede{You too, drink, just a little!}
\bg.ka miyaŋ cama py-a-ŋ-eba\\
{\sc 1sg} a\_little rice give{\sc -imp-1sg.P-pol.imp}\\
\rede{Please give me a little rice.} 
\bg. mimik,   ŋ-khot-a-n    bhoŋ=se   kaniŋ   mimik    in-u-ca-wa-m-ŋ=ha\\
a\_little {\sc neg-}be\_enough{\sc -pst-neg} {\sc cond=restr} {\sc 1pl[erg]} a\_little buy{\sc -3.P-V2.eat-npst-1pl.A-excl=nmlz.nc}\\
\rede{A little, only if is not enough we buy a little.} \source{28\_cvs\_04.038}

\ex.\ag. kam=ca cok-ma     haʔlo, mimik, \\
work{\sc =add} do{\sc -inf[deont]} {\sc excla} a\_little  \\
\rede{One also has to work a little, ...}  \source{28\_cvs\_04.326}
\bg.miyaŋ ucun ŋ-get-u-ŋa-n=na         loppi\\
a\_little nice {\sc neg-}bring\_up{\sc -3.P[pst]-excl-neg=nmlz.sg} perhaps\\
\rede{Maybe I did not recall it (a story) so well.} (lit. \rede{I slightly did not recall it nicely, perhaps.})  \source{11\_nrr\_01.038}
 \bg. miyaŋ taŋkhyaŋ mopmop  cok-t-a-by-a\\
a\_little  sky covered make{\sc -ben-imp-V2.give-imp}\\
\rede{Please make the sky a little cloudy.}  \source{37\_nrr\_07.100}

 Furthermore, \emph{miyaŋ} is also found with adjectives and  adverbs (see \Next).
 
 \exg.hoŋ=bhaŋ   miyaŋ yoʔyorok\\
 hole{\sc =abl} a\_little across\\
\rede{a little further away from the hole}  \source{04\_leg\_03.011}

The quantifier \emph{pyak} is used with count and mass nouns, and also with an intensifying function when it is combined with verbs and adverbs/adjectives. It signifies a high amount or degree of whatever is expressed by the head that it modifies. Thus, it can be rendered with  English \rede{much}, \rede{many} and \rede{very}. Examples are provided below in \Next for the nominal domain and in \NNext for verbal and adverbal/adjectival uses. In \NNext[a], \emph{pyak} is further emphasized by the deictic degree particle \emph{khiŋ}, yielding the exclamative \rede{how much!}.

\ex.\ag. pyak  sakheʔwa=ci\\
many pigeon{\sc =nsg}\\
\rede{many pigeons} \source{01\_leg\_07.013}
\bg.pyak  ŋ-geŋ-me-n\\
much {\sc neg-}bear\_fruit{\sc [3sg]-npst-neg}\\
\rede{Not much will ripen.}  \source{01\_leg\_07.122}
\bg.pyak  yaŋ  ub-wa-ŋ, \\
much money earn{\sc -npst[3.P]-1sg.A} \\
\rede{I will earn much money, ...}  \source{01\_leg\_07.190}

\ex.\ag.   ka  khiŋ     pyak  a-ma=ŋa   u-luŋma   tuŋ-me-ŋ=na\\
{\sc 1sg} this\_much much {\sc 1sg.poss-}mother{\sc =erg} {\sc 3sg.poss-}liver pour{\sc -npst-1sg.P=nmlz.sg}\\
\rede{How much my mother loves me!}  \source{01\_leg\_07.079}
\bg. suku   pyak  cond-a-sy-a-ma\\
Suku much be\_happy{\sc [3sg]-pst-mddl-pst-prf}\\
\rede{Suku was very happy.}  \source{01\_leg\_07.151}
\bg.   eko pyak thuŋdu=na        yapmi\\
one very rich{\sc =nmlz.sg} person\\
\rede{a very rich man}  \source{04\_leg\_03.014}

Examples with \emph{ibibi} (referring to an unspecific high quantity) are few; one is shown below in \Next.

\exg.wathaŋ=be      ibibi     yapmi=ci     ta-saŋ        wasi-saŋ        khe-saŋ        n-jok-ma-sy-a\\
water\_tap{\sc =loc} many\_many person{\sc =nsg} come{\sc -sim} wash{\sc -sim} go{\sc -sim} {\sc 3pl-}do{\sc -inf-aux.prog-pst}\\
\rede{At the watertap,  many, many people kept coming, bathing, going.} \source{40\_leg\_08.049}

The exhaustive quantifier \emph{ghak} \rede{all, whole} can refer to an exhaustive number or amount, as in  \Next[a], or to a complete unit, as in \Next[b] and  \Next[c]. The potential ambiguity is resolved by the verbal number agreement, which has do be plural in the exhaustive reading.

\ex.\ag.ghak limbu          m-bog-a-ma-ci=hoŋ,\\
all Limbu\_person {\sc 3pl-}get\_up{\sc -pst-prf-nsg=seq}\\
\rede{As all the Limbus woke up, ...}  \source{22\_nrr\_05.027}
\bg. ghak  ceʔya \\
whole matter\\
\rede{the whole matter}  \source{01\_leg\_07.024}
\bg.ghak ten     mag-a-khy-a,\\
whole village burn{\sc [3sg]-pst-V2.go-pst}\\
\rede{The whole village burned down.}  \source{22\_nrr\_05.026}

The deictic quantifier \emph{khiŋ} has to be interpreted with respect to the utterance context, and it can  refer to  amount or size. In most cases, its use is accompanied by  gestures that indicate the size or the amount of some entity. Occasionally,  the nominal comitative can be found attached to \emph{khiŋ}  (see \Next[c]).

\ex.\ag.khiŋ     tukkhi ŋ-aŋd-u,\\
this\_much pain {\sc 3pl.A-}endure{\sc -3.P[pst]}\\
\rede{They endured so much troubles, ...}  \source{14\_nrr\_02.07}
\bg.mi=na  chun-d-eʔ=na,   khiŋ    leŋ-d-eʔ=na, \\
small{\sc =nmlz.sg} shrink{\sc [3sg]-V2.give-npst=nmlz.sg} this\_big become{\sc [3sg]-V2.give-npst=nmlz.sg}\\
\rede{It shrinks, it becomes so small, ... }  \source{36\_cvs\_06.228}
\bg.khiŋ=nuŋ em-ma=niŋa lak=nuŋ leks-a=bi\\
this\_much{\sc =com} insert{\sc -inf=ctmp} salty{\sc =com} become{\sc [3sg]-sbjv=irr}\\
\rede{If one inserted this much, it would become salty.} 

In parallel to the demonstratives described in §\ref{dem-pron}, \emph{ŋkhiŋ} may express distal reference, i.e., \rede{that much} (compare \Next[a] and \Next[b]). In \Next[b], instead of indicating the size with his own hands, the speaker points to a piece of wood laying nearby. The distal reference is also used in general statements, as in \Next[c].

\ex.\ag.puchak khiŋ=na sa=na!\\
snake this\_much{\sc =nmlz.sg} {\sc cop.pst[3sg]=nmlz.sg}\\
\rede{The snake was this bigǃ} (The speaker is showing with own hands how big it was.)
\bg.puchak ŋkhiŋ=na sa=na!\\
snake this\_much{\sc =nmlz.sg} {\sc cop.pst[3sg]=nmlz.sg}\\
\rede{The snake was that bigǃ} (The speaker is pointing to a piece of wood laying nearby.)
\bg.cuŋ=be ŋkhiŋ ucun m-phem-me-n=ha\\
cold{\sc =loc} that\_much nice {\sc neg-}bloom{\sc [3sg]-npst-neg=nmlz.nsg}\\
\rede{In winter, it does not bloom so nicely.} (\emph{=ha} being used because of mass reference, blossoms in general, not a countable plurality of blossoms) 

 
Anaphoric deixis is possible as well, using \emph{hoŋkhiŋ}. The sentence in \Next follows a long enumeration of particular things the protagonist had to do, and \emph{hoŋkhiŋ} refers back to them.

\exg.nhaŋ    nam wandik=ŋa lom-meʔ=niŋa        hoŋkhiŋ cok-ni-ma           pʌrne    sa=bu\\
and\_then sun next\_day{\sc =ins} come\_out{\sc [3sg]-npst=ctmp} that\_much do{\sc -compl-inf[deont]} having\_to {\sc cop.pst[3sg]=rep}\\
\rede{And then, at the dawn of the next day, all that work had to be finished, people say.}  \source{11\_nrr\_01.010}

\subsection{Numerals and classifiers}\label{sec-num}

\subsubsection{Cardinal numerals}

The inherited Tibeto-Burman numerals have largely gotten lost in Kiranti \citep{Ebert1994The-structure}. In Yakkha only the numerals \emph{i} \rede{one}, \emph{hiC} \rede{two}\footnote{The capital /C/ stands for a plosive. As the numeral does not occur independently, and as it always assimilates to the following consonant, its place of articulation could not be determined.} and \emph{sum} \rede{three} are known. Another  numeral for \rede{one} is found, which is the Nepali loan \emph{eko}. It already replaces the Yakkha numeral \emph{i} in several contexts. In counting, for instance,  \emph{eko} prevails in the majority of cases. Some fixed expressions, like \emph{i len} \rede{one day}, however, contain the Yakkha form. It is quite likely  that the numeral \emph{i} and the interrogative root \emph{i} share a common origin.

Unlike in some Newari varieties,\footnote{For instance, in Dolakha Newari \citep[220]{Genetti2007_Newari} and the Newari spoken in Dulikhel (own observations).} numeral classification does not play a prominent role in Kiranti languages. Yakkha has one numeral classifier \emph{-paŋ} for human reference (cognate, e.g., with Belhare \emph{-baŋ}, Athpare \emph{-paŋ}, Camling \emph{-po}, Bantawa \emph{-pok}, Hayu \emph{-pu}). It is  used only with the Yakkha numerals \rede{two} and \rede{three}  (see \Next).  Nonsingular marking of the head noun is frequent, but optional (discussed in  §\ref{number}). For numerals above \rede{three},  borrowed Nepali numerals, as well as the Nepali classifiers \emph{jana} for humans and \emph{(w)oʈa} for things are used (see \NNext[a]). Some words for measuring units or currency may also function as classifiers (see \NNext[b]).  

\ex.\ag.eko yapmi\\
one person\\
\rede{one man/person}
\bg.hip-paŋ babu(=ci)\\
two{\sc -clf.hum} boy({\sc =nsg})\\
\rede{two boys} 
\bg.sum-baŋ mamu(=ci)\\
three{\sc -clf.hum} girl({\sc =nsg})\\
\rede{three girls} 

\ex. \ag.bis ora  khibak=ca\\
twenty {\sc clf}  rope{\sc =add}\\
\rede{twenty ropes} \source{11\_nrr\_01.012}
\bg.ah,    pãc, chʌsay     rupiya\\
yes five six\_hundred rupee\\
\rede{five, six hundred rupees} \source{28\_cvs\_04.075}

Since there is no classifier for non-human reference in Yakkha, the nonsingular marker \emph{=ci} has undergone reanalyzation in order to fill the  position of the classifier (see \Next). This is the only instance where nonsingular \emph{=ci} may occur inside a noun phrase. 

\ex.\ag.hic=ci yaŋ=ci\\
two{\sc =nsg} coin{\sc =nsg}\\
\rede{two coins}  \source{26\_tra\_02.032}
\bg.sum=ci ceʔya\\
three{\sc =nsg} word\\
\rede{three words}  \source{36\_cvs\_06.345}

Numeral expressions may also occur without a head noun; i.e., they can fill the structural position of a noun phrase (see \Next).

\ex. \ag.hip-paŋ=se\\
two{\sc -clf.hum=restr}\\
\rede{only two people} \source{36\_cvs\_06.578}
\bg.hip-paŋ=ŋa       ni-me-c-u=ha\\
two{\sc -clf.hum=erg} know{\sc -npst-du-3.P=nmlz.nc}\\
\rede{The two of them know it (how to divinate).} \source{22\_nrr\_05.081}
			
\subsubsection{Counting events}

Yakkha has a marker \emph{-ma} to individuate and count events, i.e., to express \rede{once}, \rede{twice}, \rede{three times}. It only occurs with the inherited (Tibeto-Burman) Yakkha numerals.

\ex. \ag.ka i-ma pukt-a-ŋ=na \\
{\sc 1sg} one{\sc -count} jump{\sc -pst-1sg=nmlz.sg}\\
\rede{I jumped once.}
\bg.minuma=ŋa   hip-ma     sum-ma       u-muk       hoŋ=be    end-uks-u=ca mima  lap-ma      n-yas-uks-u-n\\
cat{\sc =erg} two{\sc -count} three{\sc -count} {\sc 3sg-}hand hole{\sc =loc} insert{\sc -prf-3.P=add} mouse catch{\sc -inf} {\sc neg-}be\_able{\sc -prf-3.P-neg}\\
\rede{Although the cat tried to put its paw into the hole two or three times, it could not catch the mouse.} \source{04\_leg\_03.009}

\section{Interrogative proforms}\label{interr}

Yakkha interrogatives are based on the roots \emph{i} and \emph{heʔ}. Table \ref{int-pron} provides an overview. While \emph{i} may also occur independently, with the meaning \rede{what} (referring to events, see \Next),  \emph{heʔ} always occurs with further morphological material. Some interrogatives are easily analyzable into a base plus case marker, nominalizer or clause linkage marker, but others are not transparent. Interrogatives may also function as indefinite pronouns (see §\ref{sec-indef} above).

	\exg. i leks-a?\\
		what	happen{\sc [3sg]-pst}\\
		\rede{What happened?}


\begin{table}[htp]
\begin{center}
\begin{tabular}{ll}
\toprule
{\sc yakkha}&{\sc gloss}\\
\midrule
\emph{i \ti ina \ti iya} &\rede{what}\\
\emph{isa} &\rede{who}\\
\emph{imin} &\rede{how}\\
\emph{ikhiŋ} &\rede{how much}, \rede{how many},\\
&\rede{how big}\\
\emph{ijaŋ} &\rede{why}\\
\emph{heʔna \ti hetna}& \rede{which} ({\sc int=nmlz})\\
\emph{heʔne \ti hetne} &\rede{where}({\sc int=loc})\\
\emph{heʔnaŋ \ti heʔnhaŋ \ti} &\rede{where from} ({\sc int=abl})\\
\emph{hetnaŋ \ti hetnhaŋ} &\\
\emph{heʔniŋ \ti hetniŋ} &\rede{when} ({\sc int=ctmp})\\
\bottomrule
\end{tabular}
\end{center}
\caption{Interrogatives}\label{int-pron}
\end{table}

When the requested bit of information has a nominal nature, the base \emph{i} occurs with the nominalizers \emph{=na} or  \emph{=ha \ti =ya} (see §\ref{nmlz-uni}). For example, food is expected to consist of several different items, and will be requested with the nonsingular/non-countable form \emph{=ha \ti =ya} (see \Next[a]). Interestingly, these nominalized forms can also occur inside a noun phrase (see \Next[b]). In this example, \emph{ina} does not request  the identification of one item out of a set, as \emph{heʔna} \rede{which} would. It rather implies that nothing is presupposed. The sentence is from a dowry negotiation, and here the speakers imply that there is nothing more to give to the bride. Similarly, when the identity of a person is requested but the speaker has no set of possible answers in mind, \emph{isa} can occur inside a noun phrase (see \Next[c]). The context of this example was that some people were talking about the newly arrived researcher, and some other people who did not know about this fact (and did not see the researcher sitting around the corner) requested to know whom they were talking about.

\ex. \ag. i=ya ca-ma\\
		what{\sc =nmlz.nsg} eat{\sc -inf}\\
		\rede{What to eat?}
	\bg.  nani, i=na yubak? n-chimd-uks-u\\
		child, what{\sc =nmlz.sg}	property {\sc 3pl.A-}ask{\sc -prf-3.P}\\
		\rede{“Child, what property?” they asked her.} \source{37\_nrr\_07.006}
		\bg.isa mamu?\\
		who girl\\
		\rede{What girl (are you talking about)?}

The interrogatives \emph{ina/iya} and \emph{isa} may also head noun phrases (without modifiers), host nominal morphology and appear as predicates of interrogative copular clauses (see \Next). When a noun phrase is headed by an interrogative, modifying material is not allowed, except for clauses in which the interrogatives have an indefinite interpretation (discussed above in §\ref{sec-indef}).  The quantifying/degree interrogative \emph{ikhiŋ} (derived from the demonstrative base \emph{khiŋ} discussed in §\ref{sec-quant}) may also occur in noun-modifying position (see \NNext).

\ex.\ag. i=ga lagi ta-ya-ga=na?\\
	what{\sc =gen} for come{\sc -pst-2=nmlz.sg}\\
	\rede{What did you come for?}
\bg.na   i=ŋa      hab-a=na\\
this what{\sc =ins} cry{\sc [3sg]-pst=nmlz.sg}\\
\rede{What made her cry?/Why does she cry?} \source{13\_cvs\_02.050}
	\bg.piccha=be    isa=ŋa   ghak nis-wa=ha?\\
	child(hood){\sc =loc} who{\sc =erg} all know{\sc [3A;3.P]-npst=nmlz.nsg}\\
	\rede{Who knows everything in childhood?}  \source{40\_leg\_08.079}\\
	(a rhetorical question)
	\bg.	kha yapmi=ci isa=ci?\\
			these  person{\sc =nsg} who{\sc =nsg} \\
			\rede{Who are these people?} \\
	
\ex. \ag. a-koŋma=ga      biha     ikhiŋ   sal=be     leks-a=na?\\
{\sc 1sg.poss-}MyZ{\sc =gen}  marriage how\_much year{\sc =loc} happen{\sc [3sg]-pst=nmlz.sg}\\
\rede{In which year was  your (i.e., my aunt's) marriage?} \source{06\_cvs\_01.031}
\bg.ikhiŋ   miʔwa hond-end-u-g=ha!\\
how\_much tear uncover{\sc -V2.insert-3.P[pst]-2.A=nmlz.nsg}\\
\rede{How many tears you have shed!}\footnote{The V2 \emph{-end} indicates transitive motion downwards here.} \source{37\_nrr\_07.111}

Naturally, the same applies to \emph{heʔna} \rede{which} (see \Next); it always requests the identity of some item from a presupposed set.

\ex. \ag. heʔna des     wei-ka=na?\\
which country live{\sc [npst]-2=nmlz.sg}\\
\rede{In which country do you live?} \source{28\_cvs\_04.080}
\bg.heʔna  nis-u-ga=na?\\
which see{\sc -3.P[pst]-2.A=nmlz.sg}\\
\rede{Which one did you see?}

The interrogative \emph{ikhiŋ} is furthermore often found in exclamations about size, amount or degree, lacking the interrogative function (see \Next and \LLast[b]).

\ex. \ag.lambu ikhiŋ   mi=na, ammai  ikhiŋ   mi=na  lambu lai!\\
road how\_much small{\sc =nmlz.sg} oh\_my! how\_much small{\sc =nmlz.sg} road {\sc excla}\\
\rede{How narrow the road is, oh my, what a narrow road!} \source{36\_cvs\_06.223 }
\bg.nna  dewan-ɖhuŋga baŋna    luŋkhwak sahro cancan sa-ma=na,                pyak cancan, ikhiŋ   cancan!\\
that Dewan-stone so-called stone very high {\sc cop.pst-prf=nmlz.sg} very high, how\_much high\\
\rede{That rock called Dewan stone was really high, it was very high, how high it was!}\source{37\_nrr\_07.042}
\bg.ikhiŋ khumdu nam-my=a!\\
how\_much  tasty smell{\sc [3sg]-npst=nmlz.nc}\\
\rede{How good it smells!}

 Examples of the other interrogatives are shown in \Next.

 \ex. \ag. ɖaktar=ci=be       kheʔ-ma           pʌryo, hetniŋ,  hetne  kheʔ-ma=na=lai?\\
 doctor{\sc =nsg=loc} go{\sc -inf[deont]} having\_to when, where go{\sc -inf=nmlz.sg=excla}\\
 \rede{He has to go to the doctor; when, and where to go?} \source{36\_cvs\_06.179}
 \bg. sondu  khaʔla=na          cuŋ=be    tek   me-waʔ-le       jal kapt-uks-u-g=hoŋ                 hetnaŋ    tae-ka=na?\\
 sondu like\_this{\sc =nmlz.sg} cold{\sc =loc} clothes {\sc neg-}wear{\sc -cvb} net carry{\sc -prf-3.P[pst]-2.A=seq} where\_from come{\sc [npst]-2=nmlz.sg}\\
 \rede{Sondu, where do you come from, in this cold, without clothes, and carrying this net?} \source{01\_leg\_07.232}
 \bg. ka  ijaŋ cem-me-ŋ-ga=na?\\
{\sc 1sg} why cut{\sc -npst-1sg.P-2.A=nmlz.sg}\\
\rede{Why do you cut me?} \source{27\_nrr\_06.013}
\bg. kisa saŋ-khek-khuwa,                         hetne  sa-het-u=na                                     haʔlo?\\
deer lead\_by\_rope{\sc -V2.carry.off-nmlz} where lead\_by\_rope{\sc -V2.carry.off-3.P[pst]=nmlz.sg}  {\sc excla}\\
\rede{The one who led the deer away, where did he lead it, by the way?} \source{19\_pea\_01.024}
\bg.	aniŋ=ga ten imin et-u-ga=na?\\
			{\sc 1pl.excl.poss=gen} village how perceive{\sc -3.P[pst]-2.A=nmlz.sg}\\
			\rede{How do you like our village?}


