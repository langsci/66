\chapter{The noun phrase}\label{ch-noun}

The class of nouns is defined by the following structural features in Yakkha: nouns may head noun phrases and function as arguments of verbs without prior morphological derivations. Morphological categories typically associated with nouns are number and case. But since in Yakkha these operate on the phrasal level, the only category identifying lexical nouns is possessive inflection, marked by prefixes. Nouns typically refer to time-stable concepts like living beings, places or things, but also to some abstract or less time-stable concepts like \emph{sakmaŋ} \rede{famine} or \emph{ceʔya} \rede{language, matter, word}.

The sections of this chapter deal with the formation of nouns and some properties of lexical nouns (see \sectref{lex-noun}), nominal morphology (see \sectref{nom-morph}),  relational nouns (see \sectref{postpos-2}), and with the structure of the noun phrase (see \sectref{str-np}). 

\section{Noun formation and properties of lexical nouns}\label{lex-noun}

\subsection{Lexical nominalizations}\label{lex-noun-1}

Yakkha has three basic nominalizing devices, which will be discussed in more detail in Chapter \ref{ch-nmlz}. The common Tibeto-Burman nominalizers \emph{-pa} and \emph{-ma} are employed in lexical nominalization, deriving nouns that typically refer to types of persons, food, plants, animals and objects of material culture, e.g.,  \emph{khikpa} \rede{roasted feather dish} (literally: be bitter-{\sc nmlz}; see \tabref{table-pa} in Chapter \ref{ch-nmlz} for  more examples).\footnote{This dish consists of  roasted chicken feathers that are mixed with cooked rice.} These markers attach to verbal roots (as far as one can tell since many such nouns are opaque).  Occasionally, the marker can also attach to nominal roots, deriving nouns that are semantically associated with the meaning of the root, such as  \emph{Yakkhaba} \rede{Yakkha man, Yakkha person}. 

As is common among Tibeto-Burman languages, Yakkha does not have a gender system; the nouns are not grouped into classes receiving distinct marking or triggering agreement across the noun phrase or the clause. In lexical nouns referring to persons, \emph{-pa} marks default and male reference, and \emph{-ma} marks female reference. This is particularly prominent in occupational titles (e.g., \emph{thukkhuba/thukkhuma} referring to male and female tailors, respectively) and in kinship terms (e.g., \emph{namba} and \emph{namma} for male and female in-laws, respectively). The marker \emph{-pa} is also the default choice when a group contains members of both sexes, although another frequent option is to use co-compounds in such cases, e.g., \emph{yakkhaba-yakkhamaci} \rede{the Yakkha men and women (\ti  the Yakkha people)}. In the current nominal lexicon (with 930 entries)  there are 47 nouns ending in \emph{-pa} and 120 nouns ending in \emph{-ma},  mostly without being  etymologically transparent, though.

Various zoological and botanical terms have lexicalized the markers \emph{-ma} and \emph{-pa}, so that such nouns invariably take one or the other marker. The lexeme  for mouse is \emph{mima},  for instance,  and the lexeme for \rede{tiger} is  \emph{kiba}, regardless of whether it is a  tiger or a tigress.

There are also 73 nouns that end in \emph{-wa}, a morpheme most probably cognate with \emph{-pa}. These nouns are largely opaque; their roots cannot be determined any more. Examples are \emph{hiʔwa} \rede{wind}, \emph{chiʔwa} \rede{nettle}, \emph{lagwa} \rede{bat}, \emph{takwa} \rede{long needle}, and \emph{lupliwa} \rede{earthquake}. Many of them are, again,  botanical and zoological terms.\footnote{Nouns ending in \emph{wa} can also be related to the lexeme for water or liquid in general, as it is the case in \emph{kiwa} \rede{oil}, see below.}

Some nouns in Yakkha are lexicalized instances of headless relative clauses, e.g.,  \emph{khuncakhuba} \rede{thief} (steal-eat-{\sc nmlz}), \emph{hiŋkhuma} \rede{wife} (support-{\sc nmlz}), and \emph{chemha} \rede{liquor} (be transparent-{\sc nmlz}),  \emph{tumna} \rede{senior} (ripen-{\sc nmlz}), \emph{pakna} \rede{junior} (be raw-{\sc nmlz}). The nominalizers employed in these examples usually result in syntactic nominalizations, since they derive noun phrases, not nouns. They may either link attributive material to a head noun, or  construct headless relative clauses  (see  Chapter \ref{ch-nmlz} for a detailed description and abundant examples). 


\subsection{Compounding}\label{lex-noun-2}

Some kinds of nouns, particularly toponyms and nouns referring to kinship relations, botanical items, and objects of material culture tend to be multimorphemic. The most common pattern found is nominal compounding. Verb-noun compounds are found marginally, but the verbal roots always show some  additional morphological material which can be traced back to nominalizations or infinitives. 

\subsubsection{Co-compounds and sub-compounds}

Both co-compounds (symmetric compounds, \emph{dvandva} compounds) and sub-com\-pounds (hierarchical compounds, \emph{tatpurusha} compounds) can be found in Yakkha.\footnote{The terms \emph{dvandva} and \emph{tatpurusha} come from the Sanskrit grammatical tradition.} In sub-compounds, the first noun modifies the second, e.g., \emph{laŋ-sup} \rede{sock} (literally: foot-sheath). In co-compounds, two conceptually close nouns stand as representatives of a concept or group that is more general than these two nouns, e.g., \emph{pa-pum} for \rede{male ancestor} (literally: father-grandfather). The co-compounds generally refer to kinship relations or other groups of people. \tabref{table-nomcomp1} and \ref{table-nomcomp2} provide more examples of each type.\footnote{In current activities of language promotion, many neologisms are coined by some engaged speakers, like \emph{mitniŋwa} \rede{belief} (literally: think-mind). It cannot be said with certainty which of them will become established in the language. So far, they are only used in written materials. Nevertheless these neologisms show that nominal compounding is a productive strategy to create new lexemes in Yakkha as it is spoken today.} Nepali nouns may also participate in nominal compounding (marked by [\textsc{nep}] in the table).\footnote{The lexeme \emph{macchi} most probably has a Maithili origin: \emph{marchāi} \rede{chili plant}. But it has undergone a substantial semantic shift, meaning \rede{chili plant}, \rede{chili powder}, and \rede{hot sauce or pickles} in Yakkha. In Belhare, its form is \emph{marci} \citep{Bickel1997Dictionary}.} Only sub-compounds combine Nepali roots with Yakkha roots.\footnote{The nouns \emph{muk} and \emph{laŋ} refer to arm/hand  and leg/foot, respectively.}

{\small 
\begin{table}[htp]
\begin{centering}
\begin{tabular}{lll}
\lsptoprule
{\sc Yakkha} & {\sc gloss} & {\sc components} \\
\midrule
\emph{cottu-kektu}&\rede{ancestors}&great-grandfather\\
&&great-great-grandfather\\
\emph{pa-pum}&\rede{male ancestor}&father-grandfather\\
\emph{ma-mum}&\rede{female ancestor}&mother-grandmother\\
\emph{na-nuncha}&\rede{sisters}&elder sister-younger sibling\\
\emph{yakkhaba-yakkhama}&\rede{Yakkha people}&Y. man-Y. woman\\
\lspbottomrule
\end{tabular}\\
\caption{Co-compounds}\label{table-nomcomp1}
\end{centering}
\end{table}
}

{\small
\begin{table}[htp]
\begin{tabular}{lll}
\lsptoprule
{\sc Yakkha} & {\sc gloss} & {\sc components} \\
\midrule
\emph{yaŋchalumba-aphu}&\rede{third-born elder brother}&third-born-eB\\
\emph{laŋ-sup}&\rede{socks}&foot-sheath\\
\emph{laŋ-yok}&\rede{step, footprint}&foot-place \\
\emph{maŋme-muŋ}&(a kind of mushroom)&eagle-mushroom\\
\emph{lupme-muŋ}&(a kind of mushroom)&needle-mushroom\\
\emph{macchi-luŋkhwak}&\rede{mortar, grinding stone}&chili-stone\\
\emph{maksa-khamboʔmaŋ}&\rede{blackberry}&bear-raspberry\\
\emph{laŋ-kheʔwa}&\rede{toe}&leg-finger\\
\emph{laŋ-hup}&\rede{knee}&leg-thickening\\
\emph{lupta-kham}&\rede{landslide}&disperse/bury.\textsc{nmlz}-ground\\
\emph{hamma-tek}&\rede{blanket}&cover/spread.\textsc{inf}-cloth \\
\emph{laŋ-phila}&\rede{thigh}&leg-thigh[\textsc{nep}] \\
\emph{laŋ-tapi}&\rede{sole}&leg-hoof (probably [\textsc{nep}] ) \\
\emph{muk-tapi}&\rede{palm of hand}&arm-hoof (probably [\textsc{nep}] ) \\
\emph{dude-chepi}&\rede{milky onion}&milk[\textsc{nep}](-e)-onion\\
\lspbottomrule
\end{tabular}\\
\caption{Sub-compounds}\label{table-nomcomp2}
\end{table}
}

Co-compounds are common in the languages of the eastern regions of Eurasia. The structural difference between co-compounds and sub-compounds is also reflected in their prosody: while sub-compounds constitute one stress domain, in co-compounds each component carries its own stress.\footnote{Cf. also \citet{Waelchli2005_Co-compounds} on the intermediate position of co-compounds between words and phrases: “There are very few languages where co-compounds are undoubtedly words." \cite[3]{Waelchli2005_Co-compounds}} The components of either type of compound are treated as one phrase morphologically; case and number (both phrasal affixes in Yakkha) attach only once. Example \Next[a] shows a co-compound, \Next[b] shows a sub-compound.  In cases of obligatorily possessed nouns, the possessive prefix attaches to both components of a co-compound, as in \Next[c]. Since most co-compounds are from the domain of kinship, no instances of non-obligatorily possessed nouns with possessive marking in co-compounds  could be found. 


 \ex.\ag. tukkhuba tukkhuma=ci=ga   sewa\\ 
sick\_man sick\_woman{\sc =nsg=gen} service\\
 \rede{service for sick men and women (i.e., medical service)}\source{01\_leg\_07.300}
 \bg.kaniŋ loʔa wempha-babu=ci\\
 {\sc 1pl} like male\_teenager-boy{\sc =nsg}\\
\rede{lads like we (are)}  \source{41\_leg\_09.075}
\bg.  u-ppa u-ma=ci=ca\\
{\sc 3g.poss-}father {\sc 3g.poss-}mother{\sc =nsg=add}\\
\rede{her parents, too} \source{01\_leg\_07.152}

Some sub-compounds appear in a fossilized possessive construction, such as \emph{phak\-kusa} \rede{pork}, literally \rede{pig's meat} or \emph{wagusa} \rede{chicken meat}, literally \rede{chicken's meat}.

 In the rather complex kinship system with frequent instances of obligatory possession (cf. \sectref{inh-poss}), the prefixes marking possession usually attach to the first noun, as in  \emph{a-cya-mamu} \rede{daughter (my child + girl)} and  \emph{a-yem-namma} \rede{father-in-law's elder brother's wife} (my father's elder brother's wife + female in-law). Exceptions are found in the terminology for in-laws on the cousin level, e.g., \emph{khoknima-a-ŋoʈeŋma} \rede{father-in-law's sister’s daughter who is younger than \textsc{ego} (father's sister's younger daughter + my-female-in-law)}.
 
 
\subsubsection{Toponyms}
 
Among the toponyms, oronyms usually end in \emph{luŋ} (\textsc{ptb} *r-luŋ for \rede{stone}, \citealt[50]{Matisoff2003Handbook}). Examples are \emph{Taŋwaluŋ} (Mt. Makalu), \emph{Comluŋ} (Mt. Everest), \emph{Phakʈaŋluŋ}  (shoulder-rock, Mt. Kumbhakarna) or \emph{Namthaluŋma} (locally important rocks, connected to a mythical story).

Another syllable appearing in toponyms is \emph{liŋ}. It  is most probably related to \textsc{ptb} *b-liŋ for  \rede{forest/field} \citep[280]{Matisoff2003Handbook} and occurs in names of Yakkha villages, e.g., \emph{phakliŋ} (pig-field) or \emph{mamliŋ}  (big field), as it does in toponyms of other Tibeto-Burman languages, too.

Tibeto-Burman languages often have locational nominalizers referring to a place connected to some noun, e.g., in Classical Tibetan \citep[300]{Beyer1992_Tibetan}. In Kiranti languages, one finds e.g., \emph{-khom \ti -khop} in Thulung (cognate to Yakkha \emph{kham} \rede{ground}), and \emph{-dɛn} in Limbu (cognate to Yakkha \emph{ten} \rede{village}, \citealt[89]{Ebert1994The-structure}). Yakkha  employs another noun for this strategy, namely \emph{laŋ}, with the lexical meaning \rede{foot}. It is, however, not a nominalizer;  \emph{laŋ} cannot be used to nominalize propositions, as in \rede{the place where he cut the meat}. In compounds, \emph{laŋ} designates the area surrounding an object or characterized by it, as e.g., in \emph{khibulaŋ} \rede{area around walnut tree} or \emph{tonalaŋ} \rede{uphill area}. One also finds lexicalized instances, such as in \Next, or metaphorical extensions, as in \emph{pheksaŋlaŋ} \rede{malicious wizard} (left-foot/left-side). It does not come as surprise that toponyms contain this marker, e.g., \emph{lokphalaŋ} \rede{grove of lokpha bamboo (a huge kind of bamboo)}. However, the number of examples in the existing data do not allow conclusions about the productivity of \emph{laŋ}. 

\exg. maŋcwalaŋ=be khy-a-ŋ\\
water\_tap{\sc =loc} go{\sc -pst-1sg}\\
\rede{I went to the public water tap.} \source{40\_leg\_08.048} 

This compounding strategy has developed from a relational noun construction (see \Next and \sectref{postpos} below). The relational noun \emph{laŋ} locates an object (the {\sc figure}) next to the lower part of another object (the {\sc ground}). 

\exg.siŋ=ga u-laŋ=be\\
tree{\sc =gen} {\sc 3sg.poss-}foot{\sc =loc}\\
\rede{below the tree} (the area around the tree, not right below its roots, and not right next to the stem either)


\subsubsection{Botanical terms and nouns referring to liquids}

Many botanical terms end in \emph{siŋ} for \rede{tree} or in \emph{phuŋ} for \rede{flower}, e.g., \emph{likliŋphuŋ} \rede{mugwort} and \emph{kekpusiŋ} \rede{bull oak}. Above, in \sectref{lex-noun-1}, nouns in \emph{-wa} were discussed as fossilized nominalizations. A homophonouns morpheme with the etymological meaning of \rede{water} is found in 14 lexemes referring to liquids, such as  \emph{cuwa} \rede{beer}, \emph{naŋwa} \rede{glacier} (snow-water), \emph{casakwa} \rede{water in which uncooked rice has been washed} (rice-water), \emph{lithuʔwa} \rede{sperm} and \emph{mikwa} \rede{tear} (eye-water). 

%Even the lexeme \emph{maŋcwa} \rede{water}, is historically complex: its literal meaning is \rede{divine water} or \rede{water of the gods}.

\subsubsection{Lexical diminutives}

Diminutive markers have been reported for various Kiranti languages (see \citet[67]{Doornenbal2009A-grammar} on Bantawa; \citet[95]{Ebert1997A-grammar} on Athpare; \citet[85]{Rutgers1998Yamphu} on Yamphu). Yakkha, too, has a class of nouns ending in a morpheme \emph{-lik \ti -lek} (without any independent meaning) and referring to small things or animals, e.g., \emph{siblik} \rede{bedbug}, \emph{taŋcukulik} \rede{pigtail, tuft of hair}, \emph{yaŋlik} \rede{seed}, \emph{khelek} \rede{ant}, \emph{phokcukulik} \rede{navel}, \emph{moŋgalik} \rede{garden lizard}, \emph{makchiŋgilek} \rede{charcoal} and \emph{poŋgalik} \rede{bud}. This is not a productive derivation process, for two reasons: firstly, independent nouns like \emph{sib} or \emph{yaŋ} do not exist, and secondly, \emph{-lik} it cannot attach to any noun to indicate small size.

Another diminutive-like marker, occuring only with animate nouns, is \emph{cya \ti cyak} \rede{child}, and it is found in terms for young animals in a fossilized possessive construction, e.g., \emph{phakkucyak}  \rede{piglet} (historically: \emph{phak=ka u-cya}) or \emph{wagucya} \rede{chick} (historically: \emph{wa=ga u-cya}).

\subsubsection{Rhyming in compounds}

 Yakkha has a few nominal compounds that are built with rhymes and so-called echo words as they are known in Nepali, where this is quite a productive strategy to express associative plurality (e.g., \emph{biskuʈ-siskuʈ} \rede{cookies and the like}). In Yakkha, there is, for instance, the  name of a mythological bird, \emph{Selele-Phelele}.\footnote{Cf. file 21\_nrr\_04 of the corpus.} Further examples are \emph{kamnibak-chimnibak} \rede{friends} (no independent meaning for \emph{chimnibak} could be  established), \emph{yubak-thiŋgak} \rede{goods, property} (no independent meaning for \emph{thiŋgak} either) or \emph{sidhak-paŋdhak} \rede{traditional, herbal medicine} (\emph{sidhak} refers to medicine in general, \emph{paŋdhak} could have been derived from \emph{paŋ} \rede{house}). Rhyme-based morphology like reduplication  and also triplication is very productive in adjectives and adverbials in Yakkha (see \sectref{redup}).



\subsection{Proper nouns and teknonymy}
 
 Proper nouns identify a unique person, a place or some other entity, such as \emph{Missaŋ} (a female name), \emph{Homboŋ} (the name of a village) or \emph{Kirant Yakkha Chumma} (the name of a social association). They differ from other nouns in that they rarely form compounds, and when marked as nonsingular, they only allow associative interpretations (X and her/his folks, X and the like). 
 
 One subgroup of proper nouns are teknonyms, i.e., names of adults derived from the name of their child, usually their first child. Referring to someone as father or mother of their eldest child  is the respectful way to address or refer to older people, instead of using their names. The more frequent choice is, apparently, the name of the eldest son, but exceptions in favor of the eldest daughter's name are possible. Etymologically, teknonyms are possessive phrases, with the genitive \emph{=ga} and the third person singular possessive prefix \emph{u-} merged into a single syllable [gu], and the head nouns \emph{ma} \rede{mother} and \emph{(p)pa} \rede{father} (with geminated /p/ because of  the possessive prefix).\footnote{The nasal in the noun \emph{ma}, in contrast, does not untergo gemination. The geminated \emph{umma} that was offered by me in an elicitation earned the comment that this sounded like Limbu, not Yakkha.} The resulting word constitutes a single stress domain, with the first syllable carrying main stress. In case the child's name does not end in a vowel, an epenthetic element \emph{-e} is inserted. Examples are provided in \Next. 

\ex. \ag.Ram-e-guppa\\
		 Ram{\sc -epen-tek.gen.m}\\
	\rede{Father of Ram}
 	\bg.Bal-e-guma\\
		Bal{\sc -epen-tek.gen.f}\\
	\rede{Mother of Bal}
	
\subsection{The count/mass distinction}\label{lex-noun-4}

Mass nouns in Yakkha usually allow both readings, either referring to a concept as such, or to a unit or bounded quantity of that concept. Hence, the same lexeme may occur in different syntactic contexts without any morphological change or the addition of some classifying element. The verbal person marking, however, distinguishes the feature \rede{mass} from both singular and nonsingular. Mass nouns trigger the marker \emph{=ha} on the verb (which is also found with nonsingular number). But with regard to all other verbal markers, the mass nouns trigger singular morphology. Neither  the nonsingular marker \emph{-ci} nor the singular clitic \emph{=na} are possible on the verb when the nouns have a mass interpretation. 

Compare the two uses of the words \emph{yaŋ}  \rede{money, coin} and \emph{chem} \rede{music, song} in \Next and \NNext. In the (a) examples, these nouns have countable reference, as is evident from the presence of numerals and from the fact that they trigger number agreement on the verb (nonsingular \emph{-ci} in \Next[a] and singular \emph{=na} in \NNext[a]). In the (b) examples, the nouns have mass reference, and hence do not take the nonsingular marker \emph{=ci}.  In fact, adding \emph{=ci} would change the interpretation to nonsingular. The quantifier \emph{pyak} in \Next[b] is of no help in determining semantic or structural differences, as it may have both a mass reading \rede{much} and a nonsingular reading \rede{many}.

\ex.\ag.hic=ci yaŋ=ci     n-yuks-wa-ci=hoŋ, \\
two{\sc =nsg} coin{\sc =nsg} {\sc 3pl.A-}put\_down{\sc -npst-nsg.P=seq}\\
\rede{After they will put down two coins, ...}  \source{26\_tra\_02.032 }
\bg.pyak yaŋ ub-w=ha\\
much money earn{\sc -npst[3sg.A>3.P]=nmlz.nc}\\
\rede{She earns a lot of money.} 


\ex.\ag.ka chem chept-wa-ŋ=na\\
{\sc 1sg[erg]} song write{\sc -npst[3.P]-1sg.A=nmlz.sg}\\
\rede{I will write a song.}
\bg.chem(*=ci) end-u-g=ha=i?\\
music(*=nsg) apply{\sc -3.P[pst]-2.A=nmlz.nc=q}\\
\rede{Did you turn on the music?} \\
(It is clear from the context that the speaker did not refer to a plurality of songs, but to the sound coming out of the radio.)

As stated above, Yakkha does not have to add classifiers to distinguish between mass and  count reference. There are, however, two markers that may convey this distinction, namely the nominalizers \emph{=na} and \emph{=ha} in attributivizing function (etymologically related to the verb-final markers shown  in \Last). In \Next, while \emph{=na} implies a bounded quantity, \emph{=ha} implies mass reference. This distinction is parallel to the distinction in the demonstratives discussed in \sectref{dem-pron}.

\ex.\ag.to=na cuwa\\
uphill{\sc =nmlz.sg} beer\\
\rede{the (bowl of) beer standing uphill}
\bg.to=ha cuwa\\
uphill{\sc =nmlz.nsg} beer\\
\rede{the beer uphill (i.e., the beer of the uphill households)} 

 A non-exhaustive list of nouns that allow both count and mass reference is provided in \tabref{countmass}.
 
\begin{table}[htp]
\begin{centering}
\begin{tabular}{ll}
\lsptoprule
{\sc yakkha} & {\sc gloss}  \\
\midrule
\emph{cama}& \rede{(portion of) cooked rice}\\
\emph{ceʔya}& \rede{matter, language, word} \\
\emph{chem}& \rede{music, song} \\
\emph{chemha}& \rede{(glass of) liquor} \\
\emph{cuwa}& \rede{(glass/bowl of) beer} \\
\emph{kham}& \rede{ground, mud, (plot of) farm land} \\
\emph{khyu}& \rede{(portion of) cooked meat or vegetables} \\
\emph{maŋcwa}& \rede{(container with) water} \\
\emph{sa}& \rede{(portion of) meat}\\
\emph{yaŋ}& \rede{money, coin} \\
\emph{siŋ}& \rede{wood, tree} \\
\emph{tamphwak}& \rede{hair} \\
\lspbottomrule
\end{tabular}
\caption{Nouns with both count and mass reference}\label{countmass}
\end{centering}
\end{table}

\subsection{Inherent duality}\label{lex-noun-5}

Nouns that typically denote pairs, like legs, eyes, buttocks (but not inner organs like lungs and kidneys), usually occur with the nonsingular marker \emph{=ci}. With regard to verbal agreement, they trigger plural instead of the expected dual marking. Apparently there is no need to maintain the plural/dual distinction with referents typically occurring in sets of two (see \Next). 

\exg.a-tokcali=ci n-dug=ha=ci (*tugaciha)\\
{\sc 1sg.poss-}buttock{\sc =nsg} {\sc 3pl-}hurt{\sc -nmlz.nsg=nsg}\\*
\rede{My bottom hurts.}
 
\section{Nominal  inflectional morphology}\label{nom-morph}

Nominal inflectional categories in Yakkha are (i) number, (ii) case and (iii) possession.\footnote{\emph{Inflectional} in the sense of \rede{regularly responsive to the grammatical environment} \citep{Bickeletal2007Inflectional}.} Number and case are generally encoded by clitics (phrasal suffixes). They do not trigger agreement across the noun phrase. The case markers may also attach to nominalized phrases or to anything else in nominal function (see \sectref{nmlz-uni} for examples). The only case that may appear phrase-internally is the comitative case, coordinating two nominal heads to form a noun phrase. Since case and number markers operate on the phrasal level, the third category, possessor agreement, is the only category that applies exclusively to lexical nouns. It is encoded by prefixes attaching directly to nouns (discussed together with the pronouns in \sectref{poss-pron}). 

Further markers (particles) are  possible on noun phrases, but since they pertain to information structure, the reader is referred to Chapter \ref{particles} for their discussion.

\subsection{Number}\label{number}

Yakkha distinguishes singular, dual and plural in the verbal domain and in pronouns, but only singular and nonsingular in nouns. Singular number is unmarked. The nonsingular marker is the phrasal suffix \emph{=ci}, denoting that there are multiple instances of the item in question, or that the item/person in question is accompanied by similar items/person (associative plurality). It attaches to the rightmost element of the noun phrase (usually the nominal head), and thus has scope over the whole noun phrase. The marker does not appear inside the noun phrase, with the exception of numerals (see \sectref{sec-num}). Case markers follow the number marker (see \Next). 

\ex.\ag.kucuma\\
dog\\
\rede{a/the dog}
\bg. ghak kucuma=ci=be\\
all dog{\sc =nsg=loc}\\
\rede{at/to all the dogs}

The status of \emph{=ci} as a phrasal clitic is clearly confirmed when looking at headless noun phrases or noun phrases where the order of head and modifier is reversed for reasons  of information structure. The nonsingular marker may follow a genitive marker (see \Next[a]) or (syntactic) nominalizers (see \Next[b]), devices that would link modifying material to a head noun if there was one. In \Next[c], attributive material follows the head noun, and since it is the rightmost element, the nonsingular marker attaches to it.

\ex.\ag. heko=na         paʈi=ga=ci\\
other{\sc =nmlz.sg} side{\sc =gen=nsg}\\
\rede{those (children) from the other side (i.e., the other wife)} \source{06\_cvs\_01.054} 
\bg. hau,  kha=go,      eŋ=ga              yapmi  loʔa=ha=ci=ca\\
  {\sc excla} these{\sc =top} {\sc 1pl.incl.poss=gen} person like{\sc =nmlz.nsg=nsg=add}\\
  \rede{Oh, these guys, they are like our people, too.} \source{22\_nrr\_05.044} 
 \bg.pahuna ta-khuba=ci\\
guest come{\sc -nmlz=nsg}\\
 \rede{the guests who are coming} \source{25\_tra\_01.063}
 
\subsubsection{Omission of nonsingular \emph{=ci}}\label{number-1}

Number marking on nouns is not obligatory. With non-human  reference it is frequently omitted.  In \Next[a], it is clear from the context, from the demonstrative \emph{ŋkha} and from the verbal agreement that \emph{luŋkhwak} refers to more than one stone. With human referents, number marking cannot be omitted so easily (see \Next[b]). Another factor interacts with animacy/humanness here, namely generic vs. specific reference. In \Next[c], there is nonsingular human reference, but in a generic sense, referring to abstract classifications of people (those with whom one is/is not allowed to eat, in accordance with Hindu social law).\footnote{The Yakkha belong to the Kiranti cultural sphere, but the past centuries of Hindu dominance have left their mark on the social organization of many Tibeto-Burman groups in Nepal.} Here, the number marking can be omitted, in contrast to  (b) where the noun refers to a specific group of people, namely the speaker's friends. With specific human reference, nonsingular marking was omitted only in songs, a genre which is expected to show deviations from spoken language, due to other constraints like rhythm and rhyming.


\ex.\ag.ŋkha mamu=ci=ŋa ŋkha luŋkhwak n-leks-u-ci=ha=bu\\
 those girl{\sc =nsg=erg} those stone {\sc 3pl.A-}turn\_over{\sc -3.P[pst]-3nsg.P=nmlz.nsg=rep}\\
\rede{Those girls have turned around those rocks, it is said.}  \source{37\_nrr\_07.118}
\bg.*a-kamnibak chimd-u-ŋ-ci-ŋ=ha\\
{\sc 1sg.poss-}friend ask{\sc -3.P[pst]-1sg.A-nsg.P-1sg.A=nmlz.nsg}\\
Intended: \rede{I asked my friends.} 
\bg. ca-m=ha  yapmi  men-ja-m=ha yapmi,  kha imin=ha=ci?\\
	eat-{\sc inf[deont]=nmlz.nsg} people {\sc neg}-eat-{\sc inf[deont]=nmlz.nsg} people these how{\sc =nmlz.nsg=nsg} 		\\
	\rede{(Are they) people with whom we should eat, or with whom we should not eat, of what kind (are they)?} \source{22\_nrr\_05.040}
	
	
Number marking can also be omitted when a numeral is present in the noun phrase (see \Next[a] and \Next[b]). However, instances with overt nonsingular marking, as in  example \Next[c], are far more frequent.

\ex.\ag.  hip-paŋ     babu\\
two{\sc -clf.hum} boy\\
\rede{two boys}
\bg. hip-paŋ     paghyam-maghyam\\
two{\sc -clf.hum} old\_man-old\_woman\\
\rede{an old couple} \source{01\_leg\_07.280}
\bg. sum-baŋ       phak-khuba      yapmi=ci\\
three{\sc -clf.hum} help{\sc -nmlz} person{\sc =nsg}\\
\rede{three servants}  \source{04\_leg\_03.015}
  

\subsubsection{Associative interpretations of nonsingular marking}\label{number-2}

Nonsingular marking can be interpreted associatively, referring to people who can be associated to the respective noun (see \Next[a] and \Next[b]), a feature that is also found in other languages spoken in this area, e.g., in Newari \cite[98]{Genetti2007_Newari} and in Nepali (own observations). Occasionally, objects with nonsingular marking can also be found with an associative interpretation (see \Next[c]), but this is rare at least in the current corpus; one rather finds enumerations of various objects than associative plural marking if a plurality of items is given.
                                      

\ex. \ag. a-koŋma=ci=nuŋ=le wɛʔ=na?\\
		{\sc 1sg.poss-}MyZ{\sc =nsg=com=ctr} exist{\sc [3sg]=nmlz.sg}\\
		\rede{Oh, she lives with my aunt and her people?}  \source{06\_cvs\_01.074}
	\bg. Lila didi=ci\\
		Lila elder\_sister{\sc =nsg}\\
		\rede{Sister Lila and her family}  \source{13\_cvs\_02.059}
		\bg.  i=ha i=ha    yuncamakekek ceʔya chumma=ci     n-leks-a\\
		what{\sc =nmlz.nc} what{\sc =nmlz.nc} funny        matter assembly{\sc =nsg} {\sc 3pl-}happen{\sc -pst}\\
		\rede{Various funny incidents, meetings and the like occurred there.} \source{41\_leg\_09.008}
	
\subsection{Core case markers (Group I)}\label{case}

Case, in the classical sense, is understood as the morphological marking on a noun or a noun phrase that indicates its syntactic relatedness either to a predicate (arguments or circumstantial participants)  or to  another noun (in the case of the genitive and the comitative). Yakkha distinguishes  case clitics  that operate on the noun phrase level, marking verbal arguments  (Group I, discussed in this section), and markers that are functionally more flexible, and also less dependent phonologically (Group II, discussed in \sectref{postpos}).


Case marking (ergative, genitive, comitative, equative) may also appear on dependent clauses that are often, but not necessarily, nominalized, as will be shown below and in Chapter \ref{adv-cl} on adverbial clause linkage as well as in Chapter \ref{compl} on complementation. The parallelism between case markers and clause linkage markers is well-known in Kiranti and Tibeto-Burman in general \citep{Genetti1986The-development, DeLancey1985_Etymological, Ebert1993Kiranti}.\footnote{It is, however, not clear yet whether there  was a historical development from nominal case markers to  clause linkage markers, or whether this parallelism is original to the system.} 

Group I distinguishes seven cases, as shown in \tabref{case-markers}. Case, like number, is marked by enclitics in Yakkha, except for the nominative, which is the functionally and morphologically unmarked case in Yakkha. Since the case suffixes operate on the phrasal level, they attach to the rightmost element of the noun phrase. The case markers that start in a plosive have voiced allomorphs intervocalically and after nasals.


\begin{table}[htp]
\begin{centering}
\begin{tabular}{llp{8cm}}
\lsptoprule
{\sc case}&{\sc marker}&{\sc function}\\
\midrule
nominative& Ø&intransitive subject, transitive patient, ditransitive theme and goal, citation form, location (restricted use), copular topic and predicate\\
ergative&\emph{=ŋa}&transitive subject\\
instrumental&\emph{=ŋa}&instrument, ditransitive theme, temporal reference\\
genitive&\emph{=ka} &possession, material\\
locative&\emph{=pe} &location, ditransitive recipients and  goals, temporal reference\\
ablative&\emph{=phaŋ} &source arguments\\
comitative&\emph{=nuŋ} &coordination, associated referents, source arguments of some verbs\\
\lspbottomrule
\end{tabular} 
\caption{Case markers (Group I)}\label{case-markers}
\end{centering}
\end{table}


We know from other Kiranti languages that case markers can be stacked to yield more specific functions (e.g., \citealt[81]{Ebert1994The-structure}; \citealt[6]{Dirksmeyer2008Spatial}; \citealt[26]{Schikowski2013_Thesis}). Generally, composite case markers are common in Tibeto-Burman languages \citep[60]{DeLancey1985_Etymological}. In Yakkha, the locative or the ablative case marker can be added to the genitive of a proper noun to yield the meaning \rede{at/from X's place}. The ablative is also historically complex (see \sectref{case-abl} below).

Several Kiranti languages have a (generally optional) dative marker \emph{-lai} (e.g., Bantawa \citep{Doornenbal2009A-grammar}, Puma \citep{Bickeletal2007Two-ways}, Camling, Athpare and Thulung (\citealt{Ebert1994The-structure}), which is homonymous with the Nepali dative marker \emph{–lāī} and probably a loan. Yakkha, however, does not employ this marker. It uses other strategies to mark semantic roles typically associated with  dative marking: recipients and goals are either in the nominative or in the locative, and experiencers appear in various frames of argument realization, most prominently the Experiencer-as-Possessor frame.
In the following, the cases of Group I and their functions will be introduced. More detailed information on argument realization and transitivity is found in Chapter \ref{verb-val}.

\subsubsection{The nominative (unmarked)}\label{case-nom}

The nominative is  morphologically and functionally the unmarked case in Yakkha.\footnote{Functional unmarkedness does not imply morphological unmarkedness, as research on marked-S languages has shown \citep{Handschuh2011_thesis, Brown2001_Nias}. In the Yakkha case system, morphological and functional unmarkedness coincide.} Participants in the nominative appear in their citation form, without any further marking. Intransitive subjects (S), transitive patients (P), ditransitive theme  (T) and goal arguments  (G), topic and comment of copular clauses, and to a certain extent locations, too, can be in the nominative and thus unmarked in Yakkha.\footnote{With the discovery of ergativity, the term \rede{absolutive} came into use relatively recently to refer to the case of intransitive subjects and transitive objects when these have the same case (see \citet{McGregor2009_Ergativity} and \citet{Haspelmath2009_Terminology} for summaries of the historical gestation of the term \rede{ergative}). Since then, research on ergativity has revealed that the system is far from uniform, and optional in many languages, other factors such as reference and information structure playing a greater role than had been expected. Haspelmath mentions the problem that the terminology nominative-accusative-ergative-absolutive refers to an ideal system which is rarely found \citep[513]{Haspelmath2009_Terminology}. Both nominative and absolutive refer to the functionally unmarked case in a system, and their application usually extends well beyond marking S and P arguments. Therefore, I do not see the need to maintain the distinction between the terms \rede{nominative} and \rede{absolutive}, since the unmarked case in an ergative system and the unmarked case in an accusative system have probably more shared properties than properties distinguishing them. Since \rede{nominative} is the older term, it will be used in this work.} Example  \Next shows S, P, T and G arguments in the nominative.\footnote{To keep the glosses as short and straightforward as possible, the nominative is generally not glossed.} 


\ex. \ag. ka maŋcwa=be khe-me-ŋ=na\\
			{\sc 1sg} water{\sc =loc} go{\sc -npst-1sg=nmlz.sg}\\
		\rede{I go to fetch water.}
	\bg. nasa=ci  ŋ-und-wa-ci\\
		fish{\sc =nsg} {\sc 3pl.A-}pull\_out{\sc -npst-3nsg.P}\\
		\rede{(They) pull out the fish.}
	 \bg. ka nda cakleʈ pi-meʔ-nen=na\\
		{\sc  1sg[erg]} {\sc 2sg} sweet give{\sc -npst-1>2=nmlz.sg}\\
		\rede{I will give you a sweet.}
			

Yakkha shows a typologically common nominative/ergative syncretism: transitive subjects that are represented by a first or second person pronoun always appear unmarked (cf. \sectref{case-erg}).
	
Furthermore, both topic and comment in identificational copular constructions (see \Next), and  the figure in existential/locative copular constructions (see \Next[c]) are in the nominative.
	
\ex. \ag. na ak=ka paŋ (om)\\
		this {\sc 1sg.poss=gen} house ({\sc cop})\\
	\rede{This is my house.} 
		\bg.   ka=go      arsale          leʔlo!\\
	{\sc 1sg=top} person\_from\_year\_eight {\sc ctr.excla}		\\
	\rede{I was born in the year eight (B.S.), man!} \source{06\_cvs\_01.027}	
	\bg.nnakha=e    maŋcwa=ca        m-ma-ya-n\\
	that{\sc =loc}  water{\sc =add} {\sc neg-}be{\sc -pst[3sg]-neg}\\
	\rede{There was no water, too.}\source{42\_leg\_10.009}
	
Nominative arguments are also found in motion verb constructions, where a locative would be expected on the goal of the movement (see \Next). This option exists only for typical and frequent goals of movement, such as villages, work places, a school, a weekly market etc. The respective nouns are never modified (see \NNext[a], which was elicited in analogy to a sentence from the corpus, and which is well-formed only with a locative). Complements of verbs stating existence or location (\rede{be at X}) can generally not occur unmarked, but exceptions in the colloquial register are possible (see \NNext[b]). The nouns in the nominative thus share features with incorporated nouns, although on other grounds they are not incorporated. Since the nouns mostly refer to names of places or landmarks, they refer to highly individuated participants, while incorporated nouns are often rather generic. 

	
 	\ex.\ag. Poklabuŋ tas-a-ma-c-u=hoŋ, \\
	Poklabung{\sc [loc]} arrive{\sc -pst-prf-du-3.P=seq} 	\\
	\rede{When they arrived in Poklabung, ...} \source{22\_nrr\_05.017}	
\bg.ka  ʈhuŋkha           khy-a-ŋ=niŋ,\\
{\sc 1sg} steep\_slope{\sc [loc]} go{\sc -pst-1sg=ctmp}\\
\rede{When I was heading to the steep slopes, ...}\source{40\_leg\_08.036}

	\ex.\ag. uŋci=ga    ten*(=be)        khy-a-ma-ci,\\
	{\sc 3nsg=gen} village{\sc *(=loc)}  go{\sc -pst-prf-du}		\\
	\rede{They went to their village, ...} \source{22\_nrr\_05.037}	
	\bg.tumok waiʔ-ŋa=na\\
Tumok{\sc [loc]} be{\sc [npst]-1sg=nmlz.sg}\\
\rede{I am in Tumok.} (said on the phone)	
		
\subsubsection{The ergative \emph{=ŋa}}\label{case-erg}

Transitive and ditransitive A arguments are marked by the ergative \emph{=ŋa} (see \Next), except when they are first or second person pronouns, which display an ergative/nominative syncretism (see \NNext). 

	\ex. \ag. na,   jaba,   na   mamu=ŋa   luŋkhwak pok-ma        n-yas-u-n, \\
	this when this girl{\sc =erg} stone raise{\sc -inf}  {\sc neg-}be\_able{\sc -3.P[pst]-neg}\\
	\rede{This one, when this girl could not raise the stone, ...}\source{37\_nrr\_07.039}
	\bg.ka  a-ma=ŋa                khaʔla   ly-a-ŋ: \\
	{\sc  1sg} {\sc 1sg.poss-}mother{\sc =erg} like\_this tell{\sc [3sg.A]-pst-1sg.P}\\
	\rede{Mother told me the following: ...}\source{42\_leg\_10.011}

	
	\ex.\ag.jeppa nna  len ka       a-ma=nuŋ                a-na=ga     ceʔya y-yen-u-ŋa-n=na=ŋa, \\
	really that day {\sc 1sg[erg]} {\sc 1sg.poss-}mother{\sc =com} {\sc 1sg.poss-}eZ{\sc =gen} matter {\sc neg-}obey{\sc -3.P[pst]-1sg.A-neg=nmlz.sg=erg.cl}\\ 
\rede{Really, that day, because I did not listen to my mother's and my elder sister's warnings, ...}\source{42\_leg\_10.051}
\bg. iya nniŋda, eh,    njiŋda  yoŋ-me-c-u-ga,\\
what {\sc 2pl[erg]} oh {\sc 2du[erg]} search{\sc -npst-du-3.P-2.A}\\
\rede{Whatever you (dual) look for, ...}\footnote{The speaker is correcting himself from plural to dual pronoun.} \source{22\_nrr\_05.084}
	
	
In Yakkha, first or second person reference can also be instantiated by full nouns instead of pronouns, which is unusual from the perspective of Indo-European languages. One may have a sentence with first or second person verbal person marking, but the structural position of the pronoun is occupied by a noun, as shown in \Next.\footnote{Flexible agreement is discussed in \sectref{flex-agr}. On the principles behind agreement in Tibeto-Burman see  \citet{Bickel2000On-the-syntax}.} In such participant configurations, there is overt ergative marking on the noun. To make a long story short, the differential agent marking is mainly determined  by word class, but also by reference.
	
	\ex. \ag.phu=na mamu=ŋa yakkha ceʔya nis-wa-g=hoŋ maŋ-di-me-ŋ=naǃ\\
	white{\sc =nmlz.sg} girl{\sc =erg} Yakkha language know{\sc -npst-2=seq}  be\_surprised{\sc -V2.give-npst-1sg=nmlz.sg}\\
	\rede{I am surprised since you, a white girl, know Yakkhaǃ} 
	\bg. a-phaŋ=ŋa men=na, a-koŋma=ŋa=le   ta-ga=na           raecha\\
		{\sc 1sg.poss-}MyZH{\sc =erg}  {\sc  neg.cop[3]=nmlz.sg} {\sc 1sg.poss-}MyZ{\sc =erg=ctr} bring{\sc [pst;3.P]-2.A=nmlz.sg} {\sc mir}\\
		\rede{Not the uncle, but you, auntie, really brought her here (the second wife)!} \source{06\_cvs\_01.042}

The examples in  \Next show that the ergative marker attaches to the final element of the phrase,  whether two nouns are conjoined by a comitative (see \Next[a] and (b)) or whether the final element is a participle, as in \Next[c].\footnote{The comitative marker may function as a coordinator, much like English \rede{and}. The verbal person marking is triggered by the collective number features of both nouns (dual in (a), and nonsingular in (b)). The negated form \emph{yyogan} is found in all scenarios with third person acting on first, except for {\sc 3sg>1sg}.}
	
	\ex. \ag.lalubaŋ=nuŋ   phalubaŋ=ŋa   mamliŋ   tas-a-ma-c-u\\
	Lalubang{\sc =com} Phalubang{\sc =erg} Mamling arrive{\sc -pst-prf-du-3.P}\\
	\rede{Lalubang and Phalubang arrived in Mamling.}\source{22\_nrr\_05.041}
	\bg.a-ma=nuŋ  a-na=ŋa  y-yog-a-n=niŋ=bi, \\
	{\sc 1sg.poss-}mother{\sc =com} {\sc 1sg.poss-}sister{\sc =erg} {\sc neg-}search{\sc [3A;1.P]-sbjv-neg=ctmp=irr}\\
	\rede{If my mother and sister had not searched for me, ...}\source{42\_leg\_10.052}
	\bg.   beuli=ga=ca  u-nuncha parne=ŋa    chata    ham-met-wa\\
	bride{\sc =gen=add} {\sc 3sg.poss-}younger\_sibling  falling{\sc =erg} umbrella spread{\sc -caus-npst[3A;3.P]}\\
	\rede{Someone who is a younger sister of the bride, too, spreads an umbrella over her.}\source{25\_tra\_01.053}

	
For several Tibeto-Burman languages, ergative marking has been described as \rede{optional} and depending on pragmatic factors  (see e.g., \citet{LaPolla1995_Ergative} for a comparative account; \citet{Tournadre1991_Rhetorical} on Lhasa Tibetan; \citet{Coupe2007_Mongsen} on Mongsen Ao; \citet{Hyslop2011_Kurtop} on Kurtöp). Yakkha, however, has a strictly grammaticalized system of ergative marking; the ergative is obligatory on A arguments (under the above-mentioned conditions), which is in line with the findings on other Kiranti languages. \citet[74]{Doornenbal2009A-grammar} notes  the same for Bantawa.  \citet[549]{Bickel2003Belhare} mentions an alignment split in Belhare that leaves first person singular pronouns unmarked.\footnote{Also non-Kiranti languages like Newari, Chepang and Kham have \rede{stable} grammaticalized ergative marking \citep{LaPolla1995_Ergative}, while this is not as clear for Classical Tibetan \citep{DeLancey2011_Optional}.} The differential marking found on first and second person pronouns in Yakkha is determined by reference and word class, not by pragmatics. 

On a final note, the ergative marker is also employed in adverbial clause linkage (see Chapter \ref{adv-cl}).


\subsubsection{The instrumental \emph{=ŋ(a)}}\label{case-ins}
Yakkha exhibits an ergative-instrumental syncretism, which is not unusual, especially not in Kiranti. By formal criteria, except for one exception discussed below, the two cases cannot be distinguished.  Functionally, though, they are distinct: the ergative marks animate agent arguments, while the instrumental  typically marks inanimate participants like  instruments \Next[a], effectors, forces and causes \Next[b]. 

	\ex. \ag. chom=na phiswak=ŋa hot-haks-u=na\\
	pointed{\sc =nmlz.sg} knife{\sc =ins} pierce-{\sc V2.send-3.P[pst]=nmlz.sg}\\
	\rede{He pierced it with a pointed knife.}
	\bg. kisiʔma=ŋa solop miyaŋ eg-haks-uks-u\\
		fear{\sc =ins} immediately a\_little  break{\sc-V2.send-prf-3.P[pst] }\\
		\rede{Out of fear, he immediately broke off a little (from the stick).} \source{04\_leg\_03.023}
		
		
The medium for communication is also marked by the instrumental \Next. In this usage, an allomorph \emph{=ŋ} is possible.\footnote{Note the employment of exclusive vs. inclusive morphology in example (a). The speaker narrates the event from the perspective of the person who made the deontic statement, thus choosing the exclusive pronoun, despite the fact that the person she addresses is included. This shows that clusivity in Yakkha is not necessarily determined by including or excluding the addressee, but also by other people present in the speech situation.} In other Eastern Kiranti languages like Belhare, Chintang or Limbu, this function is taken over by a mediative/perlative marker \emph{-lam} (\citealt[549]{Bickel2003Belhare}; \citealt[83]{Schikowski2012_Morphology}; \citealt[51]{Driem1987A-grammar}). A perlative case is not attested in Yakkha, at least not in the variety spoken in Tumok.

\ex.\ag. aniŋ=ga ceʔya=ŋ=bu chem lum-biʔ-ma=na=lai\\
{\sc 1pl.excl.poss=gen} language{\sc =ins=rep} song tell{\sc -V2.give-inf[deont]=nmlz.sg=excla}\\
\rede{She says we have to sing a song in our language.} (reporting on the deontic statement of a person not included in the group) \source{06\_cvs\_01.102}
\bg. eŋ=ga            ceʔya=ŋ       sarab pi-ci=ha leks-a\\
		{\sc 1pl.incl.poss=gen} language{\sc =ins} curse give{\sc -3nsg.P[3A;pst]=nmlz.nsg} become{\sc [3sg]-pst}\\
		\rede{It happened that it (the sun) cursed them (the Linkha clan members) in our language.}\source{11\_nrr\_01.031}
		
		
The instrumental also indicates temporal reference (see \Next). On a side note,  it is very likely that the adverbial clause linkage markers  \emph{-saŋ} and \emph{=niŋ(a)} (both marking cotemporality)  are based on the ergative/instrumental case etymologically.
		
		\ex. \ag.wandik=ŋa ta-meʔ=na\\
		next\_day{\sc =ins} come{\sc [3sg]-npst=nmlz.sg}\\
		\rede{He will come tomorrow.}
		\bg. khiŋbelaʔ=ŋa\\
		this\_time{\sc =ins}\\
		\rede{at this time}
	
	
\subsubsection{The genitive \emph{=ka} }\label{case-gen}		
The genitive case is marked by the suffix \emph{=ka} (mostly realized as [ga] as  result of the voicing rule, see \sectref{voicing}). It is used for possessive constructions, linking a possessor to a head noun (see \Next). As mentioned in \sectref{poss} on possessive pronouns, the possessee may be inflected by a possessive prefix, as in \Next[b] and \Next[c]. The possessive inflection may occur in addition to a genitive-marked possessor, or may replace it, as in \Next[c].

		\ex.\ag. limbukhim=ci=ga       taŋme\\
		a\_clan{\sc =nsg=gen} daughter-in-law\\
\rede{a daughter-in-law of the Limbukhims} \source{37\_nrr\_07.002}
	\bg. isa=ga u-chya?\\
	who{\sc =gen} {\sc 3sg.poss-}child\\
		\rede{Whose child (is it)?}
		\bg. m-ba m-ma=ci\\
		{\sc 2sg.poss-}father {\sc 2sg.poss-}mother{\sc =nsg}	\\
		\rede{your parents}
	
The head noun can also be omitted. The  structure shown in \Next[a] is similar to a headless relative clause. Genitive-marked attributes may also  be linked recursively to a head noun (see \Next[b]).\footnote{The example also shows that, at least in spoken language, discontinuous phrases are possible, since the adverb \emph{uhile} belongs to the verb, but occurs inside the noun phrase.}



\ex.\ag. heko=na    patti=ga=ci\\
		other{\sc =mmlz.sg} side{\sc =gen=nsg}	\\
	\rede{those (children) from the other one (i.e., the other wife)} \source{06\_cvs\_01.033}
	\bg. aniŋ=ga liŋkha=ga uhile utpati mamliŋ=be      leks-a=na=bu\\
		{\sc 1sg.excl.poss=gen} a\_clan{\sc =gen} long\_ago origin Mamling{\sc =loc} happen{\sc [3sg]-pst=nmlz.sg=rep}\\
		\rede{Our Linkha clan originated long ago in Mamling, they say.}  \source{11\_nrr\_01.002}

Relational nouns functioning as spatial adpositions also require the genitive, illustrated by \Next . They are used in a possessive construction to which a locative must be added (see \Next[b]; cf. also \sectref{postpos}). 
	
\ex. \ag.ʈebul=ga mopparik\\
			table{\sc =gen} under\\
		\rede{under the table}
	\bg. saptakosi=ga u-lap=pe\\
		a\_river\_confluence{\sc =gen} {\sc 3sg.poss-}wing{\sc =loc}\\
	 \rede{on the shores of the Saptakosi}  \source{37\_nrr\_07.044}

The genitive is also employed to mark nominal modifiers referring to the material which the head noun is made of, as shown in \Next.

\ex. \ag. kolenluŋ=ga   cuʔlumphi\\
	marble{\sc =gen} stele\\
	\rede{a/the stele made of marble} \source{18\_nrr\_03.001}
 	\bg. siŋ=ga    saŋgoŋ\\
	wood{\sc =gen} stool	\\
	\rede{a/the wooden stool} 
\bg. plasʈik=ka    jhola=be\\
plastic{\sc =gen} bag{\sc =loc}\\
\rede{in a  plastic bag}	 \source{13\_cvs\_02.045}
	  \bg.chubuk=ka caleppa\\
	  ashes{\sc =gen} bread\\
	  \rede{bread of ashes}\footnote{A punishment for children: smearing ashes on their cheeks and slapping them.}   \source{40\_leg\_08.056}


\subsubsection{The locative \emph{=pe}}\label{case-loc}

Yakkha has only one locative case marker \emph{=pe} ([be] when voicing applies; it can be further reduced to [we] or simple [e]). Kiranti languages typically exhibit a four-fold distinction of deictic locative case markers that respond to the hilly topography of the environment.\footnote{E.g., Camling, Bantawa, Puma, Thulung, Khaling \citep{Ebert1994The-structure}; Yamphu \citep[72]{Rutgers1998Yamphu}; Belhare \citep[226]{Bickel2001Deictic}.} Such a case system consists of (i) one generic locative and three further markers to locate items (ii) above, (iii) below or (iv) on the same level as the deictic origin.\footnote{Termed \rede{vertical case} in \citet[94]{Ebert1994The-structure}; \rede{altitudinal case} in \citet[62]{Dirksmeyer2008Spatial}.} While other Eastern Kiranti languages such as Limbu and Athpare also lack those altitudinal cases (\citealt[118]{Ebert1997A-grammar}, \citealt[49]{Driem1987A-grammar}), Belhare, seemingly the closest relative of Yakkha, displays them \citep[226]{Bickel2001Deictic}. 
The locative marks the spatial coincidence of an entity defined as {\sc figure} with an environment or landmark defined as {\sc ground}  \citep[3]{Levinsonetal2006_Grammars}.  It has a very general meaning, covering relations of containment, proximity and contact, translatable as \rede{in}, \rede{at} and \rede{on}. Examples are provided in \Next. 

\ex. \ag.khorek=pe cuwa\\
		bowl{\sc =loc} beer	\\
	\rede{There is beer in the bowl.}
	 \bg.nwak=ka o-hop=pe\\
		bird{\sc =gen} {\sc 3sg.poss}-nest{\sc =loc}	\\
	\rede{in the nest of the bird}
		\bg.o-thok=pe toŋ-meʔ=na\\
		{\sc 3sg.poss}-body{\sc =loc} fit{\sc [3sg]-npst=nmlz.sg}\\
	\rede{It suits/fits on her body.}
	
The basic locative construction \cite[15]{Levinsonetal2006_Grammars}, the answer to the question \rede{Where is F?} is a copular construction with \emph{wama} (with the suppletive nonpast stems \emph{waiʔ}, \emph{wɛʔ}, \emph{wei}) \rede{be, exist} (see \Next[a]). The same construction (with different information structure) is generally used to introduce topics in the beginning of narratives (see \Next[b] and \Next[c]).


\ex.\ag.wa=ci kaŋyoŋ=be ŋ-waiʔ=ya=ci\\
chicken{\sc =ngs} chicken\_basket{\sc =loc} {\sc 3pl-}be{\sc [npst]=nmlz.nsg=nsg}\\
\rede{The chicken are in the chicken basket.} (a basket with small opening, to transport chicks)
\bg.panckapan=ga    kerabari=be        eko mãɖa luŋkhwak wɛʔ=na\\
a\_region{\sc =gen} banana\_plantation{\sc =loc} one huge stone exist{\sc [3sg;npst]=nmlz.sg}\\
\rede{In the banana plantations of Pā̃ckapan, there is a huge rock.}  \source{39\_nrr\_08.01}
\bg. eko ten=be        eko maghyam  wei-sa=na \\
one village{\sc =loc} one old\_woman exist{\sc [3sg;npst]-pst=nmlz.sg}\\
\rede{In a village, there was an old woman.} \source{01\_leg\_07.060}
	
Destinations of motion verbs and verbs of caused motion are generally marked by the locative, illustrated by \Next. As explained above in \sectref{case-nom}, in certain scenarios the locative marking on the destinations of motion verbs can be omitted. 

	 \ex. \ag.khali puŋda=we  kheʔ-m=ha\\
	only jungle{\sc =loc} go-{\sc inf[deont]=nmlz.nsg}		\\
	\rede{Their only option was to go to the forest.} \source{22\_nrr\_05.045} 
\bg.ŋkhiŋbelak=pe   phopciba=ca        ok-saŋ          hop=pe     pes-a-khy-a-ma\\
that\_time{\sc =loc} owl{\sc =add} shriek{\sc -sim} nest{\sc =loc} fly{\sc [3sg]-pst-V2.go-pst-prf}\\
\rede{That time, the owl flew back to its nest, shrieking.}	 \source{42\_leg\_10.042}
	\bg. khokpu=ga      siŋ=be    thaŋ-ma=ga       cog-a-ŋ\\
fig{\sc =gen} tree{\sc =loc} climb{\sc -inf=gen} do{\sc -pst-1sg}\\
	\rede{I tried to climb the fig tree.} \source{42\_leg\_10.020}
	\bg.beula=ga    paŋ=be     beuli ŋ-ghet-u=hoŋ, \\
	groom{\sc =gen} house{\sc =loc} bride {\sc 3pl.A-}take\_along{\sc -3.P[sbj]=seq}\\
	\rede{They take the bride into the groom's house and ...}


Example \Next shows three-argument constructions with locative-marked G arguments. Both inanimate and animate G arguments  (i.e., goals and recipients) can be in the locative. Depending on the frames of argument realization, the locative is obligatory for some verbs, but optional for others (cf. \sectref{three-arg} for a discussion of three-argument frames, alternations and differential object marking).
 
\ex.\ag. ka a-cya=ci iskul=be paks-wa-ŋ-ci-ŋ=ha \\
	{\sc 1sg[erg]} {\sc 1sg.poss}-child{\sc =nsg} school{\sc =loc} send-{\sc npst-1sg.A-nsg.P-1sg.A=nmlz.nsg}	\\
	\rede{I send my children to school.}
	\bg.uŋ=ŋa ka=be mendhwak haks-wa=na\\
	{\sc 3sg=erg} {\sc 1sg=loc} goat send{\sc [3sg.A;3.P]-npst=nmlz.sg}\\
	\rede{He sends me a goat.}

Ownership can be expressed by a verb of existence and the possessor in the locative (see \Next). The existential verb has a suppletive form \emph{ma} for negated forms \Next[b]. 

\ex.\ag. ŋga=be yaŋ waiʔ=ya?\\
{\sc 2sg.poss=loc} money exist{\sc [3sg;npst]=nmlz.nsg}\\
\rede{Do you have money?}
\bg. eŋ=ga=be yaŋ m-ma-n=ha\\
		{\sc 1pl.incl.poss=gen=loc} money {\sc neg-}exist{\sc [3;npst]-neg=nmlz.nsg}	\\
	\rede{We do not have money.} (said among own people)

It is not surprising to find the locative marking extended to temporal reference. However, the more frequent marker in this function is the instrumental \emph{=ŋa}. The locative in \Next might well be a Nepali calque, since, except for \emph{na}, all words in \Next are Nepali loans. 

\ex. \ag. na tihar din=be\\
		this a\_hindu\_festival day{\sc =loc}	\\
	\rede{on this Tihar day} \source{14\_nrr\_02.026}
 	\bg. uncas sal=be\\
	thirty-nine year{\sc =loc}		\\
	\rede{in the year thirty-nine} \source{06\_cvs\_01.013}

	There are also some fixed expressions with the locative, shown in \Next.
	
	\ex.\ag. maŋcwa=be  khe-me-ka=na=i?\\
	water{\sc =loc} go{\sc -npst-2=nmlz.sg=q}\\
	\rede{Do you go to get water (from the well)?} \source{13\_cvs\_02.066}	
	\bg.daura=be khe-me-ŋ=na\\
	fire\_wood{\sc =loc} go{\sc -npst-1sg=nmlz.sg}\\
	\rede{I go to get fire wood.}
	
There is a secondary locative marker \emph{=ge \ti =ghe},\footnote{Both forms are equally acceptable, and semantic differences could not be detected.} used only with human reference, to express the notion \rede{at X's place} (see \Next).\footnote{The word \emph{maiti} in (b) is a Nepali loan and refers to the natal home of a married woman.} The morpheme \emph{=ge} is a contraction of the genitive \emph{=ga} and the locative \emph{=pe}, a structure calqued from Nepali, where one finds e.g.,  \emph{tapāĩ-ko-mā} \rede{at your place} (you-{\sc gen-loc}), \emph{mero-mā} \rede{at my place} (mine-{\sc loc}). 

\ex. \ag.isa=ge?\\
		who{\sc =loc}\\
	\rede{At whose place?}
 	\bg. bagdata                  nak-se            kheʔ-ma           pʌryo,  mapaci=ghe,                maiti=ci=ghe             kheʔ-ma=hoŋ,\\
		marriage\_finalization ask\_for{\sc -sup} go{\sc -inf[deont]} have\_to{\sc .3sg.pst},  parents{\sc =loc} natal\_home{\sc =nsg=loc} go{\sc -inf[deont]=seq} \\
	\rede{One has to go and ask for the Bagdata (ritual), one has to go to the parents, to the wife's family, and ...} \source{26\_tra\_02.013}
	
\subsubsection{The ablative  \emph{=phaŋ}}\label{case-abl}	

The ablative \emph{=phaŋ} (or [bhaŋ] due to voicing) marks the source of movement or transfer (see \Next). Etymologically it could be the result of stacking an older ablative \emph{=haŋ} upon the locative marker \emph{=pe}. Various other Kiranti languages have such complex ablative markers based on the locative marker, too \citep[81]{Ebert1994The-structure}. In this light it might also be noteworthy that Grierson lists an ablative \emph{-bohuŋ}  for a Yakkha dialect spoken in the beginning of the 20\textsuperscript{th} century  in Darjeeling \citep{Grierson1909Linguistic}. A possible cognate to the older marker \emph{=haŋ} is the  Belhare ablative \emph{=huŋ \ti =etnahuŋ} \citep[549]{Bickel2003Belhare}.\footnote{The form \emph{=etnahuŋ} is most probably also combined of a locative and an ablative marker.}

\ex. \ag.taŋkheŋ=bhaŋ   tuknuŋ    percoʔwa uks-a-ma,\\
sky{\sc =abl} thoroughly lightning come\_down{\sc [3sg]-pst-prf}\\
\rede{Strong lightning came down from the sky.}\source{21\_nrr\_04.017}
\bg.nna=be     ŋ-hond-u-n-ci-n=oŋ                            nna  lupluŋ=bhaŋ   tumhaŋ lond-a-khy-a=na\\
that{\sc =loc} {\sc neg-}fit{\sc -3.P-neg-3nsg.P=seq} that  cave{\sc =abl} Tumhang come\_out{\sc -pst[3sg]-V2.go-pst=nmlz.sg}\\
\rede{As they did not fit there anymore, Tumhang came out of that cave.}\source{27\_nrr\_06.005}


The ablative is also used to signify the starting point for a measurement of distance, as in \Next.

\exg. i   let u-cya=ŋa              u-ma              paŋ=bhaŋ    maŋdu ta-meʔ-ma             mit-uks-u\\
one day {\sc 3sg.poss-}child{\sc =erg} {\sc 3sg.poss-}mother house{\sc =abl} far arrive{\sc -caus-inf} think{\sc -prf-3.P[pst]}\\
\rede{One day, the son wanted to bring his mother far away from the house.}\source{01\_leg\_07.067}

The medium of motion and the technical medium of communication can also be marked by  the ablative,  in parallel to the functions of the Nepali ablative \emph{bāṭa}. 

\ex. \ag.kaniŋ nawa=bhaŋ hoŋma kakt-wa-m-ŋa=na\\
{\sc 1pl[erg]} boat{\sc =abl} river cross{\sc -npst-1pl.A-excl=nmlz.sg}\\
\rede{We will cross the river by boat.}
\bg. kithrikpa=ŋa   solop       maik=phaŋ         lu-ks-u-ci\\
policeman{\sc =erg} immediately microphone{\sc =abl} call{\sc -prf-3.P[pst]-3nsgP}\\
\rede{The policeman immediately called out their names with the microphone.}\source{01\_leg\_07.166}
\bg.thawa=bhaŋ    to  ŋ-khy-a-ma=niŋ=go                    mamu nnhe=maŋ    wɛʔ=na=bu\\
ladder{\sc =abl} up {\sc 3pl-}go{\sc -prf=ctmp=top} girl there{\sc =emph} exist{\sc [3sg]=nmlz.sg=rep}\\
\rede{When he climbed up on the ladder, the girl was right there (they say)!}\source{22\_nrr\_05.111}


It is not unusual for Tibeto-Burman languages to display syncretisms between locative, allative and ablative \citep{DeLancey1985_Etymological}. In the majority of the Yakkha data, the Yakkha ablative marks the source, but there are quite a few examples with an ablative form (or an adverb derived by an ablative) marking the goal of a movement. Thus, Yakkha shows a syncretism between ablative and allative, to the exclusion of the locative.

\ex. \ag.heʔnang khe-ks-a-ga=na?\\
where{\sc [abl]} go{\sc -V2.cut-pst-2=nmlz.sg}\\
\rede{Where are you about to go?}
\bg.yondhaŋ khy-a\\
across{\sc [abl]} go{\sc -imp}\\
\rede{Go there.}\\
\rede{Go from there.}

Just like the secondary locative \emph{=ge \ti =ghe}, the ablative shows a secondary form \emph{=ghaŋ} that is used only with human reference, illustrated  by \Next[a]. Furthermore, the sentences in example \Next show that the ablative is not sensitive to topographic information. There is just one marker, used irrespective of directions and elevation levels with respect to the deictic center.

\ex. \ag. lumba=ghaŋ     ukt-u-ŋ-ci-ŋ,                               lumbapasal=bhaŋ\\
Lumba{\sc =abl} bring\_down{\sc -3.P[pst]-1sg.A-3nsg.P-1sg.A} shop\_of\_Lumba{\sc =abl}\\
\rede{I brought them down from Lumba (a person), from the Lumba shop.} \source{36\_cvs\_06.049}
\bg.yaŋliham=bhaŋ=jhen,    koi.\\
lowland{\sc =abl=top} some\\
\rede{From the lowlands (local lowlands, not the Tarai), some people.}\source{36\_cvs\_06.465}
 
 
 In some interrogative words and adverbs one can still see that they were composed of some root and an older ablative marker, e.g.,  in \emph{nhaŋ} \rede{from here/and then} and \emph{heʔnaŋ \ti heʔnhaŋ} \rede{where from}.\footnote{Both forms  \emph{heʔnaŋ} and \emph{heʔnhaŋ} are equally acceptable to the speakers, and semantic differences could not be detected.} 

\ex. \ag. heʔnaŋ tae-ka=na, mamu?\\
		where\_from come{\sc [npst]-2=nmlz.sg}, girl\\
		\rede{Where do you come from, girl?}
	\bg. mondaŋ ky-a-ŋ=na \\
		below{\sc [abl]} come\_up{\sc -pst-1sg=nmlz.sg}\\
		\rede{I came up from below.}
		
The ablative is generally not used for temporal reference. There is a postposition \emph{nhaŋto} that covers this function (cf. \sectref{postpos} below).
 

\subsubsection{The comitative  \emph{=nuŋ}}\label{case-com}

The comitative marker  \emph{=nuŋ} is cognate to Limbu \emph{-nu},  Thulung \emph{-nuŋ} \citep[81]{Ebert1994The-structure}, Wambule \emph{-no} \citep[157]{Opgenort2004A-Grammar}, Bantawa \emph{-nin} \citep[91]{Doornenbal2009A-grammar}, Chintang \emph{-nɨŋ} \citep[80]{Schikowski2012_Morphology}. It can be used as a nominal coordinator, functionally similar to English \emph{and}  (symmetrical, with nouns of the same status, as defined in  \citealt[3]{Haspelmath2004_overview}). An example is given in \Next[a], a story title of the commonly found pattern \rede{X and Y}. Thus, by its very nature, this case marker can be found inside noun phrases, coordinating two nominal heads. The other case markers attach to the coordinate structure as a whole (see \Next[b]). The marker is phonologically bound to the first component of the coordinate structure.

Examples \Next[c] and \Next[d] serve to show that both parts of the  coordinate structure contribute features to the person and number marking on the verbs. In \Next[c] the verb is marked for dual number, determined by the proper noun \emph{ɖiana} and by the omitted pronoun \emph{ka} \rede{I}. In \Next[d] the first person inclusive verbal marking is triggered by both \emph{nniŋda} and \emph{kaniŋ}.

\ex. \ag.  suku=nuŋ   kithrikpa\\
Suku{\sc =com} policeman\\
\rede{Suku (a girl's name) and the policeman} \source{01\_leg\_07.143}
		\bg. a-ma=nuŋ                a-na=ga                       ceʔya\\
{\sc 1sg.poss-}mother{\sc=com} {\sc 1sg.poss-}sister{\sc=gen} matter\\
		\rede{the warnings of my mother and sister}\source{42\_leg\_10.051}
		\bg.hakhok=ŋa  am-me-ŋ-ci-ŋ=ba,                          ɖiana=nuŋ     am-me-ŋ-ci-ŋ,                      asen=ca\\
		later{\sc =ins} come\_over{\sc -npst-excl-du-excl=emph} Diana{\sc =com} come\_over{\sc npst-excl-du-excl} yesterday{\sc =add}\\
		\rede{Later, we will come of course, Diana and I, we will come; yesterday (we came), too.}\source{36\_cvs\_06.376}
		\bg.la,    nniŋda=nuŋ   kaniŋ haku cuŋ-iǃ\\
		alright {\sc 2pl=com} {\sc 1pl} now wrestle{\sc -1pl[incl;sbjv]}\\
		\rede{Well, now let us wrestleǃ}\source{39\_nrr\_08.12}
			
The comitative is also used to mark peripheral participants that somehow accompany the main participants or that are associated with them (see \Next).

\ex. \ag.tabek,                 kacyak, mina  kondarik,               caprak                 nhaŋ    chomlaki=nuŋ       puŋda=be    lab-a-cog-a-ŋ-ci-ŋ\\
khukuri\_knife, sickle, small spade, spade and\_then split\_bamboo{\sc =com}  jungle{\sc =loc} hold{\sc -pst-V2.make-pst-excl-du-excl}\\
\rede{Carrying khukuri,  sickle, spades and split bamboo, we went into the jungle.} (literally: \rede{made into the jungle})\footnote{This V2 is only found in this one example so far, and thus, it is not treated in Chapter \ref{verb-verb} on complex predicates.} \source{40\_leg\_08.008}
\bg.ka=nuŋ kheʔ-ma=na kamnibak\\
{\sc 1sg=com} go{\sc -inf[deont]=nmlz.sg} friend\\
\rede{a friend who has to walk with me}		
		
Some frames of verbal argument realization (both intransitive and transitive) require the comitative on their arguments, such as \emph{cekma} \rede{talk}, \emph{toŋma} \rede{fit/agree/belong to}, \emph{kisiʔma} \rede{be afraid}, \emph{nakma} \rede{ask} and \emph{incama} \rede{buy (from)} (see \Next).		
				
		\ex. \ag.mimik,   ka=nuŋ    seppa,         u-ppa=nuŋ=go                   banda, n-jeŋ-me-n=na\\
		a\_little {\sc 1sg=com} {\sc restr.emph} {\sc 3sg.poss-}father{\sc =com=top} closed {\sc neg-}talk{\sc [3sg]-npst-neg=nmlz.sg}\\
		\rede{A little, just with me – with her father, nothing, she does not talk to him.} \source{36\_cvs\_06.278 }
 	\bg.limbu=ci=ga=nuŋ                  toŋ-di-me=ppa,                     eŋ=ga=go [...] aru=ga=nuŋ        n-doŋ-men\\
			Limbu\_ethnic.group{\sc =nsg=gen=com}  agree{\sc -V2.give-npst[3sg]=emph} {\sc 1pl.incl.poss=gen=top} [...] other{\sc =gen=com} {\sc neg-}agree{\sc [3sg]-npst-neg}\\
		\rede{It is like the language of the Limbus, our (language). [...] It does not fit to the others.} \source{36\_cvs\_06.256--58}
	

		
The comitative also plays a role in the derivation of some adverbs, as shown in \Next (cf.  \sectref{adv}). Furthermore, it is also found in clause linkage (cf.  \sectref{com-cl}).


\exg.khumdu=nuŋ nam-ma\\
			tasty{\sc =com} smell{\sc -inf}\\
		\rede{to smell tasty}


\subsection{Further case markers (Group II)}\label{postpos}
  
The markers of Group II are quite heterogenous; they do not define a class as such. They can appear bound to their host or independently, i.e., stressed like a separate word. Their phonological weight is also greater than that of the markers of Group I; all of them are at least disyllabic.  The case markers of Group II  have a greater flexibility with regard to hosts they can select. Not only nominals are possible, but also adverbials. Some markers of Group II  are not attested with nominal complements at all, like \emph{khaʔla} \rede{towards}.  Furthermore, a number of the markers of Group II have hybrid word class status; they can also be used as adverbs. Some markers were borrowed into the language from Nepali, like \emph{samma} \rede{until} or \emph{anusar} \rede{according to}.  \tabref{table-postpos} provides a summary of all Group II markers and their functions, described in detail in the following sections.
 
 \begin{table}[htp]
\begin{centering}
\begin{tabular}{ll}
\lsptoprule
{\sc marker}&{\sc function}\\
\midrule
\emph{khaʔla}&directional, \rede{towards}; manner \rede{like}\\
\emph{nhaŋto}&temporal ablative, \rede{since, from X on}\\
\emph{haksaŋ}&comparative, \rede{compared to}\\
\emph{haʔniŋ}&comparative, \rede{compared to}\\
\emph{loʔa}&equative, similative, \rede{like}\\
\emph{hiŋ}&equative (size) \rede{as big as}\\
\emph{maʔniŋ}&caritive, \rede{without}\\
\\
\emph{bahek} [\textsc{nep}]& exclusive, \rede{apart from}\\
\emph{samma} [\textsc{nep}]&terminative, \rede{until, towards}\\
\emph{anusar} [\textsc{nep}]&\rede{according to}\\
\emph{lagi} [\textsc{nep}]&benefactive, \rede{for}\\
\lspbottomrule
\end{tabular} 
\caption{Case markers (Group II)}\label{table-postpos}
\end{centering}
\end{table}
  	
	
\subsubsection{The direction and  manner marker  \emph{khaʔla}}

The directional/manner marker \emph{khaʔla}  \rede{towards, in the way of} is not attested with nouns, it only attaches to deictic adverbs. The directional reading is found when \emph{khaʔla} attaches to demonstrative adverbs typically occurring with motion verbs (see \Next). Etymologically, it is a combination of a demonstrative \emph{kha} with an older allative or directional case marker. Cognates of such a marker are attested in several Kiranti languages: \emph{-tni} in Bantawa \citep[84]{Doornenbal2009A-grammar} and Puma \citep{Sharma2005Case}, \emph{-baiʔni \ti -ʔni} in Chintang \citep[83]{Schikowski2012_Morphology}. %BLH 2003 - no such marker, only samma

\ex. \ag. to=khaʔla ky-a!\\
up=towards come\_up{\sc -imp}\\
\rede{Come up!}\source{01\_leg\_07.329 }
\bg.ŋkha limbu=ci             yo=khaʔla  ŋ-khy-a\\
those Limbu\_person{\sc =nsg} across=towards {\sc 3pl-}go{\sc -pst}\\
\rede{Those Limbus went away (horizontally).} \source{22\_nrr\_05.017}
\bg.nniŋga=go,          mo,  mo=khaʔla=ca        nis-uks-u-ŋ=ha\\
{\sc 2pl.poss=top} down down=towards{\sc =add} see{\sc -prf-3.P[pst]-1sg.A=nmlz.nsg}\\
\rede{Your (home), below, downwards, I have seen it, too.}\footnote{\rede{Downwards} could be any location outside the Himalayas.}\source{28\_cvs\_04.334}

The manner reading is found when \emph{khaʔla}  attaches to demonstratives (see \Next).

\ex. \ag.ijaŋ bhasa    n-jiŋ-ghom-me=ha?                        hoŋ=khaʔla=maŋ baʔlo!\\
why language {\sc 3pl-}learn{\sc -V2.roam-npst=nmlz.nsg} that\_very=like{\sc =emph} {\sc emph.excla}\\
\rede{Why do they walk around learning languages? Just like that! } \source{28\_cvs\_04.324}
\bg.nna=khaʔla, mamu, i    cok-ma=ʔlo,               hamro des?\\
that=like girl what do{\sc -inf=excla} our country\\
\rede{(It is ) like that, what to do, girl, with our country?} \source{28\_cvs\_04.163}

The marker \emph{khaʔla} also has a homonymous adverbial counterpart\footnote{Adverbial in the sense that it occurs independently, without nominal complements, and in the function of  modifying verbs.} with a purely manner reading: \rede{like this}, e.g.,  \emph{khaʔla om} \rede{It is like this.} 

\subsubsection{The temporal ablative marker \emph{nhaŋto} }
   
   The marker \emph{nhaŋto} (occasionally also \emph{bhaŋto}) usually attaches to nouns or adverbs with temporal reference and marks the beginning of time intervals, regardless of whether they extend from a point in the past, present or future, as the examples in \Next illustrate. Example \Next[d] shows that it may also attach to  demonstratives. The etymology of this marker is still transparent. It is composed of a demonstrative \emph{na} with an (older) ablative \emph{-haŋ} and the deictic adverb \emph{to} \rede{up}, yielding a phrase \rede{up from here}. This points towards a conceptualization of time as beginning below and flowing upwards. So  far, this is just an educated guess, supported by the uses of some complex predicates, such as a combination of \rede{see} and \rede{bring up}, best translated as \rede{having remembered}.
 
 \ex. \ag.asen=nhaŋto\\
 yesterday{\sc =temp.abl}\\
 \rede{since yesterday}
 \bg.mi wandik=nhaŋto\\
 a\_little later{\sc =temp.abl}\\
 \rede{from a bit later on}
   \bg.lop=nhaŋto=maŋ\\
   now{\sc =temp.abl=emph}\\
   \rede{from now on}\source{01\_leg\_07.030}
\bg.nna=nhaŋto sumphak cilleŋ              n-leks-u\\
that{\sc =temp.abl} leaf face\_up {\sc 3pl.A-}turn{\sc -3.P[pst]}\\
\rede{From that (event) on, they turned around the leaf plate to the proper side.}  \source{22\_nrr\_05.132}


This marker is occasionally also found as clause-initial coordinator used similarly to \Next, which reflects the historical stage prior to becoming a bound marker.  The previous clause is referred to by a demonstrative (not in these, but in plenty of other examples), resulting in a structure \emph{nna, nhaŋto} \rede{that, and then upwards}, and eventually  the clause-initial coordinator got reanalyzed as requiring a complement of some kind.

\ex. \ag.nhaŋto, garo n-cheŋd-et-wa=na,                            to=khaʔla\\
and\_then wall {\sc 3pl.A-}mason{\sc -V2.carry.off-npst=nmlz.sg} up=towards\\
\rede{And then they mason the wall, upwards.} \source{31\_mat\_01.093}
\bg.nhaŋto phuna=chen            seg-haks-u-ŋ=hoŋ\\
and\_then white{\sc =top} choose{\sc -V2.send-3.P[pst]-1sg.A=seq}\\
\rede{And then, I sorted out the white (bread), and ...}\source{40\_leg\_08.060}

Marginally (in one case, to be precise),  a synonymous marker \emph{nhaŋkhe}, paraphrasable as \rede{from then on hither}, was found with the same function.

\exg.nhaŋkhe        u-ma              heʔniŋ=ca        issisi n-jog-uks-u-n\\
and\_then {\sc 3sg.poss-}mother when{\sc =add} bad {\sc neg-}do{\sc -prf-3.P[pst]-neg}\\
\rede{And then, he never did his mother bad again.} \source{01\_leg\_07.082}

\subsubsection{The comparative marker \emph{haksaŋ/haʔniŋ}}
  
  The two comparative markers \emph{haksaŋ} and \emph{haʔniŋ} mark the standard in comparative and in superlative constructions. They are used interchangeably without any functional difference. Since they are treated in detail in Chapter \ref{adj-adv}, three examples shall suffice here. Examples \Next[a] and \Next[b] show comparative constructions, \Next[c] shows a superlative construction. The comparative markers can attach to all kinds of hosts, even to verbs. Etymologically they must have been converbal forms, since Yakkha has the converbal and adverbial clause linkage markers \emph{-saŋ} and \emph{=niŋ}, both indicating cotemporality. The structure of the Yakkha comparative markers could be calqued upon the structure of the  Nepali comparative marker \emph{bhanda}, which is a converbal form of the verb \emph{bhannu} \rede{to say}. The identity of a possible verbal stem \emph{hak} in Yakkha, however, could not be determined. Synchronically, the meaning of \rede{compare} is expressed by a complex verb \emph{themnima}. A likely candidate could be the verbal stem \emph{haks}, which basically means \rede{send/send up}, but is also used with the meaning \rede{weigh}.
  
  
\ex.\ag. nda haʔniŋ pak=na?\\
		2sg  {\sc compar}	be\_unripe{\sc =nmlz.sg}\\
	\rede{Is he younger than you?}
\bg. heko=ha nwak=ci haksaŋ miyaŋ alag (...) sa=na=bu\\
	other{\sc =nmlz.nsg} bird{\sc =nsg} {\sc compar}  a\_little different (...) {\sc cop.pst=nmlz.sg=rep}\\
	\rede{He was a bit different from the other birds, they say.} \source{21\_nrr\_04.002}
\bg. ghak haʔniŋ mi=na  mima\\
	all {\sc compar} small{\sc =nmlz.sg} mouse\\
\rede{the smallest mouse (of them all)} \source{01\_leg\_07.003}
  

\subsubsection{The equative and similative marker \emph{loʔa}}
  
  The equative/similative \emph{loʔa} marks the standard of an equation. It can have adverbial \Next[a] and nominal  complements  (a numeral in \Next[b]), even clausal, when they are embedded to verbs of perception or cognition. Example \Next[c] shows that the resulting equative phrase  can be “fed”  into a nominalization itself and thus made a referential phrase. The equative/similative marker is cognate to the comitative and adverbial clause linkage marker \emph{-lo \ti lok \ti loʔ} in Belhare  \citep{Bickel1993Belhare}. The same marker is known as \rede{manner suffix} in Bantawa  \citep{Doornenbal2009A-grammar}.  There is one lexicalized instance of  \emph{loʔa}, the adverb  \emph{pekloʔa \ti pyakloʔa} \rede{usual(ly)}, still morphologically transparent: its literal meaning would be \rede{like much/like many}.
 
 
  \ex. \ag.khem loʔa\\
 before like\\
  \rede{like before}
 \bg. kaniŋ ka-i-wa=niŋa eko loʔa kheps-wa-m\\
 {\sc 1pl} say{\sc -1pl-npst=ctmp} one like hear{\sc -npst[3.P]-1pl.A}\\
 \rede{When we say it, it sounds the same!} \source{36\_cvs\_06.478}
  \bg.khem loʔa=na mekan!\\
  before like{\sc =nmlz.sg} {\sc neg.cop.2sg}\\
  \rede{You are not like someone from just before!} (said to someone who was a little tipsy but claimed to come right from work)
  
  In a manner typical for Tibeto-Burman languages, this marker extends its function to clauses.\footnote{See also \citet{DeLancey1985_Etymological} and \citet{Genetti1991From}.} In \Next it takes over the function of a complementizer.
  
  \exg. ka luʔ-meʔ-nen-in=ha loʔa  cog-a-ni\\
  {\sc 1sg[erg]}  tell{\sc -npst-1>2-2pl=nmlz.nsg} like do\sc{ -imp-pl.imp}\\
  \rede{Do as I tell you.} \source{14\_nrr\_02.019}

  
\subsubsection{The equative  marker for size \emph{hiŋ}}

The equative case for size is etymologically related to the deictic adverb \emph{khiŋ} (which is etymologically composed of the demonstrative \emph{kha} and \emph{hiŋ}). Attached to a noun phrase that functions as standard of comparison, this case marker indicates that an object is as big as the  object referred to by the noun to which \emph{hiŋ} attaches, as shown in \Next. In this example, the whole phrase is nominalized and functions as the nominal predicate of a copular clause.

	\exg. m-muk a-laŋ hiŋ=na (om)\\
 {\sc 2sg.poss-}hand {\sc 1sg.poss-}foot 	as\_big\_as{\sc =nmlz.sg} ({\sc cop})	\\
\rede{Your hand is as big as my foot.}

\subsubsection{The privative marker \emph{maʔniŋ}}

The privative \emph{maʔniŋ} is historically complex, similar to \emph{haʔniŋ} above. It is composed of the negative existential copular stem \emph{ma} (in third person singular, zero-marked) and  the cotemporal adverbial clause linkage marker \emph{=niŋ} (see \Next). In the same way as we have seen above for \emph{loʔa} already, the privative phrase can be nominalized to serve as a nominal modifier, as shown in \Next[b].
  
  \ex.\ag.i=ŋa      cama        niʔ-m=ha,     maŋcwa maʔniŋ?\\
  what{\sc =ins} rice cook{\sc -inf[deont]=nmlz.nsg} water without\\
\rede{How (in what) shall we cook rice, without water?} \source{13\_cvs\_02.108}
\bg.wariŋba maʔniŋ=ha khyu\\
tomato without{\sc =nmlz.nsg} curry\_sauce\\
\rede{curry sauce without tomatoes in it} 
  
  
\subsubsection{Postpositions from Nepali}
  
The benefactive/purposive postposition \emph{lagi} (from Nepali  \emph{lāgi}), like in its source language, requires the genitive case. It can attach to proper nouns  or to nominalized clauses like the infinitive in \Next[b]. The genitive is, however, also found on purposive infinitival clauses without the postposition (see \sectref{maga}); it might well precede the point in time when \emph{lagi} entered the Yakkha language.

\ex. \ag.hoʔiǃ  ak=ka          lagi  iya=ca                       tuʔkhi n-jog-a-n\\
enough! {\sc 1sg.poss=gen} for what{\sc =add} trouble {\sc neg-}do{\sc -imp-neg}\\
\rede{No, thanks. Do not bother about me (at all).} \source{01\_leg\_07.186}
 \bg.  heʔniŋ-heʔniŋ=go    yuncama=le           cok-ma       haʔlo\\
when-when{\sc =top} laughter{\sc =ctr} do{\sc -inf[deont]} {\sc excla}\\
\rede{Sometimes one just has to joke around, man! } \source{36\_cvs\_06.263}

Another postposition from Nepali is \emph{anusar} \rede{according to} (from Nepali \emph{anusār} ). It is typically found with nominalized clauses (see \Next).


\ex. \ag.ka-ya=na                   anusar\\
say{\sc [3sg]-pst=nmlz.sg} according\_to\\
\rede{according to what was said/promised} \source{11\_nrr\_01.012}
\bg.  ka       nis-u-ŋ=ha                                  anusar\\
{\sc 1sg[erg]} see{\sc -3.P[pst]-1sg.A=nmlz.nc} according\_to\\
\rede{according to what I know/saw} \source{25\_tra\_01.169}

The terminative postposition \emph{samma}   is used to specify the endpoint of an event (see \Next). This postposition is also found in clause linkage, in combination with native adverbial subordinators.

\exg. aniŋ=ga            ceʔya   hen samma      man=ha=bu\\
{\sc 1pl.excl.poss=gen}  language now until {\sc neg.cop=nmlz.nc=rep}\\
\rede{Our language has not been established until now (they say).} \source{07\_sng\_01.06}

The exclusive postposition \emph{bahek} \rede{apart from}  serves to single out a referent to which the predication made in the sentence does not apply (see \Next). 

\exg. taŋcukulik bahek=chen,       heŋ-nhak-ni-ma,                       jammai, kha  ya-muŋ=ca,       ghak heŋ-nhaŋ-ma\\
pig-tail apart\_from{\sc =top} cut{\sc -V2.send-compl-inf[deont]} all this mouth-hair{\sc =add} all cut{\sc -V2.send-inf[deont]} \\
\rede{Apart from the pig-tail one has to cut it off, all, this beard too, all has to be cut off.} (context: funeral description) \source{29\_cvs\_05.058}


In all the postpositions from Nepali, the phonological contrast between open-mid /ʌ/ and open /a/, which is present in the source language, is neutralized to open and long /a/.

 
\section{Relational nouns}\label{postpos-2}
  
  Yakkha has a class of relational nouns, in which  specific meanings like \rede{root} are metaphorically extended to indicate more general spatial relations like \rede{under}. Usually, they occur in a possessive construction with the complement noun in the genitive and a possessive prefix attaching to the relational noun, which also hosts a locative case marker, as in  \Next[a] and \Next[b]. Relational nouns expressing spatial relations are a common source for case markers and postpositions in Tibeto-Burman \citep[62]{DeLancey1985_Etymological}.
  
	\ex. \ag. phakʈaŋluŋ=ga        u-sam=be\\
		Mount\_Kumbhakarna{\sc =gen} {\sc 3sg.poss-}root{\sc =loc}	\\
		\rede{at the foot of Mount Kumbhakarna}  \source{18\_nrr\_03.001}
		\bg. caram=ga    u-lap=pe    camokla=nuŋ   ambibu=ga    u-thap     ŋ-weʔ-ha	\\
		yard{\sc =gen} {\sc 3sg.poss-}wing{\sc =loc} banana{\sc =com} mango{\sc =gen} {\sc 3sg.poss-}plant {\sc 3pl-}exist{\sc [npst]=nmlz.nsg}\\
		\rede{At the edge of the yard there are some banana trees and mango trees.} \source{01\_leg\_07.176}
	
	
Relational nouns can also be found without the inflectional morphology between complement and relational noun, in a compound-like structure, as in  \Next[a].\footnote{The person marking for third person on the main verb here is exceptional, since it refers to a first person participant. The expected regular first person inflection (\emph{ipsamasaŋna}) would be possible as well. We know that such impersonal inflection is an alternative and frequent way to express first person nonsingular patients in Yakkha. This example is, however, the only instance in the corpus where this strategy is used for first person singular subject of an intransitive verb.} 	It is not only the locative but also the ablative which may attach to a relational noun, as shown in \Next[b]. In this particular example, the ablative marking indicates a movement along a trajectory above the table.
		
 \ex.\ag. hakhok=ŋa  ka  cend-a-ky-a-ŋ=hoŋ so-ŋ=niŋa=go ka  luŋkhwak-choŋ=be  ips-a-masa\\
	later{\sc =ins} {\sc 1sg} wake\_up{\sc -pst-V2.come\_up-pst-1sg=seq} look{\sc -1sg=ctmp=top} {\sc 1sg} stone-top{\sc =loc} sleep{\sc [3sg]-pst-pst-prf}\\
	\rede{Later, when I woke up and looked around, (I realized that) I had been sleeping on a rock.} \source{42\_leg\_10.043}
	\bg.chalumma=ŋa phuaba ʈebul-choŋ=bhaŋ bol lept-u-bi=na\\
	second\_born\_girl{\sc =erg} last\_born\_boy table-top{\sc =abl} ball throw{\sc -3.P[pst]-V2.give=nmlz.sg}\\
	\rede{Chalumma threw the ball over the table to Phuaba.} 
	
\tabref{relnoun} provides a summary and the original lexical nouns that are the bases for each relational noun. In \Next, the relational noun is reduplicated, since the relation described is not one of location at the riverside, but one of movement along the river.

	 	 
 \begin{table}[htp]
\begin{centering}
\begin{tabular}{lll}
\lsptoprule
{\sc relational noun}&{\sc gloss}&{\sc lexical meaning}\\
\midrule
\emph{choŋ \ti chom} &above, on, on top of&\rede{top, summit}\\
\emph{sam} &below&\rede{root}\\
\emph{lum} &in, between&\rede{middle}\\
\emph{yum} &next to&\rede{side}\\
\emph{hoŋ} &inside&\rede{hole}\\
\emph{lap}&next to (upper part) &\rede{wing}\\
\emph{laŋ} &next to (lower part) &\rede{leg}\\
\emph{heksaŋ} &behind, after &\rede{backside}\\
\emph{ondaŋ} &in front of, before &\rede{frontside}\\
\emph{chuptaŋ} &to the right of &\rede{right side}\\
\emph{pheksaŋ} &to the left of &\rede{left side}\\
\lspbottomrule
\end{tabular} 
\caption{Relational nouns}\label{relnoun}
\end{centering}
\end{table}
 
	
	\exg. hoŋma=ga    u-lap-ulap    lukt-a-ma\\
	river{\sc =gen} {\sc 3sg.poss-}wing{\sc -redup} run{\sc [3sg]-pst-prf} \\
	\rede{He ran along the shore of the river.} \source{01\_leg\_07.216}

The two relational nouns \emph{heksaŋ} and \emph{ondaŋ} can, additionally, occur as adverbs. In the current corpus, they are mainly used adverbially (see \Next). As these examples show, \emph{heksaŋ} and \emph{ondaŋ} , in contrast to the other relational nouns, can also be used with a temporal interpretation.

\ex.\ag.n-heksaŋ=be cuwa ta=ya\\
{\sc 2sg.poss-}behind{\sc =loc} beer come{\sc [3sg;pst]=nmlz.nsg} \\
\rede{Some beer has arrived behind you.}
\bg.tabhaŋ panc hapta heksaŋ ta-meʔ=na\\
son-in-law five week behind come{\sc [3sg]-npst=nmlz.sg}\\
\rede{The son-in-law comes five weeks later.} 
\bg.heksaŋ so-ŋ-ci-ŋ                        uŋci n-nis-u-n-ci-ŋa-n\\
later look{\sc [pst]-1sg.A-3nsg.P-1sg.A}  {\sc 3nsg} {\sc neg-}see{\sc -3.P[pst]-neg-3nsg.P-1sg.A-neg}\\
\rede{Later, when I looked for them, I did not see them.}  \source{41\_leg\_09.050}
 
Furthermore, there are spatial adpositions,  presenting an orientation system that is based on the uphill/downhill distinction. They are treated in \sectref{geomorph-postp}, together with the other word classes that are based on this topography-based system.

 	
Yakkha does not have a perlative/mediative \emph{lam} or \emph{lamma} case or postposition which is found in many of the surrounding languages.\footnote{E.g., in Chintang \citep{Schikowski2012_Morphology}; Belhare \citep{Bickel2003Belhare}; Limbu \citep{Driem1987A-grammar}, Athpare, Yamphu, Camling, Thulung \citep{Ebert2003Kiranti}.} There is also no postposition for the relation \rede{around}. This can only be expressed adverbially with \emph{ighurum} \Next.\footnote{This adverb has its origin in a noun \emph{ighurum} \rede{round}, which still exists synchronically in Yakkha.}

	\exg.  mi   em-saŋ           huŋ-ca-saŋ            ighurum yuŋ-i-misi-ŋ\\
	fire get\_warm{\sc -sim} bask{\sc -V2.eat-sim} around sit{\sc -1pl-prf.pst-excl}\\
	\rede{[...], we had sat around the fire,  getting warm.} \source{40\_leg\_08.033}
	

\section{The structure of the noun phrase}\label{str-np}

The basic function of noun phrases is to establish reference. They occur as arguments of verbs, as complements of postpositions and as predicates in copular constructions. They may host morphology such as case and number markers  and various discourse particles. Noun phrases are potentially complex; both coordinate and embedded structures can be found inside the noun phrase. Noun phrases can be headed by a lexical noun or by a pronoun, a demonstrative, a numeral, a quantifier or an adjective. Noun phrases that are not headed by a lexical noun are more restricted in the kind of modifying material they may contain. Noun phrases can also be headless, consisting just of some non-nominal material and a nominalizing device. Hence, no element in a Yakkha noun phrase is obligatory. 

 The default structure for headed  noun phrases is head-final. Deviations from this pattern reflect discourse requirements, as will be discussed below.  In noun phrases that are headed by  personal pronouns or demonstratives, modifiers follow the head. Noun phrases with more than two modfiying elements are exceedingly rare.
 
\subsection{Possessive phrases}\label{str-np-poss}

Possessive phrases minimally consist  of a noun (referring to the possessee) which is marked by a possessive prefix (indexing the possessor, see \Next[a]). If there is an overt possessor, marked by the genitive, the possessive prefix is generally optional (see \Next[b]), except for inherently possessed nouns such as core family terms and some other nouns implying part-whole relations. The possessive prefix may, however, also  co-occur with a possessive pronoun, but only when the possessor has singular reference (see unacceptable \Next[c]). Recursive embedding is possible as well, but not found beyond two levels of embedding in the currently available data \Next[e].

\ex. \ag. (ak=ka)        a-cya=ci\\
({\sc 1sg.poss=gen}) {\sc 1sg.poss-}child{\sc =nsg}\\
\rede{my children} \source{21\_nrr\_04.027 }
\bg. ghak=ka    ɖaŋgak=ci\\
all{\sc =gen} stick{\sc =nsg}\\
\rede{everyone's sticks} \source{04\_leg\_03.024 }
\bg.eŋ=ga              (*en-)na-nuncha=ci\\
{\sc 1pl.incl.poss=gen} (*{\sc 1pl.inc.poss-})eZ-yZ{\sc =nsg}\\
\rede{our sisters} \source{41\_leg\_09.015 }
\bg.beuli=ga    u-kamnibak\\ 
bride{\sc =gen} {\sc 3sg.poss-}friend\\
\rede{a friend of the bride}\source{25\_tra\_01.089}
\bg.eko khokpu=ga      u-thap=ka              u-sam=be\\
one fig{\sc =gen} {\sc 3sg.poss-}plant{\sc =gen} {\sc 3sg.poss-}root{\sc =loc}\\
\rede{below a fig tree} \source{42\_leg\_10.015}

\subsection{Other modifiers: adjectives, numerals, quantifiers, demonstratives}\label{str-np-mod}

Below, examples with numerals (see \Next[a]), demonstratives (see \Next[b] and \Next[c]), adjectives (see \Next[d] and \Next[e]) are shown. The examples also illustrate nominal morphology such as case markers, attaching to the rightmost element of the phrase, and optionally followed by discourse particles like the additive focus marker \emph{=ca} or the restrictive focus marker \emph{=se}.

\ex. \ag.eko a-muk=phaŋ\\
one {\sc 1sg.poss-}hand{\sc =abl}\\
\rede{from one of my hands} \source{ 40\_leg\_08.022}
\bg.na    tumna=ŋa\\
this elder{\sc =erg}\\
\rede{this elder one} \source{ 40\_leg\_08.055 }
\bg.ŋkha u-hiruʔwa=ci\\
those {\sc 3sg.poss-}intestine{\sc =nsg}\\
\rede{those intestines} \source{40\_leg\_08.039}
\bg.onek=ha      ceʔya=ca\\
joking{\sc =nmlz.nsg} matter{\sc =add}\\
\rede{jokes, too} \source{40\_leg\_08.057 }
\bg.heko=na         whak=pe\\
other{\sc =nmlz.sg} branch{\sc =loc}\\
\rede{on another branch} \source{42\_leg\_10.032}
\bg.honna=ga=se                          ɖaŋgak\\
that\_very{\sc =gen=restr} stick\\
\rede{only that person's stick} \source{04\_leg\_03.025}


When the head noun is a pronoun or  a demonstrative, the modifier is usually a quantifier or a numeral, and it follows the head. Occasionally other material elaborating on the identity of the pronominal referent is found as well, as in \Next[d].

\ex.\label{iyaiyadem}\ag.    iya-iya                nis-u-ga=na,                  ŋkha ghak yok-met-a-ŋ=eba\\
what-what see{\sc -3.P[pst]-2.A=nmlz.sg} that all search{\sc -caus-imp-1sg.P=pol.imp}\\
\rede{Please tell me everything you saw.} \source{19\_pea\_01.005}
\bg.kaniŋ ghak chups-i-ŋ=hoŋ\\
{\sc 1pl} all gather{\sc -1pl-excl=seq}\\
\rede{As we all had gathered, ...} \source{41\_leg\_09.054}
\bg.uŋci hip-paŋ\\
{\sc 3nsg} two{\sc -clf.hum}\\
\rede{the two of them}
\bg.kaniŋ yakkhaba yakkhama=ci\\
{\sc 1pl} Yakkha\_man Yakkha\_woman{\sc =nsg}\\
\rede{we Yakkha people}
              
			  
\subsection{Relative clauses}\label{str-np-rc}

 In \Next and \NNext, examples of relative clauses are given, constructed with the nominalizers \emph{-khuba} and \emph{=na/=ha} (treated in Chapter \ref{ch-nmlz}). They can be of considerable length and internal complexity. 
 In \Next[c], three coordinated relative clauses  serve to modify the same head noun, \emph{whaŋsa} \rede{steam}.\footnote{Enumerations of coordinated items, with the comitative marker functioning as a coordinator (between the last two items if there are more than two), are common in Yakkha. These relative clauses are not embedded into one  another; there are three different smells (or \rede{steams}), not the smell of yams that are cooked together with fried bread and sauce, which also would not make sense semantically, since \emph{whaŋma} can only refer to boiling something solid in water.} They are joined by apposition and a comitative between the latter two relative clauses. This pattern of coordination is common. In \Next[d], the relative clause is preceded by an adjective and contains a complement-taking verb with an embedded infinitive.
 
 \ex. \ag.ka  haksaŋ tum=na  yapmi\\
 {\sc 1sg} {\sc compar} elder{\sc =nmlz.sg} person\\
 \rede{a person senior to me} \source{40\_leg\_08.078}
 \bg. khaʔla   otesraŋ=ha   pachem=ci!\\
 like\_this reverse{\sc =nmlz.nsg} young\_boy{\sc =nsg}\\
 \rede{Such naughty boys!} \source{40\_leg\_08.075}
 \bg.caleppa leps-a=ha,     ni-ya=ha macchi=nuŋ    khi whaŋd=ha     whaŋsa\\
 bread deep\_fry{\sc [3sg]-pst=nmlz.nc} fry{\sc [3sg]-pst=nmlz.nc} chili\_sauce{\sc =com} yam boil{\sc [3sg;pst]=nmlz.nc} steam\\
 \rede{the steam of deep-fried bread, fried chili and boiled yams}\footnote{The lexeme \emph{macchi} is a loan from the Nepali source word \emph{marej} \rede{pepper}. In Yakkha, it refers to chili peppers, but also to hot pickles and sauces.} \source{40\_leg\_08.046 }
\bg.issisi, khem-ma=i          me-ya-m=ha                         ceʔya\\
ugly hear{\sc -inf=foc} {\sc neg-}be\_able{\sc -inf=nmlz.nc} talk\\
\rede{ugly talk that one cannot listen to} \source{36\_cvs\_06.600} 

Headless noun phrases, identical to headless relative clauses, are presented in \Next. 

\ex. \ag.  khi khoŋ-khuba=ci\\
yam dig{\sc -nmlz=nsg}\\
\rede{people digging yam} \source{40\_leg\_08.009 }
\bg.to=na\\
up{\sc =nmlz.sg}\\
\rede{the upper one}

Some nouns take clausal complements (see \sectref{noun-compl}).

\subsection{Coordination}\label{str-np-coord}

If nouns are coordinated in a noun phrase, they can either be juxtaposed (see \Next[a]), or, by means of the comitative case marker, be attached to the penultimate noun (see \Next[b]). The comitative may also coordinate adjectives. Example \Next[c] shows again that several levels of embedding are possible: the coordinated nouns may themselves be modified and these modifers may also be coordinated by \emph{=nuŋ}. Apposition is used comparatively often; instead of using some more general term, one often finds long enumerations of things. This could be a stylistic device to create suspense in narratives, as exemplified in \Next[d].

\ex. \ag.yarepmaŋ, likliŋphuŋ nam-ma=niŋ=ca   ibibi     sokma  ta-ya=na\\
fern, mugwort smell{\sc -inf=ctmp=add} very\_much breath come{\sc [3sg]-pst=nmlz.sg}\\
\rede{When we sniffed at fern and mugwort plants, we regained quite some energy.} \source{ 40\_leg\_08.018 }
\bg.paŋkhi=nuŋ   puŋdakhi     \\
cultivated\_yam{\sc =com} wild\_yam\\
\rede{cultivated yam and wild yam} \source{40\_leg\_08.025 }
\bg.paŋ=be     phu=ha=nuŋ      makhur=ha         caleppa, macchi,  khicalek=nuŋ      cuwa py-a \\
house{\sc =loc} white{\sc =nmlz.nc=com}  black{\sc =nmlz.nc} bread, pickles, rice\_dish{\sc =com} beer give{\sc -pst[1.P]}\\
\rede{At home, they gave us white and black bread, pickles, khichadi and beer.} \source{ 40\_leg\_08.051}
\bg.uŋci=ŋa   tabek,   siŋ,     phendik, lom-ma         n-darokt-u\\
{\sc 3nsg=erg} khukuri wood axe take\_out{\sc -inf} {\sc 3pl.A-}start{\sc -3.P[pst]}\\
\rede{They started to take out khukuri knives, wooden clubs and axes.} \source{41\_leg\_09.038}

Modifying material, too, can be coordinated by juxtaposition. Interestingly, when two sub-compounds are in apposition, the head noun of the first compound can be omitted, as shown in \Next[c].

\ex.\ag. phu-nuncha    na-nuncha=be     pak=na\\
elder\_brother-younger\_sibling elder\_sister-younger\_sibling{\sc =loc} be\_unripe{\sc =nmlz.sg}\\
\rede{the youngest among the brothers and sisters} \source{40\_leg\_08.052}
\bg. hoŋkhaʔla khi-ma=ha     tu-ma=ha     ceʔya \\
like\_that\_very fight{\sc -inf=nmlz.nc} wrestle{\sc -inf=nmlz.nc} matter\\
\rede{the issue of fighting and wrestling like just told} \source{41\_leg\_09.072}
 \bg.tondigaŋma liŋkhacama-puŋda=ci\\
  a\_forest\_name a\_forest\_name-forest{\sc =nsg}\\
  \rede{the Tondigangma and Linkhacama forests} \source{40\_leg\_08.011}
 
\subsection{Combinatory possibilities}

Concerning the combinatory potential inside the noun phrase, there seem to be only few restrictions. The average noun phrase, however, shows maximally two modi\-fying elements, as illustrated below: {\sc num-adj-N} in \Next[a], {\sc dem-num-N} in \Next[b], {\sc dem-adj-N} in \Next[c], {\sc poss-num-N} in \Next[d], {\sc poss-dem-N} in \Next[e], {\sc dem-Quant } in \Next[f]. Other possibilities found are {\sc poss-adj-N}, {\sc adj-RC-N}, {\sc num-RC-N}, {\sc dem-RC-N}, {\sc poss-num-N}. The only recognizable tendency found was that of putting demonstratives first, although this is not a categorical rule.
 	
 \ex. \ag.eko maɖa tiʔwa\\
 one big pheasant\\
 \rede{one big pheasant} \source{40\_leg\_08.036 }
 \bg.na   eko luŋkhwak=chen\\
 this one stone{\sc =top}\\
 \rede{as for this one stone} \source{37\_nrr\_07.007}
  \bg.na   makhruk=na caleppa \\
this black{\sc =nmlz.sg} bread\\
 \rede{this black bread} \source{40\_leg\_08.053}
  \bg.chubuk=ka    hic=ci     caleppa\\
 ashes{\sc =gen} two{\sc =nsg} bread\\
 \rede{two breads of ashes} \source{40\_leg\_08.071}
  \bg.paghyam=ga    ŋkha sala\\
 old\_man{\sc =gen} that talk\\
 \rede{that talk of the old man} \source{40\_leg\_08.076}
 \bg.      kha  ghak casak\\
 this  all uncooked\_rice\\
 \rede{all this uncooked rice} \source{01\_leg\_07.016}%34_pea_04.035 nhaŋŋa,      hoŋkha    sumbaŋ       mamucijhen
 
 When the noun phrase is headed by a pronoun, only quantifiers or numeral modifiers are possible, and they follow the head, as has been shown above in example \ref{iyaiyadem}.
 
 
From these possibilities, the following (idealized) schema for a maximal noun phrase can be inferred (see \figref{np-max}). As it was said above, the noun phrase is rather unrestricted, so that it is highly conceivable that noun phrases with an internal structure deviating from this schema can be found.
 	 
\begin{figure} 
\begin{tabular}{lllllll} 
&&&&&\textbf{N}&\\ 
\textsc{dem}&\textsc{poss}&NUM&ADJ&RC&\textbf{PRON/}&NUM/\\
&&&&&\textbf{\textsc{dem}}&QUANT\\ 
\end{tabular} 
\caption{Schema of the maximal noun phrase}\label{np-max} 
\end{figure}
 

\subsection{Information structure inside the noun phrase}

When the order of head and attribute is reversed in a noun phrase, one can notice an increase in assertiveness to the right end of the phrase. In \Next[a], for instance, an assertion is made about an old man who has the habit of making jokes, a fact which sets the scene for what is to come: the old man plays a prank at the protagonist of the story. In \Next[b], the asserted information is not so much the fact that a market takes place, because the narrative is temporally embedded in a season known for events such as markets and fun fairs, but rather the fact that it is a comparatively big market. Modifying material to the right of the head noun is restricted to one element (as in Belhare, see \citealt[562]{Bickel2003Belhare}).

\ex. \ag.nna  ighurum=be   a-pum                  laktaŋge=ca        wa-ya=na\\
that round{\sc =loc} {\sc 1sg.poss-}grandfather humorous{\sc =add} exist{\sc [3sg]-pst=nmlz.sg}\\
\rede{In that round, a humorous old man was there, too.} \source{40\_leg\_08.034}
\bg. inimma maŋpha ma=na pog-a-ma\\
market quite big{\sc =nmlz.sg} rise{\sc [3sg]-pst-prf}\\
\rede{Quite a big market took place.}\footnote{The noun \emph{inimma} is a neologism and not widely in use.} \source{01\_leg\_07.145}
  
Elements inside the noun phrase can also be focussed on or topicalized, as the following examples show. In \Next[a] \emph{akkago}  is a contrastive topic, in a (hypothetical) argument where one person brags about how many friends he has in contrast to the other person. In \Next[b], there is a contrastive focus marker inside the noun phrase, added because the assertion is made in contrast to a presupposition claiming that the opposite be true.
  
  \ex. \ag.ak=ka=go              ibebe=ha            ghak kamnibak=ci    khaʔla=hoŋ       ŋ-waiʔ=ya=ci\\
  {\sc 1sg.poss=gen=top} everywhere{\sc =nmlz.nsg} all friend{\sc =nsg} like\_this{\sc =seq} {\sc 3pl-}exist{\sc [npst]=nmlz.nsg=nsg}\\
  \rede{As for mine, I have friends everywhere, like this.} \source{36\_cvs\_06.355}
  \bg.na=go       aniŋ=ga=le                     kham, nniŋda  nhe  wa-ma      n-dokt-wa-m-ga-n=ha\\
 this{\sc =top} {\sc 1pl.excl.poss=gen=ctr} ground {\sc 2pl[erg]} here live{\sc -inf} {\sc neg-}get{\sc-npst-2pl.A-2-neg=nmlz.nsg}\\
  \rede{This is our land, you will not get  the chance to live here.} \source{22\_nrr\_05.012}
 

 
