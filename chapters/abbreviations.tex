\addchap{Abbreviations}
\label{abbreviations} 

\section*{Linguistic abbreviations}
\setlength{\parindent}{0pt}
\begin{multicols}{2}  
\largerpage
\begin{tabular}{lp{4.5cm}} 
1,2,3 	&	 person (1>3: first acting on third person, etc.)\\
A	&	 most agent-like argument of a transitive verb\\
\textsc{abl} &   ablative\\
\textsc{add} &  additive focus\\
\textsc{aff} &   affirmative\\
\textsc{alt} &   alternative\\
\textsc{aux} &  auxiliary verb\\
\textsc{ben} & 		 benefactive \\
\textsc{B.S.} & 		 Bikram Sambat calendar, as used in Nepal\\
\textsc{caus} &   causative\\
\textsc{cl} &  clause linkage marker\\
\textsc{com} &   comitative\\
\textsc{comp} &   complementizer\\
\textsc{compar} &   comparative\\
\textsc{compl} &  completive\\
\textsc{cond} &   conditional\\
\textsc{cont} &  continuative\\
\textsc{cop} &   copula\\
\textsc{ctmp} &  cotemporal (clause linkage)\\
\textsc{ctr} &   contrastive focus\\
\textsc{cvb} &   converb\\
\textsc{du} &  		 dual\\
\textsc{emph} &   emphatic\\
\textsc{erg} &   ergative \\
\textsc{excl} &   exclusive\\
\textsc{excla} &  exclamative\\
G  & most goal-like argument of a three-argument verb 
\end{tabular}

\begin{tabular}{lp{4.5cm}} 
\textsc{gen} &   genitive\\
\textsc{gsr} &   generalized semantic role\\
\textsc{hon} &  honorific\\
\textsc{hort} &   hortative\\
\textsc{rep} &  reportative marker\\
\textsc{ign} &   interjection expressing ignorance\\
\textsc{imp} &   imperative\\
\textsc{incl} &  inclusive\\
\textsc{inf} &   infinitive\\
\textsc{init} &   initiative\\
\textsc{ins} &   instrumental\\
\textsc{insist} &   insistive\\
\textsc{int} &   interjection\\
\textsc{irr} &  irrealis\\
\textsc{itp} &   interruptive clause linkage\\
\textsc{loc} &  	  locative \\
\textsc{mddl} &  middle\\
\textsc{mir} &   mirative\\
\textsc{nativ} &  nativizer\\
\textsc{nsg} &  		 nonsingular\\ 
\textsc{nc} &  non-countable\\
n.a. & not applicable\\
n.d. & no data\\
\textsc{neg} &  	negation\\
Nep. 	& Nepali\\
\textsc{nmlz} &  	nominalizer\\
\textsc{npst} &   non-past\\
\textsc{opt} &   optative\\
P &	 most patient-like argument of a transitive verb\\
\end{tabular}

\begin{tabular}{lp{4.5cm}} 
\textsc{pol} &  politeness\\
\textsc{pl} &  		 plural\\
\textsc{plu.pst} &  plupast\\
\textsc{prf} &  perfect tense\\
\textsc{poss} &   possessive (prefix or pronoun)\\
\textsc{prog} &   progressive\\
\textsc{pst} &   past tense\\
\textsc{pst.prf} &   past perfect\\
\textsc{ptb} &   Proto-Tibeto-Burman\\
\textsc{purp} &  purposive\\
\textsc{q} &   question particle\\
\textsc{quant} &   quantifier\\
\textsc{quot} &   quotative\\
\textsc{RC} &   relative clause\\
\textsc{recip} &   reciprocal\\
\textsc{redup} &  reduplication\\
\textsc{refl} &  reflexive\\
\end{tabular}

\begin{tabular}{lp{4.5cm}} 
\textsc{rep} &   reportative\\
\textsc{restr} &  restrictive focus\\
S 	& sole argument of an intransitive verb\\
\textsc{sbjv} &   subjunctive\\
\textsc{seq} &  sequential (clause linkage)\\
\textsc{sg} &  		 singular\\
\textsc{sim} &   simultaneous\\
\textsc{sup} &   supine\\
T  & most theme-like argument of a three-argument verb\\
\textsc{tag} &   tag question\\
\textsc{temp} &  temporal\\
\textsc{top} &   topic particle\\
\textsc{tripl} &  triplication\\
\textsc{V2} &  function verb (in complex predication)\\
\textsc{voc} &  vocative\\
\setlength{\parindent}{10pt}
\end{tabular}
\end{multicols}

\section*{Abbreviations of kinship terms}
\begin{multicols}{2} 
\setlength{\parindent}{0pt}

\begin{tabular}{lp{4.5cm}} 
B & brother\\
BS &  brother's son\\
BD &  brother's daughter\\
BW &  brother's wife\\
e  & elder\\
D &  daughter\\
F &   father\\
FB &   father's brother\\
FF &   father's father\\
FM &   father's mother\\
FZ &   father's sister\\
H  & husband\\
\end{tabular}

\begin{tabular}{lp{4.5cm}} 
M  &  mother\\
MB &   mother's brother\\
MF &   mother's father\\
MM &   mother's mother\\
MZ &   mother's sister\\
S  &  son\\
W  & wife\\
y  &  younger\\
Z  &  sister\\
ZS &  sister's son\\
ZD &  sister's daughter\\
ZH &  sister's husband\\
\end{tabular}
\setlength{\parindent}{10pt}

\end{multicols}

 