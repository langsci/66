\addchap{List of abbreviations}
\label{abbreviations}
\begin{refsection}

\section*{Linguistic abbreviations}

 {\small
\begin{tabbing}
banlakjhaldjkf\= \kill

1,2,3 		\> person (1>3: first acting on third person, etc.)\\
{\sc sg/du/pl/nsg} 		\> numerus: singular, dual, plural, nonsingular\\
A		\> most agent-like argument of a transitive verb\\
{\sc abl} \> ablative\\
{\sc add}\> additive focus\\
{\sc aff} \> affirmative\\
{\sc alt} \> alternative\\
{\sc aux}\> auxiliary verb\\
{\sc ben}		\> benefactive \\
{\sc B.S.}		\> Bikram Sambat calender, as used in Nepal\\
{\sc caus} \> causative\\
{\sc cl}\> clause linkage marker\\
{\sc com} \> comitative\\
{\sc comp} \> complementizer\\
{\sc compar} \> comparative\\
{\sc compl}\> completive\\
{\sc cond} \> conditional\\
{\sc cont}\>continuative\\
{\sc cop} \> copula\\
{\sc ctmp}\>cotemporal (clause linkage)\\
{\sc ctr} \> contrastive focus\\
{\sc cvb} \> converb\\
{\sc emph} \> emphatic\\
{\sc erg} \> ergative \\
{\sc excl} \> exclusive\\
{\sc excla}\> exclamative\\
G \> most goal-like argument of a three-argument verb\\
{\sc gen} \> genitive\\
{\sc gsr} \> generalized semantic role\\
{\sc hon}\> honorific\\
{\sc hort} \> hortative\\
{\sc rep}\> reportative marker\\
{\sc ign}\>  interjection expressing ignorance\\
{\sc imp} \> imperative\\
{\sc incl}\> inclusive\\
{\sc inf} \> infinitive\\
{\sc init} \> initiative\\
{\sc ins} \> instrumental\\
{\sc insist} \> insistive\\
{\sc int} \> interjection\\
{\sc irr}\>irrealis\\
{\sc itp} \> interruptive clause linkage\\
{\sc loc} 	 \> locative \\
{\sc mddl}\>middle\\
{\sc mir} \> mirative\\
{\sc nativ}\>nativizer\\
{\sc nc}\>non-countable\\
n.a.\>not applicable\\
n.d.\>no data\\
{\sc neg} \>	negation\\
Nep. \>	Nepali\\
{\sc nmlz} \>	nominalizer\\
{\sc npst} \> non-past\\
{\sc opt} \> optative\\
P 	\> most patient-like argument of a transitive verb\\
{\sc pol}\>politeness\\
{\sc plu.pst}\>plupast\\
{\sc prf}\> perfect tense\\
{\sc poss} \> possessive (prefix or pronoun)\\
{\sc prog} \> progressive\\
{\sc pst} \> past tense\\
{\sc pst.prf} \> past perfect\\
\textsc{ptb} \> Proto-Tibeto-Burman\\
{\sc purp}\> purposive\\
{\sc q} \> question particle\\
{\sc quant} \> quantifier\\
{\sc quot} \> quotative\\
{\sc RC} \> relative clause\\
{\sc recip} \> reciprocal\\
{\sc redup}\> reduplication\\
{\sc refl}\> reflexive\\
{\sc rep} \> reportative\\
{\sc restr}\> restrictive focus\\
S 	\> sole argument of an intransitive verb\\
{\sc sbjv} \> subjunctive\\
{\sc seq}\> sequential (clause linkage)\\
{\sc sim} \> simultaneous\\
{\sc sup} \> supine\\
T \> most theme-like argument of a three-argument verb\\
{\sc tag} \> tag question\\
{\sc temp}\> temporal\\
{\sc top} \> topic particle\\
{\sc tripl}\> triplication\\
{\sc V2}\> function verb (in complex predication)\\
{\sc voc}\> vocative\\
\end{tabbing}
}



\section*{Abbreviations of kinship terms}

 {\small
\begin{tabbing}
banlakjhaldjkf\= \kill 

B\> brother\\
BS\>brother's son\\
BD\>brother's daughter\\
BW\>brother's wife\\
e\> elder\\
D\> daughter\\
F \> father\\
FB \> father's brother\\
FF \> father's father\\
FM \> father's mother\\
FZ \> father's sister\\
H\> husband\\
M \> mother\\
MB \> mother's brother\\
MF \> mother's father\\
MM \> mother's mother\\
MZ \> mother's sister\\
S \> son\\
W\>wife\\
y \> younger\\
Z \> sister\\
ZS\>sister's son\\
ZD\>sister's daughter\\
ZH\> sister's husband\\

\end{tabbing}
}

\printbibliography[heading=subbibliography]
\end{refsection}

