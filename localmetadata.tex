
%%%%%%%%%%%%%%%%%%%%%%%%%%%%%%%%%%%%%%%%%%%%%%%%%%%%
%%%                                              %%%
%%%                 Metadata                     %%%
%%%          fill in as appropriate              %%%
%%%                                              %%%
%%%%%%%%%%%%%%%%%%%%%%%%%%%%%%%%%%%%%%%%%%%%%%%%%%%%

\title{A grammar of Yakkha}  %look no further, you can change those things right here.
\subtitle{}%add a subtitle between the braces if you have one
\BackTitle{A grammar of Yakkha}
\BackBody{This grammar provides the first comprehensive grammatical description of Yakkha, a Sino-Tibetan language of the Kiranti branch. Yakkha is spoken by about 14,000 speakers in eastern Nepal, in the Sankhuwa Sabha and Dhankuta districts. The grammar is based on original fieldwork in the Yakkha community. Its primary source of data is a corpus of 13,000 clauses from narratives and naturally-occurring social interaction which the author recorded and transcribed between 2009 and 2012. Corpus analyses were complemented by targeted elicitation. The grammar is written in a functional-typological framework. It focusses on morphosyntactic and semantic issues, as these present highly complex and comparatively under-researched fields in Kiranti languages. The sequence of the chapters follows the well-established order of phonological, morphological, syntactic and discourse-structural descriptions. These are supplemented by a historical and sociolinguistic introduction as well as an analysis of the complex kinship terminology. Topics such as verbal person marking, argument structure, transitivity, complex predication, grammatical relations, clause linkage, nominalization, and the topography-based orientation system have received in-depth treatment. Wherever possible, the structures found were explained in a historical-comparative perspective in order to shed more light on how their particular properties have emerged. }
%\dedication{Change dedication in localmetadata.tex}
\typesetter{Diana Schackow, Sebastian Nordhoff, Lennart Bierkandt, Felix Kopecky}
\proofreader{%
Slavomir \v{C}éplö,
Christian Döhler,
Joseph Farquharson,
Constantin Freitag,
Tom Gardner,
Eitan Grossman,
Andreas Hölzl,
Charles Ka Shing Ko,
Linda Lanz,
Timm Lichte,
Michelle Natolo,
Stephanie Natolo,
Conor Pyle,
Benjamin Saade,
Aviva Shimelman,
Aaron Sonnenschein,
João Veloso
}

 
\author{Diana Schackow}
\renewcommand{\lsSeries}{sidl} % use lowercase acronym, e.g. sidl, eotms, tgdi
\renewcommand{\lsSeriesNumber}{7} %will be assigned when the book enters the proofreading stage
\renewcommand{\lsURL}{http://langsci-press.org/catalog/book/66} % contact the coordinator for the right number

\renewcommand{\lsISBNdigital}{978-3-946234-11-1}
\renewcommand{\lsISBNhardcover}{978-3-946234-12-8}
\renewcommand{\lsISBNsoftcover}{978-3-946234-13-5}
