%%%%%%%%%%%%%%%%%%%%%%%%%%%%%%%%%%%%%%%%%%%%%%%%%%%%
%%%                                              %%%
%%%     Language Science Press Master File       %%%
%%%         follow the instructions below        %%%
%%%                                              %%%
%%%%%%%%%%%%%%%%%%%%%%%%%%%%%%%%%%%%%%%%%%%%%%%%%%%%
 
% Everything following a % is ignored
% Some lines start with %. Remove the % to include them

\documentclass[output=long             % long|short|inprep              
% 	        ,blackandwhite
% 		,smallfont
%  	        ,draftmode  
		,biblatex
		  ]{LSP/langsci}    
  
%%%%%%%%%%%%%%%%%%%%%%%%%%%%%%%%%%%%%%%%%%%%%%%%%%%%
%%%                                              %%%
%%%          additional packages                 %%%
%%%                                              %%%
%%%%%%%%%%%%%%%%%%%%%%%%%%%%%%%%%%%%%%%%%%%%%%%%%%%%

% put all additional commands you need in the 
% following files. If you do not know what this might 
% mean, you can safely ignore this section

\input{localmetadata.tex}
% add all extra packages you need to load to this file  
\usepackage{tabularx}   
\usepackage{lscape}
\usepackage{xunicode} % tex-kuerzel in unicode-zeichen umwandeln
\usepackage{linguex} % sternefeld-packung für glossen  
\usepackage{vowel}
\usepackage{qtree}
\usepackage{graphicx}
\usepackage{longtable}
\usepackage{colortbl}
\usepackage[colorlinks=true,allcolors=black]{hyperref}
\usepackage{tabularx}
\usepackage{pdfpages}



\usepackage{scrpage2}
\usepackage{multirow}
\usepackage{tikz}
\usepackage{tablefootnote}% command has the same name
%%%%%%%%%%%%%%%%%%%%%%%%%%%%%%%%%%%%%%%%%%%%%%%%%%%%
%%%                                              %%%
%%%           Examples                           %%%
%%%                                              %%%
%%%%%%%%%%%%%%%%%%%%%%%%%%%%%%%%%%%%%%%%%%%%%%%%%%%% 
%% to add additional information to the right of examples, uncomment the following line
% \usepackage{jambox}
%% if you want the source line of examples to be in italics, uncomment the following line
% \renewcommand{\exfont}{\itshape}



\input{localhyphenation.tex}
\input{localcommands.tex} 
\bibliography{MyBib} 

%%%%%%%%%%%%%%%%%%%%%%%%%%%%%%%%%%%%%%%%%%%%%%%%%%%%
%%%                                              %%%
%%%             Frontmatter                      %%%
%%%                                              %%%
%%%%%%%%%%%%%%%%%%%%%%%%%%%%%%%%%%%%%%%%%%%%%%%%%%%%
\begin{document}              
\maketitle                
\frontmatter
%% uncomment if you have preface and/or acknowledgements
%\include{chapters/preface}
\include{chapters/acknowledgments}
\addchap{List of abbreviations}
\label{abbreviations}
\begin{refsection}

\section*{Linguistic abbreviations}
\begin{multicols}{2}\enlargethispage{2\baselineskip}
\setlength{\parindent}{0pt}
1,2,3 		 person (1>3: first acting on third person, etc.)\\
{\sc sg/du/pl/nsg} 		 numerus: singular, dual, plural, nonsingular\\
A		 most agent-like argument of a transitive verb\\
{\sc abl}  ablative\\
{\sc add} additive focus\\
{\sc aff}  affirmative\\
{\sc alt}  alternative\\
{\sc aux} auxiliary verb\\
{\sc ben}		 benefactive \\
{\sc B.S.}		 Bikram Sambat calender, as used in Nepal\\
{\sc caus}  causative\\
{\sc cl} clause linkage marker\\
{\sc com}  comitative\\
{\sc comp}  complementizer\\
{\sc compar}  comparative\\
{\sc compl} completive\\
{\sc cond}  conditional\\
{\sc cont} continuative\\
{\sc cop}  copula\\
{\sc ctmp} cotemporal (clause linkage)\\
{\sc ctr}  contrastive focus\\
{\sc cvb}  converb\\
{\sc emph}  emphatic\\
{\sc erg}  ergative \\
{\sc excl}  exclusive\\
{\sc excla} exclamative\\
G  most goal-like argument of a three-argument verb\\
{\sc gen}  genitive\\
{\sc gsr}  generalized semantic role\\
{\sc hon} honorific\\
{\sc hort}  hortative\\
{\sc rep} reportative marker\\
{\sc ign}  interjection expressing ignorance\\
{\sc imp}  imperative\\
{\sc incl} inclusive\\
{\sc inf}  infinitive\\
{\sc init}  initiative\\
{\sc ins}  instrumental\\
{\sc insist}  insistive\\
{\sc int}  interjection\\
{\sc irr} irrealis\\
{\sc itp}  interruptive clause linkage\\
{\sc loc} 	  locative \\
{\sc mddl} middle\\
{\sc mir}  mirative\\
{\sc nativ} nativizer\\
{\sc nc} non-countable\\
n.a. not applicable\\
n.d. no data\\
{\sc neg} 	negation\\
Nep. 	Nepali\\
{\sc nmlz} 	nominalizer\\
{\sc npst}  non-past\\
{\sc opt}  optative\\
P 	 most patient-like argument of a transitive verb\\
{\sc pol} politeness\\
{\sc plu.pst} plupast\\
{\sc prf} perfect tense\\
{\sc poss}  possessive (prefix or pronoun)\\
{\sc prog}  progressive\\
{\sc pst}  past tense\\
{\sc pst.prf}  past perfect\\
\textsc{ptb}  Proto-Tibeto-Burman\\
{\sc purp} purposive\\
{\sc q}  question particle\\
{\sc quant}  quantifier\\
{\sc quot}  quotative\\
{\sc RC}  relative clause\\
{\sc recip}  reciprocal\\
{\sc redup} reduplication\\
{\sc refl} reflexive\\
{\sc rep}  reportative\\
{\sc restr} restrictive focus\\
S 	 sole argument of an intransitive verb\\
{\sc sbjv}  subjunctive\\
{\sc seq} sequential (clause linkage)\\
{\sc sim}  simultaneous\\
{\sc sup}  supine\\
T  most theme-like argument of a three-argument verb\\
{\sc tag}  tag question\\
{\sc temp} temporal\\
{\sc top}  topic particle\\
{\sc tripl} triplication\\
{\sc V2} function verb (in complex predication)\\
{\sc voc} vocative\\
\setlength{\parindent}{10pt}
\end{multicols}

\section*{Abbreviations of kinship terms}
\begin{multicols}{2}\enlargethispage{2\baselineskip}
\setlength{\parindent}{0pt}
B brother\\
BS brother's son\\
BD brother's daughter\\
BW brother's wife\\
e elder\\
D daughter\\
F  father\\
FB  father's brother\\
FF  father's father\\
FM  father's mother\\
FZ  father's sister\\
H husband\\
M  mother\\
MB  mother's brother\\
MF  mother's father\\
MM  mother's mother\\
MZ  mother's sister\\
S  son\\
W wife\\
y  younger\\
Z  sister\\
ZS sister's son\\
ZD sister's daughter\\
ZH sister's husband\\
\setlength{\parindent}{10pt}
\end{multicols}


\printbibliography[heading=subbibliography]
\end{refsection}
\tableofcontents      
 \mainmatter         
 

%%%%%%%%%%%%%%%%%%%%%%%%%%%%%%%%%%%%%%%%%%%%%%%%%%%%
%%%                                              %%%
%%%             Chapters                         %%%
%%%                                              %%%
%%%%%%%%%%%%%%%%%%%%%%%%%%%%%%%%%%%%%%%%%%%%%%%%%%%%

%  \input{chapters/02_Frontmatter.tex}
%  \input{chapters/03_How-to-use.tex}
%  \input{chapters/03_Intro_Yakkha.tex}
 %%\input{chapters/03_Typological_Profile.tex}
%  \input{chapters/04_Phonology.tex}
 \input{chapters/06_Pronouns_etc.tex}
 \input{chapters/06_NounPhrase.tex}
 \input{chapters/09_AdjectivesAdverbs.tex}
 \input{chapters/09_Geomorphic.tex}
 \input{chapters/07_VerbalMorphology.tex}
 \input{chapters/07_VerbalMorphology_b.tex}
 \input{chapters/08_NounVerb.tex}
 \input{chapters/08_ComplexPredicates.tex}
 \input{chapters/10a_VerbalValency.tex}
%  \input{chapters/10a_ValClasses.tex}
%  \input{chapters/10b_TransitivityOperations.tex}
%  \input{chapters/10b_TransitivityOperations_b.tex}
%  \input{chapters/11_SimpleClause.tex}
%  \input{chapters/12_Nomlz-Relativization.tex}
%  \input{chapters/13_Converbs-AdverbialClauses.tex}
%  \input{chapters/15_Complement.tex}
%  \input{chapters/14_Coordination.tex}
%  \input{chapters/18_Particles.tex}
%  \input{chapters/20_app_phopciba.tex}
%  \input{chapters/20_app_namthalungma.tex}
%  \input{chapters/20_app_linkhedhunga.tex}

\cleardoubleemptypage
\addcontentsline{toc}{chapter}{Appendix B: Yakkha kinship terms}
\includepdf{figures/90Grad_Verwandtschaft_eigeneFamilie_zum_Einbinden.pdf}
\cleardoubleemptypage
\includepdf{figures/90Grad_Verwandtschaft_Schwiegerfamilie_zum_Einbinden.pdf}


\input{chapters/21_affix_index.tex}


\input{chapters/19_Nocite.tex}

 

%%%%%%%%%%%%%%%%%%%%%%%%%%%%%%%%%%%%%%%%%%%%%%%%%%%%
%%%                                              %%%
%%%             Backmatter                       %%%
%%%                                              %%%
%%%%%%%%%%%%%%%%%%%%%%%%%%%%%%%%%%%%%%%%%%%%%%%%%%%%

% There is normally no need to change the backmatter section
\input{backmatter.tex}
\end{document} 


% you can create your book by running
% xelatex lsp-skeleton.tex
%
% you can also try a simple 
% make
% on the commandline
